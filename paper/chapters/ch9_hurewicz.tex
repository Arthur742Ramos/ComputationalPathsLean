%% ============================================================================
%% Chapter 9: The Hurewicz Theorem and Homological Algebra
%% Part II of "The Algebra of Computational Paths"
%% ============================================================================

\section{The Hurewicz Theorem and Homological Algebra}\label{sec:hurewicz}

The Hurewicz theorem establishes the fundamental connection between homotopy groups and homology, relating $\pi_1$ to the first homology group $H_1$ via abelianization. In the computational-paths framework, both sides of this connection are constructively defined: $\pi_1$ via the loop quotient, and $H_1$ as its abelianization.

\subsection{Abelianization}\label{subsec:abelianization}

\begin{definition}[Commutator]\label{def:commutator}
For a group $(G, \cdot, {}^{-1}, e)$, the \emph{commutator} of elements $a, b \in G$ is
\[
  [a, b] = a \cdot b \cdot a^{-1} \cdot b^{-1}.
\]
The \emph{commutator subgroup} $[G, G]$ is the normal subgroup generated by all commutators.
\end{definition}

\begin{definition}[Abelianization]\label{def:abelianization}
The \emph{abelianization} of a group $G$ is the quotient
\[
  G^{\mathrm{ab}} = G \;/\; [G, G],
\]
defined as the quotient by the equivalence relation generated by:
\begin{enumerate}
  \item \emph{Commutativity}: $a \cdot b \sim b \cdot a$ for all $a, b \in G$;
  \item \emph{Congruence}: if $x \sim y$ then $z \cdot x \sim z \cdot y$ and $x \cdot z \sim y \cdot z$;
  \item \emph{Group laws}: associativity, identity, and inverse laws.
\end{enumerate}
The quotient map $\eta : G \to G^{\mathrm{ab}}$ is the universal homomorphism from $G$ to an abelian group.
\end{definition}

\begin{proposition}[Universal property]\label{prop:abelianization-up}
For any abelian group $H$ and any homomorphism $\varphi : G \to H$, there exists a unique homomorphism $\bar{\varphi} : G^{\mathrm{ab}} \to H$ with $\bar{\varphi} \circ \eta = \varphi$.
\end{proposition}

\begin{theorem}[Commutators vanish in the abelianization]\label{thm:commutator-trivial}
For all $a, b \in G$, $\eta([a, b]) = e$ in $G^{\mathrm{ab}}$.
\end{theorem}

\begin{proof}
We compute in the abelianization, where $\sim$ denotes the abelianization relation:
\begin{align*}
  [a, b] &= (a \cdot b) \cdot (a^{-1} \cdot b^{-1}) \\
         &\sim (b \cdot a) \cdot (a^{-1} \cdot b^{-1}) && \text{(commutativity of } a \cdot b\text{)} \\
         &\sim b \cdot (a \cdot (a^{-1} \cdot b^{-1})) && \text{(associativity)} \\
         &\sim b \cdot ((a \cdot a^{-1}) \cdot b^{-1}) && \text{(associativity)} \\
         &\sim b \cdot (e \cdot b^{-1}) && \text{(right inverse)} \\
         &\sim b \cdot b^{-1} && \text{(left identity)} \\
         &\sim e. && \text{(right inverse)}
\end{align*}
Each step uses one of the abelianization relation generators.
\end{proof}

\subsection{First Homology via the Hurewicz Map}\label{subsec:hurewicz-map}

\begin{definition}[First homology group]\label{def:h1}
For a pointed type $(A, a)$, the \emph{first homology group} is defined as
\[
  H_1(A) \;=\; \pi_1(A, a)^{\mathrm{ab}}.
\]
This is the abelianization of the fundamental group.
\end{definition}

\begin{definition}[Hurewicz map]\label{def:hurewicz-map}
The \emph{Hurewicz homomorphism} is the quotient map
\[
  h : \pi_1(A, a) \to H_1(A), \qquad h(\alpha) = \eta(\alpha),
\]
where $\eta$ is the abelianization projection.
\end{definition}

\begin{theorem}[Hurewicz theorem, dimension~1]\label{thm:hurewicz}
For any pointed type $(A, a)$, the Hurewicz map $h$ induces an isomorphism
\[
  \pi_1(A, a)^{\mathrm{ab}} \;\cong\; H_1(A).
\]
Moreover:
\begin{enumerate}
  \item $h$ is surjective: every element of $H_1(A)$ is the image of some $\alpha \in \pi_1(A, a)$.
  \item $\ker(h) = [\pi_1(A, a), \pi_1(A, a)]$: an element $\alpha$ maps to $e \in H_1$ if and only if $\alpha$ is a product of commutators.
\end{enumerate}
\end{theorem}

\begin{proof}
The isomorphism is the identity on the underlying quotient type: $H_1(A) = \pi_1(A, a)^{\mathrm{ab}}$ by definition. Surjectivity follows because the abelianization map $\eta$ is surjective (every equivalence class has a representative). The kernel characterization follows from the definition of the abelianization relation: $\eta(\alpha) = \eta(e)$ if and only if $\alpha \sim e$ in the abelianization, which occurs precisely when $\alpha \in [G, G]$.
\end{proof}

\subsection{Examples}\label{subsec:hurewicz-examples}

\begin{example}[Circle]\label{ex:h1-circle}
$H_1(S^1) \cong \pi_1(S^1)^{\mathrm{ab}} \cong \mathbb{Z}^{\mathrm{ab}} \cong \mathbb{Z}$.
Since $\mathbb{Z}$ is already abelian, the Hurewicz map is an isomorphism.
\end{example}

\begin{example}[Torus]\label{ex:h1-torus}
$H_1(T^2) \cong (\mathbb{Z} \times \mathbb{Z})^{\mathrm{ab}} \cong \mathbb{Z} \times \mathbb{Z}$.
The product of abelian groups is abelian, so again $h$ is an isomorphism.
\end{example}

\begin{example}[Figure-eight]\label{ex:h1-figure-eight}
$H_1(S^1 \vee S^1) \cong (\mathbb{Z} * \mathbb{Z})^{\mathrm{ab}} \cong \mathbb{Z} \times \mathbb{Z}$.
Here $\pi_1 = \mathbb{Z} * \mathbb{Z}$ is non-abelian, but its abelianization is $\mathbb{Z}^2$. The Hurewicz map is not injective: the commutator $[a, b] = aba^{-1}b^{-1}$ is a non-trivial element of $\pi_1$ that maps to $0$ in $H_1$.
\end{example}

More generally, the abelianization of a free product is the direct product of the abelianizations:

\begin{proposition}\label{prop:free-product-abelianization}
$(G * H)^{\mathrm{ab}} \cong G^{\mathrm{ab}} \times H^{\mathrm{ab}}$.
\end{proposition}

\begin{proof}
Elements of $G * H$ are reduced alternating words in $G$ and $H$. In the abelianization, commutativity allows rewriting any word to gather all $G$-elements on the left and all $H$-elements on the right, yielding a pair $(g, h) \in G^{\mathrm{ab}} \times H^{\mathrm{ab}}$. The map from $(G * H)^{\mathrm{ab}}$ to $G^{\mathrm{ab}} \times H^{\mathrm{ab}}$ sends a word to the pair of component-wise sums; the inverse sends $(g, h)$ to the class of the two-letter word $g \cdot h$. These are mutually inverse homomorphisms.
\end{proof}

\begin{example}[Surface of genus~$g$]\label{ex:h1-surface}
For a closed orientable surface $\Sigma_g$ of genus $g \geq 2$,
\[
  \pi_1(\Sigma_g) = \langle a_1, b_1, \ldots, a_g, b_g \mid [a_1, b_1] \cdots [a_g, b_g] = e \rangle,
\]
and $H_1(\Sigma_g) \cong \pi_1(\Sigma_g)^{\mathrm{ab}} \cong \mathbb{Z}^{2g}$, since the relation $[a_1, b_1] \cdots [a_g, b_g] = e$ becomes trivial after abelianization.
\end{example}

\subsection{Free Groups and the Nielsen--Schreier Theorem}\label{subsec:free-groups}

The computational-paths framework provides a constructive verification of the universal property of free groups.

\begin{theorem}[Free group universal property]\label{thm:free-group-up}
The loop group $\pi_1\bigl(\bigvee_n S^1, \mathsf{base}\bigr)$ satisfies the universal property of the free group $F_n$: for any group $G$ and any function $\varphi : \{1, \ldots, n\} \to G$, there exists a unique group homomorphism $\bar{\varphi} : F_n \to G$ extending $\varphi$.
\end{theorem}

\begin{proof}
The homomorphism $\bar{\varphi}$ is defined on reduced words by substituting each generator $g_i$ with $\varphi(i)$ and each $g_i^{-1}$ with $\varphi(i)^{-1}$, then evaluating the resulting product in $G$. Uniqueness follows because a homomorphism is determined by its values on generators.
\end{proof}

\begin{theorem}[Nielsen--Schreier]\label{thm:nielsen-schreier}
Every subgroup of a free group is free.
\end{theorem}

\begin{proof}[Proof sketch]
In the computational-paths framework, a subgroup $H \leq F_n = \pi_1(\bigvee_n S^1)$ corresponds to a connected covering space of $\bigvee_n S^1$ (by the Galois correspondence, Theorem~\ref{thm:galois}). A connected covering of a graph (1-dimensional CW complex) is again a graph, and $\pi_1$ of a graph is free. Hence $H$ is free.
\end{proof}

\begin{corollary}\label{cor:free-abelianization}
$F_n^{\mathrm{ab}} \cong \mathbb{Z}^n$.
\end{corollary}

\begin{proof}
The free group on $n$ generators has abelianization $\mathbb{Z}^n$ because the generators commute in the abelianization. Equivalently, this follows from Proposition~\ref{prop:free-product-abelianization} by induction: $F_n = \mathbb{Z} * \cdots * \mathbb{Z}$ ($n$ times), and $(\mathbb{Z} * \cdots * \mathbb{Z})^{\mathrm{ab}} = \mathbb{Z}^n$.
\end{proof}

\subsection{The Higher Hurewicz Theorem}\label{subsec:higher-hurewicz}

\begin{theorem}[Higher Hurewicz, statement]\label{thm:higher-hurewicz}
For $n \geq 2$, if a pointed type $(A, a)$ is $(n-1)$-connected (i.e., $\pi_k(A, a) = 0$ for all $k < n$), then the Hurewicz map
\[
  h_n : \pi_n(A, a) \to H_n(A)
\]
is an isomorphism.
\end{theorem}

The full proof of the higher Hurewicz theorem requires the development of singular homology or an equivalent construction, which goes beyond the primary scope of the computational-paths framework as formalized. We record the statement and its key consequences.

\begin{corollary}\label{cor:sphere-homology}
$H_n(S^n) \cong \mathbb{Z}$ for $n \geq 1$.
\end{corollary}

\begin{proof}
$S^n$ is $(n-1)$-connected (Corollary~\ref{cor:pi1-higher-spheres} and its higher analogues). By the higher Hurewicz theorem, $H_n(S^n) \cong \pi_n(S^n)$. By the Freudenthal suspension theorem (Theorem~\ref{thm:freudenthal} below), $\pi_n(S^n) \cong \mathbb{Z}$.
\end{proof}

\subsection{Simple Connectivity and Triviality}\label{subsec:simply-connected}

\begin{definition}[Simply connected]\label{def:simply-connected}
A pointed type $(A, a)$ is \emph{simply connected} if $\pi_1(A, a) = \{e\}$, i.e., every loop at $a$ is rewrite-equivalent to $\mathsf{refl}(a)$.
\end{definition}

\begin{proposition}\label{prop:simply-connected-h1}
If $(A, a)$ is simply connected, then $H_1(A) = 0$.
\end{proposition}

\begin{proof}
$H_1(A) = \pi_1(A, a)^{\mathrm{ab}} = \{e\}^{\mathrm{ab}} = \{e\}$.
\end{proof}

\begin{proposition}[Detection principle]\label{prop:detection}
If $H_1(A) \neq 0$, then $\pi_1(A, a) \neq \{e\}$, i.e., $A$ is not simply connected.
\end{proposition}

\begin{proof}
This is the contrapositive of Proposition~\ref{prop:simply-connected-h1}.
\end{proof}

The detection principle provides an effective method for proving that a space has non-trivial fundamental group: it suffices to exhibit a non-trivial element in the abelianization, which is often easier to compute than the full $\pi_1$.
