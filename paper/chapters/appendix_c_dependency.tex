% ============================================================================
% Appendix C: Dependency Graph
% ============================================================================
\chapter{Dependency Graph}
\label{app:dependency}

The formalization comprises 516 Lean~4 source files organized into a
layered dependency structure. This appendix describes the principal
layers and their inter-dependencies. A schematic representation is given
in Figure~\ref{fig:dependency}.

\section*{Layer 0: Basic Path Algebra}

The foundation consists of the core path definitions and their algebraic
laws:

\begin{center}
\small
\begin{tabular}{lp{8cm}}
\texttt{Path.Basic} & Path record type, elementary operations (refl,
  symm, trans, ofEq, toEq), strict monoid/involution/anti-homomorphism
  laws. \\
\texttt{Path.Basic.Congruence} & Unary congruence (congrArg), binary
  congruence (map2, mapLeft, mapRight), and functoriality theorems. \\
\texttt{Path.Basic.Context} & Context, BiContext, DepContext,
  DepBiContext structures and their substitution operations. \\
\texttt{Path.Basic.Transport} & Transport, dependent application (apd),
  and their laws. \\
\texttt{Path.Basic.Prod} & Product type path operations and
  $\beta$/$\eta$-rules. \\
\texttt{Path.Basic.Sigma} & Sigma type path operations. \\
\texttt{Path.Basic.Sum} & Sum type path operations. \\
\texttt{Path.Basic.Fun} & Function type path operations (lamCongr,
  app). \\
\end{tabular}
\end{center}

\section*{Layer 1: The Rewrite System}

Built on Layer~0, this layer contains the TRS infrastructure:

\begin{center}
\small
\begin{tabular}{lp{8cm}}
\texttt{Path.Rewrite.Step} & The 75-rule \texttt{Step} inductive type
  and soundness proof. \\
\texttt{Path.Rewrite.Rw} & Reflexive--transitive closure. \\
\texttt{Path.Rewrite.RwEq} & Rewrite equality and all congruence
  lemmas. \\
\texttt{Path.Rewrite.LNDEQ} & Rule enumeration and mnemonic names. \\
\texttt{Path.Rewrite.Normalization} & Normal forms. \\
\texttt{Path.Rewrite.Termination} & Rule precedence and RPO. \\
\texttt{Path.Rewrite.StripLemma} & Local confluence. \\
\texttt{Path.Rewrite.Confluence*} & Global confluence and join
  witnesses. \\
\texttt{Path.Rewrite.PathExpr} & Syntactic path expressions. \\
\texttt{Path.Rewrite.Quot} & The quotient $\PathQuot$. \\
\texttt{Path.Rewrite.PathTactic} & Proof automation. \\
\end{tabular}
\end{center}

\section*{Layer 2: Groupoid and Higher-Dimensional Structure}

Built on Layers~0--1:

\begin{center}
\small
\begin{tabular}{lp{8cm}}
\texttt{Groupoid.*} & Weak and strict category/groupoid structures,
  rewrite lifts, enriched and double groupoid, symmetric monoidal
  structure. \\
\texttt{HigherDimensional.*} & Globular tower, two-cells, Derivation$_2$
  through Derivation$_4$ and DerivationHigh, contractibility theorems,
  $\omega$-groupoid assembly. \\
\end{tabular}
\end{center}

\section*{Layer 3: Homotopy Theory}

Built on Layers~0--2:

\begin{center}
\small
\begin{tabular}{lp{8cm}}
\texttt{HomotopyTheory.LoopSpace} & Loop spaces, iterated loop spaces,
  loop quotient, loop group. \\
\texttt{HomotopyTheory.PiOne} & Fundamental group, induced
  homomorphisms, functoriality. \\
\texttt{HomotopyTheory.FundamentalGroupoid} & The fundamental groupoid
  $\Pi_1(A)$. \\
\texttt{HomotopyTheory.EckmannHilton} & The Eckmann--Hilton argument
  and commutativity of $\pi_n$ for $n \ge 2$. \\
\texttt{HomotopyTheory.Circle} & The computational circle and
  $\pi_1(S^1) \cong \ZZ$. \\
\texttt{HomotopyTheory.Torus} & The torus and its fundamental group. \\
\texttt{HomotopyTheory.FigureEight} & The figure-eight, free products,
  bouquets. \\
\texttt{HomotopyTheory.VanKampen} & Seifert--van Kampen theorem. \\
\texttt{HomotopyTheory.Suspension} & Suspensions, spheres,
  Freudenthal. \\
\texttt{HomotopyTheory.KleinBottle} & Klein bottle,
  $\pi_1 \cong \ZZ \rtimes \ZZ$. \\
\texttt{HomotopyTheory.LensSpace} & Lens spaces and projective
  spaces. \\
\end{tabular}
\end{center}

\section*{Layer 4: Fibrations and Exact Sequences}

Built on Layers~0--3:

\begin{center}
\small
\begin{tabular}{lp{8cm}}
\texttt{Fibration.*} & Fiber, path lifting, fiber transport, fiber
  sequences, covering spaces, Hopf fibration. \\
\texttt{Fibration.LES} & Long exact sequence of homotopy groups. \\
\texttt{Fibration.MayerVietoris} & Mayer--Vietoris sequence. \\
\end{tabular}
\end{center}

\section*{Layer 5: Homological Algebra and Advanced Topics}

Built on Layers~0--4:

\begin{center}
\small
\begin{tabular}{lp{8cm}}
\texttt{Homological.*} & Abelianization, commutators, Hurewicz map and
  theorem, free groups, resolutions, Ext/Tor. \\
\texttt{Advanced.*} & Eilenberg--MacLane spaces, Postnikov towers,
  obstruction theory, Whitehead theorem, spectral sequences,
  characteristic classes, operads, K-theory, stable homotopy. \\
\end{tabular}
\end{center}

\section*{Dependency Diagram}

\begin{figure}[ht]
\centering
\begin{tikzcd}[column sep=small, row sep=large]
  & \fbox{\texttt{Path.Basic}} \ar[dl] \ar[d] \ar[dr] & \\
  \fbox{\texttt{Rewrite.Step}} \ar[d]
  & \fbox{\texttt{Rewrite.PathExpr}} \ar[d]
  & \fbox{\texttt{Basic.Context}} \ar[dl] \\
  \fbox{\texttt{Rw $\to$ RwEq}} \ar[d] \ar[dr]
  & \fbox{\texttt{ExprConfluence}} \ar[dl] & \\
  \fbox{\texttt{Confluence}} \ar[d]
  & \fbox{\texttt{Normalization}} \ar[l] & \\
  \fbox{\texttt{Quot}} \ar[d] \ar[dr] & & \\
  \fbox{\texttt{Groupoid}} \ar[d]
  & \fbox{\texttt{HigherDimensional}} \ar[d] & \\
  \fbox{\texttt{HomotopyTheory}} \ar[d] \ar[dr]
  & \fbox{\texttt{$\omega$-Groupoid}} \ar[l] & \\
  \fbox{\texttt{Fibration}} \ar[d]
  & \fbox{\texttt{Spaces ($\pi_1$)}} \ar[l] & \\
  \fbox{\texttt{Homological}} \ar[d] & & \\
  \fbox{\texttt{Advanced}} & &
\end{tikzcd}
\caption{Schematic dependency graph of the formalization. Arrows point
from dependency to dependent module. Boxes represent module groups.
The foundational kernel (\texttt{Path.Basic} $\to$ \texttt{Rewrite}
$\to$ \texttt{Quot}) is the critical path on which all subsequent
development depends.}
\label{fig:dependency}
\end{figure}

\begin{remark}
  The dependency graph is acyclic by construction. The longest dependency
  chain runs from \texttt{Path.Basic} through
  \texttt{Rewrite} $\to$ \texttt{Groupoid} $\to$
  \texttt{HigherDimensional} $\to$ \texttt{HomotopyTheory} $\to$
  \texttt{Fibration} $\to$ \texttt{Homological} $\to$
  \texttt{Advanced}, spanning all seven layers. Each layer adds new
  mathematical content while depending only on the layers below.
\end{remark}
