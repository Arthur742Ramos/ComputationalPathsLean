%!TEX root = ../main.tex
% Part II: Homotopy Theory (Chapters 6--10)
% The Algebra of Computational Paths
% Authors: Arthur Ferreira Ramos, Ruy J.G.B. de Queiroz, Anjolina G. de Oliveira

% ==========================================================================
%  CHAPTER 6 — Fundamental Groups and Loop Spaces
% ==========================================================================

\chapter{Fundamental Groups and Loop Spaces}\label{ch:fundamental-groups}

We now turn from the algebraic and rewriting-theoretic foundations of
computational paths to their homotopy-theoretic consequences. The
central objects of study are the \emph{loop space} and the
\emph{fundamental group}, which encode the 1-dimensional topology of
a type through the algebraic structure of its self-paths.

\section{Loop Spaces}\label{sec:loop-spaces}

\begin{definition}[Loop space]\label{def:loop-space}
Let $A$ be a type and $a : A$ a base point. The \emph{loop space} of
$A$ at $a$ is
\[
  \Omega(A, a) \;=\; \Path_A(a, a),
\]
the type of computational paths from $a$ to itself. Elements of
$\Omega(A, a)$ are called \emph{loops} at $a$.
\end{definition}

The loop space inherits three canonical operations from the path algebra:
\begin{itemize}
\item \textbf{Identity.} $\id = \refl(a)$, the reflexive loop with
  empty trace.
\item \textbf{Composition.} $p \cdot q = \trans(p, q)$, concatenation
  of the underlying traces.
\item \textbf{Inversion.} $p^{-1} = \symm(p)$, reversal of the trace.
\end{itemize}

By the monoid and involution laws established in Chapter~2, these operations
satisfy:

\begin{proposition}\label{prop:loop-space-laws}
For all $p, q, r \in \Omega(A, a)$\textup{:}
\begin{enumerate}
\item[\textup{(i)}] $\id \cdot p = p$ and $p \cdot \id = p$,
\item[\textup{(ii)}] $(p \cdot q) \cdot r = p \cdot (q \cdot r)$,
\item[\textup{(iii)}] $p^{-1} \cdot p = \id$ and $p \cdot p^{-1} = \id$,
\item[\textup{(iv)}] $(p^{-1})^{-1} = p$,
\item[\textup{(v)}] $(p \cdot q)^{-1} = q^{-1} \cdot p^{-1}$.
\end{enumerate}
All equalities hold as strict structural equalities of \textup{\texttt{Path}} records.
\end{proposition}

\begin{remark}
Proposition~\ref{prop:loop-space-laws} means that $\Omega(A, a)$ is a
\emph{strict} group \emph{before} quotienting---a distinctive feature of
the computational paths framework. In homotopy type theory, the loop
space is only a group up to higher paths; here, the group laws are
definitional by virtue of the list-based trace representation.
\end{remark}

\section{The Fundamental Group}\label{sec:fundamental-group}

Although $\Omega(A, a)$ is already a strict group, two loops with
different rewrite traces but the same underlying propositional equality
represent ``the same homotopy class.'' The fundamental group identifies
such loops.

\begin{definition}[Loop quotient]\label{def:loop-quotient}
The \emph{loop quotient} at $a$ is
\[
  \LoopQuot(A, a) \;=\; \PathRwQuot_A(a, a) \;=\;
  \Omega(A, a)\, /\, {\approx},
\]
where $\approx$ denotes rewrite equality (\texttt{RwEq}).
\end{definition}

\begin{definition}[Fundamental group]\label{def:fundamental-group}
The \emph{fundamental group} of $A$ at $a$ is
\[
  \pi_1(A, a) \;=\; \LoopQuot(A, a),
\]
equipped with multiplication $[p] \cdot [q] = [p \cdot q]$, identity
$e = [\refl(a)]$, and inversion $[p]^{-1} = [p^{-1}]$.
\end{definition}

\begin{theorem}[Group axioms]\label{thm:pi1-group}
$\pi_1(A, a)$ is a group: associativity, left and right identity, and
left and right inverse all hold as strict equalities on the quotient.
\end{theorem}

\begin{proof}
Each axiom is inherited from the corresponding strict identity on
$\Omega(A, a)$ (Proposition~\ref{prop:loop-space-laws}), which
descends to the quotient since the operations are well-defined with
respect to $\approx$ (Theorem~3.6).
\end{proof}

The group $\pi_1(A,a)$ can equivalently be described via two
intermediate structures that make the algebraic packaging explicit:

\begin{definition}[Loop monoid and loop group]\label{def:loop-monoid-group}
\hfill
\begin{itemize}
\item The \emph{loop monoid} $(\pi_1(A,a), \cdot, e)$ records
  multiplication, identity, associativity, and unit laws.
\item The \emph{loop group} extends the loop monoid with an inversion
  operation and witnesses of the inverse laws.
\end{itemize}
\end{definition}

\noindent
Both structures are \emph{canonical}: they are constructed uniformly for any
type $A$ and base point $a$, with no choices required.

\subsection*{Power operations.}
For applications to winding numbers (Chapter~\ref{ch:spaces}), we
equip $\pi_1(A,a)$ with natural and integer powers:

\begin{definition}\label{def:loop-powers}
For $x \in \pi_1(A, a)$, define:
\[
  x^0 = e, \qquad x^{n+1} = x^n \cdot x, \qquad
  x^{-n} = (x^n)^{-1} \;\text{ for } n > 0.
\]
\end{definition}

\begin{proposition}[Power laws]\label{prop:zpow-laws}
For all $x \in \pi_1(A, a)$ and $m, n \in \mathbb{Z}$\textup{:}
\begin{enumerate}
\item[\textup{(i)}] $x^{m+n} = x^m \cdot x^n$,
\item[\textup{(ii)}] $x^{-n} = (x^n)^{-1}$,
\item[\textup{(iii)}] $x^m \cdot x^n = x^n \cdot x^m$.
\end{enumerate}
\end{proposition}

\begin{proof}
Part~(i) is proved by integer induction, using the successor case
$x^{n+1} = x^n \cdot x$ and the predecessor case
$x^{n-1} = x^n \cdot x^{-1}$ as the inductive step. Part~(ii) follows
from the definition of negative powers. Part~(iii) is an immediate
consequence of (i) and the commutativity of integer addition.
\end{proof}

\section{Functoriality}\label{sec:pi1-functoriality}

\begin{theorem}[Induced homomorphism]\label{thm:induced-pi1}
Let $f : A \to B$ be a function and $a : A$. Then $f$ induces a group
homomorphism
\[
  f_* : \pi_1(A, a) \longrightarrow \pi_1(B, f(a))
\]
defined on representatives by $f_*([p]) = [\congrArg(f, p)]$.
\end{theorem}

\begin{proof}
The map $p \mapsto \congrArg(f, p)$ preserves $\approx$ since
$\congrArg$ is compatible with the rewrite system
(Theorem~3.6). Functoriality ($\congrArg(f, p \cdot q) =
\congrArg(f, p) \cdot \congrArg(f, q)$) follows from Theorem~2.8.
\end{proof}

\begin{theorem}[Functorial laws]\label{thm:pi1-functor-laws}
\hfill
\begin{enumerate}
\item[\textup{(i)}] $(\id_A)_* = \id_{\pi_1(A,a)}$.
\item[\textup{(ii)}] $(g \circ f)_* = g_* \circ f_*$ for
  $f : A \to B$ and $g : B \to C$.
\end{enumerate}
\end{theorem}

\begin{proof}
Part~(i) follows from $\congrArg(\id, p) \approx p$.
Part~(ii) follows from $\congrArg(g \circ f, p) \approx
\congrArg(g, \congrArg(f, p))$. Both are instances of rewrite rules
in the rewrite system.
\end{proof}

\begin{theorem}[Product formula]\label{thm:pi1-product}
For pointed types $(A, a)$ and $(B, b)$, there is a group isomorphism
\[
  \pi_1(A \times B,\, (a, b)) \;\cong\; \pi_1(A, a) \times \pi_1(B, b).
\]
\end{theorem}

\begin{proof}
The encoding map sends a loop $p$ in $A \times B$ to the pair
$(\fst(p),\, \snd(p))$ of its component projections. The decoding map
sends a pair $(p, q)$ to $\prodMk(p, q)$. Both maps respect $\approx$
by the congruence properties of projections and pairing. The round-trip
identities $\encode \circ \decode = \id$ and $\decode \circ \encode = \id$
follow from the product $\beta$- and $\eta$-rules
(Theorem~2.13).
\end{proof}

\section{The Fundamental Groupoid}\label{sec:fundamental-groupoid}

The fundamental group captures loops at a single base point. To handle
all points simultaneously, we pass to the fundamental groupoid.

\begin{definition}[Fundamental groupoid]\label{def:fundamental-groupoid}
The \emph{fundamental groupoid} $\Pi_1(A)$ has:
\begin{itemize}
\item \textbf{Objects}: points $a \in A$.
\item \textbf{Morphisms}: $\Hom_{\Pi_1(A)}(a, b) = \PathRwQuot_A(a, b)$.
\item \textbf{Composition}: path concatenation on the quotient.
\item \textbf{Identity}: $\id_a = [\refl(a)]$.
\item \textbf{Inverse}: $[p]^{-1} = [p^{-1}]$.
\end{itemize}
\end{definition}

\begin{theorem}\label{thm:fundamental-groupoid-strict}
$\Pi_1(A)$ is a strict groupoid: all axioms \textup{(}associativity,
identity, inverse\textup{)} hold as definitional equalities on
$\PathRwQuot$.
\end{theorem}

\begin{proof}
This is the strict groupoid $\StrictGroupoid{.}\mathrm{quotient}(A)$ from
Theorem~4.5, restricted to the morphism level.
\end{proof}

\begin{theorem}[Basepoint independence]\label{thm:basepoint-independence}
Let $\gamma : \PathRwQuot_A(a, b)$ be a path class. Then conjugation
by $\gamma$,
\[
  \varphi_\gamma : \pi_1(A, a) \xrightarrow{\;\sim\;} \pi_1(A, b),
  \qquad \varphi_\gamma(\alpha) = \gamma^{-1} \cdot \alpha \cdot \gamma,
\]
is a group isomorphism, with inverse $\varphi_{\gamma^{-1}}$.
\end{theorem}

\begin{proof}
That $\varphi_\gamma$ is a homomorphism (preserving identity,
composition, and inversion) follows by direct computation using
the strict groupoid laws. The cancellation identities
$\varphi_{\gamma^{-1}} \circ \varphi_\gamma = \id$ and
$\varphi_\gamma \circ \varphi_{\gamma^{-1}} = \id$ are verified by
expanding the definition and applying the inverse and unit laws.
\end{proof}

\begin{theorem}[Functoriality of the groupoid]\label{thm:pi1-groupoid-functor}
A function $f : A \to B$ induces a groupoid functor
$\Pi_1(f) : \Pi_1(A) \to \Pi_1(B)$ defined by
$\Pi_1(f)(a) = f(a)$ on objects and
$\Pi_1(f)([p]) = [\congrArg(f, p)]$ on morphisms. This functor
preserves identity and composition strictly.
\end{theorem}

\section{Higher Homotopy Groups and the Eckmann--Hilton Argument}%
\label{sec:higher-homotopy}

\begin{definition}[Iterated loop spaces]\label{def:iterated-loops}
The \emph{iterated loop spaces} are defined by:
\begin{gather*}
  \Omega^0(A, a) = A, \qquad
  \Omega^1(A, a) = \Path_A(a, a), \\
  \Omega^2(A, a) = \Deriv_2(\refl(a), \refl(a)), \qquad
  \Omega^3(A, a) = \Deriv_3(\refl_2, \refl_2),
\end{gather*}
where $\Deriv_k$ denotes $k$-cells in the weak $\omega$-groupoid tower.
\end{definition}

\begin{definition}[Higher homotopy groups]\label{def:higher-homotopy}
The \emph{$n$-th homotopy group} is defined by:
\begin{gather*}
  \pi_1(A, a) = \Omega^1(A, a) / {\approx}, \\
  \pi_2(A, a) = \Omega^2(A, a) / {\sim_3},
\end{gather*}
where $\alpha \sim_3 \beta$ iff there exists a 3-cell
$\Deriv_3(\alpha, \beta)$. For $n \ge 3$, the contractibility of the
tower at dimension $\ge 3$ (Theorem~5.7) implies that $\pi_n$
collapses.
\end{definition}

The double loop space $\Omega^2(A,a)$ carries two distinct composition
operations: \emph{vertical} composition (sequential concatenation of
derivations) and \emph{horizontal} composition (induced by whiskering
and path concatenation).

\begin{definition}[Whiskering]\label{def:whiskering}
Let $f : \Path_A(a,b)$ and $\alpha : p \approx q$ for
$p, q : \Path_A(b,c)$.
\begin{itemize}
\item \emph{Left whiskering}: $f \triangleright \alpha$ is the
  witness that $f \cdot p \approx f \cdot q$.
\item \emph{Right whiskering}: $\alpha \triangleleft g$ is the
  witness that $p \cdot g \approx q \cdot g$.
\end{itemize}
\emph{Horizontal composition} is defined as
$\alpha \star \beta = (\alpha \triangleleft q') \circ_v
(p \triangleright \beta)$.
\end{definition}

\begin{theorem}[Interchange law]\label{thm:interchange-EH}
For 2-cells $\alpha_1, \alpha_2, \beta_1, \beta_2$ in
$\Omega^2(A,a)$\textup{:}
\[
  (\alpha_1 \circ_v \alpha_2) \star (\beta_1 \circ_v \beta_2)
  \;=\;
  (\alpha_1 \star \beta_1) \circ_v (\alpha_2 \star \beta_2).
\]
At the level of derivations, the two sides are connected by a 3-cell
$\MetaStep_3.\mathsf{interchange}$.
\end{theorem}

\begin{theorem}[Eckmann--Hilton]\label{thm:eckmann-hilton}
On $\Omega^2(A,a)$, horizontal and vertical composition coincide
and are commutative. In particular, $\pi_2(A,a)$ is an abelian group.
\end{theorem}

\begin{proof}
The proof proceeds in three steps.

\textbf{Step 1} (Whiskering by $\refl$ is trivial). Since
$\trans(\refl(a), \refl(a))$ reduces definitionally to $\refl(a)$,
both $\alpha \triangleleft \refl$ and $\refl \triangleright \beta$
are connected to $\alpha$ and $\beta$ respectively by 3-cells.

\textbf{Step 2} ($\star = \circ_v$). From Step~1:
$\alpha \star \beta
  = (\alpha \triangleleft \refl) \circ_v (\refl \triangleright \beta)
  \sim_3 \alpha \circ_v \beta$.

\textbf{Step 3} (Commutativity). By the interchange law
(Theorem~\ref{thm:interchange}), $\alpha \star \beta \sim_3
\beta \star \alpha$ (the alternative horizontal composition
reverses the order). Combining with Step~2:
$\alpha \circ_v \beta \sim_3 \beta \circ_v \alpha$. \qedhere
\end{proof}

\begin{corollary}\label{cor:pi-n-abelian}
For every $n \ge 2$, $\pi_n(A, a)$ is abelian.
\end{corollary}

\begin{remark}[Naturality of coherence laws]
The unit laws, associativity, and symmetry involution are all
\emph{natural} with respect to rewriting: they commute with
whiskering. Concretely, for the left unit law, the diagram
$\refl \cdot p \xrightarrow{\refl \triangleright \alpha}
\refl \cdot q$ over
$p \xrightarrow{\alpha} q$ commutes: both composites through the
naturality square (applying whiskering then the unit law, or the unit
law then $\alpha$) are equal as $\RwEq$ witnesses, since both are
proofs of the same proposition.
\end{remark}


% ==========================================================================
%  CHAPTER 7 — Spaces and Their Fundamental Groups
% ==========================================================================

\chapter{Spaces and Their Fundamental Groups}\label{ch:spaces}

With the fundamental group machinery in place, we compute $\pi_1$ for
several classical spaces. Each space is modeled within the
computational-paths framework using \emph{path expressions}---a
syntactic calculus of formal generators and relations that captures the
combinatorial essence of the space.

\section{The Computational Circle \texorpdfstring{$S^1$}{S¹}}%
\label{sec:circle}

\begin{definition}[Circle]\label{def:circle}
The \emph{computational circle} $S^1$ is a one-point type
$\{*\}$ equipped with a formal loop generator via the path expression
calculus.

More precisely, define the type of \emph{circle path expressions} by
the grammar:
\[
  e \;::=\; \lp \;\mid\; \refl(a) \;\mid\; e^{-1} \;\mid\;
  e_1 \cdot e_2,
\]
where $\lp$ is a distinguished generator with source and target both
equal to $* : S^1$.
\end{definition}

The circle path expressions carry a natural notion of \emph{winding
number}:

\begin{definition}[Winding number]\label{def:winding-number}
The winding number $\wind : \text{CircleExpr}(*,*) \to \mathbb{Z}$ is
defined recursively:
\[
  \wind(\lp) = 1, \quad \wind(\refl) = 0, \quad
  \wind(e^{-1}) = -\wind(e), \quad
  \wind(e_1 \cdot e_2) = \wind(e_1) + \wind(e_2).
\]
\end{definition}

\begin{proposition}\label{prop:winding-power}
For all $n \in \mathbb{N}$,\, $\wind(\lp^n) = n$. For all
$z \in \mathbb{Z}$,\, $\wind(\lp^z) = z$.
\end{proposition}

\begin{definition}[Circle loop quotient]\label{def:circle-pi1}
Two loop expressions are identified if they have the same winding
number:
\[
  \pi_1^{\text{expr}}(S^1, *) \;=\;
  \text{CircleExpr}(*,*) \,/\, (\wind(e_1) = \wind(e_2)).
\]
\end{definition}

\begin{theorem}\label{thm:pi1-circle}
$\pi_1(S^1, *) \cong \mathbb{Z}$.
\end{theorem}

\begin{proof}
The encoding map $\encode : \pi_1^{\text{expr}}(S^1,*) \to \mathbb{Z}$
sends $[e] \mapsto \wind(e)$. The decoding map
$\decode : \mathbb{Z} \to \pi_1^{\text{expr}}(S^1,*)$ sends
$z \mapsto [\lp^z]$. These are inverse:
\begin{itemize}
\item $\encode \circ \decode$ is the identity on $\mathbb{Z}$ by
  Proposition~\ref{prop:winding-power}.
\item $\decode \circ \encode$ is the identity on the quotient: for any
  expression $e$, the canonical power $\lp^{\wind(e)}$ has the same
  winding number as $e$, hence they are identified.  \qedhere
\end{itemize}
\end{proof}

\section{The Torus \texorpdfstring{$T^2$}{T²}}\label{sec:torus}

\begin{definition}[Torus]\label{def:torus}
The \emph{torus} is defined as the product of two circles:
\[
  T^2 \;=\; S^1 \times S^1.
\]
The base point is $(*_1, *_2)$, and the two fundamental loops are:
\begin{align*}
  \alpha &= \prodMk(\lp_1, \refl(*_2)), \\
  \beta  &= \prodMk(\refl(*_1), \lp_2).
\end{align*}
\end{definition}

\begin{theorem}\label{thm:pi1-torus}
$\pi_1(T^2, (*,*)) \cong \mathbb{Z} \times \mathbb{Z}$.
\end{theorem}

\begin{proof}
By the product formula (Theorem~\ref{thm:pi1-product}):
$\pi_1(S^1 \times S^1) \cong \pi_1(S^1) \times \pi_1(S^1) \cong
\mathbb{Z} \times \mathbb{Z}$.
\end{proof}

\begin{remark}
The torus has an abelian fundamental group: $\alpha$ and $\beta$
commute. This commutativity is automatic from the product
construction---no additional relation needs to be imposed.
\end{remark}

\section{The Figure-Eight and Free Products}\label{sec:figure-eight}

\begin{definition}[Wedge sum]\label{def:wedge}
Let $(A, a_0)$ and $(B, b_0)$ be pointed types. The \emph{wedge sum}
$A \vee B$ is the pushout:
\[
\begin{tikzcd}
  \{*\} \ar[r, "a_0"] \ar[d, "b_0"'] & A \ar[d, "\inl"] \\
  B \ar[r, "\inr"'] & A \vee B
\end{tikzcd}
\]
with a glue path $\glue : \inl(a_0) = \inr(b_0)$.
\end{definition}

\begin{definition}[Figure-eight]\label{def:figure-eight}
The \emph{figure-eight} is $S^1 \vee S^1$, the wedge sum of two
circles at their base points. Its two generating loops are:
\begin{align*}
  a &= \inl_*(\lp_1), \\
  b &= \glue \cdot \inr_*(\lp_2) \cdot \glue^{-1}.
\end{align*}
\end{definition}

\begin{definition}[Free product words]\label{def:free-product-words}
Given groups $G$ and $H$, a \emph{free product word} is a finite
alternating sequence of non-identity elements from $G$ and $H$:
\[
  w = g_1 h_1 g_2 h_2 \cdots
\]
Multiplication is concatenation followed by reduction (cancelling
adjacent same-side identities). The resulting group is the
\emph{free product} $G * H$.
\end{definition}

\begin{theorem}\label{thm:pi1-figure-eight}
$\pi_1(S^1 \vee S^1, *) \cong \mathbb{Z} * \mathbb{Z}$, the free
product of $\mathbb{Z}$ with itself.
\end{theorem}

\begin{proof}
The Seifert--van~Kampen theorem (Theorem~\ref{thm:svk} below) applied
to the wedge decomposition gives
$\pi_1(S^1 \vee S^1) \cong \pi_1(S^1) * \pi_1(S^1) \cong
\mathbb{Z} * \mathbb{Z}$. Alternatively, a direct
\emph{provenance-based} equivalence identifies loop classes with free
product words over $\pi_1(S^1)$: each loop in the wedge is uniquely
decomposed into segments lying in the left or right circle, yielding
the free product structure.
\end{proof}

\begin{remark}
The fundamental group $\mathbb{Z} * \mathbb{Z}$ is non-abelian: the
loops $a$ and $b$ generate a free group on two generators. In
particular, $a \cdot b \ne b \cdot a$ as elements of $\pi_1$.
\end{remark}

\begin{definition}[Bouquet of $n$ circles]\label{def:bouquet}
The \emph{bouquet} $\bigvee_n S^1$ is the wedge of $n$ copies of
$S^1$, all sharing a common base point.
\end{definition}

\begin{corollary}\label{cor:pi1-bouquet}
$\pi_1\!\left(\bigvee_n S^1,\, *\right) \cong F_n$, the free group on
$n$ generators.
\end{corollary}

\section{The Seifert--van Kampen Theorem}\label{sec:svk}

The Seifert--van Kampen theorem is the principal computational tool
for determining fundamental groups of spaces assembled from simpler
pieces.

\begin{definition}[Pushout]\label{def:pushout}
Given functions $f : C \to A$ and $g : C \to B$, the \emph{pushout}
$A \sqcup_C B$ is the type with constructors $\inl : A \to A \sqcup_C B$,
$\inr : B \to A \sqcup_C B$, and a path constructor
$\glue : \forall c : C,\; \inl(f(c)) = \inr(g(c))$.
\end{definition}

\begin{definition}[Amalgamated free product]\label{def:amalgamated}
Given group homomorphisms $\varphi : K \to G$ and $\psi : K \to H$,
the \emph{amalgamated free product} $G *_K H$ is the quotient of the
free product $G * H$ by the normal closure of the relations
$\varphi(k) = \psi(k)$ for all $k \in K$.
\end{definition}

\begin{theorem}[Seifert--van Kampen]\label{thm:svk}
Let $f : C \to A$ and $g : C \to B$, and let $c_0 \in C$. There is an
equivalence
\[
  \pi_1(A \sqcup_C B,\; \inl(f(c_0)))
  \;\cong\;
  \pi_1(A, f(c_0)) \;*_{\pi_1(C, c_0)}\; \pi_1(B, g(c_0)),
\]
where the amalgamation is along the induced homomorphisms
$f_* : \pi_1(C, c_0) \to \pi_1(A, f(c_0))$ and
$g_* : \pi_1(C, c_0) \to \pi_1(B, g(c_0))$.
\end{theorem}

\begin{proof}[Proof sketch]
The proof follows the encode-decode method. One constructs an encoding
of loops in the pushout as words in the amalgamated free product, and
a decoding in the opposite direction, using the glue path to translate
between the two sides. The key technical ingredients are:
(i)~the naturality of the glue path with respect to loop rewriting,
(ii)~the universal property of the pushout, and
(iii)~the bijectivity of the resulting maps, established via
the provenance decomposition of pushout paths.
\end{proof}

\begin{corollary}\label{cor:svk-wedge}
For pointed types $(A, a_0)$ and $(B, b_0)$ with trivial
$\pi_1(\{*\}) = 1$\textup{:}
\[
  \pi_1(A \vee B, *) \;\cong\; \pi_1(A, a_0) * \pi_1(B, b_0).
\]
\end{corollary}

\section{Suspensions and Spheres}\label{sec:suspensions}

\begin{definition}[Suspension]\label{def:suspension}
The \emph{suspension} of a type $X$ is
\[
  \Sigma X = \{N, S\} \cup \{\merid(x) : N = S \mid x \in X\},
\]
with north pole $N$, south pole $S$, and a meridional path
$\merid(x) : \Path(N, S)$ for each $x : X$.
\end{definition}

\begin{definition}[Suspension loop]\label{def:susp-loop}
The canonical loop at the north pole induced by a point $x_0 \in X$ is
\[
  \sigma(x_0) = \merid(x_0) \cdot \merid(x_0)^{-1} : \Omega(\Sigma X, N).
\]
\end{definition}

The Freudenthal suspension theorem provides the key bridge between
loop spaces and suspensions:

\begin{theorem}[Freudenthal suspension, preview]\label{thm:freudenthal-preview}
There exists a natural map
$\Omega(X, x_0) \to \Omega(\Sigma X, N)$ that sends each loop to its
suspension and preserves the base point.
\end{theorem}

\begin{remark}
The full Freudenthal theorem states that this map is an isomorphism
on $\pi_n$ for $n < 2 \cdot \conn(X) + 1$, where $\conn(X)$ is the
connectivity of $X$. The formalization records the map and its
basepoint behavior; the connectivity bound is stated at the structural
level.
\end{remark}


% ==========================================================================
%  CHAPTER 8 — Fibrations, Covering Spaces, and Exact Sequences
% ==========================================================================

\chapter{Fibrations, Covering Spaces, and Exact Sequences}%
\label{ch:fibrations}

This chapter develops the theory of fibrations and covering spaces in
the computational-paths framework and establishes the long exact
sequence of homotopy groups.

\section{Fibers of Maps}\label{sec:fibers}

\begin{definition}[Homotopy fiber]\label{def:fiber}
Let $f : A \to B$ and $b : B$. The \emph{fiber} of $f$ over $b$ is
\[
  \Fib(f, b) \;=\; \{a : A \mid f(a) = b\}
  \;=\; \Sigma_{a : A}\, \Path_B(f(a),\, b).
\]
\end{definition}

\begin{definition}[Type families as fibrations]\label{def:family-fibration}
A type family $P : B \to \Type$ determines a fibration via the total
space construction:
\[
  E = \Sigma_{b : B}\, P(b), \qquad \proj : E \to B, \quad
  (b, p) \mapsto b.
\]
The fiber of $\proj$ over $b$ is canonically equivalent to $P(b)$:
\[
  \Fib(\proj, b) \;\simeq\; P(b).
\]
\end{definition}

\section{Path Lifting}\label{sec:path-lifting}

The fundamental property of fibrations is that paths in the base
can be lifted to the total space.

\begin{theorem}[Path lifting]\label{thm:path-lifting}
Let $P : B \to \Type$ be a type family, $p : \Path_B(b_1, b_2)$ a
path in the base, and $x : P(b_1)$ a point in the fiber over $b_1$.
Then there exists a path
\[
  \widetilde{p}(x) \;:\; \Path_E\!\big((b_1, x),\; (b_2, \transport_P(p, x))\big)
\]
in the total space $E = \Sigma P$ that lifts $p$.
\end{theorem}

\begin{definition}[Fiber transport]\label{def:fiber-transport}
The \emph{fiber transport} along $p : \Path_B(b_1, b_2)$ is the function
\[
  \transport_P(p) : P(b_1) \longrightarrow P(b_2)
\]
given by the path transport operation.
\end{definition}

\begin{proposition}[Transport composition]\label{prop:transport-composition}
For paths $p : b_1 \leadsto b_2$ and $q : b_2 \leadsto b_3$ in $B$
and $x : P(b_1)$\textup{:}
\[
  \transport_P(p \cdot q,\, x) \;=\;
  \transport_P(q,\, \transport_P(p, x)).
\]
\end{proposition}

\section{Fiber Sequences}\label{sec:fiber-sequences}

\begin{definition}[Fiber sequence]\label{def:fiber-sequence}
A \emph{fiber sequence} consists of types $F$, $E$, $B$ together with:
\begin{itemize}
\item a projection $\proj : E \to B$,
\item base points $b_0 \in B$, $e_0 \in E$ with
  $\proj(e_0) = b_0$,
\item an equivalence $F \simeq \Fib(\proj, b_0)$.
\end{itemize}
The inclusion $\incl : F \hookrightarrow E$ sends a fiber element to
its underlying point: $\incl(f) = (\text{toFiber}(f)).\text{point}$.
\end{definition}

\begin{definition}[Exactness]\label{def:exactness}
A fiber sequence $F \xrightarrow{\incl} E \xrightarrow{\proj} B$ is
\emph{exact at $E$} if:
\begin{enumerate}
\item $\proj(\incl(f)) = b_0$ for all $f \in F$
  \textup{(}image $\subseteq$ kernel\textup{)};
\item for every $e \in E$ with $\proj(e) = b_0$, there exists
  $f \in F$ with $\incl(f) = e$
  \textup{(}kernel $\subseteq$ image\textup{)}.
\end{enumerate}
\end{definition}

\begin{theorem}[Canonical fiber sequence]\label{thm:canonical-fiber-seq}
For any type family $P : B \to \Type$, base point $b \in B$, and
$x_0 : P(b)$, there is an exact fiber sequence
\[
  P(b) \;\xrightarrow{\;\incl\;}\; \textstyle\Sigma P
  \;\xrightarrow{\;\proj\;}\; B.
\]
\end{theorem}

\section{The Connecting Map}\label{sec:connecting-map}

\begin{definition}[Connecting map]\label{def:connecting-map}
Let $P : B \to \Type$ be a type family, $b \in B$, and $x_0 : P(b)$.
The \emph{connecting map}
\[
  \partial : \Omega(B, b) \longrightarrow P(b)
\]
sends a loop $\ell$ to $\transport_P(\ell,\, x_0)$.
\end{definition}

\begin{proposition}[Properties of the connecting map]\label{prop:connecting-map}
\hfill
\begin{enumerate}
\item[\textup{(i)}] $\partial(\refl(b)) = x_0$.
\item[\textup{(ii)}] $\partial(\ell_1 \cdot \ell_2) =
  \partial_{x_1}(\ell_2)$ where $x_1 = \partial(\ell_1)$
  \textup{(}i.e., the connecting map respects composition via iterated
  transport\textup{)}.
\item[\textup{(iii)}] $\partial$ respects $\approx$: if
  $\ell_1 \approx \ell_2$ then $\partial(\ell_1) = \partial(\ell_2)$.
\end{enumerate}
\end{proposition}

\begin{proof}
Part~(i) is immediate from $\transport(\refl, x_0) = x_0$. Part~(ii)
follows from the transport composition law
(Proposition~\ref{prop:transport-composition}). Part~(iii) holds
because $\ell_1 \approx \ell_2$ implies $\ell_1.\toEq = \ell_2.\toEq$
(soundness of $\approx$), and transport depends only on the
underlying propositional equality.
\end{proof}

By Part~(iii), the connecting map descends to the quotient:

\begin{corollary}\label{cor:connecting-on-pi1}
There is a well-defined map
$\partial : \pi_1(B, b) \to P(b)$ sending $[\ell] \mapsto
\transport_P(\ell, x_0)$.
\end{corollary}

\section{Covering Spaces}\label{sec:covering-spaces}

\begin{definition}[Covering space]\label{def:covering-space}
A type family $P : A \to \Type$ is a \emph{covering space} of $A$
if every fiber $P(a)$ is a \emph{set} (0-truncated): any two paths
in $P(a)$ with the same endpoints are equal.
\end{definition}

\begin{theorem}[Unique path lifting]\label{thm:unique-lifting}
For a covering space $P$, the fiber transport
$\transport_P(p) : P(a) \to P(b)$ is injective for every path
$p : a \leadsto b$.
\end{theorem}

\begin{proof}
Given $\transport_P(p, x) = \transport_P(p, y)$, apply
$\transport_P(p^{-1})$ to both sides and use the inverse law
$\transport(p^{-1}, \transport(p, x)) = x$.
\end{proof}

\subsection*{The $\pi_1$-action on fibers.}

\begin{definition}[Monodromy action]\label{def:monodromy}
Let $P : A \to \Type$ be a type family and $a \in A$. The
\emph{monodromy action} of $\pi_1(A, a)$ on the fiber $P(a)$ is the
map
\[
  \mu : \pi_1(A, a) \times P(a) \longrightarrow P(a), \qquad
  \mu([\ell], x) = \transport_P(\ell, x).
\]
\end{definition}

\begin{proposition}\label{prop:monodromy-action}
The monodromy action is a genuine group action:
\begin{enumerate}
\item[\textup{(i)}] $\mu(e, x) = x$
  \textup{(}identity acts trivially\textup{)}.
\item[\textup{(ii)}] $\mu(\alpha, \mu(\beta, x))
  = \mu(\alpha \cdot \beta, x)$
  \textup{(}compatibility with multiplication\textup{)}.
\end{enumerate}
\end{proposition}

\begin{proof}
Part~(i) is the transport identity law. Part~(ii) is the transport
composition law descending to the quotient.
\end{proof}

\subsection*{Deck transformations.}

\begin{definition}[Deck transformation]\label{def:deck-transformation}
A \emph{deck transformation} of a covering $P$ is an automorphism
$\varphi : \Sigma P \to \Sigma P$ satisfying
$\proj \circ \varphi = \proj$, together with an inverse.
\end{definition}

\begin{proposition}\label{prop:deck-group}
Deck transformations form a group under composition, with identity
$\id$ and the obvious associativity, identity, and inverse laws.
\end{proposition}

\section{The Hopf Fibration}\label{sec:hopf}

\begin{definition}[Hopf fibration data]\label{def:hopf}
The \emph{Hopf fibration} is recorded as a structure consisting of:
\begin{itemize}
\item A projection $\proj : S^3 \to S^2$.
\item Base points $b_0 \in S^2$ and $e_0 \in S^3$ with
  $\proj(e_0) = b_0$.
\item A fiber equivalence $\Fib(\proj, b_0) \simeq S^1$.
\end{itemize}
\end{definition}

\begin{theorem}\label{thm:hopf-fiber-seq}
The Hopf data assembles into an exact fiber sequence
\[
  S^1 \;\longrightarrow\; S^3 \;\longrightarrow\; S^2.
\]
\end{theorem}

\begin{remark}
The Hopf fibration is the prototypical non-trivial fiber bundle. Its
long exact sequence (Section~\ref{sec:les}) yields the classical
relation $\pi_2(S^2) \cong \mathbb{Z}$ (via the connecting
homomorphism and the fact that $\pi_1(S^1) \cong \mathbb{Z}$ and
$\pi_1(S^3) = 0$).
\end{remark}

\section{The Long Exact Sequence of Homotopy Groups}%
\label{sec:les}

\begin{theorem}[Long exact sequence]\label{thm:les}
For a type family $P : B \to \Type$ with base point $b \in B$ and
$x_0 : P(b)$, there is a long exact sequence:
\[
  \cdots \;\to\; \pi_1(P(b), x_0)
  \;\xrightarrow{\;\incl_*\;}\;
  \pi_1\!\left(\textstyle\Sigma P,\, (b, x_0)\right)
  \;\xrightarrow{\;\proj_*\;}\;
  \pi_1(B, b)
  \;\xrightarrow{\;\partial\;}\;
  P(b).
\]
Exactness holds at each term:
\begin{enumerate}
\item[\textup{(i)}] \textup{At $\pi_1(\Sigma P)$}:
  $\proj_*(\incl_*(\alpha)) = e$ for all
  $\alpha \in \pi_1(P(b), x_0)$.
\item[\textup{(ii)}] \textup{At $\pi_1(B)$}:
  $\partial(\proj_*(\beta)) = x_0$ for all
  $\beta \in \pi_1(\Sigma P, (b, x_0))$.
\end{enumerate}
\end{theorem}

\begin{proof}
\textbf{Exactness at $\pi_1(\Sigma P)$.}
The inclusion $\incl$ embeds a fiber loop as a loop in $\Sigma P$
that is constant in the base component. Applying $\proj_*$ to such a
loop produces $\congrArg(\proj, \congrArg(\iota, \ell))$ where
$\iota : P(b) \hookrightarrow \Sigma P$ is $x \mapsto (b, x)$. Since
$\proj \circ \iota$ is the constant function $b$, this reduces to
$\refl(b)$ up to $\approx$.

\textbf{Exactness at $\pi_1(B)$.}
For a loop $\ell$ in $\Sigma P$ at $(b, x_0)$, the connecting map
applied to $\proj_*(\ell)$ computes $\transport(\proj_*(\ell), x_0)$.
By the sigma-type path characterization, the base path of $\ell$ is
$\proj_*(\ell)$ and the fiber path witnesses
$\transport(\proj_*(\ell), x_0) = x_0$.
\end{proof}

\begin{remark}
The induced maps $\incl_*$, $\proj_*$, and $\partial$ are all
\emph{natural} in morphisms of fibrations: a map of fiber sequences
induces a commutative ladder of long exact sequences. The naturality
is formalized via the functoriality of $\pi_1$ (Theorem~\ref{thm:pi1-functor-laws})
and the compatibility of transport with maps.
\end{remark}

\begin{theorem}[Simply connected base]\label{thm:sc-connecting}
If the base $B$ is simply connected \textup{(}i.e.,
$\pi_1(B, b) = \{e\}$\textup{)}, then the connecting map is trivial:
$\partial(\ell) = x_0$ for all loops $\ell$ in $B$.
\end{theorem}

\begin{proof}
Every loop $\ell$ is $\approx$-equivalent to $\refl(b)$, and
$\partial(\refl(b)) = x_0$.
\end{proof}


% ==========================================================================
%  CHAPTER 9 — The Hurewicz Theorem and Homological Algebra
% ==========================================================================

\chapter{The Hurewicz Theorem and Homological Algebra}\label{ch:hurewicz}

This chapter connects the homotopy-theoretic invariant $\pi_1$ to the
homological invariant $H_1$ via the Hurewicz theorem, and develops
the algebraic machinery of abelianization.

\section{Abelianization}\label{sec:abelianization}

\begin{definition}[Commutator]\label{def:commutator}
For elements $a, b$ in a group $(G, \cdot, {}^{-1}, e)$, the
\emph{commutator} is
\[
  [a, b] = a \cdot b \cdot a^{-1} \cdot b^{-1}.
\]
\end{definition}

\begin{definition}[Abelianization]\label{def:abelianization}
The \emph{abelianization} of $G$ is the quotient
\[
  G^{\ab} = G \,/\, \langle [a, b] = e \mid a, b \in G \rangle.
\]
Concretely, $G^{\ab}$ is $G$ modulo the smallest congruence relation
that includes:
\begin{itemize}
\item commutativity: $a \cdot b \sim b \cdot a$;
\item congruence: $x \sim y$ implies $z \cdot x \sim z \cdot y$ and
  $x \cdot z \sim y \cdot z$;
\item the group laws (associativity, identity, inverse).
\end{itemize}
\end{definition}

\begin{theorem}[Commutators vanish]\label{thm:commutator-vanishes}
In $G^{\ab}$, every commutator is trivial: $[a, b] = e$.
\end{theorem}

\begin{proof}
We compute in $G^{\ab}$:
\begin{align*}
  [a, b] &= (a \cdot b) \cdot (a^{-1} \cdot b^{-1}) \\
  &\sim (b \cdot a) \cdot (a^{-1} \cdot b^{-1})
    &&\text{(commutativity of } a \cdot b\text{)} \\
  &= b \cdot (a \cdot a^{-1}) \cdot b^{-1}
    &&\text{(associativity)} \\
  &= b \cdot e \cdot b^{-1}
    &&\text{(inverse law)} \\
  &= b \cdot b^{-1} = e.
    &&\text{(identity and inverse)} \qedhere
\end{align*}
\end{proof}

\begin{corollary}\label{cor:abelianization-commutative}
$G^{\ab}$ is abelian: $[a] \cdot [b] = [b] \cdot [a]$ for all
$a, b \in G$.
\end{corollary}

\section{The First Homology Group}\label{sec:H1}

\begin{definition}[First homology]\label{def:H1}
The \emph{first homology group} of a pointed type $(A, a)$ is defined
as the abelianization of the fundamental group:
\[
  H_1(A) \;=\; \pi_1(A, a)^{\ab}.
\]
\end{definition}

\begin{definition}[Hurewicz homomorphism]\label{def:hurewicz-map}
The \emph{Hurewicz homomorphism}
\[
  h : \pi_1(A, a) \longrightarrow H_1(A)
\]
is the canonical quotient map to the abelianization.
\end{definition}

\begin{theorem}[Hurewicz theorem, dimension 1]\label{thm:hurewicz}
For any pointed type $(A, a)$, the Hurewicz homomorphism induces an
isomorphism
\[
  \pi_1(A, a)^{\ab} \;\cong\; H_1(A).
\]
\end{theorem}

\begin{proof}
This is tautological in our setup, since $H_1$ is defined as
$\pi_1^{\ab}$. The mathematical content is that $H_1$ computed via
singular homology agrees with $\pi_1^{\ab}$---a fact that in the
topological setting requires the theory of singular chains and the
identification of 1-cycles with loops.
\end{proof}

\begin{proposition}[Properties of the Hurewicz map]\label{prop:hurewicz-properties}
\hfill
\begin{enumerate}
\item[\textup{(i)}] $h$ is surjective: every element of $H_1(A)$ is
  $h(\alpha)$ for some $\alpha \in \pi_1(A, a)$.
\item[\textup{(ii)}] $\ker(h) = [\pi_1, \pi_1]$: $h(\alpha) = e$ if
  and only if $\alpha$ lies in the commutator subgroup.
\item[\textup{(iii)}] $h$ is a group homomorphism:
  $h(\alpha \cdot \beta) = h(\alpha) \cdot h(\beta)$.
\end{enumerate}
\end{proposition}

\section{Computations}\label{sec:hurewicz-computations}

\begin{example}[$H_1$ of the circle]
Since $\pi_1(S^1) \cong \mathbb{Z}$ is already abelian,
\[
  H_1(S^1) \;\cong\; \mathbb{Z}^{\ab} \;\cong\; \mathbb{Z}.
\]
The abelianization of $\mathbb{Z}$ is $\mathbb{Z}$ itself: the
abelianization relation on $\mathbb{Z}$ implies $x = y$ whenever it
relates $x$ and $y$, as verified by induction on the relation.
\end{example}

\begin{example}[$H_1$ of the torus]
Since $\pi_1(T^2) \cong \mathbb{Z} \times \mathbb{Z}$ is abelian,
\[
  H_1(T^2) \;\cong\; (\mathbb{Z} \times \mathbb{Z})^{\ab}
  \;\cong\; \mathbb{Z} \times \mathbb{Z}.
\]
\end{example}

\begin{example}[$H_1$ of the figure-eight]\label{ex:figure-eight-H1}
The figure-eight has non-abelian $\pi_1 \cong \mathbb{Z} * \mathbb{Z}$.
Its abelianization is:
\[
  H_1(S^1 \vee S^1) \;\cong\;
  (\mathbb{Z} * \mathbb{Z})^{\ab} \;\cong\;
  \mathbb{Z} \times \mathbb{Z}.
\]
The abelianization map $\mathbb{Z} * \mathbb{Z} \to
\mathbb{Z} \times \mathbb{Z}$ sends a free product word to the pair
of total exponents: a word with left-elements summing to $m$ and
right-elements summing to $n$ maps to $(m, n)$.
This is a homomorphism (it respects concatenation componentwise) and its
kernel is precisely the commutator subgroup.

This is a paradigmatic example where $\pi_1 \ne H_1$: the
non-commutativity of the free product is ``killed'' by
abelianization.
\end{example}

\begin{theorem}[Free product abelianization]\label{thm:free-product-ab}
For any groups $G$ and $H$\textup{:}
\[
  (G * H)^{\ab} \;\cong\; G^{\ab} \times H^{\ab}.
\]
\end{theorem}

\section{Simply Connected Spaces and $H_1$}\label{sec:sc-H1}

\begin{definition}[Simply connected]\label{def:simply-connected}
A type $A$ is \emph{simply connected} at $a$ if
$\pi_1(A, a) = \{e\}$, i.e., every loop at $a$ is trivial in the
rewrite quotient.
\end{definition}

\begin{theorem}\label{thm:sc-H1-trivial}
If $A$ is simply connected at $a$, then $H_1(A) = \{0\}$.
\end{theorem}

\begin{proof}
If $\pi_1(A, a)$ has a single element, then its abelianization also
has a single element.
\end{proof}

\begin{corollary}[Detection principle]\label{cor:detection}
If $H_1(A)$ contains a non-trivial element, then $\pi_1(A, a)$ is
non-trivial. Equivalently, $H_1(A) \ne 0$ implies $A$ is not simply
connected.
\end{corollary}

\section{Higher Hurewicz and Abelianization of Known Groups}%
\label{sec:higher-hurewicz}

\begin{theorem}[Higher Hurewicz, statement]\label{thm:higher-hurewicz}
For $n \ge 2$ and $(n-1)$-connected $X$, there is an isomorphism
\[
  h_n : \pi_n(X, x_0) \;\xrightarrow{\;\sim\;}\; H_n(X).
\]
In particular\textup{:}
\begin{itemize}
\item For simply connected $X$: $\pi_2(X) \cong H_2(X)$.
\item For spheres: $\pi_n(S^n) \cong H_n(S^n) \cong \mathbb{Z}$.
\end{itemize}
\end{theorem}

\begin{remark}
The higher Hurewicz theorem is stated at the structural level in the
formalization. The computational-paths framework captures $\pi_n$ for
$n = 1, 2$ concretely and records the higher statement as data.
\end{remark}


% ==========================================================================
%  CHAPTER 10 — Advanced Homotopy Theory
% ==========================================================================

\chapter{Advanced Homotopy Theory}\label{ch:advanced}

We conclude Part~II with a survey of the advanced homotopy-theoretic
structures formalized in the computational-paths library, highlighting
the interplay between the algebraic rewriting infrastructure and
classical constructions.

\section{Eilenberg--MacLane Spaces and Postnikov Towers}%
\label{sec:EM-postnikov}

\begin{definition}[Eilenberg--MacLane space]\label{def:EM-space}
An \emph{Eilenberg--MacLane space} $K(G, n)$ for a group $G$ and
$n \ge 1$ is a type satisfying:
\[
  \pi_k(K(G,n)) \;\cong\;
  \begin{cases}
    G & \text{if } k = n, \\
    0 & \text{if } k \ne n.
  \end{cases}
\]
\end{definition}

\begin{definition}[Postnikov tower]\label{def:postnikov-tower}
The \emph{Postnikov tower} of a type $A$ is a sequence of
$n$-groupoid truncations:
\[
  \cdots \;\longrightarrow\; A_{\le n+1} \;\longrightarrow\;
  A_{\le n} \;\longrightarrow\; \cdots \;\longrightarrow\;
  A_{\le 1} \;\longrightarrow\; A_{\le 0}.
\]
The $n$-th Postnikov stage $A_{\le n}$ retains all cells of the
$\omega$-groupoid tower up to dimension $n$ and collapses all higher
cells to the trivial type.
\end{definition}

\begin{theorem}[Postnikov convergence]\label{thm:postnikov-convergence}
\hfill
\begin{enumerate}
\item[\textup{(i)}] If $k \le n$, then the $k$-cells of stage $n$
  agree with the $k$-cells of the full tower.
\item[\textup{(ii)}] For $k \le n \le m$, the $k$-cells of stages
  $n$ and $m$ are equal.
\item[\textup{(iii)}] Stage $n$ kills $\pi_{n+1}$: the
  $(n+1)$-cells at stage $n$ form the trivial type.
\end{enumerate}
\end{theorem}

\begin{proof}
All three properties follow from the definition of the truncation
functor $\trunc_n$ on the $\omega$-groupoid cell types: for
$k \le n$, $\trunc_n(\text{cells}(k)) = \text{cells}(k)$;
for $k > n$, $\trunc_n(\text{cells}(k)) = \mathbf{1}$.
\end{proof}

\begin{remark}
The 1-truncation of the Postnikov tower recovers the strict
groupoid $\PathRwQuot$ (Theorem~5.11), connecting the truncation
framework to the quotient construction of Part~I.
\end{remark}

\section{The Whitehead Theorem}\label{sec:whitehead}

\begin{definition}[Weak equivalence]\label{def:weak-equiv}
A function $f : A \to B$ is a \emph{weak homotopy equivalence} if the
induced maps $f_* : \pi_n(A, a) \to \pi_n(B, f(a))$ are isomorphisms
for all $n \ge 0$ and all base points $a \in A$.
\end{definition}

\begin{theorem}[Whitehead]\label{thm:whitehead}
If $f : A \to B$ is a weak homotopy equivalence between types
satisfying appropriate finiteness conditions \textup{(}CW
approximability\textup{)}, then $f$ is a homotopy equivalence: there
exists $g : B \to A$ with $g \circ f \sim \id_A$ and
$f \circ g \sim \id_B$.
\end{theorem}

\begin{remark}
The formalization packages the Whitehead theorem as a structure:
given weak equivalence data (isomorphisms on all $\pi_n$) together
with a candidate quasi-inverse, it produces a \texttt{SimpleEquiv}
between the types. The induced map preserves the identity element
and respects loop composition.
\end{remark}

\section{Spectral Sequences}\label{sec:spectral}

The formalization includes the basic algebraic framework of spectral
sequences, providing the language for more advanced computations.

\begin{definition}[Pointed set with differential]\label{def:ptset}
A \emph{pointed set} is a pair $(X, 0)$ where $0 \in X$ is a
distinguished element. A \emph{morphism} of pointed sets is a function
preserving the distinguished element. The \emph{zero morphism}
sends everything to $0$.
\end{definition}

\begin{definition}[Spectral page]\label{def:spectral-page}
A \emph{spectral page} with bound $N$ consists of:
\begin{itemize}
\item a bigraded family of pointed sets $E^{p,q}$ for
  $0 \le p, q < N$;
\item differentials $d : E^{p,q} \to E^{p,q}$ satisfying
  $d \circ d = 0$ (the differential squares to the zero element).
\end{itemize}
\end{definition}

\begin{definition}[Spectral sequence]\label{def:spectral-seq}
A \emph{spectral sequence} is a sequence of spectral pages
$(E_r)_{r \ge 0}$ with connecting morphisms
$E_r^{p,q} \to E_{r+1}^{p,q}$ between successive pages.
\end{definition}

\begin{definition}[Degeneration]\label{def:degeneration}
A spectral sequence \emph{degenerates at page $r_0$} if all
differentials on pages $r \ge r_0$ are zero morphisms.
\end{definition}

\begin{theorem}[Convergence for finite filtrations]\label{thm:ss-convergence}
If a spectral sequence degenerates at page $r_0$, then for all
$r \ge r_0$, all differentials are trivial and the terms stabilize.
\end{theorem}

\begin{definition}[Morphism of spectral sequences]\label{def:ss-morphism}
A \emph{morphism} $\varphi : E \to F$ of spectral sequences consists
of pointed-set morphisms $\varphi_r^{p,q} : E_r^{p,q} \to F_r^{p,q}$
on each page, commuting with the differentials:
$\varphi_r \circ d^E_r = d^F_r \circ \varphi_r$.
\end{definition}

\section{Further Formalized Structures}\label{sec:further}

The computational-paths library formalizes a broad range of additional
homotopy-theoretic structures, which we briefly survey.

\subsection*{Cofiber and Puppe sequences.}
The Barratt--Puppe sequence extends the fiber sequence to the cofiber
side. For a map $f : A \to B$, the cofiber sequence
\[
  A \;\xrightarrow{f}\; B \;\longrightarrow\; \cofib(f)
  \;\longrightarrow\; \Sigma A \;\longrightarrow\; \Sigma B
  \;\longrightarrow\; \cdots
\]
is formalized with computational-path connecting maps at each stage.

\subsection*{Mayer--Vietoris sequence.}
For a pushout square, the Mayer--Vietoris exact sequence in homotopy
groups is derived as a consequence of the Seifert--van~Kampen theorem
and the long exact sequence machinery.

\subsection*{Stable homotopy and spectra.}
The colimit of iterated suspensions yields stable homotopy groups.
The library formalizes:
\begin{itemize}
\item $\Omega$-spectra: sequences of types $(E_n)_{n \ge 0}$ with
  equivalences $E_n \simeq \Omega E_{n+1}$;
\item the stable homotopy category, via the stabilization of the
  suspension map;
\item stable stems and their relation to the Freudenthal theorem.
\end{itemize}

\subsection*{Characteristic classes.}
The library includes structures for:
\begin{itemize}
\item Stiefel--Whitney classes $w_i$ for real vector bundles,
\item Chern classes $c_i$ for complex vector bundles,
\item Pontryagin classes $p_i$ for oriented real bundles.
\end{itemize}
These are defined as cohomology classes of classifying spaces and
interact with the fibration theory of Chapter~\ref{ch:fibrations}.

\subsection*{Operads and $A_\infty$-algebras.}
The algebraic structure of iterated loop spaces is captured by
operads. The formalization includes:
\begin{itemize}
\item the associahedron operad and its action on loop spaces,
\item $A_\infty$-algebra structures arising from path composition,
\item the recognition principle relating $n$-fold loop spaces to
  algebras over the little $n$-cubes operad.
\end{itemize}

\subsection*{Rational homotopy theory.}
The formalization records the rationalization functor and the
connection between rational homotopy groups and Sullivan's minimal
models, providing a bridge to the algebraic side of the theory.

\subsection*{Additional structures.}
The library also formalizes:
\begin{itemize}
\item K-theory (algebraic and topological),
\item topological Hochschild homology (THH),
\item Goodwillie calculus and polynomial functors,
\item motivic and \'etale cohomology,
\item surgery theory and bordism,
\item model categories and Quillen adjunctions,
\item higher topos theory and $\infty$-categories.
\end{itemize}
A comprehensive index mapping each structure to its Lean~4 module is
provided in Appendix~B.

\section{Summary of Part~II}\label{sec:part2-summary}

Part~II has developed the homotopy theory of computational paths from
the fundamental group through advanced topics. The key results are
summarized in Table~\ref{tab:part2-summary}.

\begin{table}[ht]
\centering
\caption{Summary of Part~II results.}\label{tab:part2-summary}
\begin{tabular}{lll}
\toprule
\textbf{Result} & \textbf{Reference} & \textbf{Key equation} \\
\midrule
$\pi_1$ is a group & Thm.~\ref{thm:pi1-group} & strict group laws \\
Product formula & Thm.~\ref{thm:pi1-product}
  & $\pi_1(A \times B) \cong \pi_1(A) \times \pi_1(B)$ \\
Eckmann--Hilton & Thm.~\ref{thm:eckmann-hilton}
  & $\pi_n$ abelian for $n \ge 2$ \\
$\pi_1(S^1) \cong \mathbb{Z}$ & Thm.~\ref{thm:pi1-circle}
  & winding number isomorphism \\
$\pi_1(T^2) \cong \mathbb{Z}^2$ & Thm.~\ref{thm:pi1-torus}
  & product of circles \\
$\pi_1(S^1 \vee S^1) \cong \mathbb{Z} * \mathbb{Z}$
  & Thm.~\ref{thm:pi1-figure-eight} & free product \\
Seifert--van Kampen & Thm.~\ref{thm:svk}
  & $\pi_1$ of pushouts \\
Long exact sequence & Thm.~\ref{thm:les}
  & $\pi_1(F) \to \pi_1(E) \to \pi_1(B) \to P(b)$ \\
Hurewicz theorem & Thm.~\ref{thm:hurewicz}
  & $H_1 \cong \pi_1^{\ab}$ \\
Whitehead theorem & Thm.~\ref{thm:whitehead}
  & weak equiv $\Rightarrow$ homotopy equiv \\
\bottomrule
\end{tabular}
\end{table}

The computational-paths framework provides a distinctive perspective
on all these results: the group, groupoid, and higher-categorical
structures are \emph{derived} from the rewrite system on path traces,
rather than being postulated as axioms. The rewriting infrastructure
of Part~I directly produces the strict algebraic identities that make
these homotopy-theoretic constructions possible.
