% ============================================================================
% Chapter 5: Higher-Dimensional Structure
% ============================================================================
\chapter{Higher-Dimensional Structure}
\label{ch:higher-dimensional}

We now ascend from the one-dimensional algebra of paths to the
higher-dimensional structure that constitutes the central contribution of this
work. Rewrite equalities between paths serve as \emph{two-cells}, and iterated
derivation structures provide cells at every dimension. The resulting tower
forms a weak $\omega$-groupoid, with a sharp contractibility threshold at
dimension~3.

\section{Two-Cells and the Bicategory of Paths}
\label{sec:two-cells}

\begin{definition}[Two-Cell]\label{def:two-cell}
  A \emph{two-cell} between paths $p, q : \Path_A(a,b)$ is a witness of
  rewrite equality:
  \[
    \eta : p \rweq q.
  \]
  Two-cells inhabit $\Prop$ (they are proof-irrelevant), since $\RwEq$ is a
  proposition.
\end{definition}

\begin{definition}[Two-Cell Operations]\label{def:two-cell-ops}
  Two-cells support the following operations:
  \begin{enumerate}[label=(\roman*)]
    \item \textbf{Identity.}\; $\mathrm{id}_p : p \rweq p$ \quad
      (via $\RwEq.\refl$).
    \item \textbf{Vertical composition.}\;
      $\eta \circ_v \theta : p \rweq r$ \quad for $\eta : p \rweq q$ and
      $\theta : q \rweq r$ \quad (via $\RwEq.\mathrm{trans}$).
    \item \textbf{Left whiskering.}\;
      $f \triangleright_L \eta : f \comp g \rweq f \comp h$ \\
      for $f : \Path_A(a,b)$ and $\eta : g \rweq h$ where
      $g, h : \Path_A(b,c)$ \\
      (via the congruence of $\mathrm{trans}$ in its second argument).
    \item \textbf{Right whiskering.}\;
      $\eta \triangleleft_R h : f \comp h \rweq g \comp h$ \\
      for $\eta : f \rweq g$ and $h : \Path_A(b,c)$ \\
      (via the congruence of $\mathrm{trans}$ in its first argument).
    \item \textbf{Horizontal composition.}\;
      $\eta \circ_h \theta : f \comp g \rweq f' \comp g'$ \\
      for $\eta : f \rweq f'$ and $\theta : g \rweq g'$ \\
      (defined as $(\eta \triangleleft_R g) \circ_v (f' \triangleright_L \theta)$).
  \end{enumerate}
\end{definition}

\subsection{Associator and Unitor Two-Cells}

The rewrite rules from Group~I provide canonical two-cells witnessing the
coherence data of a bicategory:

\begin{definition}[Associator]\label{def:associator}
  For composable paths $p : \Path_A(a,b)$, $q : \Path_A(b,c)$,
  $r : \Path_A(c,d)$, the \emph{associator} is the two-cell
  \[
    \alpha_{p,q,r} : (p \comp q) \comp r \;\rweq\; p \comp (q \comp r),
  \]
  given by $\RwEq.\mathrm{step}$ applied to rule~\ref{rule:tt}.
\end{definition}

\begin{definition}[Unitors]\label{def:unitors}
  The \emph{left} and \emph{right unitors} are:
  \begin{align*}
    \lambda_p &: \refl(a) \comp p \;\rweq\; p &
    &\text{(via rule~\ref{rule:lrr})}, \\
    \rho_p &: p \comp \refl(b) \;\rweq\; p &
    &\text{(via rule~\ref{rule:rrr})}.
  \end{align*}
\end{definition}

\subsection{Coherence Laws}

\begin{theorem}[Pentagon Coherence]\label{thm:pentagon}
  For composable paths $p, q, r, s$, the two canonical ways of
  reassociating the four-fold composite agree:
  \[
    \alpha_{p \comp q, r, s} \circ_v \alpha_{p, q, r \comp s}
    \;=\;
    (\alpha_{p,q,r} \triangleleft_R s) \circ_v
    \alpha_{p, q \comp r, s} \circ_v
    (p \triangleright_L \alpha_{q,r,s}).
  \]
  This equality of two-cells holds by proof irrelevance of $\RwEq$.
\end{theorem}

\begin{theorem}[Triangle Coherence]\label{thm:triangle}
  For composable paths $p : \Path_A(a,b)$ and $q : \Path_A(b,c)$:
  \[
    \alpha_{p, \refl(b), q} \circ_v (p \triangleright_L \lambda_q)
    \;=\;
    \rho_p \triangleleft_R q.
  \]
  Again, this holds by proof irrelevance.
\end{theorem}

\begin{theorem}[Interchange Law]\label{thm:interchange}
  For four two-cells $\eta_1, \eta_2, \theta_1, \theta_2$ arranged in a
  $2 \times 2$ grid:
  \[
    (\eta_1 \circ_h \theta_1) \circ_v (\eta_2 \circ_h \theta_2)
    \;=\;
    (\eta_1 \circ_v \eta_2) \circ_h (\theta_1 \circ_v \theta_2).
  \]
  This is the \emph{middle-four interchange}. It holds because both
  sides inhabit the same $\Prop$-valued type.
\end{theorem}

\begin{remark}\label{rem:proof-irrel-coherence}
  The pentagon, triangle, and interchange laws hold trivially (by
  $\mathsf{Subsingleton.elim}$) because two-cells are $\Prop$-valued.
  This is a feature, not a deficiency: it means that all coherence
  conditions at the two-cell level and above are \emph{automatically}
  satisfied. The non-trivial content of the theory lies at the
  one-cell level, where distinct paths encode genuinely different
  computational traces.
\end{remark}

\subsection{The Weak Bicategory and Weak 2-Groupoid}

\begin{definition}[Weak Bicategory]\label{def:weak-bicat}
  A \emph{weak bicategory} consists of:
  \begin{itemize}
    \item 0-cells (objects), 1-cells (morphisms), 2-cells.
    \item Composition and identity at the 1-cell level.
    \item Vertical and horizontal composition, whiskering, identity at
      the 2-cell level.
    \item Associator and unitors as invertible 2-cells.
    \item Pentagon and triangle coherences.
  \end{itemize}
\end{definition}

\begin{theorem}\label{thm:weak-bicat}
  Computational paths form a weak bicategory, with:
  \begin{center}
  \begin{tabular}{ll}
    0-cells: & elements of $A$, \\
    1-cells: & paths $p : \Path_A(a,b)$, \\
    2-cells: & rewrite equalities $\eta : p \rweq q$.
  \end{tabular}
  \end{center}
\end{theorem}

\begin{definition}[Weak 2-Groupoid]\label{def:weak-2-gpd}
  A \emph{weak 2-groupoid} extends a weak bicategory with:
  \begin{itemize}
    \item An inversion $\mathrm{inv}_1$ on 1-cells, with cancellation
      two-cells: $p \comp \inv{p} \rweq \refl(a)$ and $\inv{p} \comp p \rweq \refl(b)$.
    \item An inversion on 2-cells: $\eta : p \rweq q$ implies $\inv{\eta} : q \rweq p$.
  \end{itemize}
\end{definition}

\begin{theorem}\label{thm:weak-2-gpd}
  Computational paths form a weak 2-groupoid, with $\mathrm{inv}_1 = \symop$
  and inversion on 2-cells given by $\RwEq.\mathrm{symm}$.
\end{theorem}

\section{The Globular Tower}
\label{sec:globular-tower}

\begin{definition}[Globular Cell]\label{def:globular-cell}
  A \emph{globular cell} over a type $\beta$ is a triple
  \[
    c = (\mathrm{src}, \mathrm{tgt}, \mathrm{path})
    \quad\text{where}\quad
    \mathrm{src}, \mathrm{tgt} : \beta \quad\text{and}\quad
    \mathrm{path} : \Path_\beta(\mathrm{src}, \mathrm{tgt}).
  \]
  Globular cells carry reflexivity, symmetry, and composition operations
  inherited from $\Path$, satisfying the analogous algebraic laws.
\end{definition}

\begin{definition}[Globular Tower]\label{def:globular-tower}
  The \emph{globular tower} over a type $A$ is defined inductively:
  \begin{align*}
    \mathrm{Level}_0(A) &\;=\; A, \\
    \mathrm{Level}_{n+1}(A) &\;=\; \mathrm{GlobularCell}(\mathrm{Level}_n(A)).
  \end{align*}
  Each level carries $\refl$, $\symop$, and $\mathrm{trans}$ operations,
  as well as a functorial $\mathrm{map}$ operation that sends a function
  $f : A \to B$ to level-wise maps $\mathrm{Level}_n(f) :
  \mathrm{Level}_n(A) \to \mathrm{Level}_n(B)$.
\end{definition}

\begin{proposition}\label{prop:globular-tower-functorial}
  The $\mathrm{map}$ operation on globular levels satisfies:
  \begin{enumerate}[label=(\roman*)]
    \item $\mathrm{map}(\refl(x)) = \refl(\mathrm{map}(x))$.
    \item $\mathrm{map}(\symop(c)) = \symop(\mathrm{map}(c))$.
    \item $\mathrm{map}(\mathrm{trans}(p, q, h)) =
      \mathrm{trans}(\mathrm{map}(p), \mathrm{map}(q), f_*(h))$.
  \end{enumerate}
\end{proposition}

The globular tower provides the \emph{geometric} scaffolding for the
$\omega$-groupoid, but it does not encode the rewrite structure. For that,
we need the derivation cells.

\section{Derivation Cells and the Weak $\omega$-Groupoid}
\label{sec:omega-groupoid}

\subsection{Dimension 2: Derivations Between Paths}

\begin{definition}[Derivation$_2$]\label{def:derivation2}
  A \emph{derivation} (or \emph{type-valued two-cell}) between paths
  $p, q : \Path_A(a,b)$ is an element of the inductive type
  $\Derivation_2(p, q)$ with constructors:
  \begin{enumerate}[label=(\roman*)]
    \item $\Derivation_2.\refl(p) : \Derivation_2(p, p)$.
    \item $\Derivation_2.\mathrm{step}(s) : \Derivation_2(p, q)$ for
      $s : p \rew q$.
    \item $\Derivation_2.\mathrm{inv}(\delta) : \Derivation_2(q, p)$ for
      $\delta : \Derivation_2(p, q)$.
    \item $\Derivation_2.\mathrm{vcomp}(\delta_1, \delta_2) :
      \Derivation_2(p, r)$ for $\delta_1 : \Derivation_2(p, q)$ and
      $\delta_2 : \Derivation_2(q, r)$.
  \end{enumerate}
\end{definition}

Unlike $\RwEq$ (which is $\Prop$-valued), $\Derivation_2$ is
\emph{type-valued}: it carries explicit derivation structure, recording
which steps and closure operations were applied.

\begin{proposition}\label{prop:d2-projects}
  Every $\Derivation_2(p,q)$ projects to an $\RwEq(p,q)$ witness via a
  forgetful map $\Derivation_2(p,q) \to \RwEq(p,q)$. Moreover,
  $\Derivation_2(p,q)$ is inhabited if and only if $p \rweq q$.
\end{proposition}

$\Derivation_2$ supports horizontal operations:

\begin{definition}[Horizontal Operations on $\Derivation_2$]\label{def:d2-horizontal}
  \begin{enumerate}[label=(\roman*)]
    \item \textbf{Left whiskering.}\;
      $f \triangleright_L \delta : \Derivation_2(f \comp g, f \comp h)$
      for $\delta : \Derivation_2(g, h)$.
    \item \textbf{Right whiskering.}\;
      $\delta \triangleleft_R h : \Derivation_2(f \comp h, g \comp h)$
      for $\delta : \Derivation_2(f, g)$.
    \item \textbf{Horizontal composition.}\;
      $\delta_1 \circ_h \delta_2 : \Derivation_2(p \comp q, p' \comp q')$
      for $\delta_1 : \Derivation_2(p, p')$ and $\delta_2 : \Derivation_2(q, q')$.
  \end{enumerate}
\end{definition}

\subsection{Dimension 3: Meta-Steps and Derivations Between Derivations}

\begin{definition}[MetaStep$_3$]\label{def:metastep3}
  A \emph{primitive three-cell} $\mathrm{MetaStep}_3(\delta_1, \delta_2)$
  between derivations $\delta_1, \delta_2 : \Derivation_2(p, q)$ witnesses
  that $\delta_1$ and $\delta_2$ are ``equivalent as derivations.'' The
  constructors include:
  \begin{enumerate}[label=(\roman*)]
    \item \textbf{Groupoid laws for derivations:}
      \begin{align*}
        &\mathrm{vcomp}(\refl(p), \delta) \;\mapsto\; \delta, \\
        &\mathrm{vcomp}(\delta, \refl(q)) \;\mapsto\; \delta, \\
        &\mathrm{vcomp}(\mathrm{vcomp}(\delta_1, \delta_2), \delta_3)
          \;\mapsto\; \mathrm{vcomp}(\delta_1,
          \mathrm{vcomp}(\delta_2, \delta_3)), \\
        &\mathrm{inv}(\mathrm{inv}(\delta)) \;\mapsto\; \delta, \\
        &\mathrm{vcomp}(\mathrm{inv}(\delta), \delta) \;\mapsto\; \refl(q), \\
        &\mathrm{vcomp}(\delta, \mathrm{inv}(\delta)) \;\mapsto\; \refl(p), \\
        &\mathrm{inv}(\mathrm{vcomp}(\delta_1, \delta_2)) \;\mapsto\;
          \mathrm{vcomp}(\mathrm{inv}(\delta_2), \mathrm{inv}(\delta_1)).
      \end{align*}
    \item \textbf{Step coherence:} any two single-step derivations
      $\mathrm{step}(s_1)$ and $\mathrm{step}(s_2)$ for the same
      $p \rew q$ are connected.
    \item \textbf{$\RwEq$-coherence:} any two derivations projecting to
      the same $\RwEq$ witness are connected:
      $\delta_1.\mathrm{toRwEq} = \delta_2.\mathrm{toRwEq} \implies
      \mathrm{MetaStep}_3(\delta_1, \delta_2)$.
    \item \textbf{Bicategorical coherences:} pentagon and triangle for
      derivation-level associators.
    \item \textbf{Whiskering:} whiskering preserves meta-steps.
  \end{enumerate}
\end{definition}

\begin{definition}[Derivation$_3$]\label{def:derivation3}
  $\Derivation_3(\delta_1, \delta_2)$ is the free groupoid generated by
  $\mathrm{MetaStep}_3$: it has constructors $\refl$, $\mathrm{step}$
  (from $\mathrm{MetaStep}_3$), $\mathrm{inv}$, and $\mathrm{vcomp}$.
\end{definition}

\subsection{The Contractibility Theorem}

\begin{theorem}[Contractibility at Dimension $\geq 3$]\label{thm:contract3}
  For any two parallel derivations $\delta_1, \delta_2 : \Derivation_2(p,q)$,
  there exists a three-cell connecting them:
  \[
    \mathrm{contract}_3(\delta_1, \delta_2) \;:\; \Derivation_3(\delta_1, \delta_2).
  \]
\end{theorem}

\begin{proof}
  Since $\RwEq$ is $\Prop$-valued, the projections
  $\delta_1.\mathrm{toRwEq}$ and $\delta_2.\mathrm{toRwEq}$ are equal
  by $\mathsf{Subsingleton.elim}$. The $\RwEq$-coherence constructor of
  $\mathrm{MetaStep}_3$ then produces a primitive three-cell, which lifts
  to $\Derivation_3$ via the $\mathrm{step}$ constructor.
\end{proof}

\begin{corollary}[Loop Contraction]\label{cor:loop-contract}
  Every loop derivation $\delta : \Derivation_2(p, p)$ contracts to the
  identity:
  \[
    \Derivation_3(\delta,\; \Derivation_2.\refl(p)).
  \]
\end{corollary}

\begin{remark}[Critical Design Choice]\label{rem:contract-threshold}
  Contractibility starts at dimension~3, \textbf{not} at dimension~2. At
  dimension~2, $\Derivation_2(p,q)$ is inhabited only when $p \rweq q$;
  it does not connect arbitrary parallel paths. This is essential for
  capturing non-trivial fundamental groups. For example, on the circle
  $S^1$, the loop generator and $\refl$ are \emph{not} connected by a
  two-cell, which is what allows $\pi_1(S^1) \cong \ZZ$.
\end{remark}

\subsection{Dimensions 4 and Beyond}

The pattern continues uniformly:

\begin{definition}[Higher Derivation Cells]\label{def:higher-cells}
  \begin{enumerate}[label=(\roman*)]
    \item $\mathrm{MetaStep}_4$ and $\Derivation_4(\mu_1, \mu_2)$
      for $\mu_1, \mu_2 : \Derivation_3(\delta_1, \delta_2)$, with
      analogous groupoid laws, step coherence, and whiskering.
    \item For $n \geq 5$, $\mathrm{DerivationHigh}_n(c_1, c_2)$
      for $c_1, c_2 : \Derivation_4(\mu_1, \mu_2)$, parametrized by
      dimension.
  \end{enumerate}
\end{definition}

\begin{theorem}[Contractibility at Dimension $\geq 4$]\label{thm:contract4}
  For any two parallel three-cells $\mu_1, \mu_2 : \Derivation_3(\delta_1,
  \delta_2)$:
  \[
    \mathrm{contract}_4(\mu_1, \mu_2) \;:\; \Derivation_4(\mu_1, \mu_2).
  \]
  More generally, for all $k \geq 3$, any two parallel $(k-1)$-cells are
  connected by a $k$-cell.
\end{theorem}

\begin{proof}
  The same argument as \cref{thm:contract3}: the projection of each
  higher cell to its $\Prop$-valued counterpart is unique by proof
  irrelevance, and the coherence constructor lifts this to an explicit
  higher cell.
\end{proof}

\subsection{The Cell Types}

\begin{definition}[Cell Type at Each Dimension]\label{def:cell-types}
  \[
    \mathrm{Cell}_k(A) \;=\;
    \begin{cases}
      A & k = 0, \\
      \Sigma_{a,b : A}\; \Path_A(a,b) & k = 1, \\
      \Sigma_{a,b,p,q}\; \Derivation_2(p,q) & k = 2, \\
      \Sigma_{\ldots}\; \Derivation_3(\delta_1, \delta_2) & k = 3, \\
      \Sigma_{\ldots}\; \Derivation_4(\mu_1, \mu_2) & k = 4, \\
      \Sigma_{\ldots}\; \mathrm{DerivationHigh}_{k-5}(c_1, c_2) & k \geq 5.
    \end{cases}
  \]
\end{definition}

\subsection{The Main Structure Theorem}

\begin{definition}[Weak $\omega$-Groupoid]\label{def:omega-gpd}
  A \emph{weak $\omega$-groupoid} (in the sense of Batanin--Leinster
  \cite{Leinster04}) on a type $A$ consists of:
  \begin{itemize}
    \item Cells at every dimension: $\mathrm{Cell}_k(A)$ for $k \in \Nat$.
    \item Groupoid operations (identity, composition, inversion) at each
      dimension.
    \item Contractibility: for $k \geq 3$, any two parallel $(k-1)$-cells
      are connected by a $k$-cell.
    \item Bicategorical coherence: pentagon and triangle at the
      2-cell level, witnessed as 3-cells.
  \end{itemize}
\end{definition}

\begin{theorem}[Main Structure Theorem]\label{thm:omega-groupoid}
  For any type $A$, the tower
  \[
    A, \quad \Path, \quad \Derivation_2, \quad \Derivation_3, \quad
    \Derivation_4, \quad \ldots
  \]
  forms a weak $\omega$-groupoid with contractibility starting at
  dimension~3.
\end{theorem}

\begin{proof}
  The construction assembles:
  \begin{itemize}
    \item $\mathrm{contract}_3$ (\cref{thm:contract3}) for the
      contractibility at dimension~3.
    \item $\mathrm{contract}_4$ (\cref{thm:contract4}) for dimension~4.
    \item The parametrized $\mathrm{contractHigh}_n$ for dimensions
      $\geq 5$.
    \item The pentagon coherence (\cref{thm:pentagon}) as a 3-cell:
      $\mathrm{MetaStep}_3.\mathrm{pentagon}$ provides the pentagon
      equation between the two canonical reassociation derivations.
    \item The triangle coherence (\cref{thm:triangle}) as a 3-cell:
      $\mathrm{MetaStep}_3.\mathrm{triangle}$ provides the triangle
      equation.
  \end{itemize}
  Groupoid operations at each dimension are given by the constructors
  of the derivation types ($\refl$, $\mathrm{vcomp}$, $\mathrm{inv}$).
\end{proof}

\section{The Infinity-Groupoid Approximation}
\label{sec:infinity-gpd}

\begin{definition}[Coherence at Level $n$]\label{def:coherence-level}
  \[
    \mathrm{CoherenceAt}(A, n) \;=\;
    \begin{cases}
      \top & n \leq 1, \\
      \forall\, \delta_1, \delta_2 : \Derivation_2(p,q).\;
        \Derivation_3(\delta_1, \delta_2) & n = 2, \\
      \forall\, \mu_1, \mu_2 : \Derivation_3(\delta_1,\delta_2).\;
        \Derivation_4(\mu_1, \mu_2) & n = 3, \\
      \forall\, c_1, c_2.\; \mathrm{DerivationHigh}_{n-4}(c_1, c_2) & n \geq 4.
    \end{cases}
  \]
\end{definition}

\begin{theorem}\label{thm:infinity-gpd}
  For any type $A$, the canonical coherence witnesses at every level
  assemble into an $\infty$-groupoid structure:
  \[
    \mathrm{coherenceAt}(A, n) : \mathrm{CoherenceAt}(A, n)
    \quad\text{for all}\; n \in \Nat.
  \]
\end{theorem}

\begin{definition}[$n$-Groupoid Truncation]\label{def:n-truncation}
  The \emph{$n$-truncation} of the $\omega$-groupoid collapses all cells
  above dimension $n$ to the trivial type $\mathbf{1}$:
  \[
    \mathrm{TruncCell}_k(A, n) \;=\;
    \begin{cases}
      \mathrm{Cell}_k(A) & k \leq n, \\
      \mathbf{1} & k > n.
    \end{cases}
  \]
\end{definition}

\begin{theorem}\label{thm:1-truncation}
  The 1-truncation of the $\omega$-groupoid recovers the strict groupoid
  $\PathQuot$ of \cref{thm:quot-strict-gpd}.
\end{theorem}

\section{Double Groupoid and Symmetric Monoidal Structure}
\label{sec:enriched-structures}

The proof-irrelevance of two-cells enables additional algebraic structures
on the computational path space.

\subsection{Double Groupoid}

\begin{definition}[Double Groupoid]\label{def:double-gpd}
  A \emph{double groupoid} is a weak 2-groupoid equipped with an explicit
  interchange law: the two ways of composing a $2 \times 2$ grid of
  two-cells (vertical-then-horizontal vs.\ horizontal-then-vertical)
  are equal.
\end{definition}

\begin{theorem}\label{thm:double-gpd}
  Computational paths form a double groupoid. The interchange law holds
  by proof irrelevance of $\RwEq$.
\end{theorem}

\subsection{Groupoid-Enriched Category}

\begin{definition}[Groupoid-Enriched Category]\label{def:gpd-enriched}
  A \emph{groupoid-enriched category} is a weak bicategory whose
  hom-categories (the categories of 2-cells between fixed 1-cells) are
  groupoids---i.e., all 2-cells are invertible, and the groupoid axioms
  (associativity, units, inverses for vertical composition) hold as
  equalities.
\end{definition}

\begin{theorem}\label{thm:gpd-enriched}
  Computational paths form a groupoid-enriched category. All groupoid
  axioms for 2-cells hold by $\mathsf{Subsingleton.elim}$ on $\RwEq$.
\end{theorem}

\subsection{Symmetric Monoidal Structure}

\begin{definition}[Monoidal Path Algebra]\label{def:monoidal}
  Viewing path composition as a tensor product ($\otimes = \mathrm{trans}$)
  and the reflexive path as the unit ($I = \refl$), the computational path
  space carries a \emph{monoidal structure} with:
  \begin{itemize}
    \item Associator: $\alpha_{p,q,r} : (p \comp q) \comp r \rweq
      p \comp (q \comp r)$.
    \item Left unitor: $\lambda_p : \refl \comp p \rweq p$.
    \item Right unitor: $\rho_p : p \comp \refl \rweq p$.
    \item Pentagon and triangle coherences.
  \end{itemize}
\end{definition}

\begin{definition}[Braiding]\label{def:braiding}
  The \emph{braiding} is provided by the anti-homomorphism of symmetry
  (\cref{thm:strict-antihom}):
  \[
    \beta_{p,q} \;:\; \inv{(p \comp q)} \;\rweq\; \inv{q} \comp \inv{p}.
  \]
  This is a two-cell (rewrite equality) provided by rule~\ref{rule:stss}.
\end{definition}

\begin{theorem}[Symmetric Monoidal Path Algebra]\label{thm:sym-monoidal}
  The computational path space, equipped with the monoidal structure
  and braiding defined above, forms a \emph{symmetric monoidal}
  structure. The braiding satisfies:
  \begin{enumerate}[label=(\roman*)]
    \item $\inv{(\inv{q} \comp \inv{p})} \rweq p \comp q$ \;(inverse braiding).
    \item The hexagon identities (left and right) hold as equalities of
      $\Prop$-valued two-cells.
  \end{enumerate}
  Naturality of the braiding with respect to $\RwEq$ follows from the
  congruence property.
\end{theorem}

\medskip

This concludes Part~I of the paper. In Part~II, we develop the homotopy theory
built on these foundations: fundamental groups, covering spaces, fibrations,
exact sequences, and the computation of $\pi_1$ for standard spaces.
