%% ============================================================================
%% Chapter 8: Fibrations, Covering Spaces, and Exact Sequences
%% Part II of "The Algebra of Computational Paths"
%% ============================================================================

\section{Fibrations, Covering Spaces, and Exact Sequences}\label{sec:fibrations}

This chapter develops the theory of fibrations, covering spaces, and exact sequences of homotopy groups in the computational-paths framework. The central idea is that a dependent type $P : B \to \mathsf{Type}$ serves as a fibration, with the transport operation along computational paths providing the homotopy lifting property.

\subsection{Fibrations}\label{subsec:fibrations}

\begin{definition}[Fiber]\label{def:fiber}
Let $f : A \to B$ and $b : B$. The \emph{fiber} of $f$ over $b$ is
\[
  \mathrm{Fiber}(f, b) = \{a : A \mid f(a) = b\} = \Sigma(a : A).\, \mathrm{Path}(f(a), b).
\]
A fiber element consists of a point $a : A$ together with a computational path witnessing $f(a) = b$.
\end{definition}

\begin{definition}[Type family as fibration]\label{def:fibration-family}
A dependent type $P : B \to \mathsf{Type}$ is viewed as a fibration via its \emph{total space}
\[
  E = \Sigma(b : B).\, P(b)
\]
with projection $\mathrm{pr}_1 : E \to B$. The fiber of $\mathrm{pr}_1$ over $b$ is canonically equivalent to $P(b)$:
\[
  \mathrm{Fiber}(\mathrm{pr}_1, b) \;\simeq\; P(b).
\]
\end{definition}

\begin{theorem}[Path lifting]\label{thm:path-lifting}
Given a type family $P : B \to \mathsf{Type}$, a computational path $p : \mathrm{Path}_B(b_1, b_2)$, and a point $x : P(b_1)$, there exists a lifted path
\[
  \widetilde{p} : \mathrm{Path}_E\bigl((b_1, x),\; (b_2, \mathrm{transport}_P(p, x))\bigr)
\]
in the total space $E = \Sigma P$, where $\mathrm{transport}_P(p, x)$ is the fiber transport of $x$ along $p$.
\end{theorem}

\begin{proof}
The lifted path is constructed as $\widetilde{p} = \mathsf{ofEq}(h)$, where $h$ is the propositional equality $(b_1, x) = (b_2, \mathrm{transport}_P(p, x))$ obtained from $p.\mathsf{toEq} : b_1 = b_2$ by dependent elimination. The projection of $\widetilde{p}$ recovers the base path: $\mathsf{congrArg}(\mathrm{pr}_1, \widetilde{p}) = \mathsf{ofEq}(p.\mathsf{toEq})$.
\end{proof}

The transport operation satisfies the expected composition and inversion laws:

\begin{proposition}[Transport laws]\label{prop:transport-laws}
For a type family $P : B \to \mathsf{Type}$:
\begin{enumerate}
  \item $\mathrm{transport}_P(\mathsf{refl}(b), x) = x$.
  \item $\mathrm{transport}_P(\mathsf{trans}(p, q), x) = \mathrm{transport}_P(q, \mathrm{transport}_P(p, x))$.
  \item $\mathrm{transport}_P(\mathsf{symm}(p), \mathrm{transport}_P(p, x)) = x$.
  \item $\mathrm{transport}_P(p, \mathrm{transport}_P(\mathsf{symm}(p), y)) = y$.
\end{enumerate}
\end{proposition}

\begin{proof}
These follow from the transport laws of computational paths (Theorem~2.11) together with the inversion properties of the rewrite system (Rules~3--6).
\end{proof}

\begin{definition}[Fiber sequence]\label{def:fiber-sequence}
A \emph{fiber sequence} $F \to E \xrightarrow{p} B$ consists of:
\begin{itemize}
  \item A projection $p : E \to B$ with basepoints $b_0 : B$ and $e_0 : E$ satisfying $\mathrm{Path}(p(e_0), b_0)$;
  \item A type $F$ together with an equivalence $F \simeq \mathrm{Fiber}(p, b_0)$.
\end{itemize}
The \emph{inclusion} $\iota : F \to E$ sends a fiber element to its underlying point in $E$.
\end{definition}

\begin{proposition}[Exactness of the canonical fiber sequence]\label{prop:fiber-exact}
For a type family $P : B \to \mathsf{Type}$ with basepoint $b_0 : B$ and $x_0 : P(b_0)$, the canonical fiber sequence
\[
  P(b_0) \xrightarrow{\;\iota\;} \Sigma P \xrightarrow{\;\mathrm{pr}_1\;} B
\]
is \emph{exact at $\Sigma P$}: an element $e : \Sigma P$ lies in the image of $\iota$ if and only if $\mathrm{Path}(\mathrm{pr}_1(e), b_0)$.
\end{proposition}

\subsection{The Hopf Fibration}\label{subsec:hopf}

The Hopf fibration is a paradigmatic example of a non-trivial fiber bundle, and its formalization in the computational-paths framework demonstrates the system's capacity for encoding concrete topological data.

\begin{theorem}[Hopf fibration data]\label{thm:hopf}
There exists a fibration structure
\[
  S^1 \hookrightarrow S^3 \xrightarrow{\;\eta\;} S^2
\]
consisting of:
\begin{enumerate}
  \item A projection $\eta : S^3 \to S^2$;
  \item Basepoints $s_0 : S^2$ and $\tilde{s}_0 : S^3$ with $\mathrm{Path}(\eta(\tilde{s}_0), s_0)$;
  \item An equivalence $\mathrm{Fiber}(\eta, s_0) \simeq S^1$.
\end{enumerate}
The fiber inclusion $\iota : S^1 \to S^3$ and the projection $\eta$ satisfy the fiber sequence exactness: $\mathrm{Path}(\eta(\iota(x)), s_0)$ for all $x : S^1$.
\end{theorem}

\begin{proof}
The Hopf fibration data is packaged as a record containing the projection function, basepoint witnesses, and the fiber equivalence. The exactness condition follows from the definition of the fiber sequence structure: every fiber element projects to the basepoint. The computational path witnesses ensure that all conditions hold constructively.
\end{proof}

\subsection{Covering Spaces}\label{subsec:covering}

\begin{definition}[Covering space]\label{def:covering}
A type family $P : A \to \mathsf{Type}$ is a \emph{covering space} of $A$ if every fiber $P(a)$ is a set (i.e., $0$-truncated: any two paths in $P(a)$ with the same endpoints are equal).
\end{definition}

The discreteness condition ensures unique path lifting:

\begin{theorem}[Unique path lifting for coverings]\label{thm:covering-unique-lift}
If $P$ is a covering space of $A$, then for any path $p : \mathrm{Path}_A(a, b)$, the transport map $\mathrm{transport}_P(p, {-}) : P(a) \to P(b)$ is injective.
\end{theorem}

\begin{proof}
Suppose $\mathrm{transport}_P(p, x) = \mathrm{transport}_P(p, y)$. Applying $\mathrm{transport}_P(\mathsf{symm}(p), {-})$ to both sides and using the transport inversion law (Proposition~\ref{prop:transport-laws}(3)), we obtain $x = y$.
\end{proof}

\begin{definition}[Fundamental group action on fibers]\label{def:fiber-action}
For a type family $P : A \to \mathsf{Type}$ and a basepoint $a : A$, the \emph{monodromy action} of $\pi_1(A, a)$ on $P(a)$ is defined by
\[
  \alpha \cdot x = \mathrm{transport}_P(\alpha, x)
\]
for $\alpha : \pi_1(A, a)$ and $x : P(a)$. Here we identify elements of $\pi_1(A,a)$ with their representative loops via the quotient.
\end{definition}

\begin{proposition}[Action laws]\label{prop:action-laws}
The monodromy action is a well-defined group action:
\begin{enumerate}
  \item \emph{Well-definedness}: If $\mathrm{RwEq}(l_1, l_2)$ then $\mathrm{transport}_P(l_1, x) = \mathrm{transport}_P(l_2, x)$. This follows because $l_1.\mathsf{toEq} = l_2.\mathsf{toEq}$ when $\mathrm{RwEq}(l_1, l_2)$.
  \item \emph{Identity}: $e \cdot x = \mathrm{transport}_P(\mathsf{refl}(a), x) = x$.
  \item \emph{Composition}: $\alpha \cdot (\beta \cdot x) = (\alpha \cdot \beta) \cdot x$, by the transport composition law.
\end{enumerate}
\end{proposition}

\begin{theorem}[Galois correspondence]\label{thm:galois}
For a path-connected type $A$ with basepoint $a$, connected coverings of $A$ correspond to conjugacy classes of subgroups of $\pi_1(A, a)$. Specifically:
\begin{itemize}
  \item A connected covering $P$ with $x_0 : P(a)$ determines a subgroup $\mathrm{Stab}(x_0) \leq \pi_1(A, a)$, the stabilizer of $x_0$ under the monodromy action.
  \item Conversely, a subgroup $H \leq \pi_1(A, a)$ determines a covering whose fiber over $a$ is the set of cosets $\pi_1(A, a) / H$.
\end{itemize}
\end{theorem}

\subsection{The Connecting Homomorphism}\label{subsec:connecting}

\begin{definition}[Connecting map]\label{def:connecting-map}
For a type family $P : B \to \mathsf{Type}$ with basepoint $b_0 : B$ and $x_0 : P(b_0)$, the \emph{connecting map} at level~1 is
\[
  \partial : \pi_1(B, b_0) \to P(b_0), \qquad \partial([\ell]) = \mathrm{transport}_P(\ell, x_0).
\]
This map sends a loop in the base to the endpoint of its lift starting at $x_0$.
\end{definition}

\begin{proposition}[Properties of the connecting map]\label{prop:connecting-properties}
\begin{enumerate}
  \item $\partial$ is well-defined: if $\mathrm{RwEq}(\ell_1, \ell_2)$ then $\partial([\ell_1]) = \partial([\ell_2])$.
  \item $\partial(e) = x_0$: the identity loop lifts to the identity.
  \item $\partial([\ell_1] \cdot [\ell_2]) = \mathrm{transport}_P(\ell_2, \partial([\ell_1]))$: the connecting map intertwines loop composition with iterated transport.
\end{enumerate}
\end{proposition}

\begin{proof}
(1) Both loops have the same $\mathsf{toEq}$, hence induce the same transport. (2) Transport along $\mathsf{refl}$ is the identity. (3) Follows from the transport composition law: $\mathrm{transport}(\mathsf{trans}(\ell_1, \ell_2), x_0) = \mathrm{transport}(\ell_2, \mathrm{transport}(\ell_1, x_0))$.
\end{proof}

\subsection{Homotopy Fiber and Cofiber Sequences}\label{subsec:cofiber}

\begin{definition}[Homotopy fiber]\label{def:homotopy-fiber}
For $f : A \to B$ and $b : B$, the \emph{homotopy fiber} is
\[
  \mathrm{hofiber}(f, b) = \Sigma(a : A).\, \mathrm{Path}(f(a), b).
\]
This coincides with $\mathrm{Fiber}(f, b)$ as defined above.
\end{definition}

\begin{definition}[Cofiber]\label{def:cofiber}
The \emph{cofiber} (or mapping cone) of $f : A \to B$ is the pushout
\[
  \mathrm{cofib}(f) = \mathrm{Pushout}(B, \{\ast\}, A, f, !),
\]
where $! : A \to \{\ast\}$ is the unique map.
\end{definition}

\begin{theorem}[Puppe sequence]\label{thm:puppe}
For a map $f : A \to B$, there is a sequence of pointed maps
\[
  A \xrightarrow{f} B \xrightarrow{} \mathrm{cofib}(f) \xrightarrow{} \Sigma A \xrightarrow{\Sigma f} \Sigma B \xrightarrow{} \cdots
\]
where each three consecutive maps form a fiber sequence (up to equivalence). This is the \emph{Puppe sequence} (or Barratt--Puppe sequence).
\end{theorem}

\subsection{The Long Exact Sequence of Homotopy Groups}\label{subsec:long-exact}

The fiber sequence machinery yields the cornerstone result of homotopy theory:

\begin{theorem}[Long exact sequence]\label{thm:long-exact}
For a fiber sequence $F \xrightarrow{\iota} E \xrightarrow{p} B$ with basepoints, there is a long exact sequence of groups and pointed sets:
\[
  \cdots \to \pi_{n+1}(B) \xrightarrow{\;\partial\;} \pi_n(F) \xrightarrow{\;\iota_*\;} \pi_n(E) \xrightarrow{\;p_*\;} \pi_n(B) \xrightarrow{\;\partial\;} \pi_{n-1}(F) \to \cdots \to \pi_0(E) \xrightarrow{\;p_*\;} \pi_0(B).
\]
\emph{Exactness} at each term means: the image of each map equals the kernel of the next.
\end{theorem}

\begin{proof}[Proof outline]
The proof proceeds in three parts:

\medskip
\noindent\textbf{Exactness at $\pi_n(E)$}: The composition $p_* \circ \iota_*$ is trivial because $p \circ \iota$ maps every fiber element to the basepoint of $B$ (by the fiber sequence condition). Conversely, if $p_*(\beta) = e$ for $\beta \in \pi_n(E)$, then the representative loop $\beta$ in $E$ projects to a contractible loop in $B$, and the contraction provides a lift to a loop in $F$.

\medskip
\noindent\textbf{Exactness at $\pi_n(B)$}: The composition $\partial \circ p_*$ is trivial: for $\beta \in \pi_n(E)$, the connecting map $\partial(p_*(\beta))$ yields the transport of the basepoint along a loop that factors through $E$, which returns to the basepoint by exactness. The formalization verifies this using the sigma-type projection and the transport composition law.

\medskip
\noindent\textbf{The connecting homomorphism}: The map $\partial : \pi_{n+1}(B) \to \pi_n(F)$ is defined by lifting a loop in $B$ to the total space and reading off its endpoint in the fiber. For $n = 0$, $\partial$ is the connecting map of Definition~\ref{def:connecting-map}. For $n \geq 1$, $\partial$ is a group homomorphism: it sends the identity loop to the identity element and respects composition (Proposition~\ref{prop:connecting-properties}).
\end{proof}

\begin{corollary}\label{cor:les-trivial-base}
If $\pi_k(B) = 0$ for all $k \leq n$, then $\iota_* : \pi_k(F) \to \pi_k(E)$ is an isomorphism for $k < n$ and surjective for $k = n$.
\end{corollary}

\begin{corollary}\label{cor:les-simply-connected}
For a fibration with simply connected base ($\pi_1(B) = 0$), the connecting map $\partial : \pi_1(B) \to P(b_0)$ is trivial: $\partial(\ell) = x_0$ for every loop $\ell$.
\end{corollary}

\subsection{The Mayer--Vietoris Sequence}\label{subsec:mayer-vietoris}

\begin{theorem}[Mayer--Vietoris]\label{thm:mayer-vietoris}
For a pushout square $A \xleftarrow{f} C \xrightarrow{g} B$ with $P = \mathrm{Pushout}(A, B, C, f, g)$, there is an exact sequence
\[
  \cdots \to \pi_{n+1}(P) \xrightarrow{\;\partial\;} \pi_n(C) \xrightarrow{\;(f_*, g_*)\;} \pi_n(A) \oplus \pi_n(B) \xrightarrow{\;\mathsf{inl}_* - \mathsf{inr}_*\;} \pi_n(P) \xrightarrow{\;\partial\;} \cdots
\]
connecting the homotopy groups of $A$, $B$, $C$, and the pushout $P$.
\end{theorem}

\begin{proof}[Proof sketch]
The Mayer--Vietoris sequence is derived from the long exact sequence of the fiber sequence associated to the pushout. The connecting map $\partial$ is constructed from the glue paths, and exactness at each term follows from the Seifert--van Kampen analysis of pushout loops.
\end{proof}

\begin{remark}\label{rem:naturality}
The long exact sequence and Mayer--Vietoris sequence are \emph{natural}: a morphism of fiber sequences (or pushout squares) induces a morphism of the corresponding exact sequences, by the functoriality of $\pi_n$ (Theorem~\ref{thm:pi1-functor} and its higher analogues).
\end{remark}
