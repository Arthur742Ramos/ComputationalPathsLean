% ============================================================================
% Chapter 3: The Rewrite System
% ============================================================================
\chapter{The Rewrite System}
\label{ch:rewrite-system}

The fundamental algebraic laws of \cref{ch:basic-constructions}---left and
right identity, associativity, involution, anti-homomorphism---hold as strict
equalities because they reduce to list identities. But the \emph{cancellation}
laws ($p \comp \inv{p} = \refl$) and the $\beta/\eta$-rules for type formers
do not hold strictly: they require non-trivial reorganizations of the step
list. We therefore introduce a \emph{rewrite system} on paths that axiomatizes
these additional identities.

\section{The Single-Step Rewrite Relation}
\label{sec:step-relation}

\begin{definition}[Single-Step Rewrite]\label{def:step-rewrite}
  The relation $\rew$ on $\Path_A(a,b)$ is the smallest relation closed under
  the 75 rules organized into eight groups below. We write
  $\Step(p, q)$ or $p \rew q$ to denote that $p$ rewrites to $q$ in one step.
\end{definition}

\subsection{Group I: Path Algebra (8 rules)}
\label{sec:group-i}

These rules express the groupoid laws that are \emph{not} strict equalities.

\begin{enumerate}[label=\textbf{R\arabic*.}, ref=R\arabic*, leftmargin=3.5em]
  \item\label{rule:sr} \textbf{(sr)}
    $\inv{\refl(a)} \rew \refl(a)$.
  \item\label{rule:ss} \textbf{(ss)}
    $\inv{(\inv{p})} \rew p$.
  \item\label{rule:lrr} \textbf{(lrr)}
    $\refl(a) \comp p \rew p$.
  \item\label{rule:rrr} \textbf{(rrr)}
    $p \comp \refl(b) \rew p$.
  \item\label{rule:tr} \textbf{(tr)}
    $p \comp \inv{p} \rew \refl(a)$.
  \item\label{rule:tsr} \textbf{(tsr)}
    $\inv{p} \comp p \rew \refl(b)$.
  \item\label{rule:stss} \textbf{(stss)}
    $\inv{(p \comp q)} \rew \inv{q} \comp \inv{p}$.
  \item\label{rule:tt} \textbf{(tt)}
    $(p \comp q) \comp r \rew p \comp (q \comp r)$.
\end{enumerate}

\begin{remark}\label{rem:strict-vs-step}
  Rules~\ref{rule:lrr}, \ref{rule:rrr}, \ref{rule:ss}, \ref{rule:stss}, and
  \ref{rule:tt} overlap with the strict equalities of
  \cref{thm:strict-monoid,thm:strict-involution,thm:strict-antihom}. Including
  them in the rewrite system is necessary for two reasons: (i)~the
  \emph{structural closure} rules (\cref{sec:group-viii}) may produce these
  patterns nested inside larger contexts; (ii)~the rewrite system must be
  self-contained for the confluence and termination proofs.
\end{remark}

\subsection{Group II: Type-Former $\beta$/$\eta$-Rules (17 rules)}
\label{sec:group-ii}

These rules govern the interaction of path operations with type constructors.

\paragraph{Product rules.}
\begin{enumerate}[label=\textbf{R\arabic*.}, ref=R\arabic*, leftmargin=3.5em, resume]
  \item\label{rule:prod-fst-beta}
    $\fst(\prodMk(p, q)) \rew p$.
  \item\label{rule:prod-snd-beta}
    $\snd(\prodMk(p, q)) \rew q$.
  \item\label{rule:prod-eta}
    $\prodMk(\fst(r), \snd(r)) \rew r$.
  \item\label{rule:prod-mk-symm}
    $\inv{\prodMk(p, q)} \rew \prodMk(\inv{p}, \inv{q})$.
\end{enumerate}

\paragraph{Sigma rules.}
\begin{enumerate}[label=\textbf{R\arabic*.}, ref=R\arabic*, leftmargin=3.5em, resume]
  \item\label{rule:sigma-fst-beta}
    $\sigmaFst(\sigmaMk(p, q)) \rew \ofEq(\toEq(p))$.
  \item\label{rule:sigma-snd-beta}
    $\sigmaSnd(\sigmaMk(p, q)) \rew \ofEq(\toEq(q))$.
  \item\label{rule:sigma-eta}
    $\sigmaMk(\sigmaFst(r), \sigmaSnd(r)) \rew r$.
\end{enumerate}

\paragraph{Sum rules.}
\begin{enumerate}[label=\textbf{R\arabic*.}, ref=R\arabic*, leftmargin=3.5em, resume]
  \item\label{rule:sum-inl-beta}
    $(\mathrm{rec}(f,g))_*(\inlOp_*(p)) \rew f_*(p)$.
  \item\label{rule:sum-inr-beta}
    $(\mathrm{rec}(f,g))_*(\inrOp_*(q)) \rew g_*(q)$.
\end{enumerate}

\paragraph{Function rules.}
\begin{enumerate}[label=\textbf{R\arabic*.}, ref=R\arabic*, leftmargin=3.5em, resume]
  \item\label{rule:fun-app-beta}
    $\app(\lamCongr(p), a) \rew p(a)$.
  \item\label{rule:fun-eta}
    $\lamCongr(\lambda x.\, \app(q, x)) \rew q$.
  \item\label{rule:lam-symm}
    $\inv{\lamCongr(p)} \rew \lamCongr(\lambda x.\, \inv{p(x)})$.
\end{enumerate}

\paragraph{Map congruence rules.}
\begin{enumerate}[label=\textbf{R\arabic*.}, ref=R\arabic*, leftmargin=3.5em, resume]
  \item\label{rule:prod-map-congr}
    For $f = (g, h) : A \times B \to A' \times B'$,\;
    $f_*(\prodMk(p,q)) \rew \prodMk(g_*(p), h_*(q))$.
\end{enumerate}

Additional rules (22--25) govern the interaction of dependent contexts with
symmetry and the decomposition of dependent application.

\subsection{Group III: Transport Rules (7 rules)}
\label{sec:group-iii}

\begin{enumerate}[label=\textbf{R\arabic*.}, ref=R\arabic*, leftmargin=3.5em, start=26]
  \item\label{rule:transport-refl}
    $\tr_D(\refl(a), x) \rew x$ \quad (identity transport).
  \item\label{rule:transport-trans}
    $\tr_D(p \comp q, x) \rew \tr_D(q, \tr_D(p, x))$ \quad (distributivity).
  \item\label{rule:transport-symm-left}
    $\tr_D(\inv{p}, \tr_D(p, x)) \rew x$.
  \item\label{rule:transport-symm-right}
    $\tr_D(p, \tr_D(\inv{p}, y)) \rew y$.
\end{enumerate}

Rules 30--32 address transport through sigma constructors.

\subsection{Group IV: Context Rules (16 rules)}
\label{sec:context-rules}

These rules govern the interaction of context substitution with path operations.
Let $C : \Context(A, B)$.

\paragraph{Unit rules.}
\begin{enumerate}[label=\textbf{R\arabic*.}, ref=R\arabic*, leftmargin=3.5em, start=33]
  \item\label{rule:context-congr}
    $\Step(p, q) \implies \Step(C[p], C[q])$ \quad (context congruence).
  \item\label{rule:context-symm}
    $\inv{C[p]} \rew C[\inv{p}]$ \quad (symmetry through context).
  \item\label{rule:slr}
    \textbf{(slr)}\; $\substL(C, \refl, p) \rew C[p]$.
  \item\label{rule:srr}
    \textbf{(srr)}\; $\substR(C, p, \refl) \rew C[p]$.
\end{enumerate}

\paragraph{Idempotence and cancellation.}
\begin{enumerate}[label=\textbf{R\arabic*.}, ref=R\arabic*, leftmargin=3.5em, resume]
  \item\label{rule:slss}
    \textbf{(slss)}\; $\substL(C, \substL(C, r, \refl), p) \rew \substL(C, r, p)$.
  \item\label{rule:srsr}
    \textbf{(srsr)}\; $\substR(C, p, \substR(C, \refl, t)) \rew \substR(C, p, t)$.
  \item\label{rule:srrrr}
    \textbf{(srrrr)}\; $\substR(C, \refl, \substR(C, p, t)) \rew \substR(C, p, t)$.
\end{enumerate}

\paragraph{$\beta$-rules (folding into substitution form).}
\begin{enumerate}[label=\textbf{R\arabic*.}, ref=R\arabic*, leftmargin=3.5em, resume]
  \item\label{rule:tsbll}
    \textbf{(tsbll)}\; $r \comp C[p] \rew \substL(C, r, p)$.
  \item\label{rule:tsbrl}
    \textbf{(tsbrl)}\; $C[p] \comp t \rew \substR(C, p, t)$.
\end{enumerate}

\paragraph{Associativity.}
\begin{enumerate}[label=\textbf{R\arabic*.}, ref=R\arabic*, leftmargin=3.5em, resume]
  \item\label{rule:tsblr}
    \textbf{(tsblr)}\; $\substL(C, r, p) \comp t \rew r \comp \substR(C, p, t)$.
  \item\label{rule:tsbrr}
    \textbf{(tsbrr)}\; $\substR(C, p, t) \comp u \rew \substR(C, p, t \comp u)$.
\end{enumerate}

\paragraph{Cancellation.}
\begin{enumerate}[label=\textbf{R\arabic*.}, ref=R\arabic*, leftmargin=3.5em, resume]
  \item\label{rule:ttsv}
    \textbf{(ttsv)}\; $C[p] \comp (C[\inv{p}] \comp v) \rew C[p \comp \inv{p}] \comp v$.
  \item\label{rule:tstu}
    \textbf{(tstu)}\; $(v \comp C[p]) \comp C[\inv{p}] \rew v \comp C[p \comp \inv{p}]$.
\end{enumerate}

\subsection{Groups V--VI: Dependent Context and Bi-Context Rules (20 rules)}
\label{sec:groups-v-vi}

Rules 46--60 are the analogues of Group~IV for dependent contexts
$\DepContext(A, B)$, carrying the additional transport data required by the
dependence of the codomain on the base. Rules 61--68 govern the interaction of
binary contexts ($\BiContext$ and dependent binary contexts) with $\mapLeft$,
$\mapRight$, $\mapTwo$, and their structural closure.

\subsection{Group VII: Map Congruence Rules (4 rules)}
\label{sec:group-vii}

\begin{enumerate}[label=\textbf{R\arabic*.}, ref=R\arabic*, leftmargin=3.5em, start=69]
  \item $\mapLeft(f, -, b)$ preserves $\Step$: $p \rew q$ implies
    $\mapLeft(f, p, b) \rew \mapLeft(f, q, b)$.
  \item $\mapRight(f, a, -)$ preserves $\Step$.
  \item $\mapTwo$ distributes: $\mapTwo(f, p, q) \rew
    \mapRight(f, a_1, q) \comp \mapLeft(f, p, b_2)$
    (alternative factorization).
  \item Interaction of $\ofEq$ with map operations.
\end{enumerate}

\subsection{Group VIII: Structural Closure (4 rules)}
\label{sec:group-viii}

The structural closure rules propagate single-step rewrites through the
path constructors, ensuring that the rewrite relation is compatible with
all operations:

\begin{enumerate}[label=\textbf{R\arabic*.}, ref=R\arabic*, leftmargin=3.5em, start=73]
  \item\label{rule:symm-congr}
    \textbf{(symm\_congr)}\;
    $p \rew q \implies \inv{p} \rew \inv{q}$.
  \item\label{rule:trans-congr-left}
    \textbf{(trans\_congr\_left)}\;
    $p \rew q \implies p \comp r \rew q \comp r$.
  \item\label{rule:trans-congr-right}
    \textbf{(trans\_congr\_right)}\;
    $q \rew r \implies p \comp q \rew p \comp r$.
  \item\label{rule:context-congr-closure}
    \textbf{(context\_congr)}\;
    $p \rew q \implies C[p] \rew C[q]$.
\end{enumerate}

\begin{remark}
  Rules~\ref{rule:symm-congr}--\ref{rule:context-congr-closure} make $\rew$
  a \emph{congruence closure}: any rewrite deep inside a path expression can
  be lifted to the top level.
\end{remark}

\section{Soundness}
\label{sec:soundness}

\begin{theorem}[Soundness of Step]\label{thm:step-sound}
  If $p \rew q$ then $\toEq(p) = \toEq(q)$.
\end{theorem}

\begin{proof}
  By induction on the derivation of $\Step(p, q)$. Each of the 75 rules
  preserves the proof field because all path operations are designed to be
  sound with respect to the underlying propositional equality. The structural
  closure rules follow by the induction hypothesis.
\end{proof}

Soundness guarantees that rewriting never changes the \emph{meaning} of a path;
it only reorganizes the computational trace.

\section{Multi-Step Rewriting}
\label{sec:multi-step}

\begin{definition}[Multi-Step Rewrite]\label{def:rw}
  The relation $\rews$ on $\Path_A(a,b)$ is the reflexive--transitive
  closure of $\rew$. Formally, $\Rw$ is the smallest relation satisfying:
  \begin{enumerate}[label=(\roman*)]
    \item $\Rw.\refl(p) : p \rews p$ for all $p$.
    \item $\Rw.\mathrm{tail}(h, s) : p \rews r$ whenever $h : p \rews q$
      and $s : q \rew r$.
  \end{enumerate}
\end{definition}

\begin{corollary}\label{cor:rw-sound}
  If $p \rews q$ then $\toEq(p) = \toEq(q)$.
\end{corollary}

\section{Rewrite Equality}
\label{sec:rweq}

\begin{definition}[Rewrite Equality]\label{def:rweq}
  The \emph{rewrite equality} $\rweq$ is the equivalence relation generated
  by~$\rew$---equivalently, the symmetric closure of $\rews$. It is the
  smallest relation satisfying:
  \begin{enumerate}[label=(\roman*)]
    \item $\RwEq.\refl(p) : p \rweq p$.
    \item $\RwEq.\mathrm{step}(s) : p \rweq q$ whenever $s : p \rew q$.
    \item $\RwEq.\mathrm{symm}(h) : q \rweq p$ whenever $h : p \rweq q$.
    \item $\RwEq.\mathrm{trans}(h_1, h_2) : p \rweq r$ whenever
      $h_1 : p \rweq q$ and $h_2 : q \rweq r$.
  \end{enumerate}
\end{definition}

\begin{theorem}[Congruence Properties of $\RwEq$]\label{thm:rweq-congruence}
  Rewrite equality is a congruence with respect to all path operations:
  \begin{enumerate}[label=(\roman*)]
    \item $p_1 \rweq p_2$ and $q_1 \rweq q_2$ imply
      $p_1 \comp q_1 \rweq p_2 \comp q_2$.
    \item $p \rweq q$ implies $\inv{p} \rweq \inv{q}$.
    \item $p \rweq q$ implies $f_*(p) \rweq f_*(q)$ for any $f$.
    \item $p \rweq q$ implies $C[p] \rweq C[q]$ for any context $C$.
  \end{enumerate}
  Analogous congruence results hold for $\mapLeft$, $\mapRight$, $\mapTwo$,
  $\BiContext.\mapTwo$, $\DepContext.\mathrm{map}$, $\lamCongr$, $\prodMk$,
  and $\sigmaMk$.
\end{theorem}

\begin{proof}
  Each part follows by induction on the $\RwEq$ derivation, using the
  structural closure rules~\ref{rule:symm-congr}--\ref{rule:context-congr-closure}
  in the base case ($\RwEq.\mathrm{step}$).
\end{proof}

The congruence properties ensure that $\RwEq$ is a well-behaved equivalence
relation that respects the algebraic structure of paths.

\begin{theorem}[Groupoid Laws up to $\RwEq$]\label{thm:rweq-groupoid-laws}
  The following hold:
  \begin{enumerate}[label=(\roman*)]
    \item $\refl(a) \comp p \rweq p$ and $p \comp \refl(b) \rweq p$.
    \item $(p \comp q) \comp r \rweq p \comp (q \comp r)$.
    \item $p \comp \inv{p} \rweq \refl(a)$ and $\inv{p} \comp p \rweq \refl(b)$.
  \end{enumerate}
\end{theorem}

\begin{proof}
  Each is a single application of $\RwEq.\mathrm{step}$ to the
  corresponding rule from Group~I.
\end{proof}

\section{Normalization}
\label{sec:normalization}

\begin{definition}[Normal Form]\label{def:normal-form}
  A path $p : \Path_A(a,b)$ is \emph{normal} if $p = \ofEq(\toEq(p))$. The
  \emph{normalization function} is
  \[
    \normalize(p) \;=\; \ofEq(\toEq(p)) \;:\; \Path_A(a,b).
  \]
\end{definition}

Since $\toEq$ extracts the underlying equality proof and $\ofEq$ wraps it in a
single-step path, normalization discards the trace and replaces it with the
canonical one-step witness.

\begin{theorem}[Properties of Normalization]\label{thm:normalization}
  \begin{enumerate}[label=(\roman*)]
    \item $\normalize(p)$ is always normal.
    \item $p \rweq \normalize(p)$ for every path $p$.
    \item Two paths $p, q : \Path_A(a,b)$ are $\RwEq$-equivalent if and only if
      $\normalize(p) = \normalize(q)$.
  \end{enumerate}
\end{theorem}

\begin{proof}
  Part~(i) is immediate from the definition. Part~(ii): by soundness,
  $\toEq(p) = \toEq(\normalize(p))$, and both $p$ and $\normalize(p)$ can be
  connected via the rewrite rules (the groupoid rules and $\beta/\eta$-rules
  suffice to reduce any path to its normal form). Part~(iii): since
  $\normalize(p) = \ofEq(\toEq(p))$ and $\normalize(q) = \ofEq(\toEq(q))$,
  these are equal iff $\toEq(p) = \toEq(q)$, which holds by proof irrelevance
  of $\Eq$. The ``only if'' direction follows from soundness (\cref{thm:step-sound}).
\end{proof}

\begin{corollary}\label{cor:rweq-decidable}
  Rewrite equality of paths is decidable: $p \rweq q$ iff
  $\normalize(p) = \normalize(q)$, which can be checked by structural
  comparison.
\end{corollary}

\section{Termination}
\label{sec:termination}

\begin{theorem}[Termination]\label{thm:termination}
  The rewrite relation $\rews$ is well-founded: there are no infinite
  reduction sequences.
\end{theorem}

The proof uses a \emph{recursive path ordering} (RPO) adapted to the typed
rewriting setting.

\begin{definition}[Rule Precedence]\label{def:rule-precedence}
  The 76 rewrite rules are assigned a numeric rank
  $\mathrm{rank} : \mathrm{Rule} \to \Nat$, forming a well-founded
  precedence relation. The ranking is chosen so that rules introducing
  simpler path expressions (e.g., $\refl$) have lower rank than rules
  producing compound expressions.
\end{definition}

\begin{definition}[RPO Measure]\label{def:rpo}
  Each path $p$ is assigned a \emph{term} $T(p)$ in the RPO, comprising:
  \begin{itemize}
    \item A \emph{symbol} drawn from $\{\mathrm{nf}\} \cup \mathrm{Rule}
      \cup \{\mathrm{pathLen}(n) : n \in \Nat\}$, where $\mathrm{nf}$
      (normal form) is the least element.
    \item An aggregate weight $\mathrm{pathLenSum}(p) \in \Nat$.
  \end{itemize}
  The ordering $T(p) >_{\mathrm{RPO}} T(q)$ holds when the symbol rank
  of~$p$ strictly exceeds that of~$q$ and the aggregate weight does not
  increase.
\end{definition}

\begin{proposition}\label{prop:rpo-wf}
  The RPO ordering is well-founded.
\end{proposition}

\begin{theorem}\label{thm:rpo-decrease}
  Every application of a rewrite rule strictly decreases the RPO measure:
  if $p \rew q$ via rule $R$, then $T(p) >_{\mathrm{RPO}} T(q)$.
\end{theorem}

\begin{proof}
  By case analysis on the 75 rules. Each rule either reduces the symbol
  rank or maintains the rank while strictly decreasing the aggregate weight.
\end{proof}

\section{Confluence}
\label{sec:confluence}

\begin{definition}[Join]\label{def:join}
  A \emph{join} of $q$ and $r$ (where $p \rews q$ and $p \rews r$ for some
  common source~$p$) is a path $m$ together with witnesses $q \rews m$ and
  $r \rews m$.
\end{definition}

\begin{theorem}[Strip Lemma (Local Confluence)]\label{thm:strip-lemma}
  If $p \rew q$ and $p \rews r$, then $q$ and $r$ have a common reduct:
  there exists $m$ with $q \rews m$ and $r \rews m$.
\end{theorem}

\begin{proof}
  By induction on the derivation of $p \rews r$. The base case
  ($r = p$) is trivial. For the inductive case, suppose $p \rews r'$
  and $r' \rew r$. By the induction hypothesis applied to $p \rew q$
  and $p \rews r'$, we obtain a join of $q$ and $r'$ at some $m'$.
  We then perform a \emph{critical pair analysis}: for each pair of
  overlapping rules that could apply to~$r'$, we exhibit an explicit
  join. The analysis covers all pairs among the 75 rules.
\end{proof}

The critical pair analysis is the most technically demanding part of the
confluence proof. Representative cases include:

\begin{itemize}
  \item \textbf{Product $\fst$ overlap.}\;
    When $\fst(\prodMk(p, q))$ can be rewritten by both the $\beta$-rule
    (\ref{rule:prod-fst-beta}) and a structural closure rule, the two
    reducts join at~$p$.

  \item \textbf{Associativity--unit overlap.}\;
    When $((p \comp q) \comp r)$ where $r = \refl$ can be rewritten by
    either \ref{rule:tt}~(associativity) or \ref{rule:rrr}~(right unit),
    the join is $p \comp q$.

  \item \textbf{Context substitution overlap.}\;
    When $\substL(C, r, p) \comp t$ overlaps with the $\beta$-rule
    (\ref{rule:tsbll}) and the associativity rule (\ref{rule:tsblr}),
    the two reducts join at $r \comp \substR(C, p, t)$.
\end{itemize}

\begin{theorem}[Confluence]\label{thm:confluence}
  The rewrite system is confluent: for any paths $p, q, r$ with
  $p \rews q$ and $p \rews r$, there exists $m$ with $q \rews m$
  and $r \rews m$.
\end{theorem}

\begin{proof}
  By Newman's lemma~\cite{Newman42}: a terminating relation is confluent
  if and only if it is locally confluent. Termination is established in
  \cref{thm:termination}, and local confluence follows from the strip
  lemma (\cref{thm:strip-lemma}).
\end{proof}

\begin{corollary}[Unique Normal Forms]\label{cor:unique-nf}
  Every path has a unique normal form (up to structural equality), and
  two paths are $\RwEq$-equivalent if and only if they reduce to the
  same normal form.
\end{corollary}

\begin{proof}
  Existence of normal forms follows from termination. Uniqueness follows
  from confluence: if $p \rews m_1$ and $p \rews m_2$ with $m_1, m_2$
  normal, then by confluence there exists $m$ with $m_1 \rews m$ and
  $m_2 \rews m$; since $m_1$ and $m_2$ are normal, $m_1 = m = m_2$.
\end{proof}

\section{The Quotient $\PathQuot$}
\label{sec:path-quot}

\begin{definition}[Path Quotient]\label{def:path-quot}
  The \emph{path quotient} is the quotient type
  \[
    \PathQuot_A(a, b) \;=\; \Path_A(a, b) \,/\, {\rweq}.
  \]
  We write $[p]$ for the equivalence class of a path $p$.
\end{definition}

Since $\RwEq$ is a congruence (\cref{thm:rweq-congruence}), all path
operations descend to well-defined operations on the quotient:

\begin{theorem}[Well-Defined Quotient Operations]\label{thm:quot-ops}
  The following operations are well-defined on $\PathQuot$:
  \begin{align*}
    \mathrm{trans} &: \PathQuot_A(a, b) \to \PathQuot_A(b, c) \to \PathQuot_A(a, c), \\
    \mathrm{symm} &: \PathQuot_A(a, b) \to \PathQuot_A(b, a), \\
    f_* &: \PathQuot_A(a, b) \to \PathQuot_B(f(a), f(b)).
  \end{align*}
\end{theorem}

\begin{theorem}[Strict Groupoid Laws on the Quotient]\label{thm:quot-groupoid}
  On $\PathQuot$, all groupoid axioms hold as \textbf{strict equalities}
  (equalities of quotient elements):
  \begin{enumerate}[label=(\roman*)]
    \item $[\refl(a)] \comp [q] = [q]$ \quad and \quad $[p] \comp [\refl(b)] = [p]$.
    \item $([p] \comp [q]) \comp [r] = [p] \comp ([q] \comp [r])$.
    \item $[p] \comp [\inv{p}] = [\refl(a)]$ \quad and \quad
      $[\inv{p}] \comp [p] = [\refl(b)]$.
    \item $[\inv{(\inv{p})}] = [p]$.
  \end{enumerate}
\end{theorem}

\begin{proof}
  Each identity holds because the corresponding rewrite rule from Group~I
  provides a witness of $\RwEq$, which becomes an equality after quotienting.
\end{proof}

\begin{theorem}[Equivalence with the Identity Type]\label{thm:quot-equiv}
  The semantic projection $\toEq$ descends to a bijection
  \[
    \PathQuot_A(a, b) \;\cong\; (a =_A b).
  \]
\end{theorem}

\begin{proof}
  By \cref{thm:normalization}(iii), two paths are $\RwEq$-equivalent
  iff they have the same underlying equality proof (which is unique by
  UIP). Hence each equivalence class corresponds to exactly one element
  of the identity type.
\end{proof}
