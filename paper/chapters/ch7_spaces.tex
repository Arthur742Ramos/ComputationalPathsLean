%% ============================================================================
%% Chapter 7: Spaces and Their Fundamental Groups
%% Part II of "The Algebra of Computational Paths"
%% ============================================================================

\section{Spaces and Their Fundamental Groups}\label{sec:spaces}

We now compute the fundamental groups of standard spaces within the computational-paths framework. The key technical device is the \emph{path expression} construction: rather than postulating higher inductive types (as in HoTT), we model spaces as ordinary types equipped with formal loop generators at the syntactic level, then compute $\pi_1$ via winding-number arguments and free-product word decompositions.

\subsection{The Computational Circle \texorpdfstring{$S^1$}{S¹}}\label{subsec:circle}

\begin{definition}[Circle]\label{def:circle}
The \emph{computational-path circle} $S^1$ is a single-point type
\[
  S^1 = \{ \mathsf{base} \}
\]
equipped with a formal loop generator via the path expression system. Concretely, we define a type $\mathrm{CirclePathExpr}$ of \emph{path expressions} between points of $S^1$, generated by:
\begin{align*}
  \mathsf{loop} &: \mathrm{CirclePathExpr}(\mathsf{base}, \mathsf{base}), \\
  \mathsf{refl}(a) &: \mathrm{CirclePathExpr}(a, a), \\
  \mathsf{symm}(p) &: \mathrm{CirclePathExpr}(b, a) \quad\text{for } p : \mathrm{CirclePathExpr}(a, b), \\
  \mathsf{trans}(p, q) &: \mathrm{CirclePathExpr}(a, c) \quad\text{for } p : \mathrm{CirclePathExpr}(a, b),\ q : \mathrm{CirclePathExpr}(b, c).
\end{align*}
\end{definition}

Since $S^1$ has a single point, every path expression is a loop expression. The essential algebraic invariant is the \emph{winding number}.

\begin{definition}[Winding number]\label{def:winding}
The \emph{winding number} function $w : \mathrm{CirclePathExpr}(\mathsf{base}, \mathsf{base}) \to \mathbb{Z}$ is defined recursively by:
\begin{align*}
  w(\mathsf{loop}) &= 1, \\
  w(\mathsf{refl}(a)) &= 0, \\
  w(\mathsf{symm}(p)) &= -w(p), \\
  w(\mathsf{trans}(p, q)) &= w(p) + w(q).
\end{align*}
\end{definition}

The winding number is a complete invariant for path expressions modulo rewrite equality:

\begin{lemma}\label{lem:winding-invariant}
Let $p, q : \mathrm{CirclePathExpr}(\mathsf{base}, \mathsf{base})$. Then $w(p) = w(q)$ if and only if $p$ and $q$ represent the same element of $\pi_1(S^1, \mathsf{base})$.
\end{lemma}

\begin{proof}
The forward direction: the winding number is invariant under the $\mathrm{LND_{EQ}}$ rewrite rules. Each rule preserves $w$: for instance, $w(\mathsf{trans}(\mathsf{refl}, p)) = 0 + w(p) = w(p)$, confirming Rule~3. Rule~5 gives $w(\mathsf{trans}(p, \mathsf{symm}(p))) = w(p) + (-w(p)) = 0 = w(\mathsf{refl})$. All other rules are verified similarly.

The reverse direction: if $w(p) = w(q)$, then both $p$ and $q$ are rewrite-equivalent to the canonical power expression $\mathsf{loop}^{w(p)}$. This follows because every path expression $p$ rewrites to $\mathsf{loop}^{w(p)}$ via the normalization procedure: use Rule~8 (associativity) to flatten nested $\mathsf{trans}$, Rule~2 to eliminate double $\mathsf{symm}$, and Rules~5--6 to cancel adjacent $\mathsf{loop} \cdot \mathsf{loop}^{-1}$ pairs.
\end{proof}

\begin{definition}[Integer powers]\label{def:loop-powers}
For $n \in \mathbb{Z}$, the \emph{$n$-th power} of the loop generator is:
\[
  \mathsf{loop}^n = \begin{cases}
    \underbrace{\mathsf{trans}(\mathsf{loop}, \ldots, \mathsf{loop})}_{n} & \text{if } n \geq 0, \\
    \mathsf{symm}\bigl(\mathsf{loop}^{|n|}\bigr) & \text{if } n < 0.
  \end{cases}
\]
\end{definition}

\begin{theorem}\label{thm:pi1-circle}
$\pi_1(S^1, \mathsf{base}) \cong \mathbb{Z}$.
\end{theorem}

\begin{proof}
Define mutually inverse maps:
\begin{align*}
  \Phi &: \pi_1(S^1, \mathsf{base}) \to \mathbb{Z}, & \Phi([p]) &= w(p); \\
  \Psi &: \mathbb{Z} \to \pi_1(S^1, \mathsf{base}), & \Psi(n) &= [\mathsf{loop}^n].
\end{align*}
$\Phi$ is well-defined by the invariance of $w$ under rewriting. $\Phi$ is a homomorphism: $\Phi([\mathsf{trans}(p, q)]) = w(p) + w(q) = \Phi([p]) + \Phi([q])$.

The round-trip $\Phi \circ \Psi = \mathrm{id}_{\mathbb{Z}}$: $\Phi(\Psi(n)) = w(\mathsf{loop}^n) = n$, verified by induction on $n$.

The round-trip $\Psi \circ \Phi = \mathrm{id}_{\pi_1}$: for any $[p]$, $\Psi(\Phi([p])) = [\mathsf{loop}^{w(p)}] = [p]$ by Lemma~\ref{lem:winding-invariant}.
\end{proof}

\subsection{The Torus \texorpdfstring{$T^2$}{T²}}\label{subsec:torus}

\begin{definition}[Torus]\label{def:torus}
The \emph{torus} is defined as the product $T^2 = S^1 \times S^1$, with basepoint $(\mathsf{base}, \mathsf{base})$.
\end{definition}

The two generating loops of the torus are:
\begin{align*}
  \alpha &= \mathsf{prodMk}(\mathsf{loop}, \mathsf{refl}(\mathsf{base})) : \Omega(T^2, (\mathsf{base}, \mathsf{base})), \\
  \beta  &= \mathsf{prodMk}(\mathsf{refl}(\mathsf{base}), \mathsf{loop}) : \Omega(T^2, (\mathsf{base}, \mathsf{base})).
\end{align*}

\begin{theorem}\label{thm:pi1-torus}
$\pi_1(T^2, (\mathsf{base}, \mathsf{base})) \cong \mathbb{Z} \times \mathbb{Z}$.
\end{theorem}

\begin{proof}
This is an immediate consequence of the product formula (Theorem~\ref{thm:pi1-product}) and the circle computation (Theorem~\ref{thm:pi1-circle}):
\[
  \pi_1(S^1 \times S^1, (\mathsf{base}, \mathsf{base})) \cong \pi_1(S^1, \mathsf{base}) \times \pi_1(S^1, \mathsf{base}) \cong \mathbb{Z} \times \mathbb{Z}.
\]
The two generators correspond to $[\alpha] \mapsto (1, 0)$ and $[\beta] \mapsto (0, 1)$. The commutativity relation $[\alpha] \cdot [\beta] = [\beta] \cdot [\alpha]$ holds because $\mathbb{Z} \times \mathbb{Z}$ is abelian.
\end{proof}

\subsection{The Figure-Eight and Free Products}\label{subsec:figure-eight}

\begin{definition}[Wedge sum]\label{def:wedge}
For pointed types $(A, a_0)$ and $(B, b_0)$, the \emph{wedge sum} $A \vee B$ is the pushout
\[
  A \xleftarrow{\;f\;} \{\ast\} \xrightarrow{\;g\;} B
\]
where $f(\ast) = a_0$ and $g(\ast) = b_0$, with basepoint $\mathsf{inl}(a_0) = \mathsf{inr}(b_0)$.
\end{definition}

\begin{definition}[Figure-eight]\label{def:figure-eight}
The \emph{figure-eight space} is the wedge sum $S^1 \vee S^1$, with basepoint at the junction where the two circles meet.
\end{definition}

The figure-eight has two fundamental loops:
\begin{itemize}
  \item \emph{Loop $A$}: the loop of the left circle, embedded via $\mathsf{inl}$;
  \item \emph{Loop $B$}: the loop of the right circle, conjugated to the common basepoint:
  \[
    \mathsf{loopB} = \mathsf{glue} \cdot \mathsf{inr}(\mathsf{loop}) \cdot \mathsf{glue}^{-1},
  \]
  where $\mathsf{glue} : \mathrm{Path}(\mathsf{inl}(\mathsf{base}), \mathsf{inr}(\mathsf{base}))$ is the identification path of the wedge.
\end{itemize}

\begin{theorem}\label{thm:pi1-figure-eight}
$\pi_1(S^1 \vee S^1, \mathsf{base}) \cong \mathbb{Z} * \mathbb{Z}$, the free product of $\mathbb{Z}$ with itself.
\end{theorem}

The proof uses a provenance-based Seifert--van Kampen argument: loops in $S^1 \vee S^1$ are encoded as \emph{free product words}---reduced alternating sequences of elements from $\pi_1(S^1)$ (left) and $\pi_1(S^1)$ (right). The encoding and decoding maps establish a bijection between $\pi_1(S^1 \vee S^1)$ and the free product word type.

\begin{remark}\label{rem:nonabelian}
The fundamental group $\mathbb{Z} * \mathbb{Z}$ is non-abelian: the elements $[\mathsf{loopA}] \cdot [\mathsf{loopB}]$ and $[\mathsf{loopB}] \cdot [\mathsf{loopA}]$ are distinct as reduced words. This confirms that $\pi_1$ is not always abelian (cf.\ Remark~\ref{rem:pi1-nonabelian}), in contrast to $\pi_n$ for $n \geq 2$.
\end{remark}

\begin{definition}[Bouquet of circles]\label{def:bouquet}
The \emph{bouquet} $\bigvee_n S^1$ is the wedge sum of $n$ copies of the circle at a common basepoint.
\end{definition}

\begin{corollary}\label{cor:pi1-bouquet}
$\pi_1\bigl(\bigvee_n S^1, \mathsf{base}\bigr) \cong F_n$, the free group on $n$ generators.
\end{corollary}

\begin{proof}
By induction on $n$, using the Seifert--van Kampen theorem and the fact that $\pi_1(S^1) \cong \mathbb{Z}$. The base case $n = 1$ is Theorem~\ref{thm:pi1-circle}. The inductive step uses $\bigvee_{n+1} S^1 \cong \bigl(\bigvee_n S^1\bigr) \vee S^1$ and the free product decomposition.
\end{proof}

\subsection{The Seifert--van Kampen Theorem}\label{subsec:svk}

The computations above are unified by the Seifert--van Kampen theorem, which computes the fundamental group of a pushout in terms of the fundamental groups of its constituents.

\begin{definition}[Pushout]\label{def:pushout}
Given types $A$, $B$, $C$ and maps $f : C \to A$, $g : C \to B$, the \emph{pushout} $\mathrm{Pushout}(A, B, C, f, g)$ is the type generated by:
\begin{itemize}
  \item $\mathsf{inl}(a)$ for $a : A$;
  \item $\mathsf{inr}(b)$ for $b : B$;
  \item $\mathsf{glue}(c) : \mathrm{Path}(\mathsf{inl}(f(c)),\, \mathsf{inr}(g(c)))$ for $c : C$.
\end{itemize}
\end{definition}

\begin{definition}[Amalgamated free product]\label{def:amalgamated}
Given groups $G$, $H$, and $K$ with homomorphisms $\varphi : K \to G$ and $\psi : K \to H$, the \emph{amalgamated free product} $G *_K H$ is the quotient of the free product $G * H$ by the normal closure of $\{\varphi(k) \cdot \psi(k)^{-1} : k \in K\}$.
\end{definition}

\begin{theorem}[Seifert--van Kampen]\label{thm:svk}
Let $f : C \to A$ and $g : C \to B$, and fix a basepoint $c_0 : C$. Then there is an equivalence
\[
  \pi_1\bigl(\mathrm{Pushout}(A, B, C, f, g),\; \mathsf{inl}(f(c_0))\bigr) \;\simeq\; \pi_1(A, f(c_0)) *_{\pi_1(C, c_0)} \pi_1(B, g(c_0)),
\]
where the amalgamation maps are $f_* : \pi_1(C, c_0) \to \pi_1(A, f(c_0))$ and $g_* : \pi_1(C, c_0) \to \pi_1(B, g(c_0))$.
\end{theorem}

\begin{proof}[Proof sketch]
The proof follows the HoTT approach of Favonia and Shulman, adapted to the computational-paths setting. The encoding map sends a loop in the pushout to a word in the amalgamated free product by tracking provenance: each segment of the loop is recorded as belonging to $A$, $B$, or as a glue path. The decoding map builds a pushout loop from a word. The key technical conditions are:
\begin{enumerate}
  \item The glue paths provide the naturality data for the $\pi_1(C)$-action, ensuring that the amalgamation relation holds.
  \item The encode--decode round-trip uses the fact that path expressions in the pushout can be reduced to canonical word form.
  \item The decode--encode round-trip uses the pushout elimination principle.
\end{enumerate}
The formalization packages this via a \emph{provenance encoding} structure that records, for each path segment, whether it originated from the left or right injection.
\end{proof}

The Seifert--van Kampen theorem recovers the previous computations as special cases:

\begin{corollary}\label{cor:svk-wedge}
$\pi_1(A \vee B, \mathsf{base}) \cong \pi_1(A, a_0) * \pi_1(B, b_0)$ for the wedge sum of pointed types.
\end{corollary}

\begin{proof}
The wedge $A \vee B$ is the pushout of $A \xleftarrow{f} \{\ast\} \xrightarrow{g} B$ where $f(\ast) = a_0$, $g(\ast) = b_0$. Since $\pi_1(\{\ast\}, \ast) = \{e\}$ is trivial, the amalgamated free product $\pi_1(A) *_{\{e\}} \pi_1(B)$ is the ordinary free product $\pi_1(A) * \pi_1(B)$.
\end{proof}

\begin{corollary}\label{cor:svk-figure-eight}
$\pi_1(S^1 \vee S^1) \cong \mathbb{Z} * \mathbb{Z}$.
\end{corollary}

\subsection{Suspensions and Spheres}\label{subsec:suspensions}

\begin{definition}[Suspension]\label{def:suspension}
The \emph{suspension} $\Sigma X$ of a type $X$ is defined by:
\begin{itemize}
  \item Two constructors: $\mathsf{north}, \mathsf{south} : \Sigma X$;
  \item A path constructor: $\mathsf{merid}(x) : \mathrm{Path}(\mathsf{north}, \mathsf{south})$ for each $x : X$.
\end{itemize}
\end{definition}

The suspension provides a recursive construction of higher-dimensional spheres:

\begin{proposition}\label{prop:sphere-suspension}
$S^{n+1} \simeq \Sigma S^n$, where $S^0 = \{\mathsf{north}, \mathsf{south}\}$ (two discrete points).
\end{proposition}

The key loop in a suspension is the \emph{base loop} at the north pole:
\[
  \sigma(x) = \mathsf{merid}(x) \cdot \mathsf{merid}(x_0)^{-1} : \Omega(\Sigma X, \mathsf{north}).
\]

\begin{theorem}\label{thm:pi1-suspension}
For any type $X$ with at least two points, $\pi_1(\Sigma X, \mathsf{north})$ is trivial (i.e., every loop is rewrite-equivalent to $\mathsf{refl}$).
\end{theorem}

\begin{proof}[Proof sketch]
Any loop at $\mathsf{north}$ is a concatenation of paths of the form $\mathsf{merid}(x) \cdot \mathsf{merid}(y)^{-1}$. But $\mathsf{merid}(x) \cdot \mathsf{merid}(y)^{-1} \cdot \mathsf{merid}(y) \cdot \mathsf{merid}(x')^{-1}$ collapses by cancellation. By induction on the word length, every loop reduces to $\mathsf{merid}(x) \cdot \mathsf{merid}(x)^{-1} = \mathsf{refl}$.
\end{proof}

\begin{corollary}\label{cor:pi1-higher-spheres}
For $n \geq 2$, $\pi_1(S^n) = 0$.
\end{corollary}

\subsection{Further Examples}\label{subsec:further-spaces}

The computational-paths framework accommodates a range of additional spaces whose fundamental groups are computed via similar techniques.

\paragraph{Klein bottle.}
The Klein bottle $K$ is a one-point type with two formal loop generators $\alpha$ and $\beta$ subject to the relation $\alpha \cdot \beta = \beta \cdot \alpha^{-1}$. This yields
\[
  \pi_1(K) \cong \langle \alpha, \beta \mid \alpha \beta \alpha^{-1} \beta \rangle \cong \mathbb{Z} \rtimes \mathbb{Z}.
\]

\paragraph{Real projective spaces.}
The real projective space $\mathbb{R}P^n$ for $n \geq 2$ has $\pi_1(\mathbb{R}P^n) \cong \mathbb{Z}/2\mathbb{Z}$, detected by the double-cover $S^n \to \mathbb{R}P^n$.

\paragraph{Lens spaces.}
The lens space $L(p, q)$ has $\pi_1(L(p,q)) \cong \mathbb{Z}/p\mathbb{Z}$, computed from the cyclic covering.

\paragraph{Mapping cylinders and mapping cones.}
For a map $f : A \to B$, the \emph{mapping cylinder} $M_f$ deformation-retracts onto $B$, hence $\pi_1(M_f) \cong \pi_1(B)$. The \emph{mapping cone} $C_f$ fits into the Seifert--van Kampen framework as a pushout, and its fundamental group is computed by the amalgamated free product with $\pi_1(A)$ quotiented by $f_*$.
