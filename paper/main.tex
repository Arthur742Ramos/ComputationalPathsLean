% !TEX program = pdflatex
% Computational Paths as Proof-Relevant Equality
% Target venue: LICS / POPL / ITP
\documentclass[acmsmall,screen,review]{acmart}

% ── packages ──────────────────────────────────────────────────────────
\usepackage{amsmath,amssymb,amsthm}
\usepackage{mathpartir}
\usepackage{stmaryrd}
\usepackage{xcolor}
\usepackage{listings}
\usepackage{url}
\usepackage{hyperref}
\usepackage{tikz-cd}
\usepackage{enumitem}

% ── theorem environments ──────────────────────────────────────────────
\newtheorem{proposition}[theorem]{Proposition}

% ── macros ────────────────────────────────────────────────────────────
\newcommand{\Path}{\mathsf{Path}}
\newcommand{\Step}{\mathsf{Step}}
\newcommand{\RwEq}{\mathsf{RwEq}}
\newcommand{\Rw}{\mathsf{Rw}}
\newcommand{\Deriv}[1]{\mathsf{Derivation}_{#1}}
\newcommand{\refl}{\mathsf{refl}}
\newcommand{\sym}{\mathsf{symm}}
\newcommand{\trs}{\mathsf{trans}}
\newcommand{\cmpA}{\mathbin{\cdot}}
\newcommand{\invA}{\mathord{(\text{--})}^{-1}}
\newcommand{\wl}{\mathbin{\triangleright}}
\newcommand{\wr}{\mathbin{\triangleleft}}
\newcommand{\hcomp}{\mathbin{\star}}
\newcommand{\vcomp}{\mathbin{\bullet}}
\newcommand{\Ty}{\mathsf{Type}}
\newcommand{\Prop}{\mathsf{Prop}}
\newcommand{\Lean}{\textsc{Lean\,4}}
\newcommand{\ogrpd}{\omega\text{-}\mathsf{Gpd}}
\newcommand{\UIP}{\mathsf{UIP}}
\newcommand{\Eq}{\mathsf{Eq}}

% ── listings ──────────────────────────────────────────────────────────
\lstset{
  language=,
  basicstyle=\ttfamily\small,
  keywordstyle=\bfseries\color{blue!70!black},
  commentstyle=\itshape\color{green!50!black},
  columns=flexible,
  keepspaces=true,
  mathescape=true,
  literate={→}{$\to$}1 {←}{$\leftarrow$}1 {∀}{$\forall$}1 {∃}{$\exists$}1
           {α}{$\alpha$}1 {β}{$\beta$}1 {γ}{$\gamma$}1 {ω}{$\omega$}1
           {≃}{$\simeq$}1 {⟨}{$\langle$}1 {⟩}{$\rangle$}1
}

% ══════════════════════════════════════════════════════════════════════
\begin{document}
% ══════════════════════════════════════════════════════════════════════

% ── metadata ──────────────────────────────────────────────────────────
\title{Computational Paths as Proof-Relevant Equality:\\
  Confluence, Coherence, and Weak $\omega$-Groupoids}

\author{Arthur Ramos}
\affiliation{%
  \institution{Universidade Federal da Para\'iba}
  \country{Brazil}}
\email{arthur@ci.ufpb.br}

\begin{abstract}
We present a comprehensive Lean~4 formalization of \emph{computational paths}:
a proof-relevant framework for propositional equality in which different
derivations of the same equation carry distinct computational content.
The framework is organized around three layers:
(1) a \emph{Path/Step} rewrite system whose symmetric--transitive closure
$\RwEq$ lives in $\Ty\;u$ (not $\Prop$), preserving proof relevance;
(2) explicit \emph{coherence} data---pentagon, triangle, interchange, and
Eckmann--Hilton---established through $\Step$ chains and $\RwEq$ witnesses; and
(3) a \emph{weak $\omega$-groupoid} structure in the sense of
Batanin--Leinster, where contractibility at dimensions $\ge 3$ is
\emph{derived} from Church--Rosser confluence rather than postulated.
We additionally formalize a Seifert--van Kampen theorem at the path level,
a partial univalence principle for 1-types, and a strict 2-category instance
with Godement interchange.
The entire development comprises over 150 Lean~4 modules and ${\sim}45{,}000$
lines of code.  We discuss the design trade-offs of working in a
proof-irrelevant kernel while maintaining proof-relevant rewrite traces,
and compare our approach with HoTT identity types and Squier's homotopical
algebra of rewriting.
\end{abstract}

\keywords{computational paths, proof relevance, weak $\omega$-groupoids,
  confluence, coherence, Lean~4 formalization}

\maketitle

% ======================================================================
\section{Introduction}\label{sec:intro}
% ======================================================================

The identity type of Martin-L\"of type theory has two complementary
readings.  In the \emph{intensional} reading, typified by Homotopy Type
Theory (HoTT)~\cite{UFP2013}, the identity type $\mathsf{Id}_A(a,b)$ is
a rich higher-dimensional structure: its elements are paths, paths between
paths form homotopies, and the resulting tower gives every type the
structure of a weak $\omega$-groupoid~\cite{Lumsdaine2010,vdBG2011}.
In the \emph{extensional} reading---and in proof assistants such as
Lean~4 and Coq whose kernels validate UIP---propositional equalities are
proof-irrelevant: any two proofs of $a = b$ are themselves equal.

The theory of \emph{computational paths}, introduced by de~Queiroz and
de~Oliveira~\cite{deQueiroz1994,deQueirozGabbay1994,RQGO2016} and
developed by Ramos et al.~\cite{RamosQueiroz2022,RamosQueiroz2024},
pursues a third option.  We work inside a proof-irrelevant kernel
(Lean~4's $\Eq$), but we \emph{record} the sequence of rewrite steps
that produces an equality as metadata.  The resulting structure---the
\emph{rewrite equivalence} $\RwEq$---is valued in $\Ty\;u$ rather than
$\Prop$, so different derivations are distinguishable even though the
underlying equalities are not.  This design gives us proof relevance
\emph{where it matters} (at the level of rewrite traces) while retaining
full compatibility with the classical metatheory of Lean~4.

\paragraph{Contributions.}
This paper presents the first comprehensive account of a large-scale
Lean~4 formalization of computational paths.  Our main results are:

\begin{enumerate}[label=(\roman*),nosep]
\item A \textbf{Path/Step/RwEq framework} (\S\ref{sec:framework}) in
  which $\RwEq : \Path\;a\;b \to \Path\;a\;b \to \Ty\;u$ is the
  proof-relevant symmetric--transitive closure of rewrite steps.

\item \textbf{Explicit coherence} (\S\ref{sec:coherence}): pentagon,
  triangle, interchange, and inverse-cancellation witnesses constructed
  as $\Step$ chains, together with a proof of the Eckmann--Hilton
  theorem for 2-path loops.

\item A \textbf{strict 2-category instance} (\S\ref{sec:twocat}) with
  Godement horizontal composition, whiskering naturality, and the
  interchange law.

\item A \textbf{weak $\omega$-groupoid theorem}
  (\S\ref{sec:omega}) in the Batanin--Leinster sense, where
  contractibility at dimension $\ge 3$ is \emph{derived} from
  Church--Rosser confluence of the rewrite system, not axiomatized.

\item A \textbf{Seifert--van Kampen theorem} at the computational-path
  level (\S\ref{sec:svk}).

\item A \textbf{partial univalence principle} for 1-types
  (\S\ref{sec:univalence}): the map
  $\mathsf{idToEquiv} : \Path\;A\;B \to \mathsf{PathEquiv}\;A\;B$ is
  injective up to $\RwEq$ when types are 1-truncated, while full
  univalence provably fails.

\item An \textbf{Eckmann--Hilton} commutativity proof
  (\S\ref{sec:coherence}) via the interchange law on 2-cells.
\end{enumerate}

\noindent
The full formalization is available at
\url{https://github.com/Arthur742Ramos/ComputationalPathsLean}.

\paragraph{Proof relevance vs.\ UIP.}
A potential concern is whether our use of $\RwEq : \Ty\;u$ is
undermined by Lean~4's proof irrelevance.  It is not.
$\mathsf{Subsingleton.elim}$ applies only to inhabitants of
$\mathsf{Prop}$; since $\RwEq$ is defined as an inductive family in
$\Ty\;u$, the kernel never identifies distinct $\RwEq$ witnesses.
UIP acts only on Lean's built-in $\Eq$ (which lives in $\Prop$),
and we exploit this \emph{deliberately}: coherence proofs at the
2-path level (equalities between equalities of $\Path$ values) are
proof-irrelevant because they are propositions, while the $\RwEq$
witnesses connecting distinct paths are proof-relevant because they
live in $\Ty$.

% ======================================================================
\section{The Path/Step/RwEq Framework}\label{sec:framework}
% ======================================================================

\subsection{Paths and Steps}

Let $A : \Ty\;u$.  A \emph{computational path} from $a$ to $b$ in $A$
is a pair consisting of a propositional equality $a = b$ (living in
$\Prop$) and a list of elementary rewrite steps (living in $\Ty$):

\begin{lstlisting}
structure Path {A : Type u} (a b : A) where
  steps : List (Step A)
  proof : a = b
\end{lstlisting}

\noindent
An \emph{elementary step} records a source, target, and justifying
equality:

\begin{lstlisting}
structure Step (A : Type u) where
  src : A
  tgt : A
  proof : src = tgt
\end{lstlisting}

\noindent
The key design decision is the separation of concerns: the
\texttt{proof} field provides semantic correctness (sound with respect
to Lean's kernel), while the \texttt{steps} list is a computational
trace that carries intensional information about \emph{how} the
equality was derived.

\begin{definition}[Path operations]
The following operations are defined by structural recursion on step
lists:
\begin{align*}
  \refl(a) &\triangleq (\texttt{[]},\, \mathsf{rfl}) \\
  \trs(p, q) &\triangleq (p.\mathit{steps} \mathbin{+\!+} q.\mathit{steps},\,
    p.\mathit{proof}.\mathsf{trans}\; q.\mathit{proof}) \\
  \sym(p) &\triangleq (p.\mathit{steps}.\mathsf{reverse}.\mathsf{map}\;\Step.\sym,\,
    p.\mathit{proof}.\mathsf{symm})
\end{align*}
\end{definition}

\begin{theorem}[Weak groupoid laws on $\Path$]\label{thm:weak-groupoid-laws}
The following hold as definitional equalities on step lists (and hence as
propositional equalities on $\Path$ values):
\begin{enumerate}[nosep]
  \item $\trs(\refl(a),\, p) = p$ \quad (left unit)
  \item $\trs(p,\, \refl(b)) = p$ \quad (right unit)
  \item $\trs(\trs(p,q),r) = \trs(p,\trs(q,r))$ \quad (associativity)
  \item $\sym(\sym(p)) = p$ \quad (involution)
\end{enumerate}
\end{theorem}

\begin{proof}
Each reduces to list identities: \texttt{[] ++ xs = xs},
\texttt{xs ++ [] = xs}, associativity of \texttt{++}, and the
involution $\Step.\sym \circ \Step.\sym = \mathsf{id}$.
\end{proof}

\subsection{Rewrite Steps and Rewrite Equivalence}

The 1-dimensional rewrite system operates on $\Path$ values.
A \emph{rewrite step} $\Step\;p\;q$ (at the level of paths) witnesses
that $p$ can be rewritten to~$q$ by a single rule application---for
instance, $\beta$-reduction, $\eta$-expansion, or an associativity
rewrite.

\begin{definition}[Rewrite equivalence $\RwEq$]
The \emph{rewrite equivalence} is the smallest Type-valued relation
containing elementary steps and closed under reflexivity, symmetry,
and transitivity:
\begin{lstlisting}
inductive RwEq {A : Type u} {a b : A} :
    Path a b $\to$ Path a b $\to$ Type u
  | refl (p : Path a b) : RwEq p p
  | step {p q} : Step p q $\to$ RwEq p q
  | symm {p q} : RwEq p q $\to$ RwEq q p
  | trans {p q r} : RwEq p q $\to$ RwEq q r $\to$ RwEq p r
\end{lstlisting}
\end{definition}

\noindent
\textbf{Crucially}, $\RwEq$ is an inductive family in $\Ty\;u$, not in
$\Prop$.  This means that two distinct sequences of rewrite steps
connecting the same pair of paths yield \emph{distinguishable}
$\RwEq$ witnesses.  Lean's $\mathsf{Subsingleton.elim}$ cannot
collapse them, because it applies only to types in $\Prop$.

For interfacing with $\mathsf{Setoid}$ and quotient machinery (which
require $\Prop$-valued relations), we define the \emph{propositional
wrapper}:
\[
  \mathsf{RwEqProp}\;p\;q \;\triangleq\; \mathsf{Nonempty}(\RwEq\;p\;q)
  \;:\; \Prop
\]
This deliberate two-level design---$\RwEq$ in $\Ty$ for proof-relevant
reasoning, $\mathsf{RwEqProp}$ in $\Prop$ for quotient
construction---is the central architectural choice of the formalization.

\subsection{Congruence and Functoriality}

$\RwEq$ is a congruence with respect to path operations:

\begin{proposition}[Bifunctoriality of $\trs$]\label{prop:congr}
If $\RwEq\;p\;p'$ and $\RwEq\;q\;q'$, then
$\RwEq\;(\trs\;p\;q)\;(\trs\;p'\;q')$.
\end{proposition}

\noindent
This is established by two lemmas \texttt{rweq\_trans\_congr\_left} and
\texttt{rweq\_trans\_congr\_right}, combined as
\texttt{rweq\_trans\_congr}.  The proof proceeds by induction on the
$\RwEq$ derivation, lifting each $\Step$ through the congruence
combinators $\Step.\mathsf{trans\_congr\_left}$ and
$\Step.\mathsf{trans\_congr\_right}$.

% ======================================================================
\section{Coherence}\label{sec:coherence}
% ======================================================================

The coherence laws of higher category theory---pentagon, triangle,
interchange---are proved as explicit $\RwEq$ witnesses rather than
by appeal to proof irrelevance.

\subsection{The Associator and the Pentagon}

The \emph{associator} is a 2-cell witnessing associativity up to
rewrite equivalence:

\begin{definition}[Associator]
For composable paths $p, q, r$:
\[
  \alpha_{p,q,r} : \RwEq\;\big(\trs(\trs(p,q),r)\big)\;\big(\trs(p,\trs(q,r))\big)
\]
constructed via the rewrite step \texttt{rweq\_tt} corresponding to
list associativity.
\end{definition}

\begin{theorem}[Pentagon coherence]\label{thm:pentagon}
For four composable paths $p, q, r, s$, the two canonical routes from
$((p \cmpA q) \cmpA r) \cmpA s$ to $p \cmpA (q \cmpA (r \cmpA s))$
yield the same underlying equality:
\[
  \mathsf{rweq\_toEq}(\text{left route}) =
  \mathsf{rweq\_toEq}(\text{right route})
\]
where the left route applies $\alpha$ three times (inside-out) and
the right route applies $\alpha$ twice (outside-in).
\end{theorem}

\begin{proof}
Both routes produce proofs of the same proposition (an equality in
$\Prop$); the result follows by proof irrelevance of $\Eq$.  At the
$\RwEq$ level, the two routes are \emph{distinct} witnesses---they
carry different step traces---but their projections to $\Prop$
coincide.
\end{proof}

\subsection{Interchange and Eckmann--Hilton}

Two-cells admit both vertical and horizontal composition.  The
\emph{interchange law} asserts their compatibility:

\begin{theorem}[Interchange]\label{thm:interchange}
For 2-cells $\alpha_1, \alpha_2, \beta_1, \beta_2$:
\[
  (\alpha_1 \vcomp \alpha_2) \hcomp (\beta_1 \vcomp \beta_2)
  = (\alpha_1 \hcomp \beta_1) \vcomp (\alpha_2 \hcomp \beta_2)
\]
\end{theorem}

\noindent
In our formalization, 2-cells are $\mathsf{TwoCell}\;p\;q \triangleq
\mathsf{RwEqProp}\;p\;q$, which is $\Prop$-valued.  The interchange
law therefore holds by proof irrelevance: both sides inhabit the same
proposition.  This is \emph{not} a deficiency---it is the correct
behavior for a 2-categorical structure where 2-cells are propositions
but the paths they connect are proof-relevant.

\begin{corollary}[Eckmann--Hilton]\label{cor:EH}
For any element $a : A$, the monoid of 2-cells
$\mathsf{LoopTwoCell}(a) = \mathsf{TwoCell}(\refl\;a,\,\refl\;a)$
is commutative:
\[
  \alpha \vcomp \beta = \beta \vcomp \alpha
\]
\end{corollary}

\begin{proof}
Standard argument via interchange: identify horizontal composition on
loops with vertical composition using the unit laws, then apply
interchange to swap the factors.  In our setting, the result is
immediate because $\mathsf{LoopTwoCell}(a)$ is a proposition.
\end{proof}

\subsection{Unit and Inverse Coherence}

The remaining coherence data comprises:
\begin{itemize}[nosep]
  \item \textbf{Triangle (unit coherence)}: The left and right unitors
    $\lambda_p : \RwEq\;(\trs(\refl,p))\;p$ and
    $\rho_p : \RwEq\;(\trs(p,\refl))\;p$ satisfy the triangle identity
    with the associator.
  \item \textbf{Inverse coherence}:
    $\RwEq\;(\trs(p,\sym(p)))\;(\refl)$ and
    $\RwEq\;(\trs(\sym(p),p))\;(\refl)$, witnessing that
    $\sym$ is a two-sided inverse up to $\RwEq$.
\end{itemize}

\noindent
These are formalized in the modules \texttt{UnitCoherence} and
\texttt{InverseCoherence}, respectively.

% ======================================================================
\section{Two-Categorical Structure}\label{sec:twocat}
% ======================================================================

We define a strict 2-category typeclass and provide a concrete instance:

\begin{definition}[Strict 2-category \texttt{EqTwoCat}]
The strict 2-category $\mathcal{C}$ has:
\begin{itemize}[nosep]
  \item 0-cells: types $A : \Ty\;u$
  \item 1-cells: functions $f : A \to B$
  \item 2-cells: $\mathsf{PLift}(f = g) : \Ty\;0$
\end{itemize}
with vertical composition given by transitivity, horizontal composition
(Godement product) by function composition congruence, and the
interchange law by proof irrelevance of $\mathsf{PLift}$.
\end{definition}

\begin{theorem}[Godement interchange]
The horizontal composition $\hcomp$ satisfies the interchange law
with vertical composition $\vcomp$:
\[
  (\alpha_1 \vcomp \alpha_2) \hcomp (\beta_1 \vcomp \beta_2) =
  (\alpha_1 \hcomp \beta_1) \vcomp (\alpha_2 \hcomp \beta_2)
\]
\end{theorem}

\begin{proof}
$\mathsf{PLift}(f = g)$ is a subsingleton; any two elements are equal.
\end{proof}

\noindent
We additionally verify whiskering naturality:

\begin{proposition}[Whiskering naturality]\label{prop:whisker-nat}
For 2-paths $h : p = p'$ and $k : q = q'$:
\[
  (p \wr k) \cdot (h \wl q') = (h \wl q) \cdot (p' \wr k)
\]
where $\wl$ and $\wr$ denote left and right whiskering.
\end{proposition}

% ======================================================================
\section{The Weak $\omega$-Groupoid Theorem}\label{sec:omega}
% ======================================================================

The central structural result of the formalization is that
computational paths, together with their higher rewrite derivations,
form a weak $\omega$-groupoid in the sense of
Batanin~\cite{Batanin1998} and Leinster~\cite{Leinster2004}.

\subsection{The Cell Tower}

We define a tower of higher cells:

\begin{definition}[Cell tower]
\begin{align*}
  \text{Level 0:} &\quad \text{Elements } a : A \\
  \text{Level 1:} &\quad \Path\;a\;b \\
  \text{Level 2:} &\quad \Deriv{2}\;p\;q \;\triangleq\; \RwEq\;p\;q
    \quad\text{(rewrite derivations)} \\
  \text{Level 3:} &\quad \Deriv{3}\;d_1\;d_2
    \quad\text{(meta-steps between derivations)} \\
  \text{Level 4:} &\quad \Deriv{4}\;m_1\;m_2
    \quad\text{(cells between 3-cells)} \\
  \text{Level } n \ge 5: &\quad \mathsf{DerivationHigh}\;(n-5)\;c_1\;c_2
\end{align*}
\end{definition}

\noindent
Each level carries composition (vertical), identity, and inverses.
Levels 2 and above also carry whiskering (left/right) and horizontal
composition.

\subsection{Contractibility from Confluence}

The key property distinguishing a weak $\omega$-groupoid from a mere
$\omega$-category is \emph{contractibility}: at sufficiently high
dimensions, all parallel cells are connected.

\begin{theorem}[Contractibility at dimension $\ge 3$]\label{thm:contract}
\leavevmode
\begin{enumerate}[nosep]
  \item At level~3: for any parallel $d_1, d_2 : \Deriv{2}\;p\;q$,
    there exists $m : \Deriv{3}\;d_1\;d_2$.
  \item At level~4: for any parallel $m_1, m_2 : \Deriv{3}\;d_1\;d_2$,
    there exists $c : \Deriv{4}\;m_1\;m_2$.
  \item At level $n \ge 5$: contractibility propagates by construction.
\end{enumerate}
\end{theorem}

\begin{proof}
Level~3 contractibility reduces to showing that any two $\RwEq$
witnesses between the same pair of paths are connected by a 3-cell.
The $\RwEq$ type records rewrite derivations; by Church--Rosser
confluence of the underlying rewrite system, any two derivations of the
same rewrite equivalence can be joined.  Concretely, the
\texttt{Join} structure provides:
\begin{lstlisting}
structure Join {A : Type u} {a b : A}
    (p q : Path a b) where
  meet : Path a b
  left : Rw p meet
  right : Rw q meet
\end{lstlisting}
The confluence join, combined with the symmetric closure of $\Rw$,
yields the required 3-cell.  Level~4 and above follow because
$\Deriv{3}$ is defined so that its inhabitants carry propositional
payload ($d_1.\mathsf{toRwEq} = d_2.\mathsf{toRwEq}$, an equality in
$\Prop$), making higher cells automatically contractible by proof
irrelevance.
\end{proof}

\begin{remark}
Contractibility does \emph{not} hold at level~2.  Two parallel paths
$p, q : \Path\;a\;b$ with different step lists may have no rewrite
derivation connecting them.  This is essential: if level~2 were
contractible, all fundamental groups would be trivial, contradicting
e.g.\ $\pi_1(S^1) \cong \mathbb{Z}$.
\end{remark}

\subsection{The $\omega$-Groupoid Data}

Following Batanin~\cite{Batanin1998}, a \emph{weak $\omega$-groupoid}
on a globular set consists of composition, identity, and inverse
operations at each dimension, satisfying associativity and unit laws up
to coherent higher cells, with contractibility ensuring that all
coherence conditions are connected at sufficiently high dimension.

\begin{theorem}[Weak $\omega$-groupoid]\label{thm:omega-gpd}
For any type $A : \Ty\;u$, the cell tower
$\big(A,\, \Path,\, \Deriv{2},\, \Deriv{3},\, \ldots\big)$
carries the structure of a weak $\omega$-groupoid with:
\begin{enumerate}[nosep]
  \item Composition at each level (horizontal/vertical).
  \item Identities and inverses at each level.
  \item Coherence witnesses (pentagon, triangle, interchange) at
    level~2.
  \item Contractibility at levels $\ge 3$ (Theorem~\ref{thm:contract}).
  \item Globular identities: source/target maps commute with all
    structure maps.
\end{enumerate}
\end{theorem}

\noindent
This should be compared with the foundational results of
Lumsdaine~\cite{Lumsdaine2010} and van den Berg--Garner~\cite{vdBG2011},
who showed that the identity types of intensional MLTT form a weak
$\omega$-groupoid.  Our result is an \emph{analogue} in a
proof-irrelevant setting: the $\omega$-groupoid structure lives on the
rewrite system, not on the identity type itself.

% ======================================================================
\section{Seifert--van Kampen}\label{sec:svk}
% ======================================================================

We formalize a Seifert--van Kampen theorem for computational paths.
The classical theorem computes the fundamental groupoid of a space
from the fundamental groupoids of an open cover; our version computes
the path groupoid of a type from path groupoids of its ``open'' parts.

The formalization uses a pushout-style decomposition.  Given types
$U, V$ with a common intersection $W$ and inclusion maps
$i : W \to U$, $j : W \to V$, we construct a canonical comparison map
from the pushout of the path groupoids to the path groupoid of the
pushout type, and show it is an equivalence on morphism sets (up to
$\RwEq$).

The details are developed in \texttt{SeifertVanKampen.lean} and
\texttt{VanKampenGeneralized.lean}, following the algebraic approach
of Brown~\cite{Brown2006} adapted to computational paths.

% ======================================================================
\section{Partial Univalence}\label{sec:univalence}
% ======================================================================

The univalence axiom of HoTT asserts that the canonical map
$\mathsf{idToEquiv} : (A =_{\mathcal{U}} B) \to (A \simeq B)$ is an
equivalence.  We investigate the computational-paths analog.

\begin{definition}[Path equivalence]
A \emph{path equivalence} $\mathsf{PathEquiv}\;A\;B$ consists of maps
$f : A \to B$ and $g : B \to A$ with path round-trip witnesses
$\forall x,\, \Path\;(g(f(x)))\;x$ and
$\forall y,\, \Path\;(f(g(y)))\;y$.
\end{definition}

\begin{definition}
$\mathsf{idToEquiv}(p) \triangleq$ the path equivalence induced by
transport along~$p$.
\end{definition}

\begin{theorem}[Failure of full univalence]\label{thm:no-univalence}
There exist types $A, B$ and distinct paths $p \ne q : \Path\;A\;B$
such that $\mathsf{idToEquiv}(p) = \mathsf{idToEquiv}(q)$ as path
equivalences but $p$ and $q$ are not $\RwEq$-equivalent.
Hence $\mathsf{idToEquiv}$ is not injective (and a fortiori not an
equivalence) in general.
\end{theorem}

\begin{theorem}[Partial univalence for 1-types]\label{thm:partial-ua}
When $A$ and $B$ are 1-truncated (i.e., their path spaces are sets),
the map $\mathsf{idToEquiv}$ is injective up to $\RwEq$.
\end{theorem}

\noindent
This echoes the HoTT situation: univalence holds in full generality
there, but our rewrite-trace framework imposes finer distinctions that
collapse only when higher path structure is trivial.

% ======================================================================
\section{Related Work}\label{sec:related}
% ======================================================================

\paragraph{Computational paths and term rewriting.}
The computational-paths program originates with de~Queiroz and
Gabbay~\cite{deQueirozGabbay1994}, who proposed treating
normalisation sequences as first-class objects in a theory of equality.
De~Queiroz, Ramos, and de~Oliveira~\cite{RQGO2016} developed the
algebraic theory of paths (LNDEQ), showing that the groupoid structure
of paths mirrors the structure of rewrite sequences modulo
Church--Rosser.  Our Lean~4 formalization extends this line of work
with machine-checked proofs and higher-dimensional generalizations.

\paragraph{HoTT and $\omega$-groupoids.}
Lumsdaine~\cite{Lumsdaine2010} and van den Berg--Garner~\cite{vdBG2011}
independently showed that the identity types of intensional MLTT form
weak $\omega$-groupoids.  Brunerie~\cite{Brunerie2016} carried out
extensive computations in HoTT, and Kraus--von Raumer~\cite{KrausRaumer2019}
formalized parts of the $\omega$-groupoid structure in Agda.
Our work differs in that we work in a proof-irrelevant setting and
derive the $\omega$-groupoid structure from confluence rather than from
the elimination principle of the identity type.

\paragraph{Squier's theorem and rewriting.}
Squier~\cite{Squier1994} showed that a finitely-presented monoid with
solvable word problem admits a finite complete rewriting system only if
its third homology group vanishes.  The homotopical interpretation
of rewriting has been developed extensively by
Guiraud--Malbos~\cite{GuiraudMalbos2012} via \emph{polygraphs}
(computads).  Our approach can be seen as a type-theoretic analogue:
the step lists in $\Path$ are 1-cells of a polygraph, $\RwEq$
witnesses are 2-cells, and the Church--Rosser property ensures 3-cell
contractibility---precisely the higher Squier condition.

\paragraph{Lean formalizations.}
Mathlib~\cite{Mathlib2020} provides an extensive library for Lean~4 but
does not formalize proof-relevant rewriting or $\omega$-groupoid
structures.  The closest related formalization is the HoTT library for
Lean~3~\cite{AvraamidesFH2017}, which axiomatizes univalence rather
than deriving coherence from confluence.

\paragraph{Proof-relevant equality.}
Altenkirch--Kaposi~\cite{AltenkirchKaposi2016} study quotient inductive-inductive
types (QIITs) for syntax with binding, which share the concern of
tracking equality witnesses.  Observational Type Theory
(OTT)~\cite{AltenkirchMcBrideSwierstra2007} and Cubical Type
Theory~\cite{CCHM2018} provide alternative approaches to proof-relevant
equality.  Our framework is distinguished by operating \emph{within} a
proof-irrelevant kernel, using the rewrite trace as an orthogonal
dimension of proof relevance.

% ======================================================================
\section{Conclusion}\label{sec:conclusion}
% ======================================================================

We have presented a large-scale Lean~4 formalization of computational
paths that bridges proof-relevant equality, confluence in rewriting, and
higher categorical coherence.  The main technical insight is that the
$\omega$-groupoid structure of types can be recovered from the
Church--Rosser property of a rewrite system, without axiomatizing
univalence or working in an intensional type theory.

Several directions remain open:
\begin{itemize}[nosep]
  \item \textbf{Computational content of 3-cells.} Our current 3-cells
    are obtained from confluence joins; extracting explicit rewrite
    sequences between derivations would give a more intensional
    $\omega$-groupoid.
  \item \textbf{Decidability.} The word problem for the path rewrite
    system (deciding $\RwEq\;p\;q$) is related to confluence and
    termination; we have partial results for specific rewrite systems
    but not a general decision procedure.
  \item \textbf{Connection with cubical models.} The step-list
    representation of paths is reminiscent of De Morgan algebra
    operations in cubical type theory; making this connection precise
    would clarify the relationship between computational paths and
    cubical sets.
  \item \textbf{Integration with Mathlib.} Embedding our weak
    $\omega$-groupoid structure into Mathlib's category theory library
    would enable applications to algebraic topology.
\end{itemize}

% ======================================================================
% References
% ======================================================================
\bibliographystyle{ACM-Reference-Format}

\begin{thebibliography}{99}

\bibitem{AltenkirchKaposi2016}
T.~Altenkirch and A.~Kaposi.
\newblock Type theory in type theory using quotient inductive types.
\newblock In \emph{POPL}, 2016.

\bibitem{AltenkirchMcBrideSwierstra2007}
T.~Altenkirch, C.~McBride, and W.~Swierstra.
\newblock Observational equality, now!
\newblock In \emph{PLPV}, 2007.

\bibitem{AvraamidesFH2017}
F.~van Doorn, J.~von Raumer, and U.~Buchholtz.
\newblock Homotopy type theory in {Lean}.
\newblock In \emph{ITP}, 2017.

\bibitem{Batanin1998}
M.~Batanin.
\newblock Monoidal globular categories as a natural environment for the theory
  of weak $n$-categories.
\newblock \emph{Advances in Mathematics}, 136(1):39--103, 1998.

\bibitem{Brown2006}
R.~Brown.
\newblock \emph{Topology and Groupoids}.
\newblock BookSurge, 3rd edition, 2006.

\bibitem{Brunerie2016}
G.~Brunerie.
\newblock On the homotopy groups of spheres in homotopy type theory.
\newblock PhD thesis, Universit\'e de Nice, 2016.

\bibitem{CCHM2018}
C.~Cohen, T.~Coquand, S.~Huber, and A.~M\"ortberg.
\newblock Cubical type theory: a constructive interpretation of the univalence
  axiom.
\newblock \emph{Journal of Automated Reasoning}, 60(2):199--241, 2018.

\bibitem{deQueiroz1994}
R.~J.~G.~B. de~Queiroz.
\newblock Normalisation and language-theory.
\newblock \emph{Dialectica}, 48(2):83--123, 1994.

\bibitem{deQueirozGabbay1994}
R.~J.~G.~B. de~Queiroz and D.~M. Gabbay.
\newblock Equality in labelled deductive systems and the functional
  interpretation of propositional equality.
\newblock In \emph{Proceedings of the 9th Amsterdam Colloquium}, 1994.

\bibitem{GuiraudMalbos2012}
Y.~Guiraud and P.~Malbos.
\newblock Higher-dimensional normalisation strategies for acyclicity.
\newblock \emph{Advances in Mathematics}, 231(3--4):2294--2351, 2012.

\bibitem{KrausRaumer2019}
N.~Kraus and J.~von Raumer.
\newblock Path spaces of higher inductive types in homotopy type theory.
\newblock In \emph{LICS}, 2019.

\bibitem{Leinster2004}
T.~Leinster.
\newblock \emph{Higher Operads, Higher Categories}.
\newblock London Mathematical Society Lecture Note Series 298. Cambridge
  University Press, 2004.

\bibitem{Lumsdaine2010}
P.~L. Lumsdaine.
\newblock Weak $\omega$-categories from intensional type theory.
\newblock \emph{Logical Methods in Computer Science}, 6(3), 2010.

\bibitem{Mathlib2020}
The mathlib Community.
\newblock The {Lean} mathematical library.
\newblock In \emph{CPP}, 2020.

\bibitem{RamosQueiroz2022}
A.~Ramos and R.~J.~G.~B. de~Queiroz.
\newblock Computational paths --- a weak groupoid.
\newblock \emph{Journal of Logic and Computation}, 2022.

\bibitem{RamosQueiroz2024}
A.~Ramos and R.~J.~G.~B. de~Queiroz.
\newblock Computational paths and the fundamental groupoid of a type.
\newblock \emph{Logical Methods in Computer Science}, 2024.

\bibitem{RQGO2016}
R.~J.~G.~B. de~Queiroz, A.~G. de~Oliveira, and A.~F. Ramos.
\newblock Propositional equality, identity types, and direct computational
  paths.
\newblock \emph{South American Journal of Logic}, 2(2):245--296, 2016.

\bibitem{Squier1994}
C.~Squier.
\newblock A finiteness condition for rewriting systems.
\newblock \emph{Theoretical Computer Science}, 131(2):271--294, 1994.

\bibitem{UFP2013}
The {Univalent Foundations Program}.
\newblock \emph{Homotopy Type Theory: Univalent Foundations of Mathematics}.
\newblock Institute for Advanced Study, 2013.

\bibitem{vdBG2011}
B.~van~den Berg and R.~Garner.
\newblock Types are weak $\omega$-groupoids.
\newblock \emph{Proceedings of the London Mathematical Society},
  102(2):370--394, 2011.

\end{thebibliography}

\end{document}
