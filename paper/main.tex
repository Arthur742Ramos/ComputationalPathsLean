% ============================================================================
% The Calculus of Computational Paths — Complete Paper
% ============================================================================
% Compile with: pdflatex main && bibtex main && pdflatex main && pdflatex main
% ============================================================================
\documentclass[11pt,a4paper]{report}

% --- Encoding & Fonts ---
\usepackage[utf8]{inputenc}
\usepackage[T1]{fontenc}
\usepackage{lmodern}
\usepackage{microtype}

% --- Mathematics ---
\usepackage{amsmath,amssymb,amsthm,mathtools}
\usepackage{stmaryrd}        % \llbracket, \rrbracket
\usepackage{bm}

% --- Tables ---
\usepackage{booktabs}
\usepackage{longtable}

% --- Cross-references & Hyperlinks ---
\usepackage[colorlinks=true,
            linkcolor=blue!60!black,
            citecolor=green!50!black,
            urlcolor=blue!70!black]{hyperref}
\usepackage[capitalize,noabbrev]{cleveref}

% --- Layout ---
\usepackage[margin=1in]{geometry}
\usepackage{enumitem}

% --- Diagrams ---
\usepackage{tikz-cd}

% ============================================================================
% Theorem-like environments
% ============================================================================
\theoremstyle{plain}
\newtheorem{theorem}{Theorem}[chapter]
\newtheorem{proposition}[theorem]{Proposition}
\newtheorem{lemma}[theorem]{Lemma}
\newtheorem{corollary}[theorem]{Corollary}

\theoremstyle{definition}
\newtheorem{definition}[theorem]{Definition}
\newtheorem{example}[theorem]{Example}
\newtheorem{notation_def}[theorem]{Notation}

\theoremstyle{remark}
\newtheorem{remark}[theorem]{Remark}

% ============================================================================
% Custom operators and notation
% ============================================================================
\DeclareMathOperator{\refl}{refl}
\DeclareMathOperator{\symop}{symm}
\DeclareMathOperator{\ofEq}{ofEq}
\DeclareMathOperator{\toEq}{toEq}
\DeclareMathOperator{\tr}{transport}
\DeclareMathOperator{\apd}{apd}
\DeclareMathOperator{\id}{id}

\DeclareMathOperator{\congrArgOp}{congrArg}
\DeclareMathOperator{\mapLeft}{mapLeft}
\DeclareMathOperator{\mapRight}{mapRight}
\DeclareMathOperator{\mapTwo}{map_2}
\DeclareMathOperator{\prodMk}{prodMk}
\DeclareMathOperator{\fst}{fst}
\DeclareMathOperator{\snd}{snd}
\DeclareMathOperator{\inlOp}{inl}
\DeclareMathOperator{\inrOp}{inr}
\DeclareMathOperator{\sigmaMk}{sigmaMk}
\DeclareMathOperator{\sigmaFst}{sigmaFst}
\DeclareMathOperator{\sigmaSnd}{sigmaSnd}
\DeclareMathOperator{\lamCongr}{lamCongr}
\DeclareMathOperator{\app}{app}

\DeclareMathOperator{\substL}{substL}
\DeclareMathOperator{\substR}{substR}
\DeclareMathOperator{\normalize}{normalize}

\DeclareMathOperator{\Fib}{Fib}
\DeclareMathOperator{\cofib}{cofib}
\DeclareMathOperator{\hofiber}{hofiber}
\DeclareMathOperator{\conn}{conn}
\DeclareMathOperator{\trunc}{trunc}
\DeclareMathOperator{\merid}{merid}
\DeclareMathOperator{\wind}{wind}
\DeclareMathOperator{\encode}{encode}
\DeclareMathOperator{\decode}{decode}
\DeclareMathOperator{\glue}{glue}
\DeclareMathOperator{\proj}{proj}
\DeclareMathOperator{\incl}{incl}
\DeclareMathOperator{\rank}{rank}
\DeclareMathOperator{\Hom}{Hom}
\DeclareMathOperator{\Aut}{Aut}

% Aliases used in Part II (homotopy theory chapters)
\newcommand{\congrArg}{\congrArgOp}
\newcommand{\transport}{\tr}
\newcommand{\symm}{\symop}
\newcommand{\inl}{\inlOp}
\newcommand{\inr}{\inrOp}

% Short-hand
\newcommand{\Path}{\operatorname{Path}}
\newcommand{\Step}{\operatorname{Step}}
\newcommand{\Rw}{\operatorname{Rw}}
\newcommand{\RwEq}{\operatorname{RwEq}}
\newcommand{\PathQuot}{\operatorname{PathQuot}}
\newcommand{\LoopQuot}{\operatorname{LoopQuot}}
\newcommand{\StrictGroupoid}{\operatorname{StrictGroupoid}}
\newcommand{\Context}{\operatorname{Context}}
\newcommand{\BiContext}{\operatorname{BiContext}}
\newcommand{\DepContext}{\operatorname{DepContext}}
\newcommand{\Derivation}{\operatorname{D}}
\newcommand{\Deriv}{\operatorname{D}}
\newcommand{\MetaStep}{\operatorname{MetaStep}}
\newcommand{\PathRwQuot}{\operatorname{PathRwQuot}}

% Loop/homotopy
\newcommand{\lp}{\ell}

% Rewrite arrows
\newcommand{\rew}{\mathbin{\triangleright}}         % single step
\newcommand{\rews}{\mathbin{\triangleright^{*}}}    % multi-step
\newcommand{\rweq}{\mathbin{\approx}}               % rewrite equality

% Misc
\newcommand{\comp}{\mathbin{\cdot}}
\newcommand{\inv}[1]{{#1}^{-1}}
\newcommand{\ofEqfn}[1]{\ofEq(#1)}
\newcommand{\Nat}{\mathbb{N}}
\newcommand{\ZZ}{\mathbb{Z}}
\newcommand{\Sort}{\mathsf{Sort}}
\newcommand{\Type}{\mathsf{Type}}
\newcommand{\Prop}{\mathsf{Prop}}
\newcommand{\List}{\operatorname{List}}
\newcommand{\Eq}{\operatorname{Eq}}
\newcommand{\IdA}{\operatorname{Id}}
\newcommand{\ab}{\mathrm{ab}}
\newcommand{\trans}{\operatorname{trans}}

% ============================================================================
\title{\textbf{The Calculus of Computational Paths}\\[6pt]
  \large A Lean~4 Formalization of Rewrite-Trace Equality,\\
  Weak $\omega$-Groupoids, and Homotopy Theory}

\author{
  Arthur Freitas Ramos\thanks{Centro de Inform\'atica, Universidade Federal
    de Pernambuco, Recife, Brazil. \texttt{arfreita@microsoft.com}}
  \and
  Ruy J.\,G.\,B.\ de Queiroz\footnotemark[1]
  \and
  Anjolina G.\ de Oliveira\footnotemark[1]
}
\date{\today}

% ============================================================================
\begin{document}
\maketitle

% --- Abstract ---
\begin{abstract}
We present a comprehensive mathematical theory of \emph{computational
paths}---a formalism in which propositional equalities between elements of a
type carry explicit rewrite traces recording the sequence of elementary steps
by which they were derived. A computational path from $a$ to~$b$ in a type~$A$
is a pair $(s, \pi)$ where $\pi : a =_A b$ is a propositional equality and
$s$ is a finite list of elementary rewrite steps. The space of computational
paths is equipped with a confluent, terminating rewrite system whose 75~rules
are organized into eight groups covering path algebra, type-former
$\beta/\eta$-rules, transport, contexts, dependent contexts, bi-contexts, map
congruence, and structural closure. We prove that this structure gives rise to
weak $\omega$-groupoids in the sense of Batanin--Leinster, with contractibility
beginning at dimension~3---the critical threshold that preserves non-trivial
fundamental groups. Building on these foundations, we develop the associated
homotopy theory: fundamental groups and their functoriality, the Eckmann--Hilton
argument, the Seifert--van Kampen theorem, fibrations, covering spaces, the long
exact sequence of homotopy groups, the Hurewicz theorem, spectral sequences, and
computations of fundamental groups for standard spaces including the circle
($\pi_1(S^1) \cong \mathbb{Z}$), torus ($\mathbb{Z} \times \mathbb{Z}$),
figure-eight ($\mathbb{Z} * \mathbb{Z}$), Klein bottle, projective spaces, and
lens spaces. The entire development is formalized in Lean~4 (516~files,
113,024~lines, 3,258~definitions, 1,466~theorems, 2,711~structures).
\end{abstract}

% --- Table of Contents ---
\tableofcontents

% ============================================================================
%  PART I: FOUNDATIONS (Chapters 1--5)
% ============================================================================
\part{Foundations}
\label{part:foundations}

% ============================================================================
% Chapter 1: Introduction and Motivation
% ============================================================================
\chapter{Introduction and Motivation}
\label{ch:introduction}

\section{The Curry--Howard--de~Bruijn Correspondence and Propositional Equality}
\label{sec:curry-howard}

In Martin-L\"of type theory, the \emph{identity type} $\IdA_A(a,b)$ captures
propositional equality between elements $a,b : A$. Its sole introduction rule
is reflexivity: the term $\refl(a) : \IdA_A(a,a)$ witnesses that every element
is equal to itself. The elimination rule---\emph{path induction}, also known as
the~$J$-rule---states that to prove a property of an arbitrary inhabitant of
$\IdA_A(a,b)$, it suffices to verify the property for $\refl(a)$.

In proof assistants based on the Calculus of Inductive Constructions, such as
Lean~4 and Coq, the identity type \texttt{Eq} lives in $\Prop$, a universe
governed by \emph{proof irrelevance}: all inhabitants of a proposition are
definitionally identified. As a consequence, the \emph{Uniqueness of Identity
Proofs} (UIP) principle holds:

\begin{equation}\label{eq:uip}
  \forall\, p, q : \IdA_A(a,b),\quad p =_{\IdA_A(a,b)} q.
\end{equation}

This axiom collapses the entire space of identity proofs to at most one element
per pair of endpoints. From the perspective of homotopy type theory~\cite{HoTTBook},
UIP asserts that every type is a \emph{set} (a $0$-truncated type), precluding
the rich higher-dimensional structure that identity types can carry in
intensional type theory.

\section{The Computational Paths Program}
\label{sec:comp-paths-program}

Following de~Queiroz, de~Oliveira, and Ramos~\cite{DQOR16, RDQO18}, we propose
that equality proofs carry \emph{computational content}: the sequence of
rewriting steps that produced them. Even when the underlying logic satisfies
UIP---so that the propositional equality $a =_A b$ is proof-irrelevant---the
rewrite \emph{traces} are distinct combinatorial objects that can be compared,
composed, and quotiented.

The key insight is a separation of concerns:
\begin{itemize}[leftmargin=2em]
  \item The \textbf{semantic content} of an equality proof is the proposition
    $a =_A b$, which by UIP carries no information beyond its truth value.
  \item The \textbf{computational trace} is a finite sequence of elementary
    rewrite steps recording \emph{how} the equality was derived---which
    congruence rules, symmetries, transitivities, and $\beta/\eta$-reductions
    were applied, and in what order.
\end{itemize}

This separation creates a rich algebraic structure \emph{atop} the
proof-irrelevant equality, without modifying the underlying type theory.

\section{Design Principle: Path = Proof + Trace}
\label{sec:design-principle}

We now state the central definitions that constitute the formal framework.

\begin{definition}[Elementary Rewrite Step]\label{def:step}
  An \emph{elementary rewrite step} in a type~$A$ is a triple
  \[
    s = (\mathrm{src}, \mathrm{tgt}, \pi) \quad\text{where}\quad
    \mathrm{src}, \mathrm{tgt} : A \quad\text{and}\quad
    \pi : \mathrm{src} =_A \mathrm{tgt}.
  \]
  We write $\Step(A)$ for the type of all elementary rewrite steps in~$A$.
\end{definition}

Each step records a single atomic equation between two elements together with
its justification. Steps can be inverted (swapping source and target) and
mapped through functions.

\begin{definition}[Computational Path]\label{def:path}
  A \emph{computational path} from $a$ to $b$ in a type~$A$ is a pair
  \[
    p = (s, \pi) \quad\text{where}\quad
    s : \List(\Step(A)) \quad\text{and}\quad
    \pi : a =_A b.
  \]
  We write $\Path_A(a,b)$ for the type of computational paths from~$a$ to~$b$.
\end{definition}

The list $s$ is the \emph{computational trace}---it records which elementary
steps were applied. The proof $\pi$ is the \emph{semantic witness}---it
certifies that the composite rewrite is valid. Two paths with the same
endpoints but different traces are \emph{distinct} as elements of $\Path_A(a,b)$,
even though their proof fields are identified by UIP.

\begin{definition}[Semantic Projection]\label{def:toEq}
  The \emph{semantic projection} $\toEq : \Path_A(a,b) \to (a =_A b)$ extracts
  the underlying propositional equality from a computational path, discarding
  the trace:
  \[
    \toEq(s, \pi) \;=\; \pi.
  \]
\end{definition}

\begin{definition}[Canonical Witness]\label{def:ofEq}
  For any propositional equality $\pi : a =_A b$, the \emph{canonical witness}
  is the single-step path
  \[
    \ofEq(\pi) \;=\; \bigl([\langle a, b, \pi\rangle],\; \pi\bigr) \;:\; \Path_A(a,b).
  \]
\end{definition}

The function $\ofEq$ embeds the standard identity type into the computational
path space. Its image consists precisely of the one-step paths.

\section{Non-UIP for Computational Paths}
\label{sec:non-uip}

The raison d'\^etre of the computational paths framework is that it recovers
higher-dimensional structure even in a proof-irrelevant setting:

\begin{theorem}[Non-UIP for Paths]\label{thm:non-uip}
  For any nonempty type~$A$, the space of computational paths does not satisfy
  the Uniqueness of Identity Proofs:
  \[
    \neg\,\bigl(\forall\, a, b : A,\;\forall\, p, q : \Path_A(a,b),\;
    p = q\bigr).
  \]
\end{theorem}

\begin{proof}
  Let $a : A$ be any element. Consider two paths from $a$ to itself:
  \begin{align*}
    p &\;=\; \refl(a) \;=\; ([\,],\; \refl) \;:\; \Path_A(a,a), \\
    q &\;=\; \ofEq(\refl) \;=\; \bigl([\langle a, a, \refl\rangle],\;
    \refl\bigr) \;:\; \Path_A(a,a).
  \end{align*}
  These have the same proof field ($\refl$) but differ in their step lists:
  $p.\mathrm{steps} = [\,]$ while $q.\mathrm{steps} = [\langle a, a,
  \refl\rangle]$. Since the step lists are structurally distinct,
  $p \neq q$ as elements of $\Path_A(a,a)$.
\end{proof}

\begin{remark}\label{rem:key-distinction}
  \Cref{thm:non-uip} is the foundational result that enables the entire
  subsequent development. It shows that even in a type theory where all
  identity proofs are identified (UIP holds for $\Eq$), the space
  $\Path_A(a,b)$ supports a non-trivial ``higher'' structure. The extra
  information resides in the trace, not in the equality proof.
\end{remark}

\section{Related Work}
\label{sec:related-work}

Our work connects to several strands of research in type theory and
higher-dimensional algebra.

\paragraph{Homotopy Type Theory.}
The Univalent Foundations program~\cite{HoTTBook} interprets types as spaces,
identity types as path spaces, and higher identity types as higher homotopy
groups. In HoTT, the identity type \emph{itself} carries higher structure, and
UIP is rejected. Our approach is complementary: we work \emph{within} a
UIP-satisfying type theory and build the higher structure externally via
rewrite traces.

\paragraph{Types as weak $\omega$-groupoids.}
Lumsdaine~\cite{Lumsdaine10} and van~den~Berg--Garner~\cite{vdBG11}
independently showed that the tower of iterated identity types in intensional
type theory carries the structure of a weak $\omega$-groupoid. Our
\cref{thm:omega-groupoid} (\cref{ch:higher-dimensional}) establishes an
analogous result for computational paths, with the crucial difference that
contractibility begins at dimension~3 rather than dimension~1.

\paragraph{Cubical Type Theory.}
Bezem, Coquand, and Huber~\cite{BeCH14} introduced cubical sets as a
constructive model of HoTT. Computational paths offer an alternative
computational semantics: where cubical paths are built from interval
variables, computational paths are built from explicit rewrite steps.

\paragraph{Higher-Dimensional Rewriting.}
The rewriting-theoretic perspective on higher algebra, developed by Burroni,
M\'etayer, Lafont, and others, treats rewrite rules as generators of higher
cells. Our 75-rule rewrite system on paths (\cref{ch:rewrite-system}) can be
seen as an instance of this paradigm, with the strip lemma and confluence
results providing the coherence data.

\section{Outline of the Paper}
\label{sec:outline}

This paper (Part~I) develops the foundations of the theory of computational
paths in five chapters.

\begin{description}[leftmargin=2em,style=nextline]
  \item[\Cref{ch:basic-constructions}: Basic Constructions.]
    We introduce the fundamental operations on paths---reflexivity, symmetry,
    transitivity, congruence---and establish their strict algebraic laws. We
    develop the path algebra for products, sums, dependent pairs, and function
    types, and define transport and dependent application. We introduce unary,
    binary, and dependent contexts.

  \item[\Cref{ch:rewrite-system}: The Rewrite System.]
    We define the single-step rewrite relation $\Step$ with its 75 rules in
    eight groups, its reflexive--transitive closure $\Rw$, and the rewrite
    equality $\RwEq$. We prove soundness, develop normalization, establish
    termination via a recursive path ordering, prove local confluence
    (the strip lemma) and global confluence via a groupoid-algebraic
    argument, and construct the quotient $\PathQuot$.

  \item[\Cref{ch:groupoid}: The Groupoid of Computational Paths.]
    We show that every type carries a canonical weak groupoid structure
    under computational paths, that the quotient $\PathQuot$ is a strict
    groupoid, and that rewrite lifts provide functorial transport of the
    rewrite structure.

  \item[\Cref{ch:higher-dimensional}: Higher-Dimensional Structure.]
    We define two-cells (rewrite equalities between paths) and establish
    the weak bicategory structure. We construct the globular tower, define
    derivation cells at each dimension, prove contractibility at
    dimension~$\geq 3$, and establish the main structure theorem: computational
    paths form a weak $\omega$-groupoid.
\end{description}

% ============================================================================
% Chapter 2: Computational Paths — Basic Constructions
% ============================================================================
\chapter{Computational Paths: Basic Constructions}
\label{ch:basic-constructions}

Throughout this chapter, $A$, $B$, $C$ denote types in a fixed universe,
and $a, b, c, d$ denote elements of~$A$ unless otherwise stated.

\section{Fundamental Operations}
\label{sec:fundamental-ops}

We equip the space $\Path_A(a,b)$ with three primitive operations.

\begin{definition}[Reflexivity]\label{def:refl}
  The \emph{reflexive path} at $a : A$ is
  \[
    \refl(a) \;=\; ([\,],\; \refl_a) \;:\; \Path_A(a,a),
  \]
  where the step list is empty and the proof field is the reflexivity
  of~$\Eq$.
\end{definition}

\begin{definition}[Symmetry]\label{def:symm}
  For $p = (s, \pi) : \Path_A(a,b)$, the \emph{symmetric path} is
  \[
    \inv{p} \;=\; \symop(p) \;=\; \bigl(\mathrm{reverse}(\mathrm{map}(\mathrm{symm}_{\Step}, s)),\;
    \pi^{-1}\bigr) \;:\; \Path_A(b,a),
  \]
  where $\mathrm{symm}_{\Step}$ inverts each elementary step and
  $\mathrm{reverse}$ reverses the list.
\end{definition}

\begin{definition}[Transitivity / Composition]\label{def:trans}
  For $p = (s_1, \pi_1) : \Path_A(a,b)$ and $q = (s_2, \pi_2) : \Path_A(b,c)$,
  the \emph{composite path} is
  \[
    p \comp q \;=\; \mathrm{trans}(p, q) \;=\;
    (s_1 \mathbin{+\!\!+} s_2,\; \pi_1 \cdot \pi_2) \;:\; \Path_A(a,c),
  \]
  where $\mathbin{+\!\!+}$ denotes list concatenation and $\cdot$ is
  transitivity of~$\Eq$.
\end{definition}

\begin{remark}
  We use $p \comp q$ and $\mathrm{trans}(p, q)$ interchangeably,
  adopting the diagrammatic order: $p$ is followed by~$q$.
\end{remark}

\section{Strict Algebraic Laws}
\label{sec:strict-laws}

A remarkable feature of computational paths is that many algebraic identities
hold as \emph{strict} equalities---i.e., as definitional equalities of the
$\Path$ record structure---not merely up to rewriting. This is a consequence of
the list-based representation: the laws reduce to standard identities on list
concatenation and reversal.

\begin{theorem}[Strict Monoid Laws]\label{thm:strict-monoid}
  For all $p : \Path_A(a,b)$, $q : \Path_A(b,c)$, $r : \Path_A(c,d)$:
  \begin{enumerate}[label=(\roman*)]
    \item \textbf{Left identity.}\; $\refl(a) \comp p = p$.
    \item \textbf{Right identity.}\; $p \comp \refl(b) = p$.
    \item \textbf{Associativity.}\; $(p \comp q) \comp r = p \comp (q \comp r)$.
  \end{enumerate}
  All three equalities hold as structural equalities of $\Path$ records
  (not merely up to rewriting).
\end{theorem}

\begin{proof}
  \begin{enumerate}[label=(\roman*)]
    \item By definition, $\refl(a) \comp p = ([\,] \mathbin{+\!\!+} s,\;
      \refl \cdot \pi) = (s, \pi) = p$, since prepending the empty list
      is the identity on lists, and $\refl \cdot \pi = \pi$.
    \item Similarly, $p \comp \refl(b) = (s \mathbin{+\!\!+} [\,],\;
      \pi \cdot \refl) = (s, \pi) = p$.
    \item Follows from $\mathrm{List.append\_assoc}$: $(s_1 \mathbin{+\!\!+} s_2)
      \mathbin{+\!\!+} s_3 = s_1 \mathbin{+\!\!+} (s_2 \mathbin{+\!\!+} s_3)$
      and associativity of $\Eq.\mathrm{trans}$. \qedhere
  \end{enumerate}
\end{proof}

\begin{theorem}[Strict Involution]\label{thm:strict-involution}
  For all $p : \Path_A(a,b)$:
  \[
    \inv{(\inv{p})} = p.
  \]
\end{theorem}

\begin{proof}
  We compute $\inv{(\inv{p})}$: reversing the reversed list recovers the
  original list, and applying $\mathrm{symm}_{\Step}$ twice to each step
  is the identity (since swapping source and target twice returns to the
  original step). On the proof field, $(\pi^{-1})^{-1} = \pi$.
\end{proof}

\begin{theorem}[Strict Anti-Homomorphism]\label{thm:strict-antihom}
  For all $p : \Path_A(a,b)$ and $q : \Path_A(b,c)$:
  \[
    \inv{(p \comp q)} \;=\; \inv{q} \comp \inv{p}.
  \]
\end{theorem}

\begin{proof}
  By the list identity $\mathrm{reverse}(s_1 \mathbin{+\!\!+} s_2) =
  \mathrm{reverse}(s_2) \mathbin{+\!\!+} \mathrm{reverse}(s_1)$ and
  the fact that mapping commutes with reversal and append.
\end{proof}

\begin{remark}[Cancellation is not strict]\label{rem:cancel-not-strict}
  The cancellation law $p \comp \inv{p} = \refl(a)$ does \emph{not} hold
  strictly: the left-hand side has step list $s \mathbin{+\!\!+}
  \mathrm{reverse}(\mathrm{map}(\mathrm{symm}_{\Step}, s))$, which is
  non-empty whenever $s \neq [\,]$, while $\refl(a)$ has an empty step
  list. Cancellation holds only up to the rewrite relation~$\Step$
  (\cref{ch:rewrite-system}).
\end{remark}

\section{Congruence (Functorial Action)}
\label{sec:congruence}

\begin{definition}[Unary Congruence]\label{def:congrArg}
  For $f : A \to B$ and $p = (s, \pi) : \Path_A(a,b)$, the \emph{congruence}
  (or \emph{functorial action}) of~$f$ on~$p$ is
  \[
    f_*(p) \;=\; \congrArgOp(f, p) \;=\;
    \bigl(\mathrm{map}(\mathrm{map}_f, s),\; \congrArgOp(f, \pi)\bigr)
    \;:\; \Path_B(f(a), f(b)),
  \]
  where $\mathrm{map}_f$ sends a step $\langle x, y, h \rangle$ to $\langle
  f(x), f(y), \congrArgOp(f, h) \rangle$.
\end{definition}

The functorial action satisfies strict algebraic laws:

\begin{theorem}[Functoriality of Congruence]\label{thm:congrArg-functor}
  For $f : A \to B$, $g : B \to C$, and paths $p : \Path_A(a,b)$,
  $q : \Path_A(b,c)$:
  \begin{enumerate}[label=(\roman*)]
    \item \textbf{Composition.}\; $f_*(p \comp q) = f_*(p) \comp f_*(q)$.
    \item \textbf{Symmetry.}\; $f_*(\inv{p}) = \inv{f_*(p)}$.
    \item \textbf{Identity.}\; $\id_*(p) = p$ \;(where $\id = \lambda x.\, x$).
    \item \textbf{Composition of functions.}\;
      $(g \circ f)_*(p) = g_*(f_*(p))$.
  \end{enumerate}
  All equalities are strict.
\end{theorem}

\begin{proof}
  Each part reduces to a standard identity on list operations:
  \begin{enumerate}[label=(\roman*)]
    \item $\mathrm{map}(F, s_1 \mathbin{+\!\!+} s_2) = \mathrm{map}(F, s_1)
      \mathbin{+\!\!+} \mathrm{map}(F, s_2)$.
    \item $\mathrm{map}(F, \mathrm{reverse}(\mathrm{map}(G, s))) =
      \mathrm{reverse}(\mathrm{map}(F \circ G, s))$, combined with the fact
      that $\mathrm{map}_f \circ \mathrm{symm}_{\Step} = \mathrm{symm}_{\Step}
      \circ \mathrm{map}_f$.
    \item $\mathrm{map}_{\id} = \id$ on steps.
    \item $\mathrm{map}_{g \circ f} = \mathrm{map}_g \circ \mathrm{map}_f$
      on steps. \qedhere
  \end{enumerate}
\end{proof}

\begin{corollary}\label{cor:path-functor}
  The assignment $A \mapsto \Path_A$ and $f \mapsto f_*$ is a functor from
  the category of types and functions to the category of types and
  path-preserving maps (strictly preserving composition and identities).
\end{corollary}

\section{Binary Congruence}
\label{sec:binary-congruence}

For binary functions, congruence decomposes into left and right components.

\begin{definition}[Left, Right, and Binary Maps]\label{def:binary-maps}
  Let $f : A \to B \to C$.
  \begin{enumerate}[label=(\roman*)]
    \item \textbf{Left map.}\; $\mapLeft(f, p, b) = (\lambda x.\, f(x,b))_*(p)
      : \Path_C(f(a_1, b), f(a_2, b))$ \\ for $p : \Path_A(a_1, a_2)$
      and $b : B$.
    \item \textbf{Right map.}\; $\mapRight(f, a, q) = f(a)_*(q)
      : \Path_C(f(a, b_1), f(a, b_2))$ \\ for $a : A$ and $q : \Path_B(b_1, b_2)$.
    \item \textbf{Binary map.}\; $\mapTwo(f, p, q) = \mapLeft(f, p, b_1)
      \comp \mapRight(f, a_2, q)$ \\ for $p : \Path_A(a_1, a_2)$ and
      $q : \Path_B(b_1, b_2)$.
  \end{enumerate}
\end{definition}

The binary map first varies the left argument (holding the right argument
at its \emph{source} $b_1$), then varies the right argument (holding the left
argument at its \emph{target} $a_2$). This is the canonical choice that makes
the binary map compose correctly with projections.

The binary map satisfies its own functoriality laws, inherited from those of
the unary congruence. In particular, $\mapTwo(f, -, -)$ preserves composition
in each variable separately:
\[
  \mapTwo(f, p_1 \comp p_2, q_1 \comp q_2) =
  \mapLeft(f, p_1, b_1) \comp \mapLeft(f, p_2, b_1) \comp
  \mapRight(f, a_3, q_1) \comp \mapRight(f, a_3, q_2).
\]
Symmetry of $\mapTwo$ reverses the order of the components:
\[
  \inv{\mapTwo(f, p, q)} = \mapRight(f, a_2, \inv{q}) \comp
  \mapLeft(f, \inv{p}, b_1).
\]

\section{Product Paths}
\label{sec:product-paths}

\begin{definition}[Product Path Operations]\label{def:prod-paths}
  For $p : \Path_A(a_1, a_2)$ and $q : \Path_B(b_1, b_2)$:
  \begin{enumerate}[label=(\roman*)]
    \item $\prodMk(p, q) = \mapTwo(\mathrm{Prod.mk}, p, q)
      : \Path_{A \times B}((a_1, b_1), (a_2, b_2))$.
    \item $\fst(r) = (\mathrm{Prod.fst})_*(r)$ for
      $r : \Path_{A \times B}((a_1, b_1), (a_2, b_2))$.
    \item $\snd(r) = (\mathrm{Prod.snd})_*(r)$ for
      $r : \Path_{A \times B}((a_1, b_1), (a_2, b_2))$.
  \end{enumerate}
\end{definition}

\begin{theorem}[Product $\beta$/$\eta$-Rules]\label{thm:prod-beta-eta}
  The product path operations satisfy:
  \begin{enumerate}[label=(\roman*)]
    \item \textbf{$\beta$-rules.}\;
      $\fst(\prodMk(p, q)) = p$ \;(strictly, by functoriality);\;
      $\snd(\prodMk(p, q))$ reduces to~$q$ via a single rewrite step.
    \item \textbf{$\eta$-rule.}\; $\prodMk(\fst(r), \snd(r)) \rew r$
      \;(as a rewrite step, see \cref{ch:rewrite-system}).
  \end{enumerate}
\end{theorem}

The $\eta$-rule is not a strict equality because $\prodMk(\fst(r), \snd(r))$
applies the binary map construction, producing a different step list than the
original path~$r$.

\section{Sigma Paths}
\label{sec:sigma-paths}

For a dependent type $B : A \to \Type$, paths between dependent pairs
$\langle a_1, b_1 \rangle$ and $\langle a_2, b_2 \rangle$ in $\Sigma_{x:A} B(x)$
decompose into a base path and a fiber path.

\begin{definition}[Sigma Path Operations]\label{def:sigma-paths}
  \begin{enumerate}[label=(\roman*)]
    \item \textbf{Construction.}\; $\sigmaMk(p, q) : \Path_{\Sigma B}(\langle a_1,
      b_1\rangle, \langle a_2, b_2\rangle)$ where $p : \Path_A(a_1, a_2)$
      and $q : \Path_{B(a_2)}(\tr(p, b_1), b_2)$.
    \item \textbf{First projection.}\; $\sigmaFst(r) = (\Sigma.\mathrm{fst})_*(r) :
      \Path_A(a_1, a_2)$.
    \item \textbf{Second projection.}\; $\sigmaSnd(r) : \Path_{B(a_2)}(\tr(\sigmaFst(r),
      b_1), b_2)$.
  \end{enumerate}
\end{definition}

The sigma path operations satisfy analogous $\beta/\eta$-rules to the product
case. These hold as rewrite steps (not strict equalities), since the
$\sigmaMk$ constructor creates a single-step path via $\ofEq$:
\begin{align*}
  \sigmaFst(\sigmaMk(p, q)) &\rew \ofEq(\toEq(p)), \\
  \sigmaSnd(\sigmaMk(p, q)) &\rew \ofEq(\toEq(q)), \\
  \sigmaMk(\sigmaFst(r), \sigmaSnd(r)) &\rew r.
\end{align*}

\section{Sum Paths}
\label{sec:sum-paths}

\begin{definition}[Sum Path Constructors]\label{def:sum-paths}
  For $p : \Path_A(a_1, a_2)$ and $q : \Path_B(b_1, b_2)$:
  \begin{enumerate}[label=(\roman*)]
    \item $\inlOp_*(p) = (\mathrm{Sum.inl})_*(p) :
      \Path_{A + B}(\mathrm{inl}(a_1), \mathrm{inl}(a_2))$.
    \item $\inrOp_*(q) = (\mathrm{Sum.inr})_*(q) :
      \Path_{A + B}(\mathrm{inr}(b_1), \mathrm{inr}(b_2))$.
  \end{enumerate}
\end{definition}

\begin{theorem}[Sum $\beta$-Rules]\label{thm:sum-beta}
  For $f : A \to C$, $g : B \to C$, and the eliminator
  $\mathrm{rec}(f, g) : A + B \to C$:
  \begin{align*}
    (\mathrm{rec}(f,g))_*(\inlOp_*(p)) &\rew f_*(p), \\
    (\mathrm{rec}(f,g))_*(\inrOp_*(q)) &\rew g_*(q).
  \end{align*}
\end{theorem}

\section{Function Paths}
\label{sec:function-paths}

\begin{definition}[Function Path Operations]\label{def:fun-paths}
  Let $f, g : A \to B$.
  \begin{enumerate}[label=(\roman*)]
    \item \textbf{Lambda congruence.}\;
      $\lamCongr(p) : \Path_{A \to B}(f, g)$ where $p : \prod_{x:A}
      \Path_B(f(x), g(x))$ is a family of pointwise paths. The step list
      is empty and the proof field is $\mathrm{funext}(\lambda x.\,
      \toEq(p(x)))$.
    \item \textbf{Application.}\;
      $\app(r, a) = (\lambda h.\, h(a))_*(r) : \Path_B(f(a), g(a))$
      for $r : \Path_{A \to B}(f, g)$ and $a : A$.
  \end{enumerate}
\end{definition}

\begin{theorem}[Function $\beta$/$\eta$-Rules]\label{thm:fun-beta-eta}
  \begin{enumerate}[label=(\roman*)]
    \item \textbf{$\beta$.}\;
      $\app(\lamCongr(p), a) \rew p(a)$.
    \item \textbf{$\eta$.}\;
      $\lamCongr(\lambda x.\, \app(q, x)) \rew q$.
  \end{enumerate}
\end{theorem}

Additional strict identities hold for $\lamCongr$: it preserves composition
and symmetry strictly:
\begin{align*}
  \lamCongr(p) \comp \lamCongr(q) &= \lamCongr(\lambda x.\, p(x) \comp q(x)),\\
  \inv{\lamCongr(p)} &= \lamCongr(\lambda x.\, \inv{p(x)}).
\end{align*}

\section{Transport and Dependent Application}
\label{sec:transport}

\begin{definition}[Transport]\label{def:transport}
  For a type family $D : A \to \Sort$, a path $p : \Path_A(a,b)$, and
  an element $x : D(a)$, the \emph{transport} of $x$ along $p$ is
  \[
    \tr_D(p, x) \;=\; \Eq.\mathrm{rec}(\pi, x) \;:\; D(b),
  \]
  where $\pi = \toEq(p) : a =_A b$. Transport uses only the semantic
  content of~$p$, not its trace.
\end{definition}

\begin{theorem}[Transport Laws]\label{thm:transport-laws}
  The following hold strictly:
  \begin{enumerate}[label=(\roman*)]
    \item $\tr_D(\refl(a), x) = x$.
    \item $\tr_D(p \comp q, x) = \tr_D(q, \tr_D(p, x))$.
    \item $\tr_D(\inv{p}, \tr_D(p, x)) = x$.
    \item $\tr_D(p, \tr_D(\inv{p}, y)) = y$.
  \end{enumerate}
\end{theorem}

\begin{proof}
  All follow from case analysis on the path's proof field.
  Since $\Eq.\mathrm{rec}$ is definitionally the identity when the
  proof is $\refl$, (i) is immediate. Parts (ii)--(iv) follow from
  the corresponding properties of $\Eq.\mathrm{rec}$.
\end{proof}

\begin{definition}[Dependent Application]\label{def:apd}
  For a dependent function $f : \prod_{x:A} D(x)$ and a path
  $p : \Path_A(a,b)$, the \emph{dependent application} is
  \[
    \apd(f, p) \;:\; \Path_{D(b)}(\tr_D(p, f(a)),\; f(b)).
  \]
  When $p = \refl(a)$, this reduces to $\refl(f(a))$.
\end{definition}

\section{Contexts}
\label{sec:contexts}

Contexts formalize substitution of paths into sub-expressions. They are the
categorical analogue of ``evaluation in context'' from term rewriting.

\begin{definition}[Unary Context]\label{def:context}
  A \emph{context} $C : \Context(A, B)$ is a function $\mathrm{fill} : A \to B$.
  It acts on paths by
  \[
    C[p] \;=\; \congrArgOp(\mathrm{fill}, p) \;:\;
    \Path_B(C[a_1], C[a_2])
  \]
  for $p : \Path_A(a_1, a_2)$, where we write $C[a]$ for
  $\mathrm{fill}(a)$.
\end{definition}

\begin{definition}[Context Substitution]\label{def:context-subst}
  Let $C : \Context(A, B)$.
  \begin{enumerate}[label=(\roman*)]
    \item \textbf{Left substitution.}\;
      $\substL(C, r, p) = r \comp C[p] : \Path_B(x, C[a_2])$ \\
      for $r : \Path_B(x, C[a_1])$ and $p : \Path_A(a_1, a_2)$.
    \item \textbf{Right substitution.}\;
      $\substR(C, p, t) = C[p] \comp t : \Path_B(C[a_1], y)$ \\
      for $p : \Path_A(a_1, a_2)$ and $t : \Path_B(C[a_2], y)$.
  \end{enumerate}
\end{definition}

Context substitution supports an extensive algebra of identities
(cf.\ \cref{sec:context-rules} in \cref{ch:rewrite-system}), including:
\begin{itemize}
  \item Unit laws: $\substL(C, \refl, p) \rew C[p]$ and
    $\substR(C, p, \refl) \rew C[p]$.
  \item Associativity: $\substR(C, p, t) \comp u \rew \substR(C, p, t \comp u)$.
  \item Idempotence: $\substL(C, \substL(C, r, \refl), p) \rew
    \substL(C, r, p)$.
  \item $\beta$-rules: $r \comp C[p] \rew \substL(C, r, p)$ and
    $C[p] \comp t \rew \substR(C, p, t)$.
\end{itemize}

\begin{definition}[Binary Context]\label{def:bicontext}
  A \emph{binary context} $K : \BiContext(A, B, C)$ is a function
  $\mathrm{fill} : A \to B \to C$. It supports $\mapLeft$, $\mapRight$,
  and $\mapTwo$ operations obtained by freezing one argument:
  \begin{align*}
    K.\mapLeft(p, b) &= (\lambda x.\, K[x, b])_*(p), \\
    K.\mapRight(a, q) &= K[a, -]_*(q), \\
    K.\mapTwo(p, q) &= K.\mapLeft(p, b_1) \comp K.\mapRight(a_2, q).
  \end{align*}
\end{definition}

\begin{definition}[Dependent Context]\label{def:depcontext}
  A \emph{dependent context} $C : \DepContext(A, B)$ consists of
  $\mathrm{fill} : \prod_{a:A} B(a)$ for a type family $B : A \to \Type$.
  Its action on a path $p : \Path_A(a_1, a_2)$ produces
  \[
    C.\mathrm{map}(p) \;:\; \Path_{B(a_2)}(\tr_B(p, C[a_1]),\; C[a_2]),
  \]
  which is the dependent application $\apd(\mathrm{fill}, p)$.
\end{definition}

Dependent contexts admit analogues of left and right substitution
($\substL$ and $\substR$), with additional transport factors to account
for the dependence of the codomain on the base.

\begin{definition}[Dependent Binary Context]\label{def:depbicontext}
  A \emph{dependent binary context} $K : \DepContext(A, B, C)$, where
  $C : A \to B \to \Type$, consists of $\mathrm{fill} : \prod_{a:A}
  \prod_{b:B} C(a, b)$. It supports $\mapLeft$, $\mapRight$, and
  $\mapTwo$ operations that combine transport in the base with fiber
  paths, generalizing the binary context operations.
\end{definition}

% ============================================================================
% Chapter 3: The Rewrite System
% ============================================================================
\chapter{The Rewrite System}
\label{ch:rewrite-system}

The fundamental algebraic laws of \cref{ch:basic-constructions}---left and
right identity, associativity, involution, anti-homomorphism---hold as strict
equalities because they reduce to list identities. But the \emph{cancellation}
laws ($p \comp \inv{p} = \refl$) and the $\beta/\eta$-rules for type formers
do not hold strictly: they require non-trivial reorganizations of the step
list. We therefore introduce a \emph{rewrite system} on paths that axiomatizes
these additional identities.

\section{The Single-Step Rewrite Relation}
\label{sec:step-relation}

\begin{definition}[Single-Step Rewrite]\label{def:step-rewrite}
  The relation $\rew$ on $\Path_A(a,b)$ is the smallest relation closed under
  the 75 rules organized into eight groups below. We write
  $\Step(p, q)$ or $p \rew q$ to denote that $p$ rewrites to $q$ in one step.
\end{definition}

\subsection{Group I: Path Algebra (8 rules)}
\label{sec:group-i}

These rules express the groupoid laws that are \emph{not} strict equalities.

\begin{enumerate}[label=\textbf{R\arabic*.}, ref=R\arabic*, leftmargin=3.5em]
  \item\label{rule:sr} \textbf{(sr)}
    $\inv{\refl(a)} \rew \refl(a)$.
  \item\label{rule:ss} \textbf{(ss)}
    $\inv{(\inv{p})} \rew p$.
  \item\label{rule:lrr} \textbf{(lrr)}
    $\refl(a) \comp p \rew p$.
  \item\label{rule:rrr} \textbf{(rrr)}
    $p \comp \refl(b) \rew p$.
  \item\label{rule:tr} \textbf{(tr)}
    $p \comp \inv{p} \rew \refl(a)$.
  \item\label{rule:tsr} \textbf{(tsr)}
    $\inv{p} \comp p \rew \refl(b)$.
  \item\label{rule:stss} \textbf{(stss)}
    $\inv{(p \comp q)} \rew \inv{q} \comp \inv{p}$.
  \item\label{rule:tt} \textbf{(tt)}
    $(p \comp q) \comp r \rew p \comp (q \comp r)$.
\end{enumerate}

\begin{remark}\label{rem:strict-vs-step}
  Rules~\ref{rule:lrr}, \ref{rule:rrr}, \ref{rule:ss}, \ref{rule:stss}, and
  \ref{rule:tt} overlap with the strict equalities of
  \cref{thm:strict-monoid,thm:strict-involution,thm:strict-antihom}. Including
  them in the rewrite system is necessary for two reasons: (i)~the
  \emph{structural closure} rules (\cref{sec:group-viii}) may produce these
  patterns nested inside larger contexts; (ii)~the rewrite system must be
  self-contained for the confluence and termination proofs.
\end{remark}

\subsection{Group II: Type-Former $\beta$/$\eta$-Rules (17 rules)}
\label{sec:group-ii}

These rules govern the interaction of path operations with type constructors.

\paragraph{Product rules.}
\begin{enumerate}[label=\textbf{R\arabic*.}, ref=R\arabic*, leftmargin=3.5em, resume]
  \item\label{rule:prod-fst-beta}
    $\fst(\prodMk(p, q)) \rew p$.
  \item\label{rule:prod-snd-beta}
    $\snd(\prodMk(p, q)) \rew q$.
  \item\label{rule:prod-eta}
    $\prodMk(\fst(r), \snd(r)) \rew r$.
  \item\label{rule:prod-mk-symm}
    $\inv{\prodMk(p, q)} \rew \prodMk(\inv{p}, \inv{q})$.
\end{enumerate}

\paragraph{Sigma rules.}
\begin{enumerate}[label=\textbf{R\arabic*.}, ref=R\arabic*, leftmargin=3.5em, resume]
  \item\label{rule:sigma-fst-beta}
    $\sigmaFst(\sigmaMk(p, q)) \rew \ofEq(\toEq(p))$.
  \item\label{rule:sigma-snd-beta}
    $\sigmaSnd(\sigmaMk(p, q)) \rew \ofEq(\toEq(q))$.
  \item\label{rule:sigma-eta}
    $\sigmaMk(\sigmaFst(r), \sigmaSnd(r)) \rew r$.
\end{enumerate}

\paragraph{Sum rules.}
\begin{enumerate}[label=\textbf{R\arabic*.}, ref=R\arabic*, leftmargin=3.5em, resume]
  \item\label{rule:sum-inl-beta}
    $(\mathrm{rec}(f,g))_*(\inlOp_*(p)) \rew f_*(p)$.
  \item\label{rule:sum-inr-beta}
    $(\mathrm{rec}(f,g))_*(\inrOp_*(q)) \rew g_*(q)$.
\end{enumerate}

\paragraph{Function rules.}
\begin{enumerate}[label=\textbf{R\arabic*.}, ref=R\arabic*, leftmargin=3.5em, resume]
  \item\label{rule:fun-app-beta}
    $\app(\lamCongr(p), a) \rew p(a)$.
  \item\label{rule:fun-eta}
    $\lamCongr(\lambda x.\, \app(q, x)) \rew q$.
  \item\label{rule:lam-symm}
    $\inv{\lamCongr(p)} \rew \lamCongr(\lambda x.\, \inv{p(x)})$.
\end{enumerate}

\paragraph{Map congruence rules.}
\begin{enumerate}[label=\textbf{R\arabic*.}, ref=R\arabic*, leftmargin=3.5em, resume]
  \item\label{rule:prod-map-congr}
    For $f = (g, h) : A \times B \to A' \times B'$,\;
    $f_*(\prodMk(p,q)) \rew \prodMk(g_*(p), h_*(q))$.
\end{enumerate}

Additional rules (22--25) govern the interaction of dependent contexts with
symmetry and the decomposition of dependent application.

\subsection{Group III: Transport Rules (7 rules)}
\label{sec:group-iii}

\begin{enumerate}[label=\textbf{R\arabic*.}, ref=R\arabic*, leftmargin=3.5em, start=26]
  \item\label{rule:transport-refl}
    $\tr_D(\refl(a), x) \rew x$ \quad (identity transport).
  \item\label{rule:transport-trans}
    $\tr_D(p \comp q, x) \rew \tr_D(q, \tr_D(p, x))$ \quad (distributivity).
  \item\label{rule:transport-symm-left}
    $\tr_D(\inv{p}, \tr_D(p, x)) \rew x$.
  \item\label{rule:transport-symm-right}
    $\tr_D(p, \tr_D(\inv{p}, y)) \rew y$.
\end{enumerate}

Rules 30--32 address transport through sigma constructors.

\subsection{Group IV: Context Rules (16 rules)}
\label{sec:context-rules}

These rules govern the interaction of context substitution with path operations.
Let $C : \Context(A, B)$.

\paragraph{Unit rules.}
\begin{enumerate}[label=\textbf{R\arabic*.}, ref=R\arabic*, leftmargin=3.5em, start=33]
  \item\label{rule:context-congr}
    $\Step(p, q) \implies \Step(C[p], C[q])$ \quad (context congruence).
  \item\label{rule:context-symm}
    $\inv{C[p]} \rew C[\inv{p}]$ \quad (symmetry through context).
  \item\label{rule:slr}
    \textbf{(slr)}\; $\substL(C, \refl, p) \rew C[p]$.
  \item\label{rule:srr}
    \textbf{(srr)}\; $\substR(C, p, \refl) \rew C[p]$.
\end{enumerate}

\paragraph{Idempotence and cancellation.}
\begin{enumerate}[label=\textbf{R\arabic*.}, ref=R\arabic*, leftmargin=3.5em, resume]
  \item\label{rule:slss}
    \textbf{(slss)}\; $\substL(C, \substL(C, r, \refl), p) \rew \substL(C, r, p)$.
  \item\label{rule:srsr}
    \textbf{(srsr)}\; $\substR(C, p, \substR(C, \refl, t)) \rew \substR(C, p, t)$.
  \item\label{rule:srrrr}
    \textbf{(srrrr)}\; $\substR(C, \refl, \substR(C, p, t)) \rew \substR(C, p, t)$.
\end{enumerate}

\paragraph{$\beta$-rules (folding into substitution form).}
\begin{enumerate}[label=\textbf{R\arabic*.}, ref=R\arabic*, leftmargin=3.5em, resume]
  \item\label{rule:tsbll}
    \textbf{(tsbll)}\; $r \comp C[p] \rew \substL(C, r, p)$.
  \item\label{rule:tsbrl}
    \textbf{(tsbrl)}\; $C[p] \comp t \rew \substR(C, p, t)$.
\end{enumerate}

\paragraph{Associativity.}
\begin{enumerate}[label=\textbf{R\arabic*.}, ref=R\arabic*, leftmargin=3.5em, resume]
  \item\label{rule:tsblr}
    \textbf{(tsblr)}\; $\substL(C, r, p) \comp t \rew r \comp \substR(C, p, t)$.
  \item\label{rule:tsbrr}
    \textbf{(tsbrr)}\; $\substR(C, p, t) \comp u \rew \substR(C, p, t \comp u)$.
\end{enumerate}

\paragraph{Cancellation.}
\begin{enumerate}[label=\textbf{R\arabic*.}, ref=R\arabic*, leftmargin=3.5em, resume]
  \item\label{rule:ttsv}
    \textbf{(ttsv)}\; $C[p] \comp (C[\inv{p}] \comp v) \rew C[p \comp \inv{p}] \comp v$.
  \item\label{rule:tstu}
    \textbf{(tstu)}\; $(v \comp C[p]) \comp C[\inv{p}] \rew v \comp C[p \comp \inv{p}]$.
\end{enumerate}

\subsection{Groups V--VI: Dependent Context and Bi-Context Rules (20 rules)}
\label{sec:groups-v-vi}

Rules 46--60 are the analogues of Group~IV for dependent contexts
$\DepContext(A, B)$, carrying the additional transport data required by the
dependence of the codomain on the base. Rules 61--68 govern the interaction of
binary contexts ($\BiContext$ and dependent binary contexts) with $\mapLeft$,
$\mapRight$, $\mapTwo$, and their structural closure.

\subsection{Group VII: Map Congruence Rules (4 rules)}
\label{sec:group-vii}

\begin{enumerate}[label=\textbf{R\arabic*.}, ref=R\arabic*, leftmargin=3.5em, start=69]
  \item $\mapLeft(f, -, b)$ preserves $\Step$: $p \rew q$ implies
    $\mapLeft(f, p, b) \rew \mapLeft(f, q, b)$.
  \item $\mapRight(f, a, -)$ preserves $\Step$.
  \item $\mapTwo$ distributes: $\mapTwo(f, p, q) \rew
    \mapRight(f, a_1, q) \comp \mapLeft(f, p, b_2)$
    (alternative factorization).
  \item Interaction of $\ofEq$ with map operations.
\end{enumerate}

\subsection{Group VIII: Structural Closure (4 rules)}
\label{sec:group-viii}

The structural closure rules propagate single-step rewrites through the
path constructors, ensuring that the rewrite relation is compatible with
all operations:

\begin{enumerate}[label=\textbf{R\arabic*.}, ref=R\arabic*, leftmargin=3.5em, start=73]
  \item\label{rule:symm-congr}
    \textbf{(symm\_congr)}\;
    $p \rew q \implies \inv{p} \rew \inv{q}$.
  \item\label{rule:trans-congr-left}
    \textbf{(trans\_congr\_left)}\;
    $p \rew q \implies p \comp r \rew q \comp r$.
  \item\label{rule:trans-congr-right}
    \textbf{(trans\_congr\_right)}\;
    $q \rew r \implies p \comp q \rew p \comp r$.
  \item\label{rule:context-congr-closure}
    \textbf{(context\_congr)}\;
    $p \rew q \implies C[p] \rew C[q]$.
\end{enumerate}

\begin{remark}
  Rules~\ref{rule:symm-congr}--\ref{rule:context-congr-closure} make $\rew$
  a \emph{congruence closure}: any rewrite deep inside a path expression can
  be lifted to the top level.
\end{remark}

\section{Soundness}
\label{sec:soundness}

\begin{theorem}[Soundness of Step]\label{thm:step-sound}
  If $p \rew q$ then $\toEq(p) = \toEq(q)$.
\end{theorem}

\begin{proof}
  By induction on the derivation of $\Step(p, q)$. Each of the 75 rules
  preserves the proof field because all path operations are designed to be
  sound with respect to the underlying propositional equality. The structural
  closure rules follow by the induction hypothesis.
\end{proof}

Soundness guarantees that rewriting never changes the \emph{meaning} of a path;
it only reorganizes the computational trace.

\section{Multi-Step Rewriting}
\label{sec:multi-step}

\begin{definition}[Multi-Step Rewrite]\label{def:rw}
  The relation $\rews$ on $\Path_A(a,b)$ is the reflexive--transitive
  closure of $\rew$. Formally, $\Rw$ is the smallest relation satisfying:
  \begin{enumerate}[label=(\roman*)]
    \item $\Rw.\refl(p) : p \rews p$ for all $p$.
    \item $\Rw.\mathrm{tail}(h, s) : p \rews r$ whenever $h : p \rews q$
      and $s : q \rew r$.
  \end{enumerate}
\end{definition}

\begin{corollary}\label{cor:rw-sound}
  If $p \rews q$ then $\toEq(p) = \toEq(q)$.
\end{corollary}

\section{Rewrite Equality}
\label{sec:rweq}

\begin{definition}[Rewrite Equality]\label{def:rweq}
  The \emph{rewrite equality} $\rweq$ is the equivalence relation generated
  by~$\rew$---equivalently, the symmetric closure of $\rews$. It is the
  smallest relation satisfying:
  \begin{enumerate}[label=(\roman*)]
    \item $\RwEq.\refl(p) : p \rweq p$.
    \item $\RwEq.\mathrm{step}(s) : p \rweq q$ whenever $s : p \rew q$.
    \item $\RwEq.\mathrm{symm}(h) : q \rweq p$ whenever $h : p \rweq q$.
    \item $\RwEq.\mathrm{trans}(h_1, h_2) : p \rweq r$ whenever
      $h_1 : p \rweq q$ and $h_2 : q \rweq r$.
  \end{enumerate}
\end{definition}

\begin{theorem}[Congruence Properties of $\RwEq$]\label{thm:rweq-congruence}
  Rewrite equality is a congruence with respect to all path operations:
  \begin{enumerate}[label=(\roman*)]
    \item $p_1 \rweq p_2$ and $q_1 \rweq q_2$ imply
      $p_1 \comp q_1 \rweq p_2 \comp q_2$.
    \item $p \rweq q$ implies $\inv{p} \rweq \inv{q}$.
    \item $p \rweq q$ implies $f_*(p) \rweq f_*(q)$ for any $f$.
    \item $p \rweq q$ implies $C[p] \rweq C[q]$ for any context $C$.
  \end{enumerate}
  Analogous congruence results hold for $\mapLeft$, $\mapRight$, $\mapTwo$,
  $\BiContext.\mapTwo$, $\DepContext.\mathrm{map}$, $\lamCongr$, $\prodMk$,
  and $\sigmaMk$.
\end{theorem}

\begin{proof}
  Each part follows by induction on the $\RwEq$ derivation, using the
  structural closure rules~\ref{rule:symm-congr}--\ref{rule:context-congr-closure}
  in the base case ($\RwEq.\mathrm{step}$).
\end{proof}

The congruence properties ensure that $\RwEq$ is a well-behaved equivalence
relation that respects the algebraic structure of paths.

\begin{theorem}[Groupoid Laws up to $\RwEq$]\label{thm:rweq-groupoid-laws}
  The following hold:
  \begin{enumerate}[label=(\roman*)]
    \item $\refl(a) \comp p \rweq p$ and $p \comp \refl(b) \rweq p$.
    \item $(p \comp q) \comp r \rweq p \comp (q \comp r)$.
    \item $p \comp \inv{p} \rweq \refl(a)$ and $\inv{p} \comp p \rweq \refl(b)$.
  \end{enumerate}
\end{theorem}

\begin{proof}
  Each is a single application of $\RwEq.\mathrm{step}$ to the
  corresponding rule from Group~I.
\end{proof}

\section{Normalization}
\label{sec:normalization}

\begin{definition}[Normal Form]\label{def:normal-form}
  A path $p : \Path_A(a,b)$ is \emph{normal} if $p = \ofEq(\toEq(p))$. The
  \emph{normalization function} is
  \[
    \normalize(p) \;=\; \ofEq(\toEq(p)) \;:\; \Path_A(a,b).
  \]
\end{definition}

Since $\toEq$ extracts the underlying equality proof and $\ofEq$ wraps it in a
single-step path, normalization discards the trace and replaces it with the
canonical one-step witness.

\begin{theorem}[Properties of Normalization]\label{thm:normalization}
  \begin{enumerate}[label=(\roman*)]
    \item $\normalize(p)$ is always normal.
    \item $p \rweq \normalize(p)$ for every path $p$.
    \item Two paths $p, q : \Path_A(a,b)$ are $\RwEq$-equivalent if and only if
      $\normalize(p) = \normalize(q)$.
  \end{enumerate}
\end{theorem}

\begin{proof}
  Part~(i) is immediate from the definition. Part~(ii): by soundness,
  $\toEq(p) = \toEq(\normalize(p))$, and both $p$ and $\normalize(p)$ can be
  connected via the rewrite rules (the groupoid rules and $\beta/\eta$-rules
  suffice to reduce any path to its normal form). Part~(iii): since
  $\normalize(p) = \ofEq(\toEq(p))$ and $\normalize(q) = \ofEq(\toEq(q))$,
  these are equal iff $\toEq(p) = \toEq(q)$, which holds by proof irrelevance
  of $\Eq$. The ``only if'' direction follows from soundness (\cref{thm:step-sound}).
\end{proof}

\begin{corollary}\label{cor:rweq-decidable}
  Rewrite equality of paths is decidable: $p \rweq q$ iff
  $\normalize(p) = \normalize(q)$, which can be checked by structural
  comparison.
\end{corollary}

\section{Termination}
\label{sec:termination}

\begin{theorem}[Termination]\label{thm:termination}
  The rewrite relation $\rews$ is well-founded: there are no infinite
  reduction sequences.
\end{theorem}

The proof uses a \emph{recursive path ordering} (RPO) adapted to the typed
rewriting setting.

\begin{definition}[Rule Precedence]\label{def:rule-precedence}
  The 76 rewrite rules are assigned a numeric rank
  $\mathrm{rank} : \mathrm{Rule} \to \Nat$, forming a well-founded
  precedence relation. The ranking is chosen so that rules introducing
  simpler path expressions (e.g., $\refl$) have lower rank than rules
  producing compound expressions.
\end{definition}

\begin{definition}[RPO Measure]\label{def:rpo}
  Each path $p$ is assigned a \emph{term} $T(p)$ in the RPO, comprising:
  \begin{itemize}
    \item A \emph{symbol} drawn from $\{\mathrm{nf}\} \cup \mathrm{Rule}
      \cup \{\mathrm{pathLen}(n) : n \in \Nat\}$, where $\mathrm{nf}$
      (normal form) is the least element.
    \item An aggregate weight $\mathrm{pathLenSum}(p) \in \Nat$.
  \end{itemize}
  The ordering $T(p) >_{\mathrm{RPO}} T(q)$ holds when the symbol rank
  of~$p$ strictly exceeds that of~$q$ and the aggregate weight does not
  increase.
\end{definition}

\begin{proposition}\label{prop:rpo-wf}
  The RPO ordering is well-founded.
\end{proposition}

\begin{theorem}\label{thm:rpo-decrease}
  Every application of a rewrite rule strictly decreases the RPO measure:
  if $p \rew q$ via rule $R$, then $T(p) >_{\mathrm{RPO}} T(q)$.
\end{theorem}

\begin{proof}
  By case analysis on the 75 rules. Each rule either reduces the symbol
  rank or maintains the rank while strictly decreasing the aggregate weight.
\end{proof}

\section{Confluence}
\label{sec:confluence}

\begin{definition}[Join]\label{def:join}
  A \emph{join} of $q$ and $r$ (where $p \rews q$ and $p \rews r$ for some
  common source~$p$) is a path $m$ together with witnesses $q \rews m$ and
  $r \rews m$.
\end{definition}

\begin{theorem}[Strip Lemma (Local Confluence)]\label{thm:strip-lemma}
  If $p \rew q$ and $p \rews r$, then $q$ and $r$ have a common reduct:
  there exists $m$ with $q \rews m$ and $r \rews m$.
\end{theorem}

\begin{proof}
  By induction on the derivation of $p \rews r$. The base case
  ($r = p$) is trivial. For the inductive case, suppose $p \rews r'$
  and $r' \rew r$. By the induction hypothesis applied to $p \rew q$
  and $p \rews r'$, we obtain a join of $q$ and $r'$ at some $m'$.
  We then perform a \emph{critical pair analysis}: for each pair of
  overlapping rules that could apply to~$r'$, we exhibit an explicit
  join. The analysis covers all pairs among the 75 rules.
\end{proof}

The critical pair analysis is the most technically demanding part of the
confluence proof. Representative cases include:

\begin{itemize}
  \item \textbf{Product $\fst$ overlap.}\;
    When $\fst(\prodMk(p, q))$ can be rewritten by both the $\beta$-rule
    (\ref{rule:prod-fst-beta}) and a structural closure rule, the two
    reducts join at~$p$.

  \item \textbf{Associativity--unit overlap.}\;
    When $((p \comp q) \comp r)$ where $r = \refl$ can be rewritten by
    either \ref{rule:tt}~(associativity) or \ref{rule:rrr}~(right unit),
    the join is $p \comp q$.

  \item \textbf{Context substitution overlap.}\;
    When $\substL(C, r, p) \comp t$ overlaps with the $\beta$-rule
    (\ref{rule:tsbll}) and the associativity rule (\ref{rule:tsblr}),
    the two reducts join at $r \comp \substR(C, p, t)$.
\end{itemize}

\begin{theorem}[Confluence]\label{thm:confluence}
  The rewrite system is confluent: for any paths $p, q, r$ with
  $p \rews q$ and $p \rews r$, there exists $m$ with $q \rews m$
  and $r \rews m$.
\end{theorem}

\begin{proof}
  By Newman's lemma~\cite{Newman42}: a terminating relation is confluent
  if and only if it is locally confluent. Termination is established in
  \cref{thm:termination}, and local confluence follows from the strip
  lemma (\cref{thm:strip-lemma}).
\end{proof}

\begin{corollary}[Unique Normal Forms]\label{cor:unique-nf}
  Every path has a unique normal form (up to structural equality), and
  two paths are $\RwEq$-equivalent if and only if they reduce to the
  same normal form.
\end{corollary}

\begin{proof}
  Existence of normal forms follows from termination. Uniqueness follows
  from confluence: if $p \rews m_1$ and $p \rews m_2$ with $m_1, m_2$
  normal, then by confluence there exists $m$ with $m_1 \rews m$ and
  $m_2 \rews m$; since $m_1$ and $m_2$ are normal, $m_1 = m = m_2$.
\end{proof}

\section{The Quotient $\PathQuot$}
\label{sec:path-quot}

\begin{definition}[Path Quotient]\label{def:path-quot}
  The \emph{path quotient} is the quotient type
  \[
    \PathQuot_A(a, b) \;=\; \Path_A(a, b) \,/\, {\rweq}.
  \]
  We write $[p]$ for the equivalence class of a path $p$.
\end{definition}

Since $\RwEq$ is a congruence (\cref{thm:rweq-congruence}), all path
operations descend to well-defined operations on the quotient:

\begin{theorem}[Well-Defined Quotient Operations]\label{thm:quot-ops}
  The following operations are well-defined on $\PathQuot$:
  \begin{align*}
    \mathrm{trans} &: \PathQuot_A(a, b) \to \PathQuot_A(b, c) \to \PathQuot_A(a, c), \\
    \mathrm{symm} &: \PathQuot_A(a, b) \to \PathQuot_A(b, a), \\
    f_* &: \PathQuot_A(a, b) \to \PathQuot_B(f(a), f(b)).
  \end{align*}
\end{theorem}

\begin{theorem}[Strict Groupoid Laws on the Quotient]\label{thm:quot-groupoid}
  On $\PathQuot$, all groupoid axioms hold as \textbf{strict equalities}
  (equalities of quotient elements):
  \begin{enumerate}[label=(\roman*)]
    \item $[\refl(a)] \comp [q] = [q]$ \quad and \quad $[p] \comp [\refl(b)] = [p]$.
    \item $([p] \comp [q]) \comp [r] = [p] \comp ([q] \comp [r])$.
    \item $[p] \comp [\inv{p}] = [\refl(a)]$ \quad and \quad
      $[\inv{p}] \comp [p] = [\refl(b)]$.
    \item $[\inv{(\inv{p})}] = [p]$.
  \end{enumerate}
\end{theorem}

\begin{proof}
  Each identity holds because the corresponding rewrite rule from Group~I
  provides a witness of $\RwEq$, which becomes an equality after quotienting.
\end{proof}

\begin{theorem}[Equivalence with the Identity Type]\label{thm:quot-equiv}
  The semantic projection $\toEq$ descends to a bijection
  \[
    \PathQuot_A(a, b) \;\cong\; (a =_A b).
  \]
\end{theorem}

\begin{proof}
  By \cref{thm:normalization}(iii), two paths are $\RwEq$-equivalent
  iff they have the same underlying equality proof (which is unique by
  UIP). Hence each equivalence class corresponds to exactly one element
  of the identity type.
\end{proof}

% ============================================================================
% Chapter 4: The Groupoid of Computational Paths
% ============================================================================
\chapter{The Groupoid of Computational Paths}
\label{ch:groupoid}

Having established the rewrite system and its metatheoretic properties, we
now show that the algebraic structure of computational paths gives rise to
categorical structures: a \emph{weak} groupoid on the raw path space, a
\emph{strict} groupoid on the quotient, and functorial transport of the
entire rewrite structure.

\section{Weak Category and Weak Groupoid}
\label{sec:weak-groupoid}

\begin{definition}[Weak Category]\label{def:weak-cat}
  A \emph{weak category} on a type $A$ consists of:
  \begin{itemize}
    \item A composition $\mathrm{comp} : \Path_A(a,b) \to \Path_A(b,c) \to
      \Path_A(a,c)$.
    \item An identity $\mathrm{id} : (a : A) \to \Path_A(a,a)$.
    \item Witnesses (in $\Rw$) of the associativity and unit laws:
      \begin{align*}
        &\mathrm{comp}(\mathrm{comp}(p, q), r) \rews
          \mathrm{comp}(p, \mathrm{comp}(q, r)), \\
        &\mathrm{comp}(\mathrm{id}(a), p) \rews p, \\
        &\mathrm{comp}(p, \mathrm{id}(b)) \rews p.
      \end{align*}
  \end{itemize}
\end{definition}

\begin{theorem}\label{thm:type-weak-cat}
  Every type $A$ carries a canonical weak category with
  $\mathrm{comp} = \mathrm{trans}$ and $\mathrm{id} = \refl$. The unit
  and associativity laws hold via single rewrite steps (hence a fortiori
  via $\Rw$).
\end{theorem}

\begin{proof}
  The three witnesses are provided by rules~\ref{rule:lrr} (left unit),
  \ref{rule:rrr} (right unit), and \ref{rule:tt} (associativity) from
  Group~I of the rewrite system.
\end{proof}

\begin{definition}[Weak Groupoid]\label{def:weak-gpd}
  A \emph{weak groupoid} on $A$ extends a weak category with:
  \begin{itemize}
    \item An inversion $\mathrm{inv} : \Path_A(a,b) \to \Path_A(b,a)$.
    \item Witnesses of the cancellation laws:
      \begin{align*}
        &\mathrm{comp}(\mathrm{inv}(p), p) \rews \mathrm{id}(b), \\
        &\mathrm{comp}(p, \mathrm{inv}(p)) \rews \mathrm{id}(a).
      \end{align*}
  \end{itemize}
\end{definition}

\begin{theorem}\label{thm:type-weak-gpd}
  Every type $A$ is a weak groupoid under computational paths, with
  $\mathrm{inv} = \symop$.
\end{theorem}

\begin{proof}
  The cancellation laws are provided by rules~\ref{rule:tsr}
  ($\inv{p} \comp p \rew \refl(b)$) and~\ref{rule:tr}
  ($p \comp \inv{p} \rew \refl(a)$).
\end{proof}

\begin{remark}\label{rem:weakness}
  The word ``weak'' is used precisely: the groupoid laws hold only up to
  the rewrite relation $\Rw$, not as strict equalities of $\Path$ records.
  The strict equalities of \cref{thm:strict-monoid} (unit and associativity)
  provide even stronger witnesses, but the cancellation laws genuinely
  require the rewrite system.
\end{remark}

\section{Strict Category and Strict Groupoid on the Quotient}
\label{sec:strict-groupoid}

\begin{definition}[Strict Category]\label{def:strict-cat}
  A \emph{strict category} on $A$ is a category in the usual sense: the
  associativity and unit laws hold as equalities (not merely up to~$\Rw$).
\end{definition}

\begin{definition}[Strict Groupoid]\label{def:strict-gpd}
  A \emph{strict groupoid} on $A$ extends a strict category with an
  inversion satisfying the cancellation laws as equalities.
\end{definition}

\begin{theorem}\label{thm:quot-strict-gpd}
  The quotient $\PathQuot_A(-,-)$ carries the structure of a strict groupoid.
\end{theorem}

\begin{proof}
  By \cref{thm:quot-groupoid}, all groupoid axioms hold as equalities of
  quotient elements. The operations $\mathrm{trans}$, $\symop$, and $\refl$
  descend to well-defined operations on $\PathQuot$ by the congruence
  property of $\RwEq$ (\cref{thm:rweq-congruence}).
\end{proof}

This result establishes a clean separation between two levels of structure:

\begin{center}
\begin{tabular}{lll}
  \toprule
  \textbf{Level} & \textbf{Object} & \textbf{Laws} \\
  \midrule
  Raw paths & $\Path_A(a,b)$ & Weak groupoid (laws up to $\Rw$) \\
  Quotient  & $\PathQuot_A(a,b)$ & Strict groupoid (laws as equalities) \\
  \bottomrule
\end{tabular}
\end{center}

The quotient recovers the standard identity type (\cref{thm:quot-equiv}),
while the raw path space carries the additional combinatorial structure
needed for higher-dimensional algebra.

\section{Equality Functors}
\label{sec:eq-functor}

\begin{definition}[Equality Functor]\label{def:eq-functor}
  An \emph{equality functor} from $A$ to $B$ consists of:
  \begin{itemize}
    \item An object map $\mathrm{obj} : A \to B$.
    \item A path map $\mathrm{map} : \Path_A(a,b) \to \Path_B(\mathrm{obj}(a),
      \mathrm{obj}(b))$.
    \item Functoriality witnesses:
      \begin{align*}
        \mathrm{map}(\refl(a)) &= \refl(\mathrm{obj}(a)), \\
        \mathrm{map}(p \comp q) &= \mathrm{map}(p) \comp \mathrm{map}(q).
      \end{align*}
  \end{itemize}
\end{definition}

\begin{proposition}\label{prop:congrArg-functor}
  For any function $f : A \to B$, the pair $(\mathrm{obj} = f,\;
  \mathrm{map} = f_*)$ is an equality functor. The functoriality
  witnesses hold as strict equalities by \cref{thm:congrArg-functor}.
\end{proposition}

\section{Rewrite Lifts}
\label{sec:rewrite-lifts}

A \emph{rewrite lift} transports not only paths but also the rewrite
structure from one type to another.

\begin{definition}[Rewrite Lift]\label{def:rewrite-lift}
  A \emph{rewrite lift} from $A$ to $B$ consists of:
  \begin{itemize}
    \item An object map $\mathrm{obj} : A \to B$.
    \item A path map $\mathrm{mapPath} : \Path_A(a,b) \to
      \Path_B(\mathrm{obj}(a), \mathrm{obj}(b))$.
    \item A step map: $\Step(p, q) \implies \Step(\mathrm{mapPath}(p),
      \mathrm{mapPath}(q))$.
  \end{itemize}
\end{definition}

\begin{theorem}\label{thm:lift-rw-rweq}
  Any rewrite lift transports both $\Rw$ and $\RwEq$:
  \begin{enumerate}[label=(\roman*)]
    \item $p \rews q$ implies $\mathrm{mapPath}(p) \rews \mathrm{mapPath}(q)$.
    \item $p \rweq q$ implies $\mathrm{mapPath}(p) \rweq \mathrm{mapPath}(q)$.
  \end{enumerate}
\end{theorem}

\begin{proof}
  Part~(i) by induction on the $\Rw$ derivation, using the step map at
  each tail step. Part~(ii) by induction on the $\RwEq$ derivation, using
  (i) for the step case and the closure properties of $\RwEq$.
\end{proof}

\begin{proposition}\label{prop:canonical-lifts}
  Each of the following produces a canonical rewrite lift:
  \begin{enumerate}[label=(\roman*)]
    \item Any function $f : A \to B$ (via $\congrArgOp(f, -)$).
    \item Any context $C : \Context(A, B)$ (via $C[-]$).
    \item Any binary context $K : \BiContext(A, B, C)$ with a fixed
      argument (via $K.\mapLeft(-, b)$ or $K.\mapRight(a, -)$).
    \item Any dependent context $C : \DepContext(A, B)$ (via $C.\mathrm{map}$).
  \end{enumerate}
\end{proposition}

\begin{proof}
  In each case, the step map is provided by the corresponding structural
  closure rule (\ref{rule:context-congr-closure}) from Group~VIII.
\end{proof}

Rewrite lifts compose: if $L_1 : A \to B$ and $L_2 : B \to C$ are rewrite
lifts, their composition $L_2 \circ L_1$ is a rewrite lift from $A$ to $C$.
This makes the collection of types with rewrite lifts into a category,
which refines the category of types with equality functors.

% ============================================================================
% Chapter 5: Higher-Dimensional Structure
% ============================================================================
\chapter{Higher-Dimensional Structure}
\label{ch:higher-dimensional}

We now ascend from the one-dimensional algebra of paths to the
higher-dimensional structure that constitutes the central contribution of this
work. Rewrite equalities between paths serve as \emph{two-cells}, and iterated
derivation structures provide cells at every dimension. The resulting tower
forms a weak $\omega$-groupoid, with a sharp contractibility threshold at
dimension~3.

\section{Two-Cells and the Bicategory of Paths}
\label{sec:two-cells}

\begin{definition}[Two-Cell]\label{def:two-cell}
  A \emph{two-cell} between paths $p, q : \Path_A(a,b)$ is a witness of
  rewrite equality:
  \[
    \eta : p \rweq q.
  \]
  Two-cells inhabit $\Prop$ (they are proof-irrelevant), since $\RwEq$ is a
  proposition.
\end{definition}

\begin{definition}[Two-Cell Operations]\label{def:two-cell-ops}
  Two-cells support the following operations:
  \begin{enumerate}[label=(\roman*)]
    \item \textbf{Identity.}\; $\mathrm{id}_p : p \rweq p$ \quad
      (via $\RwEq.\refl$).
    \item \textbf{Vertical composition.}\;
      $\eta \circ_v \theta : p \rweq r$ \quad for $\eta : p \rweq q$ and
      $\theta : q \rweq r$ \quad (via $\RwEq.\mathrm{trans}$).
    \item \textbf{Left whiskering.}\;
      $f \triangleright_L \eta : f \comp g \rweq f \comp h$ \\
      for $f : \Path_A(a,b)$ and $\eta : g \rweq h$ where
      $g, h : \Path_A(b,c)$ \\
      (via the congruence of $\mathrm{trans}$ in its second argument).
    \item \textbf{Right whiskering.}\;
      $\eta \triangleleft_R h : f \comp h \rweq g \comp h$ \\
      for $\eta : f \rweq g$ and $h : \Path_A(b,c)$ \\
      (via the congruence of $\mathrm{trans}$ in its first argument).
    \item \textbf{Horizontal composition.}\;
      $\eta \circ_h \theta : f \comp g \rweq f' \comp g'$ \\
      for $\eta : f \rweq f'$ and $\theta : g \rweq g'$ \\
      (defined as $(\eta \triangleleft_R g) \circ_v (f' \triangleright_L \theta)$).
  \end{enumerate}
\end{definition}

\subsection{Associator and Unitor Two-Cells}

The rewrite rules from Group~I provide canonical two-cells witnessing the
coherence data of a bicategory:

\begin{definition}[Associator]\label{def:associator}
  For composable paths $p : \Path_A(a,b)$, $q : \Path_A(b,c)$,
  $r : \Path_A(c,d)$, the \emph{associator} is the two-cell
  \[
    \alpha_{p,q,r} : (p \comp q) \comp r \;\rweq\; p \comp (q \comp r),
  \]
  given by $\RwEq.\mathrm{step}$ applied to rule~\ref{rule:tt}.
\end{definition}

\begin{definition}[Unitors]\label{def:unitors}
  The \emph{left} and \emph{right unitors} are:
  \begin{align*}
    \lambda_p &: \refl(a) \comp p \;\rweq\; p &
    &\text{(via rule~\ref{rule:lrr})}, \\
    \rho_p &: p \comp \refl(b) \;\rweq\; p &
    &\text{(via rule~\ref{rule:rrr})}.
  \end{align*}
\end{definition}

\subsection{Coherence Laws}

\begin{theorem}[Pentagon Coherence]\label{thm:pentagon}
  For composable paths $p, q, r, s$, the two canonical ways of
  reassociating the four-fold composite agree:
  \[
    \alpha_{p \comp q, r, s} \circ_v \alpha_{p, q, r \comp s}
    \;=\;
    (\alpha_{p,q,r} \triangleleft_R s) \circ_v
    \alpha_{p, q \comp r, s} \circ_v
    (p \triangleright_L \alpha_{q,r,s}).
  \]
  This equality of two-cells holds by proof irrelevance of $\RwEq$.
\end{theorem}

\begin{theorem}[Triangle Coherence]\label{thm:triangle}
  For composable paths $p : \Path_A(a,b)$ and $q : \Path_A(b,c)$:
  \[
    \alpha_{p, \refl(b), q} \circ_v (p \triangleright_L \lambda_q)
    \;=\;
    \rho_p \triangleleft_R q.
  \]
  Again, this holds by proof irrelevance.
\end{theorem}

\begin{theorem}[Interchange Law]\label{thm:interchange}
  For four two-cells $\eta_1, \eta_2, \theta_1, \theta_2$ arranged in a
  $2 \times 2$ grid:
  \[
    (\eta_1 \circ_h \theta_1) \circ_v (\eta_2 \circ_h \theta_2)
    \;=\;
    (\eta_1 \circ_v \eta_2) \circ_h (\theta_1 \circ_v \theta_2).
  \]
  This is the \emph{middle-four interchange}. It holds because both
  sides inhabit the same $\Prop$-valued type.
\end{theorem}

\begin{remark}\label{rem:proof-irrel-coherence}
  The pentagon, triangle, and interchange laws hold trivially (by
  $\mathsf{Subsingleton.elim}$) because two-cells are $\Prop$-valued.
  This is a feature, not a deficiency: it means that all coherence
  conditions at the two-cell level and above are \emph{automatically}
  satisfied. The non-trivial content of the theory lies at the
  one-cell level, where distinct paths encode genuinely different
  computational traces.
\end{remark}

\subsection{The Weak Bicategory and Weak 2-Groupoid}

\begin{definition}[Weak Bicategory]\label{def:weak-bicat}
  A \emph{weak bicategory} consists of:
  \begin{itemize}
    \item 0-cells (objects), 1-cells (morphisms), 2-cells.
    \item Composition and identity at the 1-cell level.
    \item Vertical and horizontal composition, whiskering, identity at
      the 2-cell level.
    \item Associator and unitors as invertible 2-cells.
    \item Pentagon and triangle coherences.
  \end{itemize}
\end{definition}

\begin{theorem}\label{thm:weak-bicat}
  Computational paths form a weak bicategory, with:
  \begin{center}
  \begin{tabular}{ll}
    0-cells: & elements of $A$, \\
    1-cells: & paths $p : \Path_A(a,b)$, \\
    2-cells: & rewrite equalities $\eta : p \rweq q$.
  \end{tabular}
  \end{center}
\end{theorem}

\begin{definition}[Weak 2-Groupoid]\label{def:weak-2-gpd}
  A \emph{weak 2-groupoid} extends a weak bicategory with:
  \begin{itemize}
    \item An inversion $\mathrm{inv}_1$ on 1-cells, with cancellation
      two-cells: $p \comp \inv{p} \rweq \refl(a)$ and $\inv{p} \comp p \rweq \refl(b)$.
    \item An inversion on 2-cells: $\eta : p \rweq q$ implies $\inv{\eta} : q \rweq p$.
  \end{itemize}
\end{definition}

\begin{theorem}\label{thm:weak-2-gpd}
  Computational paths form a weak 2-groupoid, with $\mathrm{inv}_1 = \symop$
  and inversion on 2-cells given by $\RwEq.\mathrm{symm}$.
\end{theorem}

\section{The Globular Tower}
\label{sec:globular-tower}

\begin{definition}[Globular Cell]\label{def:globular-cell}
  A \emph{globular cell} over a type $\beta$ is a triple
  \[
    c = (\mathrm{src}, \mathrm{tgt}, \mathrm{path})
    \quad\text{where}\quad
    \mathrm{src}, \mathrm{tgt} : \beta \quad\text{and}\quad
    \mathrm{path} : \Path_\beta(\mathrm{src}, \mathrm{tgt}).
  \]
  Globular cells carry reflexivity, symmetry, and composition operations
  inherited from $\Path$, satisfying the analogous algebraic laws.
\end{definition}

\begin{definition}[Globular Tower]\label{def:globular-tower}
  The \emph{globular tower} over a type $A$ is defined inductively:
  \begin{align*}
    \mathrm{Level}_0(A) &\;=\; A, \\
    \mathrm{Level}_{n+1}(A) &\;=\; \mathrm{GlobularCell}(\mathrm{Level}_n(A)).
  \end{align*}
  Each level carries $\refl$, $\symop$, and $\mathrm{trans}$ operations,
  as well as a functorial $\mathrm{map}$ operation that sends a function
  $f : A \to B$ to level-wise maps $\mathrm{Level}_n(f) :
  \mathrm{Level}_n(A) \to \mathrm{Level}_n(B)$.
\end{definition}

\begin{proposition}\label{prop:globular-tower-functorial}
  The $\mathrm{map}$ operation on globular levels satisfies:
  \begin{enumerate}[label=(\roman*)]
    \item $\mathrm{map}(\refl(x)) = \refl(\mathrm{map}(x))$.
    \item $\mathrm{map}(\symop(c)) = \symop(\mathrm{map}(c))$.
    \item $\mathrm{map}(\mathrm{trans}(p, q, h)) =
      \mathrm{trans}(\mathrm{map}(p), \mathrm{map}(q), f_*(h))$.
  \end{enumerate}
\end{proposition}

The globular tower provides the \emph{geometric} scaffolding for the
$\omega$-groupoid, but it does not encode the rewrite structure. For that,
we need the derivation cells.

\section{Derivation Cells and the Weak $\omega$-Groupoid}
\label{sec:omega-groupoid}

\subsection{Dimension 2: Derivations Between Paths}

\begin{definition}[Derivation$_2$]\label{def:derivation2}
  A \emph{derivation} (or \emph{type-valued two-cell}) between paths
  $p, q : \Path_A(a,b)$ is an element of the inductive type
  $\Derivation_2(p, q)$ with constructors:
  \begin{enumerate}[label=(\roman*)]
    \item $\Derivation_2.\refl(p) : \Derivation_2(p, p)$.
    \item $\Derivation_2.\mathrm{step}(s) : \Derivation_2(p, q)$ for
      $s : p \rew q$.
    \item $\Derivation_2.\mathrm{inv}(\delta) : \Derivation_2(q, p)$ for
      $\delta : \Derivation_2(p, q)$.
    \item $\Derivation_2.\mathrm{vcomp}(\delta_1, \delta_2) :
      \Derivation_2(p, r)$ for $\delta_1 : \Derivation_2(p, q)$ and
      $\delta_2 : \Derivation_2(q, r)$.
  \end{enumerate}
\end{definition}

Unlike $\RwEq$ (which is $\Prop$-valued), $\Derivation_2$ is
\emph{type-valued}: it carries explicit derivation structure, recording
which steps and closure operations were applied.

\begin{proposition}\label{prop:d2-projects}
  Every $\Derivation_2(p,q)$ projects to an $\RwEq(p,q)$ witness via a
  forgetful map $\Derivation_2(p,q) \to \RwEq(p,q)$. Moreover,
  $\Derivation_2(p,q)$ is inhabited if and only if $p \rweq q$.
\end{proposition}

$\Derivation_2$ supports horizontal operations:

\begin{definition}[Horizontal Operations on $\Derivation_2$]\label{def:d2-horizontal}
  \begin{enumerate}[label=(\roman*)]
    \item \textbf{Left whiskering.}\;
      $f \triangleright_L \delta : \Derivation_2(f \comp g, f \comp h)$
      for $\delta : \Derivation_2(g, h)$.
    \item \textbf{Right whiskering.}\;
      $\delta \triangleleft_R h : \Derivation_2(f \comp h, g \comp h)$
      for $\delta : \Derivation_2(f, g)$.
    \item \textbf{Horizontal composition.}\;
      $\delta_1 \circ_h \delta_2 : \Derivation_2(p \comp q, p' \comp q')$
      for $\delta_1 : \Derivation_2(p, p')$ and $\delta_2 : \Derivation_2(q, q')$.
  \end{enumerate}
\end{definition}

\subsection{Dimension 3: Meta-Steps and Derivations Between Derivations}

\begin{definition}[MetaStep$_3$]\label{def:metastep3}
  A \emph{primitive three-cell} $\mathrm{MetaStep}_3(\delta_1, \delta_2)$
  between derivations $\delta_1, \delta_2 : \Derivation_2(p, q)$ witnesses
  that $\delta_1$ and $\delta_2$ are ``equivalent as derivations.'' The
  constructors include:
  \begin{enumerate}[label=(\roman*)]
    \item \textbf{Groupoid laws for derivations:}
      \begin{align*}
        &\mathrm{vcomp}(\refl(p), \delta) \;\mapsto\; \delta, \\
        &\mathrm{vcomp}(\delta, \refl(q)) \;\mapsto\; \delta, \\
        &\mathrm{vcomp}(\mathrm{vcomp}(\delta_1, \delta_2), \delta_3)
          \;\mapsto\; \mathrm{vcomp}(\delta_1,
          \mathrm{vcomp}(\delta_2, \delta_3)), \\
        &\mathrm{inv}(\mathrm{inv}(\delta)) \;\mapsto\; \delta, \\
        &\mathrm{vcomp}(\mathrm{inv}(\delta), \delta) \;\mapsto\; \refl(q), \\
        &\mathrm{vcomp}(\delta, \mathrm{inv}(\delta)) \;\mapsto\; \refl(p), \\
        &\mathrm{inv}(\mathrm{vcomp}(\delta_1, \delta_2)) \;\mapsto\;
          \mathrm{vcomp}(\mathrm{inv}(\delta_2), \mathrm{inv}(\delta_1)).
      \end{align*}
    \item \textbf{Step coherence:} any two single-step derivations
      $\mathrm{step}(s_1)$ and $\mathrm{step}(s_2)$ for the same
      $p \rew q$ are connected.
    \item \textbf{$\RwEq$-coherence:} any two derivations projecting to
      the same $\RwEq$ witness are connected:
      $\delta_1.\mathrm{toRwEq} = \delta_2.\mathrm{toRwEq} \implies
      \mathrm{MetaStep}_3(\delta_1, \delta_2)$.
    \item \textbf{Bicategorical coherences:} pentagon and triangle for
      derivation-level associators.
    \item \textbf{Whiskering:} whiskering preserves meta-steps.
  \end{enumerate}
\end{definition}

\begin{definition}[Derivation$_3$]\label{def:derivation3}
  $\Derivation_3(\delta_1, \delta_2)$ is the free groupoid generated by
  $\mathrm{MetaStep}_3$: it has constructors $\refl$, $\mathrm{step}$
  (from $\mathrm{MetaStep}_3$), $\mathrm{inv}$, and $\mathrm{vcomp}$.
\end{definition}

\subsection{The Contractibility Theorem}

\begin{theorem}[Contractibility at Dimension $\geq 3$]\label{thm:contract3}
  For any two parallel derivations $\delta_1, \delta_2 : \Derivation_2(p,q)$,
  there exists a three-cell connecting them:
  \[
    \mathrm{contract}_3(\delta_1, \delta_2) \;:\; \Derivation_3(\delta_1, \delta_2).
  \]
\end{theorem}

\begin{proof}
  Since $\RwEq$ is $\Prop$-valued, the projections
  $\delta_1.\mathrm{toRwEq}$ and $\delta_2.\mathrm{toRwEq}$ are equal
  by $\mathsf{Subsingleton.elim}$. The $\RwEq$-coherence constructor of
  $\mathrm{MetaStep}_3$ then produces a primitive three-cell, which lifts
  to $\Derivation_3$ via the $\mathrm{step}$ constructor.
\end{proof}

\begin{corollary}[Loop Contraction]\label{cor:loop-contract}
  Every loop derivation $\delta : \Derivation_2(p, p)$ contracts to the
  identity:
  \[
    \Derivation_3(\delta,\; \Derivation_2.\refl(p)).
  \]
\end{corollary}

\begin{remark}[Critical Design Choice]\label{rem:contract-threshold}
  Contractibility starts at dimension~3, \textbf{not} at dimension~2. At
  dimension~2, $\Derivation_2(p,q)$ is inhabited only when $p \rweq q$;
  it does not connect arbitrary parallel paths. This is essential for
  capturing non-trivial fundamental groups. For example, on the circle
  $S^1$, the loop generator and $\refl$ are \emph{not} connected by a
  two-cell, which is what allows $\pi_1(S^1) \cong \ZZ$.
\end{remark}

\subsection{Dimensions 4 and Beyond}

The pattern continues uniformly:

\begin{definition}[Higher Derivation Cells]\label{def:higher-cells}
  \begin{enumerate}[label=(\roman*)]
    \item $\mathrm{MetaStep}_4$ and $\Derivation_4(\mu_1, \mu_2)$
      for $\mu_1, \mu_2 : \Derivation_3(\delta_1, \delta_2)$, with
      analogous groupoid laws, step coherence, and whiskering.
    \item For $n \geq 5$, $\mathrm{DerivationHigh}_n(c_1, c_2)$
      for $c_1, c_2 : \Derivation_4(\mu_1, \mu_2)$, parametrized by
      dimension.
  \end{enumerate}
\end{definition}

\begin{theorem}[Contractibility at Dimension $\geq 4$]\label{thm:contract4}
  For any two parallel three-cells $\mu_1, \mu_2 : \Derivation_3(\delta_1,
  \delta_2)$:
  \[
    \mathrm{contract}_4(\mu_1, \mu_2) \;:\; \Derivation_4(\mu_1, \mu_2).
  \]
  More generally, for all $k \geq 3$, any two parallel $(k-1)$-cells are
  connected by a $k$-cell.
\end{theorem}

\begin{proof}
  The same argument as \cref{thm:contract3}: the projection of each
  higher cell to its $\Prop$-valued counterpart is unique by proof
  irrelevance, and the coherence constructor lifts this to an explicit
  higher cell.
\end{proof}

\subsection{The Cell Types}

\begin{definition}[Cell Type at Each Dimension]\label{def:cell-types}
  \[
    \mathrm{Cell}_k(A) \;=\;
    \begin{cases}
      A & k = 0, \\
      \Sigma_{a,b : A}\; \Path_A(a,b) & k = 1, \\
      \Sigma_{a,b,p,q}\; \Derivation_2(p,q) & k = 2, \\
      \Sigma_{\ldots}\; \Derivation_3(\delta_1, \delta_2) & k = 3, \\
      \Sigma_{\ldots}\; \Derivation_4(\mu_1, \mu_2) & k = 4, \\
      \Sigma_{\ldots}\; \mathrm{DerivationHigh}_{k-5}(c_1, c_2) & k \geq 5.
    \end{cases}
  \]
\end{definition}

\subsection{The Main Structure Theorem}

\begin{definition}[Weak $\omega$-Groupoid]\label{def:omega-gpd}
  A \emph{weak $\omega$-groupoid} (in the sense of Batanin--Leinster
  \cite{Leinster04}) on a type $A$ consists of:
  \begin{itemize}
    \item Cells at every dimension: $\mathrm{Cell}_k(A)$ for $k \in \Nat$.
    \item Groupoid operations (identity, composition, inversion) at each
      dimension.
    \item Contractibility: for $k \geq 3$, any two parallel $(k-1)$-cells
      are connected by a $k$-cell.
    \item Bicategorical coherence: pentagon and triangle at the
      2-cell level, witnessed as 3-cells.
  \end{itemize}
\end{definition}

\begin{theorem}[Main Structure Theorem]\label{thm:omega-groupoid}
  For any type $A$, the tower
  \[
    A, \quad \Path, \quad \Derivation_2, \quad \Derivation_3, \quad
    \Derivation_4, \quad \ldots
  \]
  forms a weak $\omega$-groupoid with contractibility starting at
  dimension~3.
\end{theorem}

\begin{proof}
  The construction assembles:
  \begin{itemize}
    \item $\mathrm{contract}_3$ (\cref{thm:contract3}) for the
      contractibility at dimension~3.
    \item $\mathrm{contract}_4$ (\cref{thm:contract4}) for dimension~4.
    \item The parametrized $\mathrm{contractHigh}_n$ for dimensions
      $\geq 5$.
    \item The pentagon coherence (\cref{thm:pentagon}) as a 3-cell:
      $\mathrm{MetaStep}_3.\mathrm{pentagon}$ provides the pentagon
      equation between the two canonical reassociation derivations.
    \item The triangle coherence (\cref{thm:triangle}) as a 3-cell:
      $\mathrm{MetaStep}_3.\mathrm{triangle}$ provides the triangle
      equation.
  \end{itemize}
  Groupoid operations at each dimension are given by the constructors
  of the derivation types ($\refl$, $\mathrm{vcomp}$, $\mathrm{inv}$).
\end{proof}

\section{The Infinity-Groupoid Approximation}
\label{sec:infinity-gpd}

\begin{definition}[Coherence at Level $n$]\label{def:coherence-level}
  \[
    \mathrm{CoherenceAt}(A, n) \;=\;
    \begin{cases}
      \top & n \leq 1, \\
      \forall\, \delta_1, \delta_2 : \Derivation_2(p,q).\;
        \Derivation_3(\delta_1, \delta_2) & n = 2, \\
      \forall\, \mu_1, \mu_2 : \Derivation_3(\delta_1,\delta_2).\;
        \Derivation_4(\mu_1, \mu_2) & n = 3, \\
      \forall\, c_1, c_2.\; \mathrm{DerivationHigh}_{n-4}(c_1, c_2) & n \geq 4.
    \end{cases}
  \]
\end{definition}

\begin{theorem}\label{thm:infinity-gpd}
  For any type $A$, the canonical coherence witnesses at every level
  assemble into an $\infty$-groupoid structure:
  \[
    \mathrm{coherenceAt}(A, n) : \mathrm{CoherenceAt}(A, n)
    \quad\text{for all}\; n \in \Nat.
  \]
\end{theorem}

\begin{definition}[$n$-Groupoid Truncation]\label{def:n-truncation}
  The \emph{$n$-truncation} of the $\omega$-groupoid collapses all cells
  above dimension $n$ to the trivial type $\mathbf{1}$:
  \[
    \mathrm{TruncCell}_k(A, n) \;=\;
    \begin{cases}
      \mathrm{Cell}_k(A) & k \leq n, \\
      \mathbf{1} & k > n.
    \end{cases}
  \]
\end{definition}

\begin{theorem}\label{thm:1-truncation}
  The 1-truncation of the $\omega$-groupoid recovers the strict groupoid
  $\PathQuot$ of \cref{thm:quot-strict-gpd}.
\end{theorem}

\section{Double Groupoid and Symmetric Monoidal Structure}
\label{sec:enriched-structures}

The proof-irrelevance of two-cells enables additional algebraic structures
on the computational path space.

\subsection{Double Groupoid}

\begin{definition}[Double Groupoid]\label{def:double-gpd}
  A \emph{double groupoid} is a weak 2-groupoid equipped with an explicit
  interchange law: the two ways of composing a $2 \times 2$ grid of
  two-cells (vertical-then-horizontal vs.\ horizontal-then-vertical)
  are equal.
\end{definition}

\begin{theorem}\label{thm:double-gpd}
  Computational paths form a double groupoid. The interchange law holds
  by proof irrelevance of $\RwEq$.
\end{theorem}

\subsection{Groupoid-Enriched Category}

\begin{definition}[Groupoid-Enriched Category]\label{def:gpd-enriched}
  A \emph{groupoid-enriched category} is a weak bicategory whose
  hom-categories (the categories of 2-cells between fixed 1-cells) are
  groupoids---i.e., all 2-cells are invertible, and the groupoid axioms
  (associativity, units, inverses for vertical composition) hold as
  equalities.
\end{definition}

\begin{theorem}\label{thm:gpd-enriched}
  Computational paths form a groupoid-enriched category. All groupoid
  axioms for 2-cells hold by $\mathsf{Subsingleton.elim}$ on $\RwEq$.
\end{theorem}

\subsection{Symmetric Monoidal Structure}

\begin{definition}[Monoidal Path Algebra]\label{def:monoidal}
  Viewing path composition as a tensor product ($\otimes = \mathrm{trans}$)
  and the reflexive path as the unit ($I = \refl$), the computational path
  space carries a \emph{monoidal structure} with:
  \begin{itemize}
    \item Associator: $\alpha_{p,q,r} : (p \comp q) \comp r \rweq
      p \comp (q \comp r)$.
    \item Left unitor: $\lambda_p : \refl \comp p \rweq p$.
    \item Right unitor: $\rho_p : p \comp \refl \rweq p$.
    \item Pentagon and triangle coherences.
  \end{itemize}
\end{definition}

\begin{definition}[Braiding]\label{def:braiding}
  The \emph{braiding} is provided by the anti-homomorphism of symmetry
  (\cref{thm:strict-antihom}):
  \[
    \beta_{p,q} \;:\; \inv{(p \comp q)} \;\rweq\; \inv{q} \comp \inv{p}.
  \]
  This is a two-cell (rewrite equality) provided by rule~\ref{rule:stss}.
\end{definition}

\begin{theorem}[Symmetric Monoidal Path Algebra]\label{thm:sym-monoidal}
  The computational path space, equipped with the monoidal structure
  and braiding defined above, forms a \emph{symmetric monoidal}
  structure. The braiding satisfies:
  \begin{enumerate}[label=(\roman*)]
    \item $\inv{(\inv{q} \comp \inv{p})} \rweq p \comp q$ \;(inverse braiding).
    \item The hexagon identities (left and right) hold as equalities of
      $\Prop$-valued two-cells.
  \end{enumerate}
  Naturality of the braiding with respect to $\RwEq$ follows from the
  congruence property.
\end{theorem}

\medskip

This concludes Part~I of the paper. In Part~II, we develop the homotopy theory
built on these foundations: fundamental groups, covering spaces, fibrations,
exact sequences, and the computation of $\pi_1$ for standard spaces.


% ============================================================================
%  PART II: HOMOTOPY THEORY (Chapters 6--10)
% ============================================================================
\part{Homotopy Theory}
\label{part:homotopy}

%!TEX root = ../main.tex
% Part II: Homotopy Theory (Chapters 6--10)
% The Algebra of Computational Paths
% Authors: Arthur Ferreira Ramos, Ruy J.G.B. de Queiroz, Anjolina G. de Oliveira

% ==========================================================================
%  CHAPTER 6 — Fundamental Groups and Loop Spaces
% ==========================================================================

\chapter{Fundamental Groups and Loop Spaces}\label{ch:fundamental-groups}

We now turn from the algebraic and rewriting-theoretic foundations of
computational paths to their homotopy-theoretic consequences. The
central objects of study are the \emph{loop space} and the
\emph{fundamental group}, which encode the 1-dimensional topology of
a type through the algebraic structure of its self-paths.

\section{Loop Spaces}\label{sec:loop-spaces}

\begin{definition}[Loop space]\label{def:loop-space}
Let $A$ be a type and $a : A$ a base point. The \emph{loop space} of
$A$ at $a$ is
\[
  \Omega(A, a) \;=\; \Path_A(a, a),
\]
the type of computational paths from $a$ to itself. Elements of
$\Omega(A, a)$ are called \emph{loops} at $a$.
\end{definition}

The loop space inherits three canonical operations from the path algebra:
\begin{itemize}
\item \textbf{Identity.} $\id = \refl(a)$, the reflexive loop with
  empty trace.
\item \textbf{Composition.} $p \cdot q = \trans(p, q)$, concatenation
  of the underlying traces.
\item \textbf{Inversion.} $p^{-1} = \symm(p)$, reversal of the trace.
\end{itemize}

By the monoid and involution laws established in Chapter~2, these operations
satisfy:

\begin{proposition}\label{prop:loop-space-laws}
For all $p, q, r \in \Omega(A, a)$\textup{:}
\begin{enumerate}
\item[\textup{(i)}] $\id \cdot p = p$ and $p \cdot \id = p$,
\item[\textup{(ii)}] $(p \cdot q) \cdot r = p \cdot (q \cdot r)$,
\item[\textup{(iii)}] $p^{-1} \cdot p = \id$ and $p \cdot p^{-1} = \id$,
\item[\textup{(iv)}] $(p^{-1})^{-1} = p$,
\item[\textup{(v)}] $(p \cdot q)^{-1} = q^{-1} \cdot p^{-1}$.
\end{enumerate}
All equalities hold as strict structural equalities of \textup{\texttt{Path}} records.
\end{proposition}

\begin{remark}
Proposition~\ref{prop:loop-space-laws} means that $\Omega(A, a)$ is a
\emph{strict} group \emph{before} quotienting---a distinctive feature of
the computational paths framework. In homotopy type theory, the loop
space is only a group up to higher paths; here, the group laws are
definitional by virtue of the list-based trace representation.
\end{remark}

\section{The Fundamental Group}\label{sec:fundamental-group}

Although $\Omega(A, a)$ is already a strict group, two loops with
different rewrite traces but the same underlying propositional equality
represent ``the same homotopy class.'' The fundamental group identifies
such loops.

\begin{definition}[Loop quotient]\label{def:loop-quotient}
The \emph{loop quotient} at $a$ is
\[
  \LoopQuot(A, a) \;=\; \PathRwQuot_A(a, a) \;=\;
  \Omega(A, a)\, /\, {\approx},
\]
where $\approx$ denotes rewrite equality (\texttt{RwEq}).
\end{definition}

\begin{definition}[Fundamental group]\label{def:fundamental-group}
The \emph{fundamental group} of $A$ at $a$ is
\[
  \pi_1(A, a) \;=\; \LoopQuot(A, a),
\]
equipped with multiplication $[p] \cdot [q] = [p \cdot q]$, identity
$e = [\refl(a)]$, and inversion $[p]^{-1} = [p^{-1}]$.
\end{definition}

\begin{theorem}[Group axioms]\label{thm:pi1-group}
$\pi_1(A, a)$ is a group: associativity, left and right identity, and
left and right inverse all hold as strict equalities on the quotient.
\end{theorem}

\begin{proof}
Each axiom is inherited from the corresponding strict identity on
$\Omega(A, a)$ (Proposition~\ref{prop:loop-space-laws}), which
descends to the quotient since the operations are well-defined with
respect to $\approx$ (Theorem~3.6).
\end{proof}

The group $\pi_1(A,a)$ can equivalently be described via two
intermediate structures that make the algebraic packaging explicit:

\begin{definition}[Loop monoid and loop group]\label{def:loop-monoid-group}
\hfill
\begin{itemize}
\item The \emph{loop monoid} $(\pi_1(A,a), \cdot, e)$ records
  multiplication, identity, associativity, and unit laws.
\item The \emph{loop group} extends the loop monoid with an inversion
  operation and witnesses of the inverse laws.
\end{itemize}
\end{definition}

\noindent
Both structures are \emph{canonical}: they are constructed uniformly for any
type $A$ and base point $a$, with no choices required.

\subsection*{Power operations.}
For applications to winding numbers (Chapter~\ref{ch:spaces}), we
equip $\pi_1(A,a)$ with natural and integer powers:

\begin{definition}\label{def:loop-powers}
For $x \in \pi_1(A, a)$, define:
\[
  x^0 = e, \qquad x^{n+1} = x^n \cdot x, \qquad
  x^{-n} = (x^n)^{-1} \;\text{ for } n > 0.
\]
\end{definition}

\begin{proposition}[Power laws]\label{prop:zpow-laws}
For all $x \in \pi_1(A, a)$ and $m, n \in \mathbb{Z}$\textup{:}
\begin{enumerate}
\item[\textup{(i)}] $x^{m+n} = x^m \cdot x^n$,
\item[\textup{(ii)}] $x^{-n} = (x^n)^{-1}$,
\item[\textup{(iii)}] $x^m \cdot x^n = x^n \cdot x^m$.
\end{enumerate}
\end{proposition}

\begin{proof}
Part~(i) is proved by integer induction, using the successor case
$x^{n+1} = x^n \cdot x$ and the predecessor case
$x^{n-1} = x^n \cdot x^{-1}$ as the inductive step. Part~(ii) follows
from the definition of negative powers. Part~(iii) is an immediate
consequence of (i) and the commutativity of integer addition.
\end{proof}

\section{Functoriality}\label{sec:pi1-functoriality}

\begin{theorem}[Induced homomorphism]\label{thm:induced-pi1}
Let $f : A \to B$ be a function and $a : A$. Then $f$ induces a group
homomorphism
\[
  f_* : \pi_1(A, a) \longrightarrow \pi_1(B, f(a))
\]
defined on representatives by $f_*([p]) = [\congrArg(f, p)]$.
\end{theorem}

\begin{proof}
The map $p \mapsto \congrArg(f, p)$ preserves $\approx$ since
$\congrArg$ is compatible with the rewrite system
(Theorem~3.6). Functoriality ($\congrArg(f, p \cdot q) =
\congrArg(f, p) \cdot \congrArg(f, q)$) follows from Theorem~2.8.
\end{proof}

\begin{theorem}[Functorial laws]\label{thm:pi1-functor-laws}
\hfill
\begin{enumerate}
\item[\textup{(i)}] $(\id_A)_* = \id_{\pi_1(A,a)}$.
\item[\textup{(ii)}] $(g \circ f)_* = g_* \circ f_*$ for
  $f : A \to B$ and $g : B \to C$.
\end{enumerate}
\end{theorem}

\begin{proof}
Part~(i) follows from $\congrArg(\id, p) \approx p$.
Part~(ii) follows from $\congrArg(g \circ f, p) \approx
\congrArg(g, \congrArg(f, p))$. Both are instances of rewrite rules
in the rewrite system.
\end{proof}

\begin{theorem}[Product formula]\label{thm:pi1-product}
For pointed types $(A, a)$ and $(B, b)$, there is a group isomorphism
\[
  \pi_1(A \times B,\, (a, b)) \;\cong\; \pi_1(A, a) \times \pi_1(B, b).
\]
\end{theorem}

\begin{proof}
The encoding map sends a loop $p$ in $A \times B$ to the pair
$(\fst(p),\, \snd(p))$ of its component projections. The decoding map
sends a pair $(p, q)$ to $\prodMk(p, q)$. Both maps respect $\approx$
by the congruence properties of projections and pairing. The round-trip
identities $\encode \circ \decode = \id$ and $\decode \circ \encode = \id$
follow from the product $\beta$- and $\eta$-rules
(Theorem~2.13).
\end{proof}

\section{The Fundamental Groupoid}\label{sec:fundamental-groupoid}

The fundamental group captures loops at a single base point. To handle
all points simultaneously, we pass to the fundamental groupoid.

\begin{definition}[Fundamental groupoid]\label{def:fundamental-groupoid}
The \emph{fundamental groupoid} $\Pi_1(A)$ has:
\begin{itemize}
\item \textbf{Objects}: points $a \in A$.
\item \textbf{Morphisms}: $\Hom_{\Pi_1(A)}(a, b) = \PathRwQuot_A(a, b)$.
\item \textbf{Composition}: path concatenation on the quotient.
\item \textbf{Identity}: $\id_a = [\refl(a)]$.
\item \textbf{Inverse}: $[p]^{-1} = [p^{-1}]$.
\end{itemize}
\end{definition}

\begin{theorem}\label{thm:fundamental-groupoid-strict}
$\Pi_1(A)$ is a strict groupoid: all axioms \textup{(}associativity,
identity, inverse\textup{)} hold as definitional equalities on
$\PathRwQuot$.
\end{theorem}

\begin{proof}
This is the strict groupoid $\StrictGroupoid{.}\mathrm{quotient}(A)$ from
Theorem~4.5, restricted to the morphism level.
\end{proof}

\begin{theorem}[Basepoint independence]\label{thm:basepoint-independence}
Let $\gamma : \PathRwQuot_A(a, b)$ be a path class. Then conjugation
by $\gamma$,
\[
  \varphi_\gamma : \pi_1(A, a) \xrightarrow{\;\sim\;} \pi_1(A, b),
  \qquad \varphi_\gamma(\alpha) = \gamma^{-1} \cdot \alpha \cdot \gamma,
\]
is a group isomorphism, with inverse $\varphi_{\gamma^{-1}}$.
\end{theorem}

\begin{proof}
That $\varphi_\gamma$ is a homomorphism (preserving identity,
composition, and inversion) follows by direct computation using
the strict groupoid laws. The cancellation identities
$\varphi_{\gamma^{-1}} \circ \varphi_\gamma = \id$ and
$\varphi_\gamma \circ \varphi_{\gamma^{-1}} = \id$ are verified by
expanding the definition and applying the inverse and unit laws.
\end{proof}

\begin{theorem}[Functoriality of the groupoid]\label{thm:pi1-groupoid-functor}
A function $f : A \to B$ induces a groupoid functor
$\Pi_1(f) : \Pi_1(A) \to \Pi_1(B)$ defined by
$\Pi_1(f)(a) = f(a)$ on objects and
$\Pi_1(f)([p]) = [\congrArg(f, p)]$ on morphisms. This functor
preserves identity and composition strictly.
\end{theorem}

\section{Higher Homotopy Groups and the Eckmann--Hilton Argument}%
\label{sec:higher-homotopy}

\begin{definition}[Iterated loop spaces]\label{def:iterated-loops}
The \emph{iterated loop spaces} are defined by:
\begin{gather*}
  \Omega^0(A, a) = A, \qquad
  \Omega^1(A, a) = \Path_A(a, a), \\
  \Omega^2(A, a) = \Deriv_2(\refl(a), \refl(a)), \qquad
  \Omega^3(A, a) = \Deriv_3(\refl_2, \refl_2),
\end{gather*}
where $\Deriv_k$ denotes $k$-cells in the weak $\omega$-groupoid tower.
\end{definition}

\begin{definition}[Higher homotopy groups]\label{def:higher-homotopy}
The \emph{$n$-th homotopy group} is defined by:
\begin{gather*}
  \pi_1(A, a) = \Omega^1(A, a) / {\approx}, \\
  \pi_2(A, a) = \Omega^2(A, a) / {\sim_3},
\end{gather*}
where $\alpha \sim_3 \beta$ iff there exists a 3-cell
$\Deriv_3(\alpha, \beta)$. For $n \ge 3$, the contractibility of the
tower at dimension $\ge 3$ (Theorem~5.7) implies that $\pi_n$
collapses.
\end{definition}

The double loop space $\Omega^2(A,a)$ carries two distinct composition
operations: \emph{vertical} composition (sequential concatenation of
derivations) and \emph{horizontal} composition (induced by whiskering
and path concatenation).

\begin{definition}[Whiskering]\label{def:whiskering}
Let $f : \Path_A(a,b)$ and $\alpha : p \approx q$ for
$p, q : \Path_A(b,c)$.
\begin{itemize}
\item \emph{Left whiskering}: $f \triangleright \alpha$ is the
  witness that $f \cdot p \approx f \cdot q$.
\item \emph{Right whiskering}: $\alpha \triangleleft g$ is the
  witness that $p \cdot g \approx q \cdot g$.
\end{itemize}
\emph{Horizontal composition} is defined as
$\alpha \star \beta = (\alpha \triangleleft q') \circ_v
(p \triangleright \beta)$.
\end{definition}

\begin{theorem}[Interchange law]\label{thm:interchange-EH}
For 2-cells $\alpha_1, \alpha_2, \beta_1, \beta_2$ in
$\Omega^2(A,a)$\textup{:}
\[
  (\alpha_1 \circ_v \alpha_2) \star (\beta_1 \circ_v \beta_2)
  \;=\;
  (\alpha_1 \star \beta_1) \circ_v (\alpha_2 \star \beta_2).
\]
At the level of derivations, the two sides are connected by a 3-cell
$\MetaStep_3.\mathsf{interchange}$.
\end{theorem}

\begin{theorem}[Eckmann--Hilton]\label{thm:eckmann-hilton}
On $\Omega^2(A,a)$, horizontal and vertical composition coincide
and are commutative. In particular, $\pi_2(A,a)$ is an abelian group.
\end{theorem}

\begin{proof}
The proof proceeds in three steps.

\textbf{Step 1} (Whiskering by $\refl$ is trivial). Since
$\trans(\refl(a), \refl(a))$ reduces definitionally to $\refl(a)$,
both $\alpha \triangleleft \refl$ and $\refl \triangleright \beta$
are connected to $\alpha$ and $\beta$ respectively by 3-cells.

\textbf{Step 2} ($\star = \circ_v$). From Step~1:
$\alpha \star \beta
  = (\alpha \triangleleft \refl) \circ_v (\refl \triangleright \beta)
  \sim_3 \alpha \circ_v \beta$.

\textbf{Step 3} (Commutativity). By the interchange law
(Theorem~\ref{thm:interchange}), $\alpha \star \beta \sim_3
\beta \star \alpha$ (the alternative horizontal composition
reverses the order). Combining with Step~2:
$\alpha \circ_v \beta \sim_3 \beta \circ_v \alpha$. \qedhere
\end{proof}

\begin{corollary}\label{cor:pi-n-abelian}
For every $n \ge 2$, $\pi_n(A, a)$ is abelian.
\end{corollary}

\begin{remark}[Naturality of coherence laws]
The unit laws, associativity, and symmetry involution are all
\emph{natural} with respect to rewriting: they commute with
whiskering. Concretely, for the left unit law, the diagram
$\refl \cdot p \xrightarrow{\refl \triangleright \alpha}
\refl \cdot q$ over
$p \xrightarrow{\alpha} q$ commutes: both composites through the
naturality square (applying whiskering then the unit law, or the unit
law then $\alpha$) are equal as $\RwEq$ witnesses, since both are
proofs of the same proposition.
\end{remark}


% ==========================================================================
%  CHAPTER 7 — Spaces and Their Fundamental Groups
% ==========================================================================

\chapter{Spaces and Their Fundamental Groups}\label{ch:spaces}

With the fundamental group machinery in place, we compute $\pi_1$ for
several classical spaces. Each space is modeled within the
computational-paths framework using \emph{path expressions}---a
syntactic calculus of formal generators and relations that captures the
combinatorial essence of the space.

\section{The Computational Circle \texorpdfstring{$S^1$}{S¹}}%
\label{sec:circle}

\begin{definition}[Circle]\label{def:circle}
The \emph{computational circle} $S^1$ is a one-point type
$\{*\}$ equipped with a formal loop generator via the path expression
calculus.

More precisely, define the type of \emph{circle path expressions} by
the grammar:
\[
  e \;::=\; \lp \;\mid\; \refl(a) \;\mid\; e^{-1} \;\mid\;
  e_1 \cdot e_2,
\]
where $\lp$ is a distinguished generator with source and target both
equal to $* : S^1$.
\end{definition}

The circle path expressions carry a natural notion of \emph{winding
number}:

\begin{definition}[Winding number]\label{def:winding-number}
The winding number $\wind : \text{CircleExpr}(*,*) \to \mathbb{Z}$ is
defined recursively:
\[
  \wind(\lp) = 1, \quad \wind(\refl) = 0, \quad
  \wind(e^{-1}) = -\wind(e), \quad
  \wind(e_1 \cdot e_2) = \wind(e_1) + \wind(e_2).
\]
\end{definition}

\begin{proposition}\label{prop:winding-power}
For all $n \in \mathbb{N}$,\, $\wind(\lp^n) = n$. For all
$z \in \mathbb{Z}$,\, $\wind(\lp^z) = z$.
\end{proposition}

\begin{definition}[Circle loop quotient]\label{def:circle-pi1}
Two loop expressions are identified if they have the same winding
number:
\[
  \pi_1^{\text{expr}}(S^1, *) \;=\;
  \text{CircleExpr}(*,*) \,/\, (\wind(e_1) = \wind(e_2)).
\]
\end{definition}

\begin{theorem}\label{thm:pi1-circle}
$\pi_1(S^1, *) \cong \mathbb{Z}$.
\end{theorem}

\begin{proof}
The encoding map $\encode : \pi_1^{\text{expr}}(S^1,*) \to \mathbb{Z}$
sends $[e] \mapsto \wind(e)$. The decoding map
$\decode : \mathbb{Z} \to \pi_1^{\text{expr}}(S^1,*)$ sends
$z \mapsto [\lp^z]$. These are inverse:
\begin{itemize}
\item $\encode \circ \decode$ is the identity on $\mathbb{Z}$ by
  Proposition~\ref{prop:winding-power}.
\item $\decode \circ \encode$ is the identity on the quotient: for any
  expression $e$, the canonical power $\lp^{\wind(e)}$ has the same
  winding number as $e$, hence they are identified.  \qedhere
\end{itemize}
\end{proof}

\section{The Torus \texorpdfstring{$T^2$}{T²}}\label{sec:torus}

\begin{definition}[Torus]\label{def:torus}
The \emph{torus} is defined as the product of two circles:
\[
  T^2 \;=\; S^1 \times S^1.
\]
The base point is $(*_1, *_2)$, and the two fundamental loops are:
\begin{align*}
  \alpha &= \prodMk(\lp_1, \refl(*_2)), \\
  \beta  &= \prodMk(\refl(*_1), \lp_2).
\end{align*}
\end{definition}

\begin{theorem}\label{thm:pi1-torus}
$\pi_1(T^2, (*,*)) \cong \mathbb{Z} \times \mathbb{Z}$.
\end{theorem}

\begin{proof}
By the product formula (Theorem~\ref{thm:pi1-product}):
$\pi_1(S^1 \times S^1) \cong \pi_1(S^1) \times \pi_1(S^1) \cong
\mathbb{Z} \times \mathbb{Z}$.
\end{proof}

\begin{remark}
The torus has an abelian fundamental group: $\alpha$ and $\beta$
commute. This commutativity is automatic from the product
construction---no additional relation needs to be imposed.
\end{remark}

\section{The Figure-Eight and Free Products}\label{sec:figure-eight}

\begin{definition}[Wedge sum]\label{def:wedge}
Let $(A, a_0)$ and $(B, b_0)$ be pointed types. The \emph{wedge sum}
$A \vee B$ is the pushout:
\[
\begin{tikzcd}
  \{*\} \ar[r, "a_0"] \ar[d, "b_0"'] & A \ar[d, "\inl"] \\
  B \ar[r, "\inr"'] & A \vee B
\end{tikzcd}
\]
with a glue path $\glue : \inl(a_0) = \inr(b_0)$.
\end{definition}

\begin{definition}[Figure-eight]\label{def:figure-eight}
The \emph{figure-eight} is $S^1 \vee S^1$, the wedge sum of two
circles at their base points. Its two generating loops are:
\begin{align*}
  a &= \inl_*(\lp_1), \\
  b &= \glue \cdot \inr_*(\lp_2) \cdot \glue^{-1}.
\end{align*}
\end{definition}

\begin{definition}[Free product words]\label{def:free-product-words}
Given groups $G$ and $H$, a \emph{free product word} is a finite
alternating sequence of non-identity elements from $G$ and $H$:
\[
  w = g_1 h_1 g_2 h_2 \cdots
\]
Multiplication is concatenation followed by reduction (cancelling
adjacent same-side identities). The resulting group is the
\emph{free product} $G * H$.
\end{definition}

\begin{theorem}\label{thm:pi1-figure-eight}
$\pi_1(S^1 \vee S^1, *) \cong \mathbb{Z} * \mathbb{Z}$, the free
product of $\mathbb{Z}$ with itself.
\end{theorem}

\begin{proof}
The Seifert--van~Kampen theorem (Theorem~\ref{thm:svk} below) applied
to the wedge decomposition gives
$\pi_1(S^1 \vee S^1) \cong \pi_1(S^1) * \pi_1(S^1) \cong
\mathbb{Z} * \mathbb{Z}$. Alternatively, a direct
\emph{provenance-based} equivalence identifies loop classes with free
product words over $\pi_1(S^1)$: each loop in the wedge is uniquely
decomposed into segments lying in the left or right circle, yielding
the free product structure.
\end{proof}

\begin{remark}
The fundamental group $\mathbb{Z} * \mathbb{Z}$ is non-abelian: the
loops $a$ and $b$ generate a free group on two generators. In
particular, $a \cdot b \ne b \cdot a$ as elements of $\pi_1$.
\end{remark}

\begin{definition}[Bouquet of $n$ circles]\label{def:bouquet}
The \emph{bouquet} $\bigvee_n S^1$ is the wedge of $n$ copies of
$S^1$, all sharing a common base point.
\end{definition}

\begin{corollary}\label{cor:pi1-bouquet}
$\pi_1\!\left(\bigvee_n S^1,\, *\right) \cong F_n$, the free group on
$n$ generators.
\end{corollary}

\section{The Seifert--van Kampen Theorem}\label{sec:svk}

The Seifert--van Kampen theorem is the principal computational tool
for determining fundamental groups of spaces assembled from simpler
pieces.

\begin{definition}[Pushout]\label{def:pushout}
Given functions $f : C \to A$ and $g : C \to B$, the \emph{pushout}
$A \sqcup_C B$ is the type with constructors $\inl : A \to A \sqcup_C B$,
$\inr : B \to A \sqcup_C B$, and a path constructor
$\glue : \forall c : C,\; \inl(f(c)) = \inr(g(c))$.
\end{definition}

\begin{definition}[Amalgamated free product]\label{def:amalgamated}
Given group homomorphisms $\varphi : K \to G$ and $\psi : K \to H$,
the \emph{amalgamated free product} $G *_K H$ is the quotient of the
free product $G * H$ by the normal closure of the relations
$\varphi(k) = \psi(k)$ for all $k \in K$.
\end{definition}

\begin{theorem}[Seifert--van Kampen]\label{thm:svk}
Let $f : C \to A$ and $g : C \to B$, and let $c_0 \in C$. There is an
equivalence
\[
  \pi_1(A \sqcup_C B,\; \inl(f(c_0)))
  \;\cong\;
  \pi_1(A, f(c_0)) \;*_{\pi_1(C, c_0)}\; \pi_1(B, g(c_0)),
\]
where the amalgamation is along the induced homomorphisms
$f_* : \pi_1(C, c_0) \to \pi_1(A, f(c_0))$ and
$g_* : \pi_1(C, c_0) \to \pi_1(B, g(c_0))$.
\end{theorem}

\begin{proof}[Proof sketch]
The proof follows the encode-decode method. One constructs an encoding
of loops in the pushout as words in the amalgamated free product, and
a decoding in the opposite direction, using the glue path to translate
between the two sides. The key technical ingredients are:
(i)~the naturality of the glue path with respect to loop rewriting,
(ii)~the universal property of the pushout, and
(iii)~the bijectivity of the resulting maps, established via
the provenance decomposition of pushout paths.
\end{proof}

\begin{corollary}\label{cor:svk-wedge}
For pointed types $(A, a_0)$ and $(B, b_0)$ with trivial
$\pi_1(\{*\}) = 1$\textup{:}
\[
  \pi_1(A \vee B, *) \;\cong\; \pi_1(A, a_0) * \pi_1(B, b_0).
\]
\end{corollary}

\section{Suspensions and Spheres}\label{sec:suspensions}

\begin{definition}[Suspension]\label{def:suspension}
The \emph{suspension} of a type $X$ is
\[
  \Sigma X = \{N, S\} \cup \{\merid(x) : N = S \mid x \in X\},
\]
with north pole $N$, south pole $S$, and a meridional path
$\merid(x) : \Path(N, S)$ for each $x : X$.
\end{definition}

\begin{definition}[Suspension loop]\label{def:susp-loop}
The canonical loop at the north pole induced by a point $x_0 \in X$ is
\[
  \sigma(x_0) = \merid(x_0) \cdot \merid(x_0)^{-1} : \Omega(\Sigma X, N).
\]
\end{definition}

The Freudenthal suspension theorem provides the key bridge between
loop spaces and suspensions:

\begin{theorem}[Freudenthal suspension, preview]\label{thm:freudenthal-preview}
There exists a natural map
$\Omega(X, x_0) \to \Omega(\Sigma X, N)$ that sends each loop to its
suspension and preserves the base point.
\end{theorem}

\begin{remark}
The full Freudenthal theorem states that this map is an isomorphism
on $\pi_n$ for $n < 2 \cdot \conn(X) + 1$, where $\conn(X)$ is the
connectivity of $X$. The formalization records the map and its
basepoint behavior; the connectivity bound is stated at the structural
level.
\end{remark}


% ==========================================================================
%  CHAPTER 8 — Fibrations, Covering Spaces, and Exact Sequences
% ==========================================================================

\chapter{Fibrations, Covering Spaces, and Exact Sequences}%
\label{ch:fibrations}

This chapter develops the theory of fibrations and covering spaces in
the computational-paths framework and establishes the long exact
sequence of homotopy groups.

\section{Fibers of Maps}\label{sec:fibers}

\begin{definition}[Homotopy fiber]\label{def:fiber}
Let $f : A \to B$ and $b : B$. The \emph{fiber} of $f$ over $b$ is
\[
  \Fib(f, b) \;=\; \{a : A \mid f(a) = b\}
  \;=\; \Sigma_{a : A}\, \Path_B(f(a),\, b).
\]
\end{definition}

\begin{definition}[Type families as fibrations]\label{def:family-fibration}
A type family $P : B \to \Type$ determines a fibration via the total
space construction:
\[
  E = \Sigma_{b : B}\, P(b), \qquad \proj : E \to B, \quad
  (b, p) \mapsto b.
\]
The fiber of $\proj$ over $b$ is canonically equivalent to $P(b)$:
\[
  \Fib(\proj, b) \;\simeq\; P(b).
\]
\end{definition}

\section{Path Lifting}\label{sec:path-lifting}

The fundamental property of fibrations is that paths in the base
can be lifted to the total space.

\begin{theorem}[Path lifting]\label{thm:path-lifting}
Let $P : B \to \Type$ be a type family, $p : \Path_B(b_1, b_2)$ a
path in the base, and $x : P(b_1)$ a point in the fiber over $b_1$.
Then there exists a path
\[
  \widetilde{p}(x) \;:\; \Path_E\!\big((b_1, x),\; (b_2, \transport_P(p, x))\big)
\]
in the total space $E = \Sigma P$ that lifts $p$.
\end{theorem}

\begin{definition}[Fiber transport]\label{def:fiber-transport}
The \emph{fiber transport} along $p : \Path_B(b_1, b_2)$ is the function
\[
  \transport_P(p) : P(b_1) \longrightarrow P(b_2)
\]
given by the path transport operation.
\end{definition}

\begin{proposition}[Transport composition]\label{prop:transport-composition}
For paths $p : b_1 \leadsto b_2$ and $q : b_2 \leadsto b_3$ in $B$
and $x : P(b_1)$\textup{:}
\[
  \transport_P(p \cdot q,\, x) \;=\;
  \transport_P(q,\, \transport_P(p, x)).
\]
\end{proposition}

\section{Fiber Sequences}\label{sec:fiber-sequences}

\begin{definition}[Fiber sequence]\label{def:fiber-sequence}
A \emph{fiber sequence} consists of types $F$, $E$, $B$ together with:
\begin{itemize}
\item a projection $\proj : E \to B$,
\item base points $b_0 \in B$, $e_0 \in E$ with
  $\proj(e_0) = b_0$,
\item an equivalence $F \simeq \Fib(\proj, b_0)$.
\end{itemize}
The inclusion $\incl : F \hookrightarrow E$ sends a fiber element to
its underlying point: $\incl(f) = (\text{toFiber}(f)).\text{point}$.
\end{definition}

\begin{definition}[Exactness]\label{def:exactness}
A fiber sequence $F \xrightarrow{\incl} E \xrightarrow{\proj} B$ is
\emph{exact at $E$} if:
\begin{enumerate}
\item $\proj(\incl(f)) = b_0$ for all $f \in F$
  \textup{(}image $\subseteq$ kernel\textup{)};
\item for every $e \in E$ with $\proj(e) = b_0$, there exists
  $f \in F$ with $\incl(f) = e$
  \textup{(}kernel $\subseteq$ image\textup{)}.
\end{enumerate}
\end{definition}

\begin{theorem}[Canonical fiber sequence]\label{thm:canonical-fiber-seq}
For any type family $P : B \to \Type$, base point $b \in B$, and
$x_0 : P(b)$, there is an exact fiber sequence
\[
  P(b) \;\xrightarrow{\;\incl\;}\; \textstyle\Sigma P
  \;\xrightarrow{\;\proj\;}\; B.
\]
\end{theorem}

\section{The Connecting Map}\label{sec:connecting-map}

\begin{definition}[Connecting map]\label{def:connecting-map}
Let $P : B \to \Type$ be a type family, $b \in B$, and $x_0 : P(b)$.
The \emph{connecting map}
\[
  \partial : \Omega(B, b) \longrightarrow P(b)
\]
sends a loop $\ell$ to $\transport_P(\ell,\, x_0)$.
\end{definition}

\begin{proposition}[Properties of the connecting map]\label{prop:connecting-map}
\hfill
\begin{enumerate}
\item[\textup{(i)}] $\partial(\refl(b)) = x_0$.
\item[\textup{(ii)}] $\partial(\ell_1 \cdot \ell_2) =
  \partial_{x_1}(\ell_2)$ where $x_1 = \partial(\ell_1)$
  \textup{(}i.e., the connecting map respects composition via iterated
  transport\textup{)}.
\item[\textup{(iii)}] $\partial$ respects $\approx$: if
  $\ell_1 \approx \ell_2$ then $\partial(\ell_1) = \partial(\ell_2)$.
\end{enumerate}
\end{proposition}

\begin{proof}
Part~(i) is immediate from $\transport(\refl, x_0) = x_0$. Part~(ii)
follows from the transport composition law
(Proposition~\ref{prop:transport-composition}). Part~(iii) holds
because $\ell_1 \approx \ell_2$ implies $\ell_1.\toEq = \ell_2.\toEq$
(soundness of $\approx$), and transport depends only on the
underlying propositional equality.
\end{proof}

By Part~(iii), the connecting map descends to the quotient:

\begin{corollary}\label{cor:connecting-on-pi1}
There is a well-defined map
$\partial : \pi_1(B, b) \to P(b)$ sending $[\ell] \mapsto
\transport_P(\ell, x_0)$.
\end{corollary}

\section{Covering Spaces}\label{sec:covering-spaces}

\begin{definition}[Covering space]\label{def:covering-space}
A type family $P : A \to \Type$ is a \emph{covering space} of $A$
if every fiber $P(a)$ is a \emph{set} (0-truncated): any two paths
in $P(a)$ with the same endpoints are equal.
\end{definition}

\begin{theorem}[Unique path lifting]\label{thm:unique-lifting}
For a covering space $P$, the fiber transport
$\transport_P(p) : P(a) \to P(b)$ is injective for every path
$p : a \leadsto b$.
\end{theorem}

\begin{proof}
Given $\transport_P(p, x) = \transport_P(p, y)$, apply
$\transport_P(p^{-1})$ to both sides and use the inverse law
$\transport(p^{-1}, \transport(p, x)) = x$.
\end{proof}

\subsection*{The $\pi_1$-action on fibers.}

\begin{definition}[Monodromy action]\label{def:monodromy}
Let $P : A \to \Type$ be a type family and $a \in A$. The
\emph{monodromy action} of $\pi_1(A, a)$ on the fiber $P(a)$ is the
map
\[
  \mu : \pi_1(A, a) \times P(a) \longrightarrow P(a), \qquad
  \mu([\ell], x) = \transport_P(\ell, x).
\]
\end{definition}

\begin{proposition}\label{prop:monodromy-action}
The monodromy action is a genuine group action:
\begin{enumerate}
\item[\textup{(i)}] $\mu(e, x) = x$
  \textup{(}identity acts trivially\textup{)}.
\item[\textup{(ii)}] $\mu(\alpha, \mu(\beta, x))
  = \mu(\alpha \cdot \beta, x)$
  \textup{(}compatibility with multiplication\textup{)}.
\end{enumerate}
\end{proposition}

\begin{proof}
Part~(i) is the transport identity law. Part~(ii) is the transport
composition law descending to the quotient.
\end{proof}

\subsection*{Deck transformations.}

\begin{definition}[Deck transformation]\label{def:deck-transformation}
A \emph{deck transformation} of a covering $P$ is an automorphism
$\varphi : \Sigma P \to \Sigma P$ satisfying
$\proj \circ \varphi = \proj$, together with an inverse.
\end{definition}

\begin{proposition}\label{prop:deck-group}
Deck transformations form a group under composition, with identity
$\id$ and the obvious associativity, identity, and inverse laws.
\end{proposition}

\section{The Hopf Fibration}\label{sec:hopf}

\begin{definition}[Hopf fibration data]\label{def:hopf}
The \emph{Hopf fibration} is recorded as a structure consisting of:
\begin{itemize}
\item A projection $\proj : S^3 \to S^2$.
\item Base points $b_0 \in S^2$ and $e_0 \in S^3$ with
  $\proj(e_0) = b_0$.
\item A fiber equivalence $\Fib(\proj, b_0) \simeq S^1$.
\end{itemize}
\end{definition}

\begin{theorem}\label{thm:hopf-fiber-seq}
The Hopf data assembles into an exact fiber sequence
\[
  S^1 \;\longrightarrow\; S^3 \;\longrightarrow\; S^2.
\]
\end{theorem}

\begin{remark}
The Hopf fibration is the prototypical non-trivial fiber bundle. Its
long exact sequence (Section~\ref{sec:les}) yields the classical
relation $\pi_2(S^2) \cong \mathbb{Z}$ (via the connecting
homomorphism and the fact that $\pi_1(S^1) \cong \mathbb{Z}$ and
$\pi_1(S^3) = 0$).
\end{remark}

\section{The Long Exact Sequence of Homotopy Groups}%
\label{sec:les}

\begin{theorem}[Long exact sequence]\label{thm:les}
For a type family $P : B \to \Type$ with base point $b \in B$ and
$x_0 : P(b)$, there is a long exact sequence:
\[
  \cdots \;\to\; \pi_1(P(b), x_0)
  \;\xrightarrow{\;\incl_*\;}\;
  \pi_1\!\left(\textstyle\Sigma P,\, (b, x_0)\right)
  \;\xrightarrow{\;\proj_*\;}\;
  \pi_1(B, b)
  \;\xrightarrow{\;\partial\;}\;
  P(b).
\]
Exactness holds at each term:
\begin{enumerate}
\item[\textup{(i)}] \textup{At $\pi_1(\Sigma P)$}:
  $\proj_*(\incl_*(\alpha)) = e$ for all
  $\alpha \in \pi_1(P(b), x_0)$.
\item[\textup{(ii)}] \textup{At $\pi_1(B)$}:
  $\partial(\proj_*(\beta)) = x_0$ for all
  $\beta \in \pi_1(\Sigma P, (b, x_0))$.
\end{enumerate}
\end{theorem}

\begin{proof}
\textbf{Exactness at $\pi_1(\Sigma P)$.}
The inclusion $\incl$ embeds a fiber loop as a loop in $\Sigma P$
that is constant in the base component. Applying $\proj_*$ to such a
loop produces $\congrArg(\proj, \congrArg(\iota, \ell))$ where
$\iota : P(b) \hookrightarrow \Sigma P$ is $x \mapsto (b, x)$. Since
$\proj \circ \iota$ is the constant function $b$, this reduces to
$\refl(b)$ up to $\approx$.

\textbf{Exactness at $\pi_1(B)$.}
For a loop $\ell$ in $\Sigma P$ at $(b, x_0)$, the connecting map
applied to $\proj_*(\ell)$ computes $\transport(\proj_*(\ell), x_0)$.
By the sigma-type path characterization, the base path of $\ell$ is
$\proj_*(\ell)$ and the fiber path witnesses
$\transport(\proj_*(\ell), x_0) = x_0$.
\end{proof}

\begin{remark}
The induced maps $\incl_*$, $\proj_*$, and $\partial$ are all
\emph{natural} in morphisms of fibrations: a map of fiber sequences
induces a commutative ladder of long exact sequences. The naturality
is formalized via the functoriality of $\pi_1$ (Theorem~\ref{thm:pi1-functor-laws})
and the compatibility of transport with maps.
\end{remark}

\begin{theorem}[Simply connected base]\label{thm:sc-connecting}
If the base $B$ is simply connected \textup{(}i.e.,
$\pi_1(B, b) = \{e\}$\textup{)}, then the connecting map is trivial:
$\partial(\ell) = x_0$ for all loops $\ell$ in $B$.
\end{theorem}

\begin{proof}
Every loop $\ell$ is $\approx$-equivalent to $\refl(b)$, and
$\partial(\refl(b)) = x_0$.
\end{proof}


% ==========================================================================
%  CHAPTER 9 — The Hurewicz Theorem and Homological Algebra
% ==========================================================================

\chapter{The Hurewicz Theorem and Homological Algebra}\label{ch:hurewicz}

This chapter connects the homotopy-theoretic invariant $\pi_1$ to the
homological invariant $H_1$ via the Hurewicz theorem, and develops
the algebraic machinery of abelianization.

\section{Abelianization}\label{sec:abelianization}

\begin{definition}[Commutator]\label{def:commutator}
For elements $a, b$ in a group $(G, \cdot, {}^{-1}, e)$, the
\emph{commutator} is
\[
  [a, b] = a \cdot b \cdot a^{-1} \cdot b^{-1}.
\]
\end{definition}

\begin{definition}[Abelianization]\label{def:abelianization}
The \emph{abelianization} of $G$ is the quotient
\[
  G^{\ab} = G \,/\, \langle [a, b] = e \mid a, b \in G \rangle.
\]
Concretely, $G^{\ab}$ is $G$ modulo the smallest congruence relation
that includes:
\begin{itemize}
\item commutativity: $a \cdot b \sim b \cdot a$;
\item congruence: $x \sim y$ implies $z \cdot x \sim z \cdot y$ and
  $x \cdot z \sim y \cdot z$;
\item the group laws (associativity, identity, inverse).
\end{itemize}
\end{definition}

\begin{theorem}[Commutators vanish]\label{thm:commutator-vanishes}
In $G^{\ab}$, every commutator is trivial: $[a, b] = e$.
\end{theorem}

\begin{proof}
We compute in $G^{\ab}$:
\begin{align*}
  [a, b] &= (a \cdot b) \cdot (a^{-1} \cdot b^{-1}) \\
  &\sim (b \cdot a) \cdot (a^{-1} \cdot b^{-1})
    &&\text{(commutativity of } a \cdot b\text{)} \\
  &= b \cdot (a \cdot a^{-1}) \cdot b^{-1}
    &&\text{(associativity)} \\
  &= b \cdot e \cdot b^{-1}
    &&\text{(inverse law)} \\
  &= b \cdot b^{-1} = e.
    &&\text{(identity and inverse)} \qedhere
\end{align*}
\end{proof}

\begin{corollary}\label{cor:abelianization-commutative}
$G^{\ab}$ is abelian: $[a] \cdot [b] = [b] \cdot [a]$ for all
$a, b \in G$.
\end{corollary}

\section{The First Homology Group}\label{sec:H1}

\begin{definition}[First homology]\label{def:H1}
The \emph{first homology group} of a pointed type $(A, a)$ is defined
as the abelianization of the fundamental group:
\[
  H_1(A) \;=\; \pi_1(A, a)^{\ab}.
\]
\end{definition}

\begin{definition}[Hurewicz homomorphism]\label{def:hurewicz-map}
The \emph{Hurewicz homomorphism}
\[
  h : \pi_1(A, a) \longrightarrow H_1(A)
\]
is the canonical quotient map to the abelianization.
\end{definition}

\begin{theorem}[Hurewicz theorem, dimension 1]\label{thm:hurewicz}
For any pointed type $(A, a)$, the Hurewicz homomorphism induces an
isomorphism
\[
  \pi_1(A, a)^{\ab} \;\cong\; H_1(A).
\]
\end{theorem}

\begin{proof}
This is tautological in our setup, since $H_1$ is defined as
$\pi_1^{\ab}$. The mathematical content is that $H_1$ computed via
singular homology agrees with $\pi_1^{\ab}$---a fact that in the
topological setting requires the theory of singular chains and the
identification of 1-cycles with loops.
\end{proof}

\begin{proposition}[Properties of the Hurewicz map]\label{prop:hurewicz-properties}
\hfill
\begin{enumerate}
\item[\textup{(i)}] $h$ is surjective: every element of $H_1(A)$ is
  $h(\alpha)$ for some $\alpha \in \pi_1(A, a)$.
\item[\textup{(ii)}] $\ker(h) = [\pi_1, \pi_1]$: $h(\alpha) = e$ if
  and only if $\alpha$ lies in the commutator subgroup.
\item[\textup{(iii)}] $h$ is a group homomorphism:
  $h(\alpha \cdot \beta) = h(\alpha) \cdot h(\beta)$.
\end{enumerate}
\end{proposition}

\section{Computations}\label{sec:hurewicz-computations}

\begin{example}[$H_1$ of the circle]
Since $\pi_1(S^1) \cong \mathbb{Z}$ is already abelian,
\[
  H_1(S^1) \;\cong\; \mathbb{Z}^{\ab} \;\cong\; \mathbb{Z}.
\]
The abelianization of $\mathbb{Z}$ is $\mathbb{Z}$ itself: the
abelianization relation on $\mathbb{Z}$ implies $x = y$ whenever it
relates $x$ and $y$, as verified by induction on the relation.
\end{example}

\begin{example}[$H_1$ of the torus]
Since $\pi_1(T^2) \cong \mathbb{Z} \times \mathbb{Z}$ is abelian,
\[
  H_1(T^2) \;\cong\; (\mathbb{Z} \times \mathbb{Z})^{\ab}
  \;\cong\; \mathbb{Z} \times \mathbb{Z}.
\]
\end{example}

\begin{example}[$H_1$ of the figure-eight]\label{ex:figure-eight-H1}
The figure-eight has non-abelian $\pi_1 \cong \mathbb{Z} * \mathbb{Z}$.
Its abelianization is:
\[
  H_1(S^1 \vee S^1) \;\cong\;
  (\mathbb{Z} * \mathbb{Z})^{\ab} \;\cong\;
  \mathbb{Z} \times \mathbb{Z}.
\]
The abelianization map $\mathbb{Z} * \mathbb{Z} \to
\mathbb{Z} \times \mathbb{Z}$ sends a free product word to the pair
of total exponents: a word with left-elements summing to $m$ and
right-elements summing to $n$ maps to $(m, n)$.
This is a homomorphism (it respects concatenation componentwise) and its
kernel is precisely the commutator subgroup.

This is a paradigmatic example where $\pi_1 \ne H_1$: the
non-commutativity of the free product is ``killed'' by
abelianization.
\end{example}

\begin{theorem}[Free product abelianization]\label{thm:free-product-ab}
For any groups $G$ and $H$\textup{:}
\[
  (G * H)^{\ab} \;\cong\; G^{\ab} \times H^{\ab}.
\]
\end{theorem}

\section{Simply Connected Spaces and $H_1$}\label{sec:sc-H1}

\begin{definition}[Simply connected]\label{def:simply-connected}
A type $A$ is \emph{simply connected} at $a$ if
$\pi_1(A, a) = \{e\}$, i.e., every loop at $a$ is trivial in the
rewrite quotient.
\end{definition}

\begin{theorem}\label{thm:sc-H1-trivial}
If $A$ is simply connected at $a$, then $H_1(A) = \{0\}$.
\end{theorem}

\begin{proof}
If $\pi_1(A, a)$ has a single element, then its abelianization also
has a single element.
\end{proof}

\begin{corollary}[Detection principle]\label{cor:detection}
If $H_1(A)$ contains a non-trivial element, then $\pi_1(A, a)$ is
non-trivial. Equivalently, $H_1(A) \ne 0$ implies $A$ is not simply
connected.
\end{corollary}

\section{Higher Hurewicz and Abelianization of Known Groups}%
\label{sec:higher-hurewicz}

\begin{theorem}[Higher Hurewicz, statement]\label{thm:higher-hurewicz}
For $n \ge 2$ and $(n-1)$-connected $X$, there is an isomorphism
\[
  h_n : \pi_n(X, x_0) \;\xrightarrow{\;\sim\;}\; H_n(X).
\]
In particular\textup{:}
\begin{itemize}
\item For simply connected $X$: $\pi_2(X) \cong H_2(X)$.
\item For spheres: $\pi_n(S^n) \cong H_n(S^n) \cong \mathbb{Z}$.
\end{itemize}
\end{theorem}

\begin{remark}
The higher Hurewicz theorem is stated at the structural level in the
formalization. The computational-paths framework captures $\pi_n$ for
$n = 1, 2$ concretely and records the higher statement as data.
\end{remark}


% ==========================================================================
%  CHAPTER 10 — Advanced Homotopy Theory
% ==========================================================================

\chapter{Advanced Homotopy Theory}\label{ch:advanced}

We conclude Part~II with a survey of the advanced homotopy-theoretic
structures formalized in the computational-paths library, highlighting
the interplay between the algebraic rewriting infrastructure and
classical constructions.

\section{Eilenberg--MacLane Spaces and Postnikov Towers}%
\label{sec:EM-postnikov}

\begin{definition}[Eilenberg--MacLane space]\label{def:EM-space}
An \emph{Eilenberg--MacLane space} $K(G, n)$ for a group $G$ and
$n \ge 1$ is a type satisfying:
\[
  \pi_k(K(G,n)) \;\cong\;
  \begin{cases}
    G & \text{if } k = n, \\
    0 & \text{if } k \ne n.
  \end{cases}
\]
\end{definition}

\begin{definition}[Postnikov tower]\label{def:postnikov-tower}
The \emph{Postnikov tower} of a type $A$ is a sequence of
$n$-groupoid truncations:
\[
  \cdots \;\longrightarrow\; A_{\le n+1} \;\longrightarrow\;
  A_{\le n} \;\longrightarrow\; \cdots \;\longrightarrow\;
  A_{\le 1} \;\longrightarrow\; A_{\le 0}.
\]
The $n$-th Postnikov stage $A_{\le n}$ retains all cells of the
$\omega$-groupoid tower up to dimension $n$ and collapses all higher
cells to the trivial type.
\end{definition}

\begin{theorem}[Postnikov convergence]\label{thm:postnikov-convergence}
\hfill
\begin{enumerate}
\item[\textup{(i)}] If $k \le n$, then the $k$-cells of stage $n$
  agree with the $k$-cells of the full tower.
\item[\textup{(ii)}] For $k \le n \le m$, the $k$-cells of stages
  $n$ and $m$ are equal.
\item[\textup{(iii)}] Stage $n$ kills $\pi_{n+1}$: the
  $(n+1)$-cells at stage $n$ form the trivial type.
\end{enumerate}
\end{theorem}

\begin{proof}
All three properties follow from the definition of the truncation
functor $\trunc_n$ on the $\omega$-groupoid cell types: for
$k \le n$, $\trunc_n(\text{cells}(k)) = \text{cells}(k)$;
for $k > n$, $\trunc_n(\text{cells}(k)) = \mathbf{1}$.
\end{proof}

\begin{remark}
The 1-truncation of the Postnikov tower recovers the strict
groupoid $\PathRwQuot$ (Theorem~5.11), connecting the truncation
framework to the quotient construction of Part~I.
\end{remark}

\section{The Whitehead Theorem}\label{sec:whitehead}

\begin{definition}[Weak equivalence]\label{def:weak-equiv}
A function $f : A \to B$ is a \emph{weak homotopy equivalence} if the
induced maps $f_* : \pi_n(A, a) \to \pi_n(B, f(a))$ are isomorphisms
for all $n \ge 0$ and all base points $a \in A$.
\end{definition}

\begin{theorem}[Whitehead]\label{thm:whitehead}
If $f : A \to B$ is a weak homotopy equivalence between types
satisfying appropriate finiteness conditions \textup{(}CW
approximability\textup{)}, then $f$ is a homotopy equivalence: there
exists $g : B \to A$ with $g \circ f \sim \id_A$ and
$f \circ g \sim \id_B$.
\end{theorem}

\begin{remark}
The formalization packages the Whitehead theorem as a structure:
given weak equivalence data (isomorphisms on all $\pi_n$) together
with a candidate quasi-inverse, it produces a \texttt{SimpleEquiv}
between the types. The induced map preserves the identity element
and respects loop composition.
\end{remark}

\section{Spectral Sequences}\label{sec:spectral}

The formalization includes the basic algebraic framework of spectral
sequences, providing the language for more advanced computations.

\begin{definition}[Pointed set with differential]\label{def:ptset}
A \emph{pointed set} is a pair $(X, 0)$ where $0 \in X$ is a
distinguished element. A \emph{morphism} of pointed sets is a function
preserving the distinguished element. The \emph{zero morphism}
sends everything to $0$.
\end{definition}

\begin{definition}[Spectral page]\label{def:spectral-page}
A \emph{spectral page} with bound $N$ consists of:
\begin{itemize}
\item a bigraded family of pointed sets $E^{p,q}$ for
  $0 \le p, q < N$;
\item differentials $d : E^{p,q} \to E^{p,q}$ satisfying
  $d \circ d = 0$ (the differential squares to the zero element).
\end{itemize}
\end{definition}

\begin{definition}[Spectral sequence]\label{def:spectral-seq}
A \emph{spectral sequence} is a sequence of spectral pages
$(E_r)_{r \ge 0}$ with connecting morphisms
$E_r^{p,q} \to E_{r+1}^{p,q}$ between successive pages.
\end{definition}

\begin{definition}[Degeneration]\label{def:degeneration}
A spectral sequence \emph{degenerates at page $r_0$} if all
differentials on pages $r \ge r_0$ are zero morphisms.
\end{definition}

\begin{theorem}[Convergence for finite filtrations]\label{thm:ss-convergence}
If a spectral sequence degenerates at page $r_0$, then for all
$r \ge r_0$, all differentials are trivial and the terms stabilize.
\end{theorem}

\begin{definition}[Morphism of spectral sequences]\label{def:ss-morphism}
A \emph{morphism} $\varphi : E \to F$ of spectral sequences consists
of pointed-set morphisms $\varphi_r^{p,q} : E_r^{p,q} \to F_r^{p,q}$
on each page, commuting with the differentials:
$\varphi_r \circ d^E_r = d^F_r \circ \varphi_r$.
\end{definition}

\section{Further Formalized Structures}\label{sec:further}

The computational-paths library formalizes a broad range of additional
homotopy-theoretic structures, which we briefly survey.

\subsection*{Cofiber and Puppe sequences.}
The Barratt--Puppe sequence extends the fiber sequence to the cofiber
side. For a map $f : A \to B$, the cofiber sequence
\[
  A \;\xrightarrow{f}\; B \;\longrightarrow\; \cofib(f)
  \;\longrightarrow\; \Sigma A \;\longrightarrow\; \Sigma B
  \;\longrightarrow\; \cdots
\]
is formalized with computational-path connecting maps at each stage.

\subsection*{Mayer--Vietoris sequence.}
For a pushout square, the Mayer--Vietoris exact sequence in homotopy
groups is derived as a consequence of the Seifert--van~Kampen theorem
and the long exact sequence machinery.

\subsection*{Stable homotopy and spectra.}
The colimit of iterated suspensions yields stable homotopy groups.
The library formalizes:
\begin{itemize}
\item $\Omega$-spectra: sequences of types $(E_n)_{n \ge 0}$ with
  equivalences $E_n \simeq \Omega E_{n+1}$;
\item the stable homotopy category, via the stabilization of the
  suspension map;
\item stable stems and their relation to the Freudenthal theorem.
\end{itemize}

\subsection*{Characteristic classes.}
The library includes structures for:
\begin{itemize}
\item Stiefel--Whitney classes $w_i$ for real vector bundles,
\item Chern classes $c_i$ for complex vector bundles,
\item Pontryagin classes $p_i$ for oriented real bundles.
\end{itemize}
These are defined as cohomology classes of classifying spaces and
interact with the fibration theory of Chapter~\ref{ch:fibrations}.

\subsection*{Operads and $A_\infty$-algebras.}
The algebraic structure of iterated loop spaces is captured by
operads. The formalization includes:
\begin{itemize}
\item the associahedron operad and its action on loop spaces,
\item $A_\infty$-algebra structures arising from path composition,
\item the recognition principle relating $n$-fold loop spaces to
  algebras over the little $n$-cubes operad.
\end{itemize}

\subsection*{Rational homotopy theory.}
The formalization records the rationalization functor and the
connection between rational homotopy groups and Sullivan's minimal
models, providing a bridge to the algebraic side of the theory.

\subsection*{Additional structures.}
The library also formalizes:
\begin{itemize}
\item K-theory (algebraic and topological),
\item topological Hochschild homology (THH),
\item Goodwillie calculus and polynomial functors,
\item motivic and \'etale cohomology,
\item surgery theory and bordism,
\item model categories and Quillen adjunctions,
\item higher topos theory and $\infty$-categories.
\end{itemize}
A comprehensive index mapping each structure to its Lean~4 module is
provided in Appendix~B.

\section{Summary of Part~II}\label{sec:part2-summary}

Part~II has developed the homotopy theory of computational paths from
the fundamental group through advanced topics. The key results are
summarized in Table~\ref{tab:part2-summary}.

\begin{table}[ht]
\centering
\caption{Summary of Part~II results.}\label{tab:part2-summary}
\begin{tabular}{lll}
\toprule
\textbf{Result} & \textbf{Reference} & \textbf{Key equation} \\
\midrule
$\pi_1$ is a group & Thm.~\ref{thm:pi1-group} & strict group laws \\
Product formula & Thm.~\ref{thm:pi1-product}
  & $\pi_1(A \times B) \cong \pi_1(A) \times \pi_1(B)$ \\
Eckmann--Hilton & Thm.~\ref{thm:eckmann-hilton}
  & $\pi_n$ abelian for $n \ge 2$ \\
$\pi_1(S^1) \cong \mathbb{Z}$ & Thm.~\ref{thm:pi1-circle}
  & winding number isomorphism \\
$\pi_1(T^2) \cong \mathbb{Z}^2$ & Thm.~\ref{thm:pi1-torus}
  & product of circles \\
$\pi_1(S^1 \vee S^1) \cong \mathbb{Z} * \mathbb{Z}$
  & Thm.~\ref{thm:pi1-figure-eight} & free product \\
Seifert--van Kampen & Thm.~\ref{thm:svk}
  & $\pi_1$ of pushouts \\
Long exact sequence & Thm.~\ref{thm:les}
  & $\pi_1(F) \to \pi_1(E) \to \pi_1(B) \to P(b)$ \\
Hurewicz theorem & Thm.~\ref{thm:hurewicz}
  & $H_1 \cong \pi_1^{\ab}$ \\
Whitehead theorem & Thm.~\ref{thm:whitehead}
  & weak equiv $\Rightarrow$ homotopy equiv \\
\bottomrule
\end{tabular}
\end{table}

The computational-paths framework provides a distinctive perspective
on all these results: the group, groupoid, and higher-categorical
structures are \emph{derived} from the rewrite system on path traces,
rather than being postulated as axioms. The rewriting infrastructure
of Part~I directly produces the strict algebraic identities that make
these homotopy-theoretic constructions possible.


% ============================================================================
%  PART III: METALOGICAL PROPERTIES AND CONCLUSIONS (Chapters 11--12)
% ============================================================================
\part{Metalogical Properties and Conclusions}
\label{part:metatheory}

% ============================================================================
% Chapter 11: The Rewrite System — Metatheory
% Part III of "The Algebra of Computational Paths"
% ============================================================================
\chapter{The Rewrite System: Metatheory}
\label{ch:metatheory}

In Part~I (Chapter~\ref{ch:rewrite-system}), we introduced the 75-rule
rewrite system on computational paths, established its soundness,
normalization properties, termination, and confluence, and constructed
the quotient $\PathQuot$. In this chapter, we examine the rewrite system
from a \emph{metatheoretic} perspective: we situate it within the
framework of typed term rewriting, develop the syntactic path expression
language \texttt{PathExpr}, analyze the strip lemma in detail, establish
the decidability of rewrite equality, describe the \texttt{path\_simp}
tactic for proof automation, and clarify the connection between
computational paths and the identity types of Homotopy Type Theory.

\section{The Typed Rewriting Perspective}
\label{sec:typed-rewriting}

\begin{definition}[Typed Rewriting System]\label{def:typed-trs}
  A \emph{typed term rewriting system} (typed TRS) consists of:
  \begin{enumerate}[label=(\roman*)]
    \item A set of \emph{sorts} (types), here the endpoint pairs $(a, b)$
      for $a, b : A$.
    \item A set of \emph{typed terms} (path expressions), well-typed by
      their source and target.
    \item A \emph{reduction relation} $\rew$ on terms of the same sort,
      given by the 75 rules of Chapter~\ref{ch:rewrite-system}.
  \end{enumerate}
  The rewrite system on $\Path_A(a, b)$ is thus a many-sorted first-order
  TRS in the sense of Klop~\cite{Klop92} and Terese~\cite{Terese03}, with
  sorts indexed by pairs of elements.
\end{definition}

\begin{remark}\label{rem:trs-features}
  Our TRS has several distinctive features compared to classical term
  rewriting:
  \begin{enumerate}[label=(\roman*)]
    \item \textbf{Dependent sorts.} The sort of a path depends on its
      endpoints, which are themselves elements of the ambient type~$A$.
      Rules like associativity (Rule~\ref{rule:tt}) change the
      intermediate typing---$(p \comp q) \comp r$ has the same endpoints
      as $p \comp (q \comp r)$, but the subterms have different types.
    \item \textbf{Higher-order constructors.} The $\lamCongr$ and
      $\congrArgOp$ constructors take functions as arguments, making the
      system a \emph{higher-order} TRS in the sense of
      Nipkow~\cite{Nipkow91}.
    \item \textbf{Proof-irrelevant semantics.} While the syntactic terms
      (path expressions) are first-order or higher-order, the semantic
      interpretation collapses via $\toEq$ to the proof-irrelevant
      identity type. This separation between syntax and semantics is
      fundamental to the theory.
  \end{enumerate}
\end{remark}

\begin{definition}[The $\mathrm{LND}_{\mathrm{EQ}}$-TRS]\label{def:lndeq}
  The complete rewrite system is called the $\mathrm{LND}_{\mathrm{EQ}}$-TRS
  (following the terminology of~\cite{RDQO18}). The rules are enumerated
  using the mnemonic names of \cref{tab:lndeq-rules}.
\end{definition}

\begin{table}[ht]
\centering
\caption{The $\mathrm{LND}_{\mathrm{EQ}}$-TRS rule mnemonics and their
  correspondence to the constructors of the \texttt{Step} inductive type.}
\label{tab:lndeq-rules}
\small
\begin{tabular}{lllp{5.4cm}}
\toprule
\textbf{Mnemonic} & \textbf{Lean name} & \textbf{Group} & \textbf{Description} \\
\midrule
sr   & \texttt{symm\_refl}  & I   & $\inv{\refl} \rew \refl$ \\
ss   & \texttt{symm\_symm}  & I   & $\inv{(\inv{p})} \rew p$ \\
lrr  & \texttt{trans\_refl\_left}  & I & $\refl \comp p \rew p$ \\
rrr  & \texttt{trans\_refl\_right} & I & $p \comp \refl \rew p$ \\
tr   & \texttt{trans\_symm} & I   & $p \comp \inv{p} \rew \refl$ \\
tsr  & \texttt{symm\_trans} & I   & $\inv{p} \comp p \rew \refl$ \\
stss & \texttt{symm\_trans\_congr} & I & $\inv{(p \comp q)} \rew \inv{q} \comp \inv{p}$ \\
tt   & \texttt{trans\_assoc} & I  & $(p \comp q) \comp r \rew p \comp (q \comp r)$ \\
\midrule
mx2l1 & \texttt{map2\_subst} & II & $\mapTwo$ factorization \\
mx2l2--mx2r2 & \texttt{prod\_*\_beta} & II & Product $\beta$-rules \\
mxetaProd & \texttt{prod\_eta} & II & Product $\eta$-rule \\
mxsigmaFst/Snd & \texttt{sigma\_*\_beta} & II & $\Sigma$-type $\beta$-rules \\
mxetaSigma & \texttt{sigma\_eta} & II & $\Sigma$-type $\eta$-rule \\
mxcase & \texttt{sum\_rec\_*\_beta} & II & Sum $\beta$-rules \\
mxetaFun & \texttt{fun\_eta} & II & Function $\eta$-rule \\
mxlam & \texttt{fun\_app\_beta} & II & Function $\beta$-rule \\
\midrule
slr/srr & \texttt{context\_subst\_*\_refl\_*} & IV & Context unit rules \\
slss/srsr & \texttt{context\_subst\_*\_idempotent} & IV & Context idempotence \\
tsbll/tsbrl & \texttt{context\_subst\_*\_beta} & IV & Context $\beta$-rules \\
tsblr/tsbrr & \texttt{context\_subst\_*\_assoc} & IV & Context associativity \\
ttsv/tstu & \texttt{context\_tt\_cancel\_*} & IV & Context cancellation \\
\bottomrule
\end{tabular}
\end{table}

The precedence ranking assigns each rule a natural number via the function
$\mathrm{rank} : \mathrm{Rule} \to \Nat$ (see
\texttt{Rewrite.Termination.Rule.rank} in the formalization). The ranking
is compatible with the termination ordering: rules producing simpler
expressions receive lower ranks.

\section{PathExpr: A First-Order Term Language}
\label{sec:pathexpr}

While the type $\Path_A(a, b)$ is a \emph{record} (a list of steps paired
with a proof), reasoning about rewrites requires a \emph{syntactic}
representation of path expressions. The \texttt{PathExpr} type provides
this.

\begin{definition}[Path Expression]\label{def:pathexpr}
  The type $\mathrm{PathExpr}_{A}(a, b)$ is an inductive type with
  constructors:
  \begin{align*}
    \mathrm{atom} &: \Path_A(a, b) \to \mathrm{PathExpr}_A(a, b), \\
    \refl &: (a : A) \to \mathrm{PathExpr}_A(a, a), \\
    \mathrm{symm} &: \mathrm{PathExpr}_A(a, b) \to \mathrm{PathExpr}_A(b, a), \\
    \mathrm{trans} &: \mathrm{PathExpr}_A(a, b) \to \mathrm{PathExpr}_A(b, c)
      \to \mathrm{PathExpr}_A(a, c), \\
    \congrArgOp &: (f : A \to B) \to \mathrm{PathExpr}_A(a, b)
      \to \mathrm{PathExpr}_B(f(a), f(b)), \\
    \mapTwo &: (f : A \to B \to C) \to \mathrm{PathExpr}_A(a_1, a_2)
      \to \mathrm{PathExpr}_B(b_1, b_2) \\
      &\qquad\qquad\qquad\qquad\quad
      \to \mathrm{PathExpr}_C(f\,a_1\,b_1,\; f\,a_2\,b_2), \\
    \mathrm{context\_map} &: \Context(A, B) \to \mathrm{PathExpr}_A(a, b)
      \to \mathrm{PathExpr}_B(C(a), C(b)), \\
    \mathrm{context\_subst\_left} &: \Context(A, B)
      \to \mathrm{PathExpr}_B(x, C(a_1))
      \to \mathrm{PathExpr}_A(a_1, a_2) \\
      &\qquad\qquad\qquad\qquad\quad
      \to \mathrm{PathExpr}_B(x, C(a_2)), \\
    \mathrm{context\_subst\_right} &: \Context(A, B)
      \to \mathrm{PathExpr}_A(a_1, a_2)
      \to \mathrm{PathExpr}_B(C(a_2), y) \\
      &\qquad\qquad\qquad\qquad\quad
      \to \mathrm{PathExpr}_B(C(a_1), y).
  \end{align*}
\end{definition}

The key property of $\mathrm{PathExpr}$ is that it separates the
\emph{syntactic structure} of a path from its \emph{semantic content}.

\begin{definition}[Evaluation]\label{def:pathexpr-eval}
  The \emph{evaluation function}
  $\mathrm{eval} : \mathrm{PathExpr}_A(a, b) \to \Path_A(a, b)$
  interprets each constructor by the corresponding path operation:
  \begin{align*}
    \mathrm{eval}(\mathrm{atom}(p)) &= p, \\
    \mathrm{eval}(\refl(a)) &= \Path.\refl(a), \\
    \mathrm{eval}(\mathrm{symm}(e)) &= \Path.\mathrm{symm}(\mathrm{eval}(e)), \\
    \mathrm{eval}(\mathrm{trans}(e_1, e_2)) &= \Path.\mathrm{trans}(\mathrm{eval}(e_1),
      \mathrm{eval}(e_2)),
  \end{align*}
  and similarly for the remaining constructors.
\end{definition}

\begin{definition}[Size Measure]\label{def:pathexpr-size}
  The \emph{size} of a path expression is defined recursively:
  \begin{align*}
    \mathrm{size}(\mathrm{atom}(p)) &= 1, \qquad
    \mathrm{size}(\refl(a)) = 1, \\
    \mathrm{size}(\mathrm{symm}(e)) &= \mathrm{size}(e) + 1, \\
    \mathrm{size}(\mathrm{trans}(e_1, e_2)) &= \mathrm{size}(e_1)
      + \mathrm{size}(e_2) + 1,
  \end{align*}
  and analogously for the remaining constructors. The size provides a
  termination measure for recursive functions on path expressions.
\end{definition}

\subsection{Rewriting on PathExpr}

The single-step and multi-step rewrite relations lift from $\Path$ to
$\mathrm{PathExpr}$, with the same 75 rules applied syntactically. The
critical property is that evaluation commutes with rewriting:

\begin{theorem}[Evaluation Preserves Rewriting]\label{thm:eval-rw}
  If $e_1 \rew e_2$ as path expressions, then
  $\mathrm{eval}(e_1) \rew \mathrm{eval}(e_2)$ as paths (or
  $\mathrm{eval}(e_1) \rweq \mathrm{eval}(e_2)$).
\end{theorem}

\begin{proof}
  By induction on the derivation of $e_1 \rew e_2$. Each syntactic rule
  application maps to the corresponding semantic rule via $\mathrm{eval}$.
\end{proof}

\subsection{Confluence of PathExpr}

The main payoff of the $\mathrm{PathExpr}$ language is that confluence can
be stated and proved at the syntactic level, then transferred to $\Path$.

\begin{definition}[Join for PathExpr]\label{def:pathexpr-join}
  A \emph{join} of path expressions $e_1$ and $e_2$ is a triple
  $(m, h_1, h_2)$ where $m$ is a path expression, $h_1 : e_1 \rews m$,
  and $h_2 : e_2 \rews m$.
\end{definition}

\begin{theorem}[PathExpr Confluence]\label{thm:pathexpr-confluence}
  The rewrite system on $\mathrm{PathExpr}$ is confluent: for any
  $e_1 \rews e$ and $e_2 \rews e$ (or, equivalently, any two rewrites
  from a common source), explicit join witnesses can be computed by the
  reduction strategy.
\end{theorem}

The formalization packages this result as the typeclass
\texttt{HasJoinOfRwExpr}, which provides:
\[
  \mathsf{join\_of\_rw} : \Rw(p, q) \to \Rw(p, r) \to
  \mathrm{Join}(q, r),
\]
where the join includes an explicit meet term and the two reduction
certificates.

\begin{corollary}[Confluence Transfer]\label{cor:confluence-transfer}
  The confluence of $\mathrm{PathExpr}$ rewrites transfers to
  $\Path$-level joins: if $\Rw(\mathrm{eval}(e), p)$ and
  $\Rw(\mathrm{eval}(e), q)$, then $p$ and $q$ have a common reduct.
\end{corollary}

\begin{proof}
  Apply $\mathrm{eval}$ to the $\mathrm{PathExpr}$-level join and use
  \cref{thm:eval-rw}.
\end{proof}

\section{The Strip Lemma and Local Confluence in Detail}
\label{sec:strip-detail}

The strip lemma (\cref{thm:strip-lemma}) is the technical heart of the
confluence proof. We now describe its structure in more detail.

\begin{theorem}[Strip Lemma, Detailed]\label{thm:strip-detailed}
  Let $p \rew q$ (a single step) and $p \rews r$ (a multi-step
  reduction). Then there exists a path $m$ with $q \rews m$ and
  $r \rews m$. The proof proceeds by induction on the length of
  $p \rews r$:
  \begin{enumerate}[label=(\roman*)]
    \item \textbf{Base case} ($r = p$): take $m = q$.
    \item \textbf{Inductive case} ($p \rews r'$ and $r' \rew r$):
      by the induction hypothesis applied to $p \rew q$ and $p \rews r'$,
      obtain a join $m'$ of $q$ and $r'$. Then analyze the critical pair
      $(m' \leftarrow r' \rew r)$ to produce the final join~$m$.
  \end{enumerate}
\end{theorem}

\subsection{Critical Pair Analysis}

The critical pairs of the $\mathrm{LND}_{\mathrm{EQ}}$-TRS arise when
two rules overlap---i.e., the left-hand side of one rule is a subterm of
the left-hand side of another, or the same term matches two distinct rules.

\begin{definition}[Critical Pair]\label{def:critical-pair}
  A \emph{critical pair} is a pair $(q_1, q_2)$ of paths obtained by
  applying two different rules (or the same rule at two different
  positions) to a common redex $p$, yielding $p \rew q_1$ and
  $p \rew q_2$.
\end{definition}

The 75 rules generate a finite (though large) set of critical pairs.
The principal families of critical pairs are:

\begin{enumerate}[label=(\arabic*)]
  \item \textbf{Associativity--unit overlap.}\;
    The term $(p \comp \refl) \comp r$ matches both Rule~\ref{rule:rrr}
    (reducing to $p \comp r$) and Rule~\ref{rule:tt} (reducing to
    $p \comp (\refl \comp r)$). The latter then reduces by
    Rule~\ref{rule:lrr} to $p \comp r$. The join is $p \comp r$.

  \item \textbf{Associativity--inverse overlap.}\;
    The term $(p \comp \inv{p}) \comp r$ matches both Rule~\ref{rule:tr}
    composed with Rule~\ref{rule:lrr} (reducing to $r$) and
    Rule~\ref{rule:tt} (reducing to $p \comp (\inv{p} \comp r)$). The
    latter further reduces by the inverse law and left unit.

  \item \textbf{Product $\beta$--$\eta$ overlap.}\;
    The term $\prodMk(\fst(r), \snd(r))$ where $r = \prodMk(p, q)$:
    applying $\eta$ first gives $r = \prodMk(p, q)$; applying $\beta$ to
    each component gives $\prodMk(p, q)$.

  \item \textbf{Context--structural overlap.}\;
    A context substitution $\substL(C, r, p)$ where $p \rew p'$: the
    structural rule and the context-specific rule may both apply. The
    joins are produced via the context congruence property.

  \item \textbf{Symmetry--composition overlap.}\;
    The term $\inv{(p \comp \refl)}$ matches both Rule~\ref{rule:rrr}
    (under symmetry congruence, giving $\inv{p}$) and
    Rule~\ref{rule:stss} (giving $\inv{\refl} \comp \inv{p}$, which
    reduces to $\refl \comp \inv{p}$ and then to $\inv{p}$).
\end{enumerate}

\begin{proposition}[All Critical Pairs are Joinable]\label{prop:critical-pairs}
  Every critical pair $(q_1, q_2)$ of the $\mathrm{LND}_{\mathrm{EQ}}$-TRS
  is joinable: there exists $m$ with $q_1 \rews m$ and $q_2 \rews m$.
  The join witnesses are explicitly constructed in the formalization.
\end{proposition}

\begin{proof}
  By exhaustive case analysis. The formalization in
  \texttt{Rewrite.ConfluenceProof} and
  \texttt{Rewrite.ConfluenceProofPathExpr} covers all cases, with each
  critical pair resolved by exhibiting a concrete join term and the
  reduction sequences reaching it.
\end{proof}

\begin{theorem}[Constructive Confluence]\label{thm:constructive-confluence}
  The confluence proof is \emph{constructive}: given two multi-step
  reductions from a common source, the join witness (the common reduct
  together with both reduction sequences) is computable. This is
  formalized in \texttt{Rewrite.ConfluenceConstructive} and packaged via
  the \texttt{HasJoinOfRw} typeclass.
\end{theorem}

\section{Decidability and the Path Tactic}
\label{sec:decidability}

\begin{theorem}[Decidability of $\RwEq$]\label{thm:rweq-decidable}
  Rewrite equality of computational paths is decidable: there is an
  algorithm that, given paths $p, q : \Path_A(a, b)$, determines whether
  $p \rweq q$.
\end{theorem}

\begin{proof}
  By \cref{thm:normalization}(iii), $p \rweq q$ if and only if
  $\normalize(p) = \normalize(q)$. Since $\normalize(p) = \ofEq(\toEq(p))$
  and $\normalize(q) = \ofEq(\toEq(q))$, equality of the normal forms
  reduces to equality of the underlying propositional equalities
  $\toEq(p)$ and $\toEq(q)$, which are equal by proof irrelevance.
  Thus $p \rweq q$ holds for \emph{all} paths with the same endpoints,
  and the decision procedure is trivial: always return ``yes.''
\end{proof}

\begin{remark}\label{rem:trivial-decidability}
  The triviality of the decision procedure is a consequence of proof
  irrelevance in the ambient type theory. In an intensional type theory
  without UIP (such as the core of HoTT), $\RwEq$-equivalence would be a
  genuinely non-trivial decision problem. In our setting, the interest
  lies not in the \emph{decision} but in the \emph{witnesses}: the
  explicit rewrite sequences connecting $p$ to its normal form.
\end{remark}

\subsection{The \texttt{path\_simp} Tactic}

The formalization provides a suite of Lean~4 tactics for automating
proofs involving rewrite equality:

\begin{definition}[Path Tactics]\label{def:path-tactics}
  The following tactics are defined in \texttt{Rewrite.PathTactic}:
  \begin{itemize}
    \item \texttt{path\_rfl}: closes $\RwEq$ goals that are reflexive
      ($p \rweq p$), via $\RwEq.\refl$.
    \item \texttt{path\_symm}: transforms a goal $p \rweq q$ into
      $q \rweq p$, via $\RwEq.\mathrm{symm}$.
    \item \texttt{path\_simp}: applies the $\mathtt{simp}$ tactic with the
      library of $\RwEq$ lemmas (registered as \texttt{@[simp]}
      attributes on all 75 rules and their derived consequences).
    \item \texttt{path\_trans h}: applies transitivity with an
      intermediate path from hypothesis~$h$.
    \item \texttt{path\_congr\_left h} / \texttt{path\_congr\_right h}:
      applies the congruence property of $\mathrm{trans}$ in the left or
      right argument.
    \item \texttt{path\_assoc}: reassociates $\mathrm{trans}$ chains to
      the right.
    \item \texttt{path\_canon}: uses the canonicalization rule to close
      goals by reducing both sides to $\ofEq$.
  \end{itemize}
\end{definition}

\begin{example}
  The following Lean proof demonstrates the tactic suite:
  \begin{center}
  \begin{minipage}{0.8\textwidth}
  \begin{verbatim}
  example (p : Path a a) :
      RwEq (trans (refl a) p) p := by
    path_simp

  example (h : RwEq p q) :
      RwEq (trans p r) (trans q r) := by
    path_congr_left h
  \end{verbatim}
  \end{minipage}
  \end{center}
\end{example}

\section{Connection to HoTT Identity Types}
\label{sec:hott-connection}

The computational paths framework has a precise relationship with
the identity types of Homotopy Type Theory.

\begin{theorem}[J-Elimination]\label{thm:j-elim}
  The computational path structure satisfies the $J$-elimination rule
  (path induction): for any type family
  $D : \Pi_{a,b : A}.\, \Path_A(a, b) \to \Type$ and any
  $d : \Pi_{a : A}.\, D(a, a, \refl(a))$, there exists
  \[
    J(D, d) : \Pi_{a, b : A}.\, \Pi_{p : \Path_A(a,b)}.\, D(a, b, p).
  \]
\end{theorem}

\begin{proof}
  Define $J(D, d, a, b, p) = \mathrm{transport}_D(p, d(a))$, where the
  transport is along the path from $(a, a, \refl(a))$ to $(a, b, p)$
  induced by the underlying equality $\toEq(p) : a =_A b$.
\end{proof}

\begin{theorem}[Function Extensionality]\label{thm:funext}
  For functions $f, g : A \to B$, if $h : \Pi_{x : A}.\, \Path_B(f(x), g(x))$
  (a pointwise path), then there exists a path
  $\mathrm{funext}(h) : \Path_{A \to B}(f, g)$.
\end{theorem}

\begin{proof}
  Construct $\mathrm{funext}(h) = \lamCongr(h)$, the function congruence
  path. Its trace records the pointwise rewrite sequence; its proof field
  is the propositional function extensionality of the ambient theory.
\end{proof}

\begin{theorem}[HoTT Compatibility]\label{thm:hott-compat}
  The path quotient $\PathQuot_A(a, b)$ is equivalent (as a type) to the
  propositional identity type $a =_A b$
  (Theorem~\ref{thm:quot-equiv}). Under this equivalence:
  \begin{enumerate}[label=(\roman*)]
    \item Path composition corresponds to transitivity of equality.
    \item Path inversion corresponds to symmetry.
    \item Congruence corresponds to $\mathrm{ap}$ (the action on paths).
    \item Transport corresponds to $\mathrm{transport}$.
    \item The $J$-rule (\cref{thm:j-elim}) corresponds to the standard
      $J$-eliminator.
  \end{enumerate}
\end{theorem}

\begin{remark}[The Univalence Question]\label{rem:univalence}
  The computational paths framework does \emph{not} require the
  univalence axiom or higher inductive types (HITs) as primitive axioms.
  Instead:
  \begin{itemize}
    \item The higher-dimensional structure (weak $\omega$-groupoid) is
      \emph{derived} from the rewrite system on traces, not postulated.
    \item Non-trivial fundamental groups arise from the path expression
      calculus (formal generators and relations), not from HITs.
    \item The contractibility at dimension~$\geq 3$
      (\cref{thm:contract3}) is a \emph{consequence} of proof irrelevance,
      not an axiom.
  \end{itemize}
  This provides a complementary approach to HoTT: where HoTT enriches
  the identity type by rejecting UIP, computational paths enrich the
  \emph{computational content} of equality proofs while retaining UIP at
  the propositional level.
\end{remark}

\section{Module Organization and Dependency Structure}
\label{sec:module-organization}

The metatheoretic results are organized across the following modules in
the \texttt{Rewrite} directory:

\begin{center}
\small
\begin{tabular}{lp{7.5cm}}
\toprule
\textbf{Module} & \textbf{Contents} \\
\midrule
\texttt{Step.lean} & The 75-rule \texttt{Step} inductive type and soundness proof \\
\texttt{Rw.lean} & Reflexive--transitive closure $\Rw$ \\
\texttt{RwEq.lean} & Rewrite equality $\RwEq$ and all congruence lemmas \\
\texttt{Normalization.lean} & Normal forms and the $\normalize$ function \\
\texttt{LNDEQ.lean} & Rule enumeration and mnemonic names \\
\texttt{Termination.lean} & Rule precedence and RPO measure \\
\texttt{StripLemma.lean} & The strip lemma (local confluence) \\
\texttt{Confluence.lean} & Join structure and \texttt{HasJoinOfRw} interface \\
\texttt{ConfluenceProof.lean} & Concrete join witnesses \\
\texttt{ConfluenceFull.lean} & Full confluence theorem \\
\texttt{ConfluenceConstructive.lean} & Constructive confluence certificates \\
\texttt{PathExpr.lean} & Syntactic path expressions (1,228 lines) \\
\texttt{ExprConfluence.lean} & \texttt{HasJoinOfRwExpr} typeclass \\
\texttt{PathExprConfluence.lean} & Confluence transfer to $\Path$ \\
\texttt{PathTactic.lean} & The \texttt{path\_simp} tactic suite \\
\texttt{Quot.lean} & The quotient $\PathQuot$ and its operations \\
\texttt{SimpleEquiv.lean} & Equivalence $\PathQuot \simeq \mathrm{Eq}$ \\
\texttt{MinimalAxioms.lean} & Minimal axiom sets for the TRS \\
\bottomrule
\end{tabular}
\end{center}

The dependency chain for the full metatheory is:
\[
  \texttt{Step} \to \texttt{Rw} \to \texttt{RwEq} \to
  \texttt{Normalization} \to \texttt{Termination} \to
  \texttt{StripLemma} \to \texttt{Confluence} \to \texttt{Quot}.
\]
The $\mathrm{PathExpr}$ branch runs in parallel:
\[
  \texttt{PathExpr} \to \texttt{ExprConfluence} \to
  \texttt{PathExprConfluence},
\]
joining the main branch at the confluence level.

% ============================================================================
% Chapter 12: Conclusion and Future Directions
% Part III of "The Algebra of Computational Paths"
% ============================================================================
\chapter{Conclusion and Future Directions}
\label{ch:conclusion}

\section{Summary of Contributions}
\label{sec:summary}

This paper has developed a comprehensive mathematical theory of
\emph{computational paths}---a framework in which propositional
equalities carry explicit rewrite traces recording the computational
steps by which they were derived. The principal contributions are:

\paragraph{A complete algebraic framework (Part~I).}
Starting from the definition of a computational path as a pair
$(s, \pi)$ of a step list and a propositional equality
(\cref{def:path}), we constructed the full path algebra: reflexivity,
symmetry, transitivity, congruence, transport, dependent application,
and operations for products, sums, dependent pairs, and function types
(Chapter~\ref{ch:basic-constructions}). The key insight is that even in a
proof-irrelevant setting where UIP holds for $\Eq$, the computational
traces are distinct combinatorial objects that support a rich algebraic
structure (\cref{thm:non-uip}).

\paragraph{A confluent, terminating rewrite system.}
The 76 rules of the $\mathrm{LND}_{\mathrm{EQ}}$-TRS
(Chapter~\ref{ch:rewrite-system}), organized into eight groups---path
algebra, type-former $\beta$/$\eta$-rules, transport, contexts,
dependent contexts, bi-contexts, map congruence, and structural
closure---axiomatize the identities that do not hold strictly on
computational traces. We proved soundness
(\cref{thm:step-sound}), termination (\cref{thm:termination}),
local confluence via the strip lemma (\cref{thm:strip-lemma}),
and global confluence via Newman's lemma (\cref{thm:confluence}).
The quotient $\PathQuot$ by rewrite equality recovers the standard
identity type (\cref{thm:quot-equiv}).

\paragraph{Weak $\omega$-groupoid structure.}
The tower of iterated derivation cells---paths, derivations between
paths, meta-derivations between derivations, and so on---forms a weak
$\omega$-groupoid in the sense of Batanin--Leinster
(\cref{thm:omega-groupoid}), with contractibility beginning at
dimension~3 (\cref{thm:contract3}). This threshold is critical: it
preserves non-trivial fundamental groups while ensuring that all
higher coherence conditions are automatically satisfied via proof
irrelevance (\cref{rem:contract-threshold}).

\paragraph{Comprehensive homotopy theory (Part~II).}
Building on the algebraic foundations, we developed:
\begin{itemize}
  \item Fundamental groups and their functoriality
    (Chapter~\ref{ch:fundamental-groups}), including the product formula,
    Eckmann--Hilton argument, and basepoint independence.
  \item Computations of $\pi_1$ for standard spaces
    (Chapter~\ref{ch:spaces}): the circle ($\pi_1 \cong \ZZ$), torus
    ($\ZZ \times \ZZ$), figure-eight ($\ZZ * \ZZ$), Klein bottle,
    projective spaces, lens spaces, and bouquets.
  \item The Seifert--van Kampen theorem for pushouts
    (\cref{thm:svk}), yielding $\pi_1$ of wedge sums and
    amalgamated free products.
  \item Fibrations, covering spaces, and the long exact sequence
    of homotopy groups (Chapter~\ref{ch:fibrations}), including the Hopf
    fibration.
  \item The Hurewicz theorem ($H_1 \cong \pi_1^{\mathrm{ab}}$) and
    homological algebra (Chapter~\ref{ch:hurewicz}).
  \item Advanced structures: Eilenberg--MacLane spaces, Postnikov
    towers, the Whitehead theorem, spectral sequences, characteristic
    classes, and operads (Chapter~\ref{ch:advanced}).
\end{itemize}

\paragraph{Metatheory and automation (Part~III).}
The syntactic path expression language $\mathrm{PathExpr}$
(Chapter~\ref{ch:metatheory}) enables constructive confluence proofs
with explicit join witnesses. The \texttt{path\_simp} tactic suite
automates rewrite-equality reasoning in Lean~4.

\section{The Formalization}
\label{sec:formalization}

The entire development is formalized in Lean~4, comprising:

\begin{center}
\begin{tabular}{lr}
\toprule
\textbf{Metric} & \textbf{Count} \\
\midrule
Lean 4 source files & 302 \\
Total lines of code & 60,860 \\
Definitions (\texttt{def}, \texttt{noncomputable def}) & 1,710 \\
Theorems (\texttt{theorem}) & 885 \\
Structures (\texttt{structure}) & 701 \\
Inductive types (\texttt{inductive}) & 69 \\
\bottomrule
\end{tabular}
\end{center}

\paragraph{Design decisions.}
Several design choices shaped the formalization:
\begin{enumerate}[label=(\roman*)]
  \item \textbf{Path as record, not inductive.} Computational paths are
    represented as records (a step list plus a proof), not as an inductive
    family. This makes the monoid laws (left/right unit, associativity)
    hold definitionally---a significant advantage for proof ergonomics.
  \item \textbf{Step as an inductive proposition.} The single-step
    rewrite relation $\Step$ is defined as an inductive type in $\Prop$,
    making it proof-irrelevant. This simplifies the higher-dimensional
    theory: two-cells and above are automatically well-behaved.
  \item \textbf{Quotient types for the fundamental group.} The quotient
    $\PathQuot$ uses Lean's built-in \texttt{Quot} type, ensuring that the
    fundamental group inherits decidable equality and supports pattern
    matching via \texttt{Quot.lift} and \texttt{Quot.ind}.
  \item \textbf{Typeclass-based confluence.} The confluence interface
    \texttt{HasJoinOfRw} is a typeclass, allowing different confluence
    strategies (constructive, classical, syntactic) to be plugged in
    transparently.
  \item \textbf{Universe polymorphism.} The entire development is
    universe-polymorphic, with $\Path$, $\Step$, $\Rw$, $\RwEq$, and
    $\PathQuot$ defined at universe level~$u$.
\end{enumerate}

\paragraph{Proof automation.}
The \texttt{path\_simp} tactic (Section~\ref{sec:decidability}) proved
essential for managing the combinatorial complexity of the rewrite system.
With all 76 rules and their congruence variants registered as
\texttt{@[simp]} lemmas, \texttt{path\_simp} resolves most $\RwEq$ goals
automatically. For more complex goals, the dedicated tactics
(\texttt{path\_trans}, \texttt{path\_congr\_left/right},
\texttt{path\_assoc}) provide fine-grained control.

\paragraph{Module structure.}
The formalization is organized into a hierarchy reflecting the paper's
structure:
\begin{itemize}
  \item \texttt{Path.Basic}: paths, steps, fundamental operations,
    contexts.
  \item \texttt{Path.Rewrite}: the TRS, confluence, normalization,
    quotient.
  \item \texttt{Groupoid}: weak and strict groupoid structures,
    rewrite lifts.
  \item \texttt{HigherDimensional}: globular tower, derivation cells,
    $\omega$-groupoid.
  \item \texttt{HomotopyTheory}: fundamental groups, loop spaces, spaces.
  \item \texttt{Fibration}: fibrations, covering spaces, exact sequences.
  \item \texttt{Homological}: abelianization, Hurewicz, spectral sequences.
  \item \texttt{Advanced}: Eilenberg--MacLane, Postnikov, operads,
    K-theory.
\end{itemize}

A complete dependency graph is provided in Appendix~\ref{app:dependency}.

\section{Future Directions}
\label{sec:future}

Several natural extensions of this work are under investigation:

\paragraph{Cubical computational paths.}
The current framework uses \emph{list-based} traces (sequences of
elementary steps). An alternative representation uses
\emph{cubical} traces, where a path in dimension~$n$ is an $n$-cube
with specified boundary. This would align computational paths more
closely with cubical type theory~\cite{BeCH14}, while retaining the
rewrite-based computational content. The key challenge is defining the
appropriate notion of ``cubical step'' and establishing confluence for
the resulting system.

\paragraph{Machine-checked higher-categorical coherence.}
While contractibility at dimension~$\geq 3$ is established
(\cref{thm:contract3}), the explicit coherence data at dimension~2
(pentagon, triangle, interchange) is only verified via proof
irrelevance. A finer analysis---producing explicit derivation witnesses
for each coherence law without appealing to proof irrelevance---would
yield a fully constructive weak 2-category structure, potentially
useful in settings without proof irrelevance.

\paragraph{Computational content extraction.}
The rewrite traces carried by computational paths contain
\emph{algorithmic information}: the sequence of rewrites encodes a
proof strategy. Extracting this information systematically could yield:
\begin{itemize}
  \item Certified program transformations guided by equality proofs.
  \item Optimization strategies based on trace analysis (e.g.,
    detecting redundant steps, finding shorter traces).
  \item Connections to cost semantics and complexity of equational
    reasoning.
\end{itemize}

\paragraph{Applications to automated reasoning.}
The $\mathrm{LND}_{\mathrm{EQ}}$-TRS is a self-contained equational
theory with good metatheoretic properties (confluent, terminating,
decidable). It could serve as a decision procedure kernel for
automated theorem provers dealing with equality reasoning in
dependent type theories.

\paragraph{Connections to higher-dimensional rewriting.}
The rewrite rules on paths are themselves generators of 2-cells
(derivations), and the critical-pair analysis produces 3-cells. This
is an instance of Squier's theory~\cite{Squier94} of higher-dimensional
rewriting, where the homotopical properties of a rewrite system
(e.g., finite derivation type) are captured by higher-dimensional
cells. Establishing a precise connection between the
$\mathrm{LND}_{\mathrm{EQ}}$-TRS and Squier's framework would link
computational paths to the broader program of higher-dimensional
algebra.

\paragraph{Synthetic homotopy theory.}
The computational-paths approach provides a new setting for
\emph{synthetic} homotopy theory---developing homotopy theory directly
from the algebraic structure of paths, without reference to topological
spaces. Extending the current development to cover more of synthetic
homotopy theory (e.g., Blakers--Massey, the James construction, the
Hopf invariant) would demonstrate the power of the framework.

\paragraph{Scaling the formalization.}
The current formalization of 60,860 lines covers a substantial portion
of basic algebraic topology. Scaling to more advanced topics
(e.g., stable homotopy theory, chromatic homotopy theory, derived
algebraic geometry) will require:
\begin{itemize}
  \item More sophisticated automation (e.g., a dedicated tactic for
    spectral sequence computations).
  \item Better library management and modular design patterns.
  \item Potential integration with other Lean~4 mathematics libraries
    (e.g., Mathlib) for the algebraic prerequisites.
\end{itemize}

\bigskip

\noindent
In summary, computational paths offer a distinctive perspective on the
algebra of equality: by recording \emph{how} an equality was derived
(not just \emph{that} it holds), we recover higher-dimensional structure
in a proof-irrelevant setting, build a complete homotopy theory with
explicit combinatorial foundations, and provide a framework where all
coherence conditions are either computationally verified or automatically
satisfied. The formalization in Lean~4 demonstrates that this program can
be carried out rigorously and at scale.


% ============================================================================
%  APPENDICES
% ============================================================================
\appendix

% ============================================================================
% Appendix A: Complete List of Rewrite Rules
% ============================================================================
\chapter{Complete List of Rewrite Rules}
\label{app:rules}

We list all 76 rewrite rules of the $\mathrm{LND}_{\mathrm{EQ}}$-TRS,
organized by group. Each rule is given with its formal name (matching the
constructor of the \texttt{Step} inductive type in the Lean~4
formalization), its $\mathrm{LND}_{\mathrm{EQ}}$ mnemonic, and a
mathematical description. The notation follows Chapter~\ref{ch:rewrite-system}:
$p \rew q$ denotes a single rewrite step from~$p$ to~$q$.

\section*{Group I: Path Algebra (8 rules)}

\begin{enumerate}[label=\textbf{R\arabic*.}, leftmargin=3.5em]
  \item \texttt{symm\_refl} \hfill \textbf{(sr)} \\
    $\inv{\refl(a)} \;\rew\; \refl(a)$.

  \item \texttt{symm\_symm} \hfill \textbf{(ss)} \\
    $\inv{(\inv{p})} \;\rew\; p$.

  \item \texttt{trans\_refl\_left} \hfill \textbf{(lrr)} \\
    $\refl(a) \comp p \;\rew\; p$.

  \item \texttt{trans\_refl\_right} \hfill \textbf{(rrr)} \\
    $p \comp \refl(b) \;\rew\; p$.

  \item \texttt{trans\_symm} \hfill \textbf{(tr)} \\
    $p \comp \inv{p} \;\rew\; \refl(a)$.

  \item \texttt{symm\_trans} \hfill \textbf{(tsr)} \\
    $\inv{p} \comp p \;\rew\; \refl(b)$.

  \item \texttt{symm\_trans\_congr} \hfill \textbf{(stss)} \\
    $\inv{(p \comp q)} \;\rew\; \inv{q} \comp \inv{p}$.

  \item \texttt{trans\_assoc} \hfill \textbf{(tt)} \\
    $(p \comp q) \comp r \;\rew\; p \comp (q \comp r)$.
\end{enumerate}

\section*{Group II: Type-Former $\beta$/$\eta$-Rules (17 rules)}

\paragraph{Map decomposition.}
\begin{enumerate}[label=\textbf{R\arabic*.}, leftmargin=3.5em, start=9]
  \item \texttt{map2\_subst} \hfill \textbf{(mx2l1)} \\
    $\mapTwo(f, p, q) \;\rew\; \mapRight(f, a_1, q) \comp \mapLeft(f, p, b_2)$.
\end{enumerate}

\paragraph{Product rules.}
\begin{enumerate}[label=\textbf{R\arabic*.}, leftmargin=3.5em, resume]
  \item \texttt{prod\_fst\_beta} \hfill \textbf{(mx2l2)} \\
    $\fst(\prodMk(p, q)) \;\rew\; p$.

  \item \texttt{prod\_snd\_beta} \hfill \textbf{(mx2r1)} \\
    $\snd(\prodMk(p, q)) \;\rew\; q$.

  \item \texttt{prod\_rec\_beta} \hfill \textbf{(mx2r2)} \\
    $\mathrm{rec}(f) \circ \prodMk(p, q) \;\rew\; \mapTwo(f, p, q)$.

  \item \texttt{prod\_eta} \hfill \textbf{(mxetaProd)} \\
    $\prodMk(\fst(r), \snd(r)) \;\rew\; r$.

  \item \texttt{prod\_mk\_symm} \hfill \textbf{(mx3l)} \\
    $\inv{\prodMk(p, q)} \;\rew\; \prodMk(\inv{p}, \inv{q})$.

  \item \texttt{prod\_map\_congrArg} \hfill \textbf{(mxc)} \\
    $\congrArgOp(f, \prodMk(p, q)) \;\rew\; \prodMk(\congrArgOp(\fst \circ f, p),
    \congrArgOp(\snd \circ f, q))$.
\end{enumerate}

\paragraph{Sigma type rules.}
\begin{enumerate}[label=\textbf{R\arabic*.}, leftmargin=3.5em, resume]
  \item \texttt{sigma\_fst\_beta} \hfill \textbf{(mxsigmaFst)} \\
    $\sigmaFst(\sigmaMk(p, q)) \;\rew\; \ofEq(\toEq(p))$.

  \item \texttt{sigma\_snd\_beta} \hfill \textbf{(mxsigmaSnd)} \\
    $\sigmaSnd(\sigmaMk(p, q)) \;\rew\; \ofEq(\toEq(q))$.

  \item \texttt{sigma\_eta} \hfill \textbf{(mxetaSigma)} \\
    $\sigmaMk(\sigmaFst(r), \sigmaSnd(r)) \;\rew\; r$.

  \item \texttt{sigma\_mk\_symm} \hfill \textbf{(smsigma)} \\
    $\inv{\sigmaMk(p, q)} \;\rew\; \sigmaMk(\inv{p}, \inv{q'})$
    where $q'$ incorporates the transport correction.
\end{enumerate}

\paragraph{Sum type rules.}
\begin{enumerate}[label=\textbf{R\arabic*.}, leftmargin=3.5em, resume]
  \item \texttt{sum\_rec\_inl\_beta} \hfill \textbf{(mxcase)} \\
    $(\mathrm{rec}(f, g))_*(\inlOp_*(p)) \;\rew\; f_*(p)$.

  \item \texttt{sum\_rec\_inr\_beta} \hfill \textbf{(mxcase)} \\
    $(\mathrm{rec}(f, g))_*(\inrOp_*(q)) \;\rew\; g_*(q)$.
\end{enumerate}

\paragraph{Function type rules.}
\begin{enumerate}[label=\textbf{R\arabic*.}, leftmargin=3.5em, resume]
  \item \texttt{fun\_app\_beta} \hfill \textbf{(mxlam)} \\
    $\app(\lamCongr(p), a) \;\rew\; p(a)$.

  \item \texttt{fun\_eta} \hfill \textbf{(mxetaFun)} \\
    $\lamCongr(\lambda x.\, \app(q, x)) \;\rew\; q$.

  \item \texttt{lam\_congr\_symm} \hfill \textbf{(mx3r)} \\
    $\inv{\lamCongr(p)} \;\rew\; \lamCongr(\lambda x.\, \inv{p(x)})$.
\end{enumerate}

\paragraph{Dependent application.}
\begin{enumerate}[label=\textbf{R\arabic*.}, leftmargin=3.5em, resume]
  \item \texttt{apd\_refl} \\
    $\apd(f, \refl(a)) \;\rew\; \refl(f(a))$.
\end{enumerate}

\section*{Group III: Transport Rules (5 rules)}

\begin{enumerate}[label=\textbf{R\arabic*.}, leftmargin=3.5em, start=26]
  \item \texttt{transport\_refl\_beta} \\
    $\tr_D(\refl(a), x) \;\rew\; x$.

  \item \texttt{transport\_trans\_beta} \\
    $\tr_D(p \comp q, x) \;\rew\; \tr_D(q, \tr_D(p, x))$.

  \item \texttt{transport\_symm\_left\_beta} \\
    $\tr_D(\inv{p}, \tr_D(p, x)) \;\rew\; x$.

  \item \texttt{transport\_symm\_right\_beta} \\
    $\tr_D(p, \tr_D(\inv{p}, y)) \;\rew\; y$.

  \item \texttt{transport\_sigmaMk\_fst\_beta} \\
    $\fst(\tr_\Sigma(\sigmaMk(p, q), x)) \;\rew\; \tr_D(p, \fst(x))$.
\end{enumerate}

\section*{Group IV: Context Rules (16 rules)}

Let $C : \Context(A, B)$ throughout.

\begin{enumerate}[label=\textbf{R\arabic*.}, leftmargin=3.5em, start=31]
  \item \texttt{context\_congr} \\
    $p \rew q \implies C[p] \;\rew\; C[q]$ \hfill (context congruence).

  \item \texttt{context\_map\_symm} \hfill \textbf{(smss)} \\
    $\inv{C[p]} \;\rew\; C[\inv{p}]$.

  \item \texttt{context\_tt\_cancel\_left} \hfill \textbf{(ttsv)} \\
    $C[p] \comp (C[\inv{p}] \comp v) \;\rew\; C[p \comp \inv{p}] \comp v$.

  \item \texttt{context\_tt\_cancel\_right} \hfill \textbf{(tstu)} \\
    $(v \comp C[p]) \comp C[\inv{p}] \;\rew\; v \comp C[p \comp \inv{p}]$.

  \item \texttt{context\_subst\_left\_beta} \hfill \textbf{(tsbll)} \\
    $r \comp C[p] \;\rew\; \substL(C, r, p)$.

  \item \texttt{context\_subst\_left\_of\_right} \\
    $\substL(C, \substR(C, p, t), q) \;\rew\; \substR(C, p, t \comp C[q])$.

  \item \texttt{context\_subst\_left\_assoc} \hfill \textbf{(tsblr)} \\
    $\substL(C, r, p) \comp t \;\rew\; r \comp \substR(C, p, t)$.

  \item \texttt{context\_subst\_right\_beta} \hfill \textbf{(tsbrl)} \\
    $C[p] \comp t \;\rew\; \substR(C, p, t)$.

  \item \texttt{context\_subst\_right\_assoc} \hfill \textbf{(tsbrr)} \\
    $\substR(C, p, t) \comp u \;\rew\; \substR(C, p, t \comp u)$.

  \item \texttt{context\_subst\_left\_refl\_right} \hfill \textbf{(slr)} \\
    $\substL(C, \refl, p) \;\rew\; C[p]$.

  \item \texttt{context\_subst\_left\_refl\_left} \hfill \textbf{(slss)} \\
    $\substL(C, \substL(C, r, \refl), p) \;\rew\; \substL(C, r, p)$.

  \item \texttt{context\_subst\_right\_refl\_left} \hfill \textbf{(srr)} \\
    $\substR(C, p, \refl) \;\rew\; C[p]$.

  \item \texttt{context\_subst\_right\_refl\_right} \hfill \textbf{(srrrr)} \\
    $\substR(C, \refl, \substR(C, p, t)) \;\rew\; \substR(C, p, t)$.

  \item \texttt{context\_subst\_left\_idempotent} \hfill \textbf{(slsss)} \\
    $\substL(C, C[\refl], p) \;\rew\; C[p]$.

  \item \texttt{context\_subst\_right\_cancel\_inner} \hfill \textbf{(srsr)} \\
    $\substR(C, p, \substR(C, \refl, t)) \;\rew\; \substR(C, p, t)$.

  \item \texttt{context\_subst\_right\_cancel\_outer} \\
    $\substR(C, p, \substR(C, q, t)) \;\rew\; \substR(C, p \comp q, t)$
    \hfill (under appropriate conditions).
\end{enumerate}

\section*{Group V: Dependent Context Rules (12 rules)}

Let $D : \DepContext(A, B)$ throughout. The rules are direct analogues of
Group~IV with the additional transport data required by the dependency:

\begin{enumerate}[label=\textbf{R\arabic*.}, leftmargin=3.5em, start=47]
  \item \texttt{depContext\_congr} \\
    $p \rew q \implies D[p] \;\rew\; D[q]$.

  \item \texttt{depContext\_map\_symm} \\
    $\inv{D[p]} \;\rew\; D[\inv{p}]$.

  \item \texttt{depContext\_subst\_left\_beta} \\
    $r \comp D[p] \;\rew\; \substL(D, r, p)$.

  \item \texttt{depContext\_subst\_left\_assoc} \\
    $\substL(D, r, p) \comp t \;\rew\; r \comp \substR(D, p, t)$.

  \item \texttt{depContext\_subst\_right\_beta} \\
    $D[p] \comp t \;\rew\; \substR(D, p, t)$.

  \item \texttt{depContext\_subst\_right\_assoc} \\
    $\substR(D, p, t) \comp u \;\rew\; \substR(D, p, t \comp u)$.

  \item \texttt{depContext\_subst\_left\_refl\_right} \\
    $\substL(D, \refl, p) \;\rew\; D[p]$.

  \item \texttt{depContext\_subst\_left\_refl\_left} \\
    $\substL(D, \substL(D, r, \refl), p) \;\rew\; \substL(D, r, p)$.

  \item \texttt{depContext\_subst\_right\_refl\_left} \\
    $\substR(D, p, \refl) \;\rew\; D[p]$.

  \item \texttt{depContext\_subst\_right\_refl\_right} \\
    $\substR(D, \refl, \substR(D, p, t)) \;\rew\; \substR(D, p, t)$.

  \item \texttt{depContext\_subst\_left\_idempotent} \\
    $\substL(D, D[\refl], p) \;\rew\; D[p]$.

  \item \texttt{depContext\_subst\_right\_cancel\_inner} \\
    $\substR(D, p, \substR(D, \refl, t)) \;\rew\; \substR(D, p, t)$.
\end{enumerate}

\section*{Group VI: Bi-Context Rules (8 rules)}

These rules govern the congruence properties of binary contexts
$\BiContext(A, B, C)$ and dependent binary contexts $\mathrm{DepBiContext}$:

\begin{enumerate}[label=\textbf{R\arabic*.}, leftmargin=3.5em, start=59]
  \item \texttt{depBiContext\_mapLeft\_congr} \\
    $p \rew q \implies \mapLeft(D_2, p, b) \;\rew\; \mapLeft(D_2, q, b)$.

  \item \texttt{depBiContext\_mapRight\_congr} \\
    $p \rew q \implies \mapRight(D_2, a, p) \;\rew\; \mapRight(D_2, a, q)$.

  \item \texttt{depBiContext\_map2\_congr\_left} \\
    $p \rew q \implies \mapTwo(D_2, p, r) \;\rew\; \mapTwo(D_2, q, r)$.

  \item \texttt{depBiContext\_map2\_congr\_right} \\
    $p \rew q \implies \mapTwo(D_2, r, p) \;\rew\; \mapTwo(D_2, r, q)$.

  \item \texttt{biContext\_mapLeft\_congr} \\
    $p \rew q \implies \mapLeft(B_2, p, b) \;\rew\; \mapLeft(B_2, q, b)$.

  \item \texttt{biContext\_mapRight\_congr} \\
    $p \rew q \implies \mapRight(B_2, a, p) \;\rew\; \mapRight(B_2, a, q)$.

  \item \texttt{biContext\_map2\_congr\_left} \\
    $p \rew q \implies \mapTwo(B_2, p, r) \;\rew\; \mapTwo(B_2, q, r)$.

  \item \texttt{biContext\_map2\_congr\_right} \\
    $p \rew q \implies \mapTwo(B_2, r, p) \;\rew\; \mapTwo(B_2, r, q)$.
\end{enumerate}

\section*{Group VII: Map Congruence Rules (4 rules)}

\begin{enumerate}[label=\textbf{R\arabic*.}, leftmargin=3.5em, start=67]
  \item \texttt{mapLeft\_congr} \\
    $p \rew q \implies \mapLeft(f, p, b) \;\rew\; \mapLeft(f, q, b)$.

  \item \texttt{mapRight\_congr} \\
    $p \rew q \implies \mapRight(f, a, p) \;\rew\; \mapRight(f, a, q)$.

  \item \texttt{mapLeft\_ofEq} \hfill \textbf{(mxp)} \\
    $\mapLeft(f, \ofEq(h), b) \;\rew\; \ofEq(\congrArgOp(\lambda x.\, f\,x\,b,\, h))$.

  \item \texttt{mapRight\_ofEq} \hfill \textbf{(nxp)} \\
    $\mapRight(f, a, \ofEq(h)) \;\rew\; \ofEq(\congrArgOp(f\,a,\, h))$.
\end{enumerate}

\section*{Group VIII: Structural Closure (4 rules)}

These rules propagate single-step rewrites through all path constructors,
ensuring that the rewrite relation is a congruence:

\begin{enumerate}[label=\textbf{R\arabic*.}, leftmargin=3.5em, start=71]
  \item \texttt{symm\_congr} \\
    $p \rew q \implies \inv{p} \;\rew\; \inv{q}$.

  \item \texttt{trans\_congr\_left} \\
    $p \rew q \implies p \comp r \;\rew\; q \comp r$.

  \item \texttt{trans\_congr\_right} \\
    $q \rew r \implies p \comp q \;\rew\; p \comp r$.

  \item \texttt{context\_congr} \\
    $p \rew q \implies C[p] \;\rew\; C[q]$ \hfill (already listed as R31).
\end{enumerate}

\bigskip

\begin{remark}
  The total count of 76 includes Rule~R31/R74 once (context congruence
  appears in both Group~IV as a context-specific rule and in Group~VIII
  as a structural closure principle; it is a single constructor
  \texttt{context\_congr} in the \texttt{Step} type). The formalization
  contains exactly 74 distinct constructors, with context congruence
  serving double duty. The paper count of 76 follows the convention
  of~\cite{RDQO18}, which lists certain rules in multiple groups for
  expository clarity.
\end{remark}

% ============================================================================
% Appendix B: Index of Definitions and Theorems
% ============================================================================
\chapter{Index of Definitions and Theorems}
\label{app:index}

The following table maps the principal definitions and theorems in this
paper to their corresponding locations in the Lean~4 formalization.
Module paths are relative to the root namespace
\texttt{ComputationalPaths}.

\section*{Part I: Foundations}

\small
\begin{longtable}{p{3.8cm}p{2.5cm}p{5.5cm}}
\toprule
\textbf{Paper Reference} & \textbf{Type} & \textbf{Lean Module / Name} \\
\midrule
\endfirsthead
\toprule
\textbf{Paper Reference} & \textbf{Type} & \textbf{Lean Module / Name} \\
\midrule
\endhead
\bottomrule
\endfoot

\multicolumn{3}{l}{\textbf{Chapter 1: Introduction}} \\
\cmidrule(l){1-3}
Def.~\ref{def:step} (Step) & structure & \texttt{Path.Basic.ElemStep} \\
Def.~\ref{def:path} (Path) & structure & \texttt{Path.Basic.Path} \\
Def.~\ref{def:toEq} (toEq) & def & \texttt{Path.Basic.Path.toEq} \\
Def.~\ref{def:ofEq} (ofEq) & def & \texttt{Path.Basic.Path.ofEq} \\
Thm.~\ref{thm:non-uip} (Non-UIP) & theorem & \texttt{Path.Basic.Path.nonUIP} \\
\midrule

\multicolumn{3}{l}{\textbf{Chapter 2: Basic Constructions}} \\
\cmidrule(l){1-3}
Def.~2.3 (refl, symm, trans) & def & \texttt{Path.Basic.Path.refl/symm/trans} \\
Thm.~2.4 (Monoid laws) & theorem & \texttt{Path.Basic.trans\_assoc}, etc. \\
Thm.~2.5 (Involution) & theorem & \texttt{Path.Basic.symm\_symm} \\
Thm.~2.6 (Anti-homomorphism) & theorem & \texttt{Path.Basic.symm\_trans} \\
Def.~2.7 (congrArg) & def & \texttt{Path.Basic.Congruence.congrArg} \\
Thm.~2.8 (Functoriality) & theorem & \texttt{Path.Basic.Congruence.congrArg\_trans} \\
Def.~2.9 (Binary congr.) & def & \texttt{Path.Basic.Congruence.map2} \\
Def.~2.10 (Transport) & def & \texttt{Path.Basic.Path.transport} \\
Def.~2.12 (apd) & def & \texttt{Path.Basic.Path.apd} \\
Thm.~2.13 (Product $\beta$/$\eta$) & theorem & \texttt{Path.Basic.Prod.*} \\
Def.~2.17 (Context) & structure & \texttt{Path.Basic.Context.Context} \\
Def.~2.18 (substLeft/Right) & def & \texttt{Path.Basic.Context.substLeft/Right} \\
\midrule

\multicolumn{3}{l}{\textbf{Chapter 3: The Rewrite System}} \\
\cmidrule(l){1-3}
Def.~\ref{def:step-rewrite} (Step relation) & inductive & \texttt{Path.Rewrite.Step.Step} \\
Thm.~\ref{thm:step-sound} (Soundness) & theorem & \texttt{Path.Rewrite.Step.step\_toEq} \\
Def.~\ref{def:rw} (Rw) & inductive & \texttt{Path.Rewrite.Rw.Rw} \\
Def.~\ref{def:rweq} (RwEq) & inductive & \texttt{Path.Rewrite.RwEq.RwEq} \\
Thm.~\ref{thm:rweq-congruence} (Congruence) & theorem & \texttt{Path.Rewrite.RwEq.rweq\_*} \\
Def.~\ref{def:normal-form} (Normal form) & def & \texttt{Path.Rewrite.Normalization.normalize} \\
Thm.~\ref{thm:normalization} (Normalization) & theorem & \texttt{Path.Rewrite.Normalization.*} \\
Thm.~\ref{thm:termination} (Termination) & theorem & \texttt{Path.Rewrite.Termination.*} \\
Thm.~\ref{thm:strip-lemma} (Strip lemma) & theorem & \texttt{Path.Rewrite.StripLemma.*} \\
Thm.~\ref{thm:confluence} (Confluence) & theorem & \texttt{Path.Rewrite.Confluence.*} \\
Def.~\ref{def:path-quot} (PathQuot) & def & \texttt{Path.Rewrite.Quot.PathRwQuot} \\
Thm.~\ref{thm:quot-equiv} (Quotient equiv.) & theorem & \texttt{Path.Rewrite.SimpleEquiv.*} \\
\midrule

\multicolumn{3}{l}{\textbf{Chapter 4: The Groupoid}} \\
\cmidrule(l){1-3}
Def.~4.1 (WeakCategory) & structure & \texttt{Groupoid.WeakCategory} \\
Thm.~4.2 (Canonical weak cat.) & theorem & \texttt{Groupoid.weakCategory\_of\_path} \\
Def.~4.3 (WeakGroupoid) & structure & \texttt{Groupoid.WeakGroupoid} \\
Thm.~4.4 (Type is weak gpd.) & theorem & \texttt{Groupoid.weakGroupoid\_of\_path} \\
Thm.~4.5 (Strict quotient gpd.) & theorem & \texttt{Groupoid.strictGroupoid\_quot} \\
Def.~4.6 (RewriteLift) & structure & \texttt{Groupoid.RewriteLift} \\
\midrule

\multicolumn{3}{l}{\textbf{Chapter 5: Higher-Dimensional Structure}} \\
\cmidrule(l){1-3}
Def.~\ref{def:two-cell} (Two-cell) & def & \texttt{HigherDimensional.TwoCell} \\
Def.~\ref{def:globular-cell} (GlobularCell) & structure & \texttt{HigherDimensional.GlobularCell} \\
Def.~\ref{def:derivation2} (Derivation$_2$) & inductive & \texttt{HigherDimensional.Derivation2} \\
Def.~\ref{def:derivation3} (Derivation$_3$) & inductive & \texttt{HigherDimensional.Derivation3} \\
Thm.~\ref{thm:contract3} (Contractibility $\ge 3$) & theorem & \texttt{HigherDimensional.contract3} \\
Thm.~\ref{thm:omega-groupoid} ($\omega$-groupoid) & theorem & \texttt{HigherDimensional.omegaGroupoid} \\
\end{longtable}

\section*{Part II: Homotopy Theory}

\small
\begin{longtable}{p{3.8cm}p{2.5cm}p{5.5cm}}
\toprule
\textbf{Paper Reference} & \textbf{Type} & \textbf{Lean Module / Name} \\
\midrule
\endfirsthead
\toprule
\textbf{Paper Reference} & \textbf{Type} & \textbf{Lean Module / Name} \\
\midrule
\endhead
\bottomrule
\endfoot

\multicolumn{3}{l}{\textbf{Chapter 6: Fundamental Groups}} \\
\cmidrule(l){1-3}
Def.~\ref{def:loop-space} (Loop space) & def & \texttt{HomotopyTheory.LoopSpace} \\
Def.~\ref{def:fundamental-group} ($\pi_1$) & def & \texttt{HomotopyTheory.PiOne} \\
Thm.~\ref{thm:pi1-group} (Group axioms) & theorem & \texttt{HomotopyTheory.PiOne.group} \\
Thm.~\ref{thm:induced-pi1} (Induced hom.) & theorem & \texttt{HomotopyTheory.PiOne.induced} \\
Thm.~\ref{thm:pi1-product} (Product formula) & theorem & \texttt{HomotopyTheory.PiOne.prod\_iso} \\
Thm.~\ref{thm:eckmann-hilton} (Eckmann--Hilton) & theorem & \texttt{HomotopyTheory.EckmannHilton} \\
Def.~\ref{def:fundamental-groupoid} ($\Pi_1$) & structure & \texttt{HomotopyTheory.FundamentalGroupoid} \\
\midrule

\multicolumn{3}{l}{\textbf{Chapter 7: Spaces}} \\
\cmidrule(l){1-3}
Def.~\ref{def:circle} (Circle) & structure & \texttt{HomotopyTheory.Circle.CircleCompPath} \\
Thm.~\ref{thm:pi1-circle} ($\pi_1(S^1) \cong \ZZ$) & theorem & \texttt{HomotopyTheory.Circle.pi1\_iso} \\
Thm.~\ref{thm:pi1-torus} ($\pi_1(T^2)$) & theorem & \texttt{HomotopyTheory.Torus.pi1\_iso} \\
Thm.~\ref{thm:pi1-figure-eight} ($\pi_1(S^1 \vee S^1)$) & theorem & \texttt{HomotopyTheory.FigureEight.pi1\_iso} \\
Thm.~\ref{thm:svk} (Seifert--van Kampen) & theorem & \texttt{HomotopyTheory.VanKampen.svk} \\
\midrule

\multicolumn{3}{l}{\textbf{Chapter 8: Fibrations}} \\
\cmidrule(l){1-3}
Def.~\ref{def:fiber} (Fiber) & def & \texttt{Fibration.Fiber} \\
Thm.~\ref{thm:path-lifting} (Path lifting) & theorem & \texttt{Fibration.pathLifting} \\
Def.~\ref{def:covering-space} (Covering space) & structure & \texttt{Fibration.CoveringSpace} \\
Def.~\ref{def:hopf} (Hopf fibration) & structure & \texttt{Fibration.Hopf.HopfData} \\
Thm.~\ref{thm:les} (Long exact sequence) & theorem & \texttt{Fibration.LES.les} \\
\midrule

\multicolumn{3}{l}{\textbf{Chapter 9: Hurewicz Theorem}} \\
\cmidrule(l){1-3}
Def.~\ref{def:abelianization} (Abelianization) & def & \texttt{Homological.Abelianization} \\
Def.~\ref{def:hurewicz-map} (Hurewicz map) & def & \texttt{Homological.Hurewicz.hurewiczMap} \\
Thm.~\ref{thm:hurewicz} (Hurewicz theorem) & theorem & \texttt{Homological.Hurewicz.hurewicz\_iso} \\
\midrule

\multicolumn{3}{l}{\textbf{Chapter 10: Advanced}} \\
\cmidrule(l){1-3}
Def.~\ref{def:EM-space} (EM space) & structure & \texttt{Advanced.EilenbergMacLane} \\
Def.~\ref{def:postnikov-tower} (Postnikov) & structure & \texttt{Advanced.Postnikov.Tower} \\
Thm.~\ref{thm:whitehead} (Whitehead) & theorem & \texttt{Advanced.Whitehead.whitehead} \\
Def.~\ref{def:spectral-seq} (Spectral seq.) & structure & \texttt{Advanced.SpectralSeq.Sequence} \\
\midrule

\multicolumn{3}{l}{\textbf{Chapter 11: Metatheory}} \\
\cmidrule(l){1-3}
Def.~\ref{def:pathexpr} (PathExpr) & inductive & \texttt{Path.Rewrite.PathExpr.PathExpr} \\
Def.~\ref{def:pathexpr-eval} (Evaluation) & def & \texttt{Path.Rewrite.PathExpr.eval} \\
Thm.~\ref{thm:pathexpr-confluence} (Expr conf.) & theorem & \texttt{Path.Rewrite.ExprConfluence.*} \\
Thm.~\ref{thm:rweq-decidable} (Decidability) & theorem & \texttt{Path.Rewrite.Normalization.rweq\_dec} \\
Def.~\ref{def:path-tactics} (Tactics) & tactic & \texttt{Path.Rewrite.PathTactic.*} \\
\end{longtable}

\normalsize
\begin{remark}
  Module names are indicative and may vary slightly across
  formalization versions. The definitive module paths are recorded in the
  \texttt{ComputationalPaths.lean} manifest file at the repository root.
\end{remark}

% ============================================================================
% Appendix C: Dependency Graph
% ============================================================================
\chapter{Dependency Graph}
\label{app:dependency}

The formalization comprises 516 Lean~4 source files organized into a
layered dependency structure. This appendix describes the principal
layers and their inter-dependencies. A schematic representation is given
in Figure~\ref{fig:dependency}.

\section*{Layer 0: Basic Path Algebra}

The foundation consists of the core path definitions and their algebraic
laws:

\begin{center}
\small
\begin{tabular}{lp{8cm}}
\texttt{Path.Basic} & Path record type, elementary operations (refl,
  symm, trans, ofEq, toEq), strict monoid/involution/anti-homomorphism
  laws. \\
\texttt{Path.Basic.Congruence} & Unary congruence (congrArg), binary
  congruence (map2, mapLeft, mapRight), and functoriality theorems. \\
\texttt{Path.Basic.Context} & Context, BiContext, DepContext,
  DepBiContext structures and their substitution operations. \\
\texttt{Path.Basic.Transport} & Transport, dependent application (apd),
  and their laws. \\
\texttt{Path.Basic.Prod} & Product type path operations and
  $\beta$/$\eta$-rules. \\
\texttt{Path.Basic.Sigma} & Sigma type path operations. \\
\texttt{Path.Basic.Sum} & Sum type path operations. \\
\texttt{Path.Basic.Fun} & Function type path operations (lamCongr,
  app). \\
\end{tabular}
\end{center}

\section*{Layer 1: The Rewrite System}

Built on Layer~0, this layer contains the TRS infrastructure:

\begin{center}
\small
\begin{tabular}{lp{8cm}}
\texttt{Path.Rewrite.Step} & The 75-rule \texttt{Step} inductive type
  and soundness proof. \\
\texttt{Path.Rewrite.Rw} & Reflexive--transitive closure. \\
\texttt{Path.Rewrite.RwEq} & Rewrite equality and all congruence
  lemmas. \\
\texttt{Path.Rewrite.LNDEQ} & Rule enumeration and mnemonic names. \\
\texttt{Path.Rewrite.Normalization} & Normal forms. \\
\texttt{Path.Rewrite.Termination} & Rule precedence and RPO. \\
\texttt{Path.Rewrite.StripLemma} & Local confluence. \\
\texttt{Path.Rewrite.Confluence*} & Global confluence and join
  witnesses. \\
\texttt{Path.Rewrite.PathExpr} & Syntactic path expressions. \\
\texttt{Path.Rewrite.Quot} & The quotient $\PathQuot$. \\
\texttt{Path.Rewrite.PathTactic} & Proof automation. \\
\end{tabular}
\end{center}

\section*{Layer 2: Groupoid and Higher-Dimensional Structure}

Built on Layers~0--1:

\begin{center}
\small
\begin{tabular}{lp{8cm}}
\texttt{Groupoid.*} & Weak and strict category/groupoid structures,
  rewrite lifts, enriched and double groupoid, symmetric monoidal
  structure. \\
\texttt{HigherDimensional.*} & Globular tower, two-cells, Derivation$_2$
  through Derivation$_4$ and DerivationHigh, contractibility theorems,
  $\omega$-groupoid assembly. \\
\end{tabular}
\end{center}

\section*{Layer 3: Homotopy Theory}

Built on Layers~0--2:

\begin{center}
\small
\begin{tabular}{lp{8cm}}
\texttt{HomotopyTheory.LoopSpace} & Loop spaces, iterated loop spaces,
  loop quotient, loop group. \\
\texttt{HomotopyTheory.PiOne} & Fundamental group, induced
  homomorphisms, functoriality. \\
\texttt{HomotopyTheory.FundamentalGroupoid} & The fundamental groupoid
  $\Pi_1(A)$. \\
\texttt{HomotopyTheory.EckmannHilton} & The Eckmann--Hilton argument
  and commutativity of $\pi_n$ for $n \ge 2$. \\
\texttt{HomotopyTheory.Circle} & The computational circle and
  $\pi_1(S^1) \cong \ZZ$. \\
\texttt{HomotopyTheory.Torus} & The torus and its fundamental group. \\
\texttt{HomotopyTheory.FigureEight} & The figure-eight, free products,
  bouquets. \\
\texttt{HomotopyTheory.VanKampen} & Seifert--van Kampen theorem. \\
\texttt{HomotopyTheory.Suspension} & Suspensions, spheres,
  Freudenthal. \\
\texttt{HomotopyTheory.KleinBottle} & Klein bottle,
  $\pi_1 \cong \ZZ \rtimes \ZZ$. \\
\texttt{HomotopyTheory.LensSpace} & Lens spaces and projective
  spaces. \\
\end{tabular}
\end{center}

\section*{Layer 4: Fibrations and Exact Sequences}

Built on Layers~0--3:

\begin{center}
\small
\begin{tabular}{lp{8cm}}
\texttt{Fibration.*} & Fiber, path lifting, fiber transport, fiber
  sequences, covering spaces, Hopf fibration. \\
\texttt{Fibration.LES} & Long exact sequence of homotopy groups. \\
\texttt{Fibration.MayerVietoris} & Mayer--Vietoris sequence. \\
\end{tabular}
\end{center}

\section*{Layer 5: Homological Algebra and Advanced Topics}

Built on Layers~0--4:

\begin{center}
\small
\begin{tabular}{lp{8cm}}
\texttt{Homological.*} & Abelianization, commutators, Hurewicz map and
  theorem, free groups, resolutions, Ext/Tor. \\
\texttt{Advanced.*} & Eilenberg--MacLane spaces, Postnikov towers,
  obstruction theory, Whitehead theorem, spectral sequences,
  characteristic classes, operads, K-theory, stable homotopy. \\
\end{tabular}
\end{center}

\section*{Dependency Diagram}

\begin{figure}[ht]
\centering
\begin{tikzcd}[column sep=small, row sep=large]
  & \fbox{\texttt{Path.Basic}} \ar[dl] \ar[d] \ar[dr] & \\
  \fbox{\texttt{Rewrite.Step}} \ar[d]
  & \fbox{\texttt{Rewrite.PathExpr}} \ar[d]
  & \fbox{\texttt{Basic.Context}} \ar[dl] \\
  \fbox{\texttt{Rw $\to$ RwEq}} \ar[d] \ar[dr]
  & \fbox{\texttt{ExprConfluence}} \ar[dl] & \\
  \fbox{\texttt{Confluence}} \ar[d]
  & \fbox{\texttt{Normalization}} \ar[l] & \\
  \fbox{\texttt{Quot}} \ar[d] \ar[dr] & & \\
  \fbox{\texttt{Groupoid}} \ar[d]
  & \fbox{\texttt{HigherDimensional}} \ar[d] & \\
  \fbox{\texttt{HomotopyTheory}} \ar[d] \ar[dr]
  & \fbox{\texttt{$\omega$-Groupoid}} \ar[l] & \\
  \fbox{\texttt{Fibration}} \ar[d]
  & \fbox{\texttt{Spaces ($\pi_1$)}} \ar[l] & \\
  \fbox{\texttt{Homological}} \ar[d] & & \\
  \fbox{\texttt{Advanced}} & &
\end{tikzcd}
\caption{Schematic dependency graph of the formalization. Arrows point
from dependency to dependent module. Boxes represent module groups.
The foundational kernel (\texttt{Path.Basic} $\to$ \texttt{Rewrite}
$\to$ \texttt{Quot}) is the critical path on which all subsequent
development depends.}
\label{fig:dependency}
\end{figure}

\begin{remark}
  The dependency graph is acyclic by construction. The longest dependency
  chain runs from \texttt{Path.Basic} through
  \texttt{Rewrite} $\to$ \texttt{Groupoid} $\to$
  \texttt{HigherDimensional} $\to$ \texttt{HomotopyTheory} $\to$
  \texttt{Fibration} $\to$ \texttt{Homological} $\to$
  \texttt{Advanced}, spanning all seven layers. Each layer adds new
  mathematical content while depending only on the layers below.
\end{remark}


% ============================================================================
%  BIBLIOGRAPHY
% ============================================================================
\bibliographystyle{alpha}
\bibliography{references}

\end{document}
