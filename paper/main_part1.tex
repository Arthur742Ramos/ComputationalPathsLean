% ============================================================================
% The Calculus of Computational Paths — Part I: Foundations
% ============================================================================
\documentclass[11pt,a4paper]{report}

% --- Encoding & Fonts ---
\usepackage[utf8]{inputenc}
\usepackage[T1]{fontenc}
\usepackage{lmodern}
\usepackage{microtype}

% --- Mathematics ---
\usepackage{amsmath,amssymb,amsthm,mathtools}
\usepackage{stmaryrd}        % \llbracket, \rrbracket
\usepackage{bm}

% --- Cross-references & Hyperlinks ---
\usepackage[colorlinks=true,linkcolor=blue!60!black,citecolor=green!50!black,urlcolor=blue!70!black]{hyperref}
\usepackage[capitalize,noabbrev]{cleveref}

% --- Layout ---
\usepackage[margin=1in]{geometry}
\usepackage{enumitem}

% --- Diagrams ---
\usepackage{tikz-cd}

% ============================================================================
% Theorem-like environments
% ============================================================================
\theoremstyle{plain}
\newtheorem{theorem}{Theorem}[chapter]
\newtheorem{proposition}[theorem]{Proposition}
\newtheorem{lemma}[theorem]{Lemma}
\newtheorem{corollary}[theorem]{Corollary}

\theoremstyle{definition}
\newtheorem{definition}[theorem]{Definition}
\newtheorem{example}[theorem]{Example}
\newtheorem{notation_def}[theorem]{Notation}

\theoremstyle{remark}
\newtheorem{remark}[theorem]{Remark}

% ============================================================================
% Custom operators and notation
% ============================================================================
\DeclareMathOperator{\refl}{refl}
\DeclareMathOperator{\symop}{symm}
\DeclareMathOperator{\ofEq}{ofEq}
\DeclareMathOperator{\toEq}{toEq}
\DeclareMathOperator{\tr}{transport}
\DeclareMathOperator{\apd}{apd}
\DeclareMathOperator{\id}{id}

\DeclareMathOperator{\congrArgOp}{congrArg}
\DeclareMathOperator{\mapLeft}{mapLeft}
\DeclareMathOperator{\mapRight}{mapRight}
\DeclareMathOperator{\mapTwo}{map_2}
\DeclareMathOperator{\prodMk}{prodMk}
\DeclareMathOperator{\fst}{fst}
\DeclareMathOperator{\snd}{snd}
\DeclareMathOperator{\inlOp}{inl}
\DeclareMathOperator{\inrOp}{inr}
\DeclareMathOperator{\sigmaMk}{sigmaMk}
\DeclareMathOperator{\sigmaFst}{sigmaFst}
\DeclareMathOperator{\sigmaSnd}{sigmaSnd}
\DeclareMathOperator{\lamCongr}{lamCongr}
\DeclareMathOperator{\app}{app}

\DeclareMathOperator{\substL}{substL}
\DeclareMathOperator{\substR}{substR}
\DeclareMathOperator{\normalize}{normalize}

% Short-hand
\newcommand{\Path}{\operatorname{Path}}
\newcommand{\Step}{\operatorname{Step}}
\newcommand{\Rw}{\operatorname{Rw}}
\newcommand{\RwEq}{\operatorname{RwEq}}
\newcommand{\PathQuot}{\operatorname{PathQuot}}
\newcommand{\Context}{\operatorname{Context}}
\newcommand{\BiContext}{\operatorname{BiContext}}
\newcommand{\DepContext}{\operatorname{DepContext}}
\newcommand{\Derivation}{\operatorname{D}}

% Rewrite arrows
\newcommand{\rew}{\mathbin{\triangleright}}         % single step
\newcommand{\rews}{\mathbin{\triangleright^{*}}}    % multi-step
\newcommand{\rweq}{\mathbin{\approx}}               % rewrite equality

% Misc
\newcommand{\comp}{\mathbin{\cdot}}
\newcommand{\inv}[1]{{#1}^{-1}}
\newcommand{\ofEqfn}[1]{\ofEq(#1)}
\newcommand{\Nat}{\mathbb{N}}
\newcommand{\ZZ}{\mathbb{Z}}
\newcommand{\Sort}{\mathsf{Sort}}
\newcommand{\Type}{\mathsf{Type}}
\newcommand{\Prop}{\mathsf{Prop}}
\newcommand{\List}{\operatorname{List}}
\newcommand{\Eq}{\operatorname{Eq}}
\newcommand{\IdA}{\operatorname{Id}}

% ============================================================================
\title{\textbf{The Calculus of Computational Paths}\\[6pt]
  \large Part I: Foundations}

\author{
  Arthur Ferreira Ramos\thanks{Centro de Informática, Universidade Federal de Pernambuco, Recife, Brazil.}
  \and
  Ruy J.\,G.\,B.\ de Queiroz\footnotemark[1]
  \and
  Anjolina G.\ de Oliveira\footnotemark[1]
}
\date{\today}

% ============================================================================
\begin{document}
\maketitle

\begin{abstract}
We develop a comprehensive mathematical theory of \emph{computational paths}---a
formalism in which propositional equalities between elements of a type carry
explicit rewrite traces recording the sequence of elementary steps by which they
were derived. A computational path from $a$ to~$b$ in a type~$A$ is a pair
$(s,\pi)$ where $\pi : a =_A b$ is a propositional equality and $s$ is a finite
list of elementary rewrite steps. The space of computational paths is equipped
with a confluent, terminating rewrite system whose 76~rules are organized into
eight groups covering path algebra, type-former $\beta/\eta$-rules, transport,
contexts, and structural closure. We prove that the quotient by rewrite equality
recovers the standard identity type as a strict groupoid, while the
non-quotiented path space forms a weak groupoid. Most significantly, we establish
that the tower of iterated derivation cells forms a weak $\omega$-groupoid in the
sense of Batanin--Leinster, with contractibility beginning at dimension~3---the
critical threshold that preserves non-trivial fundamental groups. The entire
development is formalized in Lean~4 (302~files, 60,860~lines, 1,710~definitions,
885~theorems).
\end{abstract}

\tableofcontents

% ============================================================================
% ============================================================================
% Chapter 1: Introduction and Motivation
% ============================================================================
\chapter{Introduction and Motivation}
\label{ch:introduction}

\section{The Curry--Howard--de~Bruijn Correspondence and Propositional Equality}
\label{sec:curry-howard}

In Martin-L\"of type theory, the \emph{identity type} $\IdA_A(a,b)$ captures
propositional equality between elements $a,b : A$. Its sole introduction rule
is reflexivity: the term $\refl(a) : \IdA_A(a,a)$ witnesses that every element
is equal to itself. The elimination rule---\emph{path induction}, also known as
the~$J$-rule---states that to prove a property of an arbitrary inhabitant of
$\IdA_A(a,b)$, it suffices to verify the property for $\refl(a)$.

In proof assistants based on the Calculus of Inductive Constructions, such as
Lean~4 and Coq, the identity type \texttt{Eq} lives in $\Prop$, a universe
governed by \emph{proof irrelevance}: all inhabitants of a proposition are
definitionally identified. As a consequence, the \emph{Uniqueness of Identity
Proofs} (UIP) principle holds:

\begin{equation}\label{eq:uip}
  \forall\, p, q : \IdA_A(a,b),\quad p =_{\IdA_A(a,b)} q.
\end{equation}

This axiom collapses the entire space of identity proofs to at most one element
per pair of endpoints. From the perspective of homotopy type theory~\cite{HoTTBook},
UIP asserts that every type is a \emph{set} (a $0$-truncated type), precluding
the rich higher-dimensional structure that identity types can carry in
intensional type theory.

\section{The Computational Paths Program}
\label{sec:comp-paths-program}

Following de~Queiroz, de~Oliveira, and Ramos~\cite{DQOR16, RDQO18}, we propose
that equality proofs carry \emph{computational content}: the sequence of
rewriting steps that produced them. Even when the underlying logic satisfies
UIP---so that the propositional equality $a =_A b$ is proof-irrelevant---the
rewrite \emph{traces} are distinct combinatorial objects that can be compared,
composed, and quotiented.

The key insight is a separation of concerns:
\begin{itemize}[leftmargin=2em]
  \item The \textbf{semantic content} of an equality proof is the proposition
    $a =_A b$, which by UIP carries no information beyond its truth value.
  \item The \textbf{computational trace} is a finite sequence of elementary
    rewrite steps recording \emph{how} the equality was derived---which
    congruence rules, symmetries, transitivities, and $\beta/\eta$-reductions
    were applied, and in what order.
\end{itemize}

This separation creates a rich algebraic structure \emph{atop} the
proof-irrelevant equality, without modifying the underlying type theory.

\section{Design Principle: Path = Proof + Trace}
\label{sec:design-principle}

We now state the central definitions that constitute the formal framework.

\begin{definition}[Elementary Rewrite Step]\label{def:step}
  An \emph{elementary rewrite step} in a type~$A$ is a triple
  \[
    s = (\mathrm{src}, \mathrm{tgt}, \pi) \quad\text{where}\quad
    \mathrm{src}, \mathrm{tgt} : A \quad\text{and}\quad
    \pi : \mathrm{src} =_A \mathrm{tgt}.
  \]
  We write $\Step(A)$ for the type of all elementary rewrite steps in~$A$.
\end{definition}

Each step records a single atomic equation between two elements together with
its justification. Steps can be inverted (swapping source and target) and
mapped through functions.

\begin{definition}[Computational Path]\label{def:path}
  A \emph{computational path} from $a$ to $b$ in a type~$A$ is a pair
  \[
    p = (s, \pi) \quad\text{where}\quad
    s : \List(\Step(A)) \quad\text{and}\quad
    \pi : a =_A b.
  \]
  We write $\Path_A(a,b)$ for the type of computational paths from~$a$ to~$b$.
\end{definition}

The list $s$ is the \emph{computational trace}---it records which elementary
steps were applied. The proof $\pi$ is the \emph{semantic witness}---it
certifies that the composite rewrite is valid. Two paths with the same
endpoints but different traces are \emph{distinct} as elements of $\Path_A(a,b)$,
even though their proof fields are identified by UIP.

\begin{definition}[Semantic Projection]\label{def:toEq}
  The \emph{semantic projection} $\toEq : \Path_A(a,b) \to (a =_A b)$ extracts
  the underlying propositional equality from a computational path, discarding
  the trace:
  \[
    \toEq(s, \pi) \;=\; \pi.
  \]
\end{definition}

\begin{definition}[Canonical Witness]\label{def:ofEq}
  For any propositional equality $\pi : a =_A b$, the \emph{canonical witness}
  is the single-step path
  \[
    \ofEq(\pi) \;=\; \bigl([\langle a, b, \pi\rangle],\; \pi\bigr) \;:\; \Path_A(a,b).
  \]
\end{definition}

The function $\ofEq$ embeds the standard identity type into the computational
path space. Its image consists precisely of the one-step paths.

\section{Non-UIP for Computational Paths}
\label{sec:non-uip}

The raison d'\^etre of the computational paths framework is that it recovers
higher-dimensional structure even in a proof-irrelevant setting:

\begin{theorem}[Non-UIP for Paths]\label{thm:non-uip}
  For any nonempty type~$A$, the space of computational paths does not satisfy
  the Uniqueness of Identity Proofs:
  \[
    \neg\,\bigl(\forall\, a, b : A,\;\forall\, p, q : \Path_A(a,b),\;
    p = q\bigr).
  \]
\end{theorem}

\begin{proof}
  Let $a : A$ be any element. Consider two paths from $a$ to itself:
  \begin{align*}
    p &\;=\; \refl(a) \;=\; ([\,],\; \refl) \;:\; \Path_A(a,a), \\
    q &\;=\; \ofEq(\refl) \;=\; \bigl([\langle a, a, \refl\rangle],\;
    \refl\bigr) \;:\; \Path_A(a,a).
  \end{align*}
  These have the same proof field ($\refl$) but differ in their step lists:
  $p.\mathrm{steps} = [\,]$ while $q.\mathrm{steps} = [\langle a, a,
  \refl\rangle]$. Since the step lists are structurally distinct,
  $p \neq q$ as elements of $\Path_A(a,a)$.
\end{proof}

\begin{remark}\label{rem:key-distinction}
  \Cref{thm:non-uip} is the foundational result that enables the entire
  subsequent development. It shows that even in a type theory where all
  identity proofs are identified (UIP holds for $\Eq$), the space
  $\Path_A(a,b)$ supports a non-trivial ``higher'' structure. The extra
  information resides in the trace, not in the equality proof.
\end{remark}

\section{Related Work}
\label{sec:related-work}

Our work connects to several strands of research in type theory and
higher-dimensional algebra.

\paragraph{Homotopy Type Theory.}
The Univalent Foundations program~\cite{HoTTBook} interprets types as spaces,
identity types as path spaces, and higher identity types as higher homotopy
groups. In HoTT, the identity type \emph{itself} carries higher structure, and
UIP is rejected. Our approach is complementary: we work \emph{within} a
UIP-satisfying type theory and build the higher structure externally via
rewrite traces.

\paragraph{Types as weak $\omega$-groupoids.}
Lumsdaine~\cite{Lumsdaine10} and van~den~Berg--Garner~\cite{vdBG11}
independently showed that the tower of iterated identity types in intensional
type theory carries the structure of a weak $\omega$-groupoid. Our
\cref{thm:omega-groupoid} (\cref{ch:higher-dimensional}) establishes an
analogous result for computational paths, with the crucial difference that
contractibility begins at dimension~3 rather than dimension~1.

\paragraph{Cubical Type Theory.}
Bezem, Coquand, and Huber~\cite{BeCH14} introduced cubical sets as a
constructive model of HoTT. Computational paths offer an alternative
computational semantics: where cubical paths are built from interval
variables, computational paths are built from explicit rewrite steps.

\paragraph{Higher-Dimensional Rewriting.}
The rewriting-theoretic perspective on higher algebra, developed by Burroni,
M\'etayer, Lafont, and others, treats rewrite rules as generators of higher
cells. Our 75-rule rewrite system on paths (\cref{ch:rewrite-system}) can be
seen as an instance of this paradigm, with the strip lemma and confluence
results providing the coherence data.

\section{Outline of the Paper}
\label{sec:outline}

This paper (Part~I) develops the foundations of the theory of computational
paths in five chapters.

\begin{description}[leftmargin=2em,style=nextline]
  \item[\Cref{ch:basic-constructions}: Basic Constructions.]
    We introduce the fundamental operations on paths---reflexivity, symmetry,
    transitivity, congruence---and establish their strict algebraic laws. We
    develop the path algebra for products, sums, dependent pairs, and function
    types, and define transport and dependent application. We introduce unary,
    binary, and dependent contexts.

  \item[\Cref{ch:rewrite-system}: The Rewrite System.]
    We define the single-step rewrite relation $\Step$ with its 75 rules in
    eight groups, its reflexive--transitive closure $\Rw$, and the rewrite
    equality $\RwEq$. We prove soundness, develop normalization, establish
    termination via a recursive path ordering, prove local confluence
    (the strip lemma) and global confluence via a groupoid-algebraic
    argument, and construct the quotient $\PathQuot$.

  \item[\Cref{ch:groupoid}: The Groupoid of Computational Paths.]
    We show that every type carries a canonical weak groupoid structure
    under computational paths, that the quotient $\PathQuot$ is a strict
    groupoid, and that rewrite lifts provide functorial transport of the
    rewrite structure.

  \item[\Cref{ch:higher-dimensional}: Higher-Dimensional Structure.]
    We define two-cells (rewrite equalities between paths) and establish
    the weak bicategory structure. We construct the globular tower, define
    derivation cells at each dimension, prove contractibility at
    dimension~$\geq 3$, and establish the main structure theorem: computational
    paths form a weak $\omega$-groupoid.
\end{description}

% ============================================================================
% Chapter 2: Computational Paths — Basic Constructions
% ============================================================================
\chapter{Computational Paths: Basic Constructions}
\label{ch:basic-constructions}

Throughout this chapter, $A$, $B$, $C$ denote types in a fixed universe,
and $a, b, c, d$ denote elements of~$A$ unless otherwise stated.

\section{Fundamental Operations}
\label{sec:fundamental-ops}

We equip the space $\Path_A(a,b)$ with three primitive operations.

\begin{definition}[Reflexivity]\label{def:refl}
  The \emph{reflexive path} at $a : A$ is
  \[
    \refl(a) \;=\; ([\,],\; \refl_a) \;:\; \Path_A(a,a),
  \]
  where the step list is empty and the proof field is the reflexivity
  of~$\Eq$.
\end{definition}

\begin{definition}[Symmetry]\label{def:symm}
  For $p = (s, \pi) : \Path_A(a,b)$, the \emph{symmetric path} is
  \[
    \inv{p} \;=\; \symop(p) \;=\; \bigl(\mathrm{reverse}(\mathrm{map}(\mathrm{symm}_{\Step}, s)),\;
    \pi^{-1}\bigr) \;:\; \Path_A(b,a),
  \]
  where $\mathrm{symm}_{\Step}$ inverts each elementary step and
  $\mathrm{reverse}$ reverses the list.
\end{definition}

\begin{definition}[Transitivity / Composition]\label{def:trans}
  For $p = (s_1, \pi_1) : \Path_A(a,b)$ and $q = (s_2, \pi_2) : \Path_A(b,c)$,
  the \emph{composite path} is
  \[
    p \comp q \;=\; \mathrm{trans}(p, q) \;=\;
    (s_1 \mathbin{+\!\!+} s_2,\; \pi_1 \cdot \pi_2) \;:\; \Path_A(a,c),
  \]
  where $\mathbin{+\!\!+}$ denotes list concatenation and $\cdot$ is
  transitivity of~$\Eq$.
\end{definition}

\begin{remark}
  We use $p \comp q$ and $\mathrm{trans}(p, q)$ interchangeably,
  adopting the diagrammatic order: $p$ is followed by~$q$.
\end{remark}

\section{Strict Algebraic Laws}
\label{sec:strict-laws}

A remarkable feature of computational paths is that many algebraic identities
hold as \emph{strict} equalities---i.e., as definitional equalities of the
$\Path$ record structure---not merely up to rewriting. This is a consequence of
the list-based representation: the laws reduce to standard identities on list
concatenation and reversal.

\begin{theorem}[Strict Monoid Laws]\label{thm:strict-monoid}
  For all $p : \Path_A(a,b)$, $q : \Path_A(b,c)$, $r : \Path_A(c,d)$:
  \begin{enumerate}[label=(\roman*)]
    \item \textbf{Left identity.}\; $\refl(a) \comp p = p$.
    \item \textbf{Right identity.}\; $p \comp \refl(b) = p$.
    \item \textbf{Associativity.}\; $(p \comp q) \comp r = p \comp (q \comp r)$.
  \end{enumerate}
  All three equalities hold as structural equalities of $\Path$ records
  (not merely up to rewriting).
\end{theorem}

\begin{proof}
  \begin{enumerate}[label=(\roman*)]
    \item By definition, $\refl(a) \comp p = ([\,] \mathbin{+\!\!+} s,\;
      \refl \cdot \pi) = (s, \pi) = p$, since prepending the empty list
      is the identity on lists, and $\refl \cdot \pi = \pi$.
    \item Similarly, $p \comp \refl(b) = (s \mathbin{+\!\!+} [\,],\;
      \pi \cdot \refl) = (s, \pi) = p$.
    \item Follows from $\mathrm{List.append\_assoc}$: $(s_1 \mathbin{+\!\!+} s_2)
      \mathbin{+\!\!+} s_3 = s_1 \mathbin{+\!\!+} (s_2 \mathbin{+\!\!+} s_3)$
      and associativity of $\Eq.\mathrm{trans}$. \qedhere
  \end{enumerate}
\end{proof}

\begin{theorem}[Strict Involution]\label{thm:strict-involution}
  For all $p : \Path_A(a,b)$:
  \[
    \inv{(\inv{p})} = p.
  \]
\end{theorem}

\begin{proof}
  We compute $\inv{(\inv{p})}$: reversing the reversed list recovers the
  original list, and applying $\mathrm{symm}_{\Step}$ twice to each step
  is the identity (since swapping source and target twice returns to the
  original step). On the proof field, $(\pi^{-1})^{-1} = \pi$.
\end{proof}

\begin{theorem}[Strict Anti-Homomorphism]\label{thm:strict-antihom}
  For all $p : \Path_A(a,b)$ and $q : \Path_A(b,c)$:
  \[
    \inv{(p \comp q)} \;=\; \inv{q} \comp \inv{p}.
  \]
\end{theorem}

\begin{proof}
  By the list identity $\mathrm{reverse}(s_1 \mathbin{+\!\!+} s_2) =
  \mathrm{reverse}(s_2) \mathbin{+\!\!+} \mathrm{reverse}(s_1)$ and
  the fact that mapping commutes with reversal and append.
\end{proof}

\begin{remark}[Cancellation is not strict]\label{rem:cancel-not-strict}
  The cancellation law $p \comp \inv{p} = \refl(a)$ does \emph{not} hold
  strictly: the left-hand side has step list $s \mathbin{+\!\!+}
  \mathrm{reverse}(\mathrm{map}(\mathrm{symm}_{\Step}, s))$, which is
  non-empty whenever $s \neq [\,]$, while $\refl(a)$ has an empty step
  list. Cancellation holds only up to the rewrite relation~$\Step$
  (\cref{ch:rewrite-system}).
\end{remark}

\section{Congruence (Functorial Action)}
\label{sec:congruence}

\begin{definition}[Unary Congruence]\label{def:congrArg}
  For $f : A \to B$ and $p = (s, \pi) : \Path_A(a,b)$, the \emph{congruence}
  (or \emph{functorial action}) of~$f$ on~$p$ is
  \[
    f_*(p) \;=\; \congrArgOp(f, p) \;=\;
    \bigl(\mathrm{map}(\mathrm{map}_f, s),\; \congrArgOp(f, \pi)\bigr)
    \;:\; \Path_B(f(a), f(b)),
  \]
  where $\mathrm{map}_f$ sends a step $\langle x, y, h \rangle$ to $\langle
  f(x), f(y), \congrArgOp(f, h) \rangle$.
\end{definition}

The functorial action satisfies strict algebraic laws:

\begin{theorem}[Functoriality of Congruence]\label{thm:congrArg-functor}
  For $f : A \to B$, $g : B \to C$, and paths $p : \Path_A(a,b)$,
  $q : \Path_A(b,c)$:
  \begin{enumerate}[label=(\roman*)]
    \item \textbf{Composition.}\; $f_*(p \comp q) = f_*(p) \comp f_*(q)$.
    \item \textbf{Symmetry.}\; $f_*(\inv{p}) = \inv{f_*(p)}$.
    \item \textbf{Identity.}\; $\id_*(p) = p$ \;(where $\id = \lambda x.\, x$).
    \item \textbf{Composition of functions.}\;
      $(g \circ f)_*(p) = g_*(f_*(p))$.
  \end{enumerate}
  All equalities are strict.
\end{theorem}

\begin{proof}
  Each part reduces to a standard identity on list operations:
  \begin{enumerate}[label=(\roman*)]
    \item $\mathrm{map}(F, s_1 \mathbin{+\!\!+} s_2) = \mathrm{map}(F, s_1)
      \mathbin{+\!\!+} \mathrm{map}(F, s_2)$.
    \item $\mathrm{map}(F, \mathrm{reverse}(\mathrm{map}(G, s))) =
      \mathrm{reverse}(\mathrm{map}(F \circ G, s))$, combined with the fact
      that $\mathrm{map}_f \circ \mathrm{symm}_{\Step} = \mathrm{symm}_{\Step}
      \circ \mathrm{map}_f$.
    \item $\mathrm{map}_{\id} = \id$ on steps.
    \item $\mathrm{map}_{g \circ f} = \mathrm{map}_g \circ \mathrm{map}_f$
      on steps. \qedhere
  \end{enumerate}
\end{proof}

\begin{corollary}\label{cor:path-functor}
  The assignment $A \mapsto \Path_A$ and $f \mapsto f_*$ is a functor from
  the category of types and functions to the category of types and
  path-preserving maps (strictly preserving composition and identities).
\end{corollary}

\section{Binary Congruence}
\label{sec:binary-congruence}

For binary functions, congruence decomposes into left and right components.

\begin{definition}[Left, Right, and Binary Maps]\label{def:binary-maps}
  Let $f : A \to B \to C$.
  \begin{enumerate}[label=(\roman*)]
    \item \textbf{Left map.}\; $\mapLeft(f, p, b) = (\lambda x.\, f(x,b))_*(p)
      : \Path_C(f(a_1, b), f(a_2, b))$ \\ for $p : \Path_A(a_1, a_2)$
      and $b : B$.
    \item \textbf{Right map.}\; $\mapRight(f, a, q) = f(a)_*(q)
      : \Path_C(f(a, b_1), f(a, b_2))$ \\ for $a : A$ and $q : \Path_B(b_1, b_2)$.
    \item \textbf{Binary map.}\; $\mapTwo(f, p, q) = \mapLeft(f, p, b_1)
      \comp \mapRight(f, a_2, q)$ \\ for $p : \Path_A(a_1, a_2)$ and
      $q : \Path_B(b_1, b_2)$.
  \end{enumerate}
\end{definition}

The binary map first varies the left argument (holding the right argument
at its \emph{source} $b_1$), then varies the right argument (holding the left
argument at its \emph{target} $a_2$). This is the canonical choice that makes
the binary map compose correctly with projections.

The binary map satisfies its own functoriality laws, inherited from those of
the unary congruence. In particular, $\mapTwo(f, -, -)$ preserves composition
in each variable separately:
\[
  \mapTwo(f, p_1 \comp p_2, q_1 \comp q_2) =
  \mapLeft(f, p_1, b_1) \comp \mapLeft(f, p_2, b_1) \comp
  \mapRight(f, a_3, q_1) \comp \mapRight(f, a_3, q_2).
\]
Symmetry of $\mapTwo$ reverses the order of the components:
\[
  \inv{\mapTwo(f, p, q)} = \mapRight(f, a_2, \inv{q}) \comp
  \mapLeft(f, \inv{p}, b_1).
\]

\section{Product Paths}
\label{sec:product-paths}

\begin{definition}[Product Path Operations]\label{def:prod-paths}
  For $p : \Path_A(a_1, a_2)$ and $q : \Path_B(b_1, b_2)$:
  \begin{enumerate}[label=(\roman*)]
    \item $\prodMk(p, q) = \mapTwo(\mathrm{Prod.mk}, p, q)
      : \Path_{A \times B}((a_1, b_1), (a_2, b_2))$.
    \item $\fst(r) = (\mathrm{Prod.fst})_*(r)$ for
      $r : \Path_{A \times B}((a_1, b_1), (a_2, b_2))$.
    \item $\snd(r) = (\mathrm{Prod.snd})_*(r)$ for
      $r : \Path_{A \times B}((a_1, b_1), (a_2, b_2))$.
  \end{enumerate}
\end{definition}

\begin{theorem}[Product $\beta$/$\eta$-Rules]\label{thm:prod-beta-eta}
  The product path operations satisfy:
  \begin{enumerate}[label=(\roman*)]
    \item \textbf{$\beta$-rules.}\;
      $\fst(\prodMk(p, q)) = p$ \;(strictly, by functoriality);\;
      $\snd(\prodMk(p, q))$ reduces to~$q$ via a single rewrite step.
    \item \textbf{$\eta$-rule.}\; $\prodMk(\fst(r), \snd(r)) \rew r$
      \;(as a rewrite step, see \cref{ch:rewrite-system}).
  \end{enumerate}
\end{theorem}

The $\eta$-rule is not a strict equality because $\prodMk(\fst(r), \snd(r))$
applies the binary map construction, producing a different step list than the
original path~$r$.

\section{Sigma Paths}
\label{sec:sigma-paths}

For a dependent type $B : A \to \Type$, paths between dependent pairs
$\langle a_1, b_1 \rangle$ and $\langle a_2, b_2 \rangle$ in $\Sigma_{x:A} B(x)$
decompose into a base path and a fiber path.

\begin{definition}[Sigma Path Operations]\label{def:sigma-paths}
  \begin{enumerate}[label=(\roman*)]
    \item \textbf{Construction.}\; $\sigmaMk(p, q) : \Path_{\Sigma B}(\langle a_1,
      b_1\rangle, \langle a_2, b_2\rangle)$ where $p : \Path_A(a_1, a_2)$
      and $q : \Path_{B(a_2)}(\tr(p, b_1), b_2)$.
    \item \textbf{First projection.}\; $\sigmaFst(r) = (\Sigma.\mathrm{fst})_*(r) :
      \Path_A(a_1, a_2)$.
    \item \textbf{Second projection.}\; $\sigmaSnd(r) : \Path_{B(a_2)}(\tr(\sigmaFst(r),
      b_1), b_2)$.
  \end{enumerate}
\end{definition}

The sigma path operations satisfy analogous $\beta/\eta$-rules to the product
case. These hold as rewrite steps (not strict equalities), since the
$\sigmaMk$ constructor creates a single-step path via $\ofEq$:
\begin{align*}
  \sigmaFst(\sigmaMk(p, q)) &\rew \ofEq(\toEq(p)), \\
  \sigmaSnd(\sigmaMk(p, q)) &\rew \ofEq(\toEq(q)), \\
  \sigmaMk(\sigmaFst(r), \sigmaSnd(r)) &\rew r.
\end{align*}

\section{Sum Paths}
\label{sec:sum-paths}

\begin{definition}[Sum Path Constructors]\label{def:sum-paths}
  For $p : \Path_A(a_1, a_2)$ and $q : \Path_B(b_1, b_2)$:
  \begin{enumerate}[label=(\roman*)]
    \item $\inlOp_*(p) = (\mathrm{Sum.inl})_*(p) :
      \Path_{A + B}(\mathrm{inl}(a_1), \mathrm{inl}(a_2))$.
    \item $\inrOp_*(q) = (\mathrm{Sum.inr})_*(q) :
      \Path_{A + B}(\mathrm{inr}(b_1), \mathrm{inr}(b_2))$.
  \end{enumerate}
\end{definition}

\begin{theorem}[Sum $\beta$-Rules]\label{thm:sum-beta}
  For $f : A \to C$, $g : B \to C$, and the eliminator
  $\mathrm{rec}(f, g) : A + B \to C$:
  \begin{align*}
    (\mathrm{rec}(f,g))_*(\inlOp_*(p)) &\rew f_*(p), \\
    (\mathrm{rec}(f,g))_*(\inrOp_*(q)) &\rew g_*(q).
  \end{align*}
\end{theorem}

\section{Function Paths}
\label{sec:function-paths}

\begin{definition}[Function Path Operations]\label{def:fun-paths}
  Let $f, g : A \to B$.
  \begin{enumerate}[label=(\roman*)]
    \item \textbf{Lambda congruence.}\;
      $\lamCongr(p) : \Path_{A \to B}(f, g)$ where $p : \prod_{x:A}
      \Path_B(f(x), g(x))$ is a family of pointwise paths. The step list
      is empty and the proof field is $\mathrm{funext}(\lambda x.\,
      \toEq(p(x)))$.
    \item \textbf{Application.}\;
      $\app(r, a) = (\lambda h.\, h(a))_*(r) : \Path_B(f(a), g(a))$
      for $r : \Path_{A \to B}(f, g)$ and $a : A$.
  \end{enumerate}
\end{definition}

\begin{theorem}[Function $\beta$/$\eta$-Rules]\label{thm:fun-beta-eta}
  \begin{enumerate}[label=(\roman*)]
    \item \textbf{$\beta$.}\;
      $\app(\lamCongr(p), a) \rew p(a)$.
    \item \textbf{$\eta$.}\;
      $\lamCongr(\lambda x.\, \app(q, x)) \rew q$.
  \end{enumerate}
\end{theorem}

Additional strict identities hold for $\lamCongr$: it preserves composition
and symmetry strictly:
\begin{align*}
  \lamCongr(p) \comp \lamCongr(q) &= \lamCongr(\lambda x.\, p(x) \comp q(x)),\\
  \inv{\lamCongr(p)} &= \lamCongr(\lambda x.\, \inv{p(x)}).
\end{align*}

\section{Transport and Dependent Application}
\label{sec:transport}

\begin{definition}[Transport]\label{def:transport}
  For a type family $D : A \to \Sort$, a path $p : \Path_A(a,b)$, and
  an element $x : D(a)$, the \emph{transport} of $x$ along $p$ is
  \[
    \tr_D(p, x) \;=\; \Eq.\mathrm{rec}(\pi, x) \;:\; D(b),
  \]
  where $\pi = \toEq(p) : a =_A b$. Transport uses only the semantic
  content of~$p$, not its trace.
\end{definition}

\begin{theorem}[Transport Laws]\label{thm:transport-laws}
  The following hold strictly:
  \begin{enumerate}[label=(\roman*)]
    \item $\tr_D(\refl(a), x) = x$.
    \item $\tr_D(p \comp q, x) = \tr_D(q, \tr_D(p, x))$.
    \item $\tr_D(\inv{p}, \tr_D(p, x)) = x$.
    \item $\tr_D(p, \tr_D(\inv{p}, y)) = y$.
  \end{enumerate}
\end{theorem}

\begin{proof}
  All follow from case analysis on the path's proof field.
  Since $\Eq.\mathrm{rec}$ is definitionally the identity when the
  proof is $\refl$, (i) is immediate. Parts (ii)--(iv) follow from
  the corresponding properties of $\Eq.\mathrm{rec}$.
\end{proof}

\begin{definition}[Dependent Application]\label{def:apd}
  For a dependent function $f : \prod_{x:A} D(x)$ and a path
  $p : \Path_A(a,b)$, the \emph{dependent application} is
  \[
    \apd(f, p) \;:\; \Path_{D(b)}(\tr_D(p, f(a)),\; f(b)).
  \]
  When $p = \refl(a)$, this reduces to $\refl(f(a))$.
\end{definition}

\section{Contexts}
\label{sec:contexts}

Contexts formalize substitution of paths into sub-expressions. They are the
categorical analogue of ``evaluation in context'' from term rewriting.

\begin{definition}[Unary Context]\label{def:context}
  A \emph{context} $C : \Context(A, B)$ is a function $\mathrm{fill} : A \to B$.
  It acts on paths by
  \[
    C[p] \;=\; \congrArgOp(\mathrm{fill}, p) \;:\;
    \Path_B(C[a_1], C[a_2])
  \]
  for $p : \Path_A(a_1, a_2)$, where we write $C[a]$ for
  $\mathrm{fill}(a)$.
\end{definition}

\begin{definition}[Context Substitution]\label{def:context-subst}
  Let $C : \Context(A, B)$.
  \begin{enumerate}[label=(\roman*)]
    \item \textbf{Left substitution.}\;
      $\substL(C, r, p) = r \comp C[p] : \Path_B(x, C[a_2])$ \\
      for $r : \Path_B(x, C[a_1])$ and $p : \Path_A(a_1, a_2)$.
    \item \textbf{Right substitution.}\;
      $\substR(C, p, t) = C[p] \comp t : \Path_B(C[a_1], y)$ \\
      for $p : \Path_A(a_1, a_2)$ and $t : \Path_B(C[a_2], y)$.
  \end{enumerate}
\end{definition}

Context substitution supports an extensive algebra of identities
(cf.\ \cref{sec:context-rules} in \cref{ch:rewrite-system}), including:
\begin{itemize}
  \item Unit laws: $\substL(C, \refl, p) \rew C[p]$ and
    $\substR(C, p, \refl) \rew C[p]$.
  \item Associativity: $\substR(C, p, t) \comp u \rew \substR(C, p, t \comp u)$.
  \item Idempotence: $\substL(C, \substL(C, r, \refl), p) \rew
    \substL(C, r, p)$.
  \item $\beta$-rules: $r \comp C[p] \rew \substL(C, r, p)$ and
    $C[p] \comp t \rew \substR(C, p, t)$.
\end{itemize}

\begin{definition}[Binary Context]\label{def:bicontext}
  A \emph{binary context} $K : \BiContext(A, B, C)$ is a function
  $\mathrm{fill} : A \to B \to C$. It supports $\mapLeft$, $\mapRight$,
  and $\mapTwo$ operations obtained by freezing one argument:
  \begin{align*}
    K.\mapLeft(p, b) &= (\lambda x.\, K[x, b])_*(p), \\
    K.\mapRight(a, q) &= K[a, -]_*(q), \\
    K.\mapTwo(p, q) &= K.\mapLeft(p, b_1) \comp K.\mapRight(a_2, q).
  \end{align*}
\end{definition}

\begin{definition}[Dependent Context]\label{def:depcontext}
  A \emph{dependent context} $C : \DepContext(A, B)$ consists of
  $\mathrm{fill} : \prod_{a:A} B(a)$ for a type family $B : A \to \Type$.
  Its action on a path $p : \Path_A(a_1, a_2)$ produces
  \[
    C.\mathrm{map}(p) \;:\; \Path_{B(a_2)}(\tr_B(p, C[a_1]),\; C[a_2]),
  \]
  which is the dependent application $\apd(\mathrm{fill}, p)$.
\end{definition}

Dependent contexts admit analogues of left and right substitution
($\substL$ and $\substR$), with additional transport factors to account
for the dependence of the codomain on the base.

\begin{definition}[Dependent Binary Context]\label{def:depbicontext}
  A \emph{dependent binary context} $K : \DepContext(A, B, C)$, where
  $C : A \to B \to \Type$, consists of $\mathrm{fill} : \prod_{a:A}
  \prod_{b:B} C(a, b)$. It supports $\mapLeft$, $\mapRight$, and
  $\mapTwo$ operations that combine transport in the base with fiber
  paths, generalizing the binary context operations.
\end{definition}

% ============================================================================
% Chapter 3: The Rewrite System
% ============================================================================
\chapter{The Rewrite System}
\label{ch:rewrite-system}

The fundamental algebraic laws of \cref{ch:basic-constructions}---left and
right identity, associativity, involution, anti-homomorphism---hold as strict
equalities because they reduce to list identities. But the \emph{cancellation}
laws ($p \comp \inv{p} = \refl$) and the $\beta/\eta$-rules for type formers
do not hold strictly: they require non-trivial reorganizations of the step
list. We therefore introduce a \emph{rewrite system} on paths that axiomatizes
these additional identities.

\section{The Single-Step Rewrite Relation}
\label{sec:step-relation}

\begin{definition}[Single-Step Rewrite]\label{def:step-rewrite}
  The relation $\rew$ on $\Path_A(a,b)$ is the smallest relation closed under
  the 75 rules organized into eight groups below. We write
  $\Step(p, q)$ or $p \rew q$ to denote that $p$ rewrites to $q$ in one step.
\end{definition}

\subsection{Group I: Path Algebra (8 rules)}
\label{sec:group-i}

These rules express the groupoid laws that are \emph{not} strict equalities.

\begin{enumerate}[label=\textbf{R\arabic*.}, ref=R\arabic*, leftmargin=3.5em]
  \item\label{rule:sr} \textbf{(sr)}
    $\inv{\refl(a)} \rew \refl(a)$.
  \item\label{rule:ss} \textbf{(ss)}
    $\inv{(\inv{p})} \rew p$.
  \item\label{rule:lrr} \textbf{(lrr)}
    $\refl(a) \comp p \rew p$.
  \item\label{rule:rrr} \textbf{(rrr)}
    $p \comp \refl(b) \rew p$.
  \item\label{rule:tr} \textbf{(tr)}
    $p \comp \inv{p} \rew \refl(a)$.
  \item\label{rule:tsr} \textbf{(tsr)}
    $\inv{p} \comp p \rew \refl(b)$.
  \item\label{rule:stss} \textbf{(stss)}
    $\inv{(p \comp q)} \rew \inv{q} \comp \inv{p}$.
  \item\label{rule:tt} \textbf{(tt)}
    $(p \comp q) \comp r \rew p \comp (q \comp r)$.
\end{enumerate}

\begin{remark}\label{rem:strict-vs-step}
  Rules~\ref{rule:lrr}, \ref{rule:rrr}, \ref{rule:ss}, \ref{rule:stss}, and
  \ref{rule:tt} overlap with the strict equalities of
  \cref{thm:strict-monoid,thm:strict-involution,thm:strict-antihom}. Including
  them in the rewrite system is necessary for two reasons: (i)~the
  \emph{structural closure} rules (\cref{sec:group-viii}) may produce these
  patterns nested inside larger contexts; (ii)~the rewrite system must be
  self-contained for the confluence and termination proofs.
\end{remark}

\subsection{Group II: Type-Former $\beta$/$\eta$-Rules (17 rules)}
\label{sec:group-ii}

These rules govern the interaction of path operations with type constructors.

\paragraph{Product rules.}
\begin{enumerate}[label=\textbf{R\arabic*.}, ref=R\arabic*, leftmargin=3.5em, resume]
  \item\label{rule:prod-fst-beta}
    $\fst(\prodMk(p, q)) \rew p$.
  \item\label{rule:prod-snd-beta}
    $\snd(\prodMk(p, q)) \rew q$.
  \item\label{rule:prod-eta}
    $\prodMk(\fst(r), \snd(r)) \rew r$.
  \item\label{rule:prod-mk-symm}
    $\inv{\prodMk(p, q)} \rew \prodMk(\inv{p}, \inv{q})$.
\end{enumerate}

\paragraph{Sigma rules.}
\begin{enumerate}[label=\textbf{R\arabic*.}, ref=R\arabic*, leftmargin=3.5em, resume]
  \item\label{rule:sigma-fst-beta}
    $\sigmaFst(\sigmaMk(p, q)) \rew \ofEq(\toEq(p))$.
  \item\label{rule:sigma-snd-beta}
    $\sigmaSnd(\sigmaMk(p, q)) \rew \ofEq(\toEq(q))$.
  \item\label{rule:sigma-eta}
    $\sigmaMk(\sigmaFst(r), \sigmaSnd(r)) \rew r$.
\end{enumerate}

\paragraph{Sum rules.}
\begin{enumerate}[label=\textbf{R\arabic*.}, ref=R\arabic*, leftmargin=3.5em, resume]
  \item\label{rule:sum-inl-beta}
    $(\mathrm{rec}(f,g))_*(\inlOp_*(p)) \rew f_*(p)$.
  \item\label{rule:sum-inr-beta}
    $(\mathrm{rec}(f,g))_*(\inrOp_*(q)) \rew g_*(q)$.
\end{enumerate}

\paragraph{Function rules.}
\begin{enumerate}[label=\textbf{R\arabic*.}, ref=R\arabic*, leftmargin=3.5em, resume]
  \item\label{rule:fun-app-beta}
    $\app(\lamCongr(p), a) \rew p(a)$.
  \item\label{rule:fun-eta}
    $\lamCongr(\lambda x.\, \app(q, x)) \rew q$.
  \item\label{rule:lam-symm}
    $\inv{\lamCongr(p)} \rew \lamCongr(\lambda x.\, \inv{p(x)})$.
\end{enumerate}

\paragraph{Map congruence rules.}
\begin{enumerate}[label=\textbf{R\arabic*.}, ref=R\arabic*, leftmargin=3.5em, resume]
  \item\label{rule:prod-map-congr}
    For $f = (g, h) : A \times B \to A' \times B'$,\;
    $f_*(\prodMk(p,q)) \rew \prodMk(g_*(p), h_*(q))$.
\end{enumerate}

Additional rules (22--25) govern the interaction of dependent contexts with
symmetry and the decomposition of dependent application.

\subsection{Group III: Transport Rules (7 rules)}
\label{sec:group-iii}

\begin{enumerate}[label=\textbf{R\arabic*.}, ref=R\arabic*, leftmargin=3.5em, start=26]
  \item\label{rule:transport-refl}
    $\tr_D(\refl(a), x) \rew x$ \quad (identity transport).
  \item\label{rule:transport-trans}
    $\tr_D(p \comp q, x) \rew \tr_D(q, \tr_D(p, x))$ \quad (distributivity).
  \item\label{rule:transport-symm-left}
    $\tr_D(\inv{p}, \tr_D(p, x)) \rew x$.
  \item\label{rule:transport-symm-right}
    $\tr_D(p, \tr_D(\inv{p}, y)) \rew y$.
\end{enumerate}

Rules 30--32 address transport through sigma constructors.

\subsection{Group IV: Context Rules (16 rules)}
\label{sec:context-rules}

These rules govern the interaction of context substitution with path operations.
Let $C : \Context(A, B)$.

\paragraph{Unit rules.}
\begin{enumerate}[label=\textbf{R\arabic*.}, ref=R\arabic*, leftmargin=3.5em, start=33]
  \item\label{rule:context-congr}
    $\Step(p, q) \implies \Step(C[p], C[q])$ \quad (context congruence).
  \item\label{rule:context-symm}
    $\inv{C[p]} \rew C[\inv{p}]$ \quad (symmetry through context).
  \item\label{rule:slr}
    \textbf{(slr)}\; $\substL(C, \refl, p) \rew C[p]$.
  \item\label{rule:srr}
    \textbf{(srr)}\; $\substR(C, p, \refl) \rew C[p]$.
\end{enumerate}

\paragraph{Idempotence and cancellation.}
\begin{enumerate}[label=\textbf{R\arabic*.}, ref=R\arabic*, leftmargin=3.5em, resume]
  \item\label{rule:slss}
    \textbf{(slss)}\; $\substL(C, \substL(C, r, \refl), p) \rew \substL(C, r, p)$.
  \item\label{rule:srsr}
    \textbf{(srsr)}\; $\substR(C, p, \substR(C, \refl, t)) \rew \substR(C, p, t)$.
  \item\label{rule:srrrr}
    \textbf{(srrrr)}\; $\substR(C, \refl, \substR(C, p, t)) \rew \substR(C, p, t)$.
\end{enumerate}

\paragraph{$\beta$-rules (folding into substitution form).}
\begin{enumerate}[label=\textbf{R\arabic*.}, ref=R\arabic*, leftmargin=3.5em, resume]
  \item\label{rule:tsbll}
    \textbf{(tsbll)}\; $r \comp C[p] \rew \substL(C, r, p)$.
  \item\label{rule:tsbrl}
    \textbf{(tsbrl)}\; $C[p] \comp t \rew \substR(C, p, t)$.
\end{enumerate}

\paragraph{Associativity.}
\begin{enumerate}[label=\textbf{R\arabic*.}, ref=R\arabic*, leftmargin=3.5em, resume]
  \item\label{rule:tsblr}
    \textbf{(tsblr)}\; $\substL(C, r, p) \comp t \rew r \comp \substR(C, p, t)$.
  \item\label{rule:tsbrr}
    \textbf{(tsbrr)}\; $\substR(C, p, t) \comp u \rew \substR(C, p, t \comp u)$.
\end{enumerate}

\paragraph{Cancellation.}
\begin{enumerate}[label=\textbf{R\arabic*.}, ref=R\arabic*, leftmargin=3.5em, resume]
  \item\label{rule:ttsv}
    \textbf{(ttsv)}\; $C[p] \comp (C[\inv{p}] \comp v) \rew C[p \comp \inv{p}] \comp v$.
  \item\label{rule:tstu}
    \textbf{(tstu)}\; $(v \comp C[p]) \comp C[\inv{p}] \rew v \comp C[p \comp \inv{p}]$.
\end{enumerate}

\subsection{Groups V--VI: Dependent Context and Bi-Context Rules (20 rules)}
\label{sec:groups-v-vi}

Rules 46--60 are the analogues of Group~IV for dependent contexts
$\DepContext(A, B)$, carrying the additional transport data required by the
dependence of the codomain on the base. Rules 61--68 govern the interaction of
binary contexts ($\BiContext$ and dependent binary contexts) with $\mapLeft$,
$\mapRight$, $\mapTwo$, and their structural closure.

\subsection{Group VII: Map Congruence Rules (4 rules)}
\label{sec:group-vii}

\begin{enumerate}[label=\textbf{R\arabic*.}, ref=R\arabic*, leftmargin=3.5em, start=69]
  \item $\mapLeft(f, -, b)$ preserves $\Step$: $p \rew q$ implies
    $\mapLeft(f, p, b) \rew \mapLeft(f, q, b)$.
  \item $\mapRight(f, a, -)$ preserves $\Step$.
  \item $\mapTwo$ distributes: $\mapTwo(f, p, q) \rew
    \mapRight(f, a_1, q) \comp \mapLeft(f, p, b_2)$
    (alternative factorization).
  \item Interaction of $\ofEq$ with map operations.
\end{enumerate}

\subsection{Group VIII: Structural Closure (4 rules)}
\label{sec:group-viii}

The structural closure rules propagate single-step rewrites through the
path constructors, ensuring that the rewrite relation is compatible with
all operations:

\begin{enumerate}[label=\textbf{R\arabic*.}, ref=R\arabic*, leftmargin=3.5em, start=73]
  \item\label{rule:symm-congr}
    \textbf{(symm\_congr)}\;
    $p \rew q \implies \inv{p} \rew \inv{q}$.
  \item\label{rule:trans-congr-left}
    \textbf{(trans\_congr\_left)}\;
    $p \rew q \implies p \comp r \rew q \comp r$.
  \item\label{rule:trans-congr-right}
    \textbf{(trans\_congr\_right)}\;
    $q \rew r \implies p \comp q \rew p \comp r$.
  \item\label{rule:context-congr-closure}
    \textbf{(context\_congr)}\;
    $p \rew q \implies C[p] \rew C[q]$.
\end{enumerate}

\begin{remark}
  Rules~\ref{rule:symm-congr}--\ref{rule:context-congr-closure} make $\rew$
  a \emph{congruence closure}: any rewrite deep inside a path expression can
  be lifted to the top level.
\end{remark}

\section{Soundness}
\label{sec:soundness}

\begin{theorem}[Soundness of Step]\label{thm:step-sound}
  If $p \rew q$ then $\toEq(p) = \toEq(q)$.
\end{theorem}

\begin{proof}
  By induction on the derivation of $\Step(p, q)$. Each of the 75 rules
  preserves the proof field because all path operations are designed to be
  sound with respect to the underlying propositional equality. The structural
  closure rules follow by the induction hypothesis.
\end{proof}

Soundness guarantees that rewriting never changes the \emph{meaning} of a path;
it only reorganizes the computational trace.

\section{Multi-Step Rewriting}
\label{sec:multi-step}

\begin{definition}[Multi-Step Rewrite]\label{def:rw}
  The relation $\rews$ on $\Path_A(a,b)$ is the reflexive--transitive
  closure of $\rew$. Formally, $\Rw$ is the smallest relation satisfying:
  \begin{enumerate}[label=(\roman*)]
    \item $\Rw.\refl(p) : p \rews p$ for all $p$.
    \item $\Rw.\mathrm{tail}(h, s) : p \rews r$ whenever $h : p \rews q$
      and $s : q \rew r$.
  \end{enumerate}
\end{definition}

\begin{corollary}\label{cor:rw-sound}
  If $p \rews q$ then $\toEq(p) = \toEq(q)$.
\end{corollary}

\section{Rewrite Equality}
\label{sec:rweq}

\begin{definition}[Rewrite Equality]\label{def:rweq}
  The \emph{rewrite equality} $\rweq$ is the equivalence relation generated
  by~$\rew$---equivalently, the symmetric closure of $\rews$. It is the
  smallest relation satisfying:
  \begin{enumerate}[label=(\roman*)]
    \item $\RwEq.\refl(p) : p \rweq p$.
    \item $\RwEq.\mathrm{step}(s) : p \rweq q$ whenever $s : p \rew q$.
    \item $\RwEq.\mathrm{symm}(h) : q \rweq p$ whenever $h : p \rweq q$.
    \item $\RwEq.\mathrm{trans}(h_1, h_2) : p \rweq r$ whenever
      $h_1 : p \rweq q$ and $h_2 : q \rweq r$.
  \end{enumerate}
\end{definition}

\begin{theorem}[Congruence Properties of $\RwEq$]\label{thm:rweq-congruence}
  Rewrite equality is a congruence with respect to all path operations:
  \begin{enumerate}[label=(\roman*)]
    \item $p_1 \rweq p_2$ and $q_1 \rweq q_2$ imply
      $p_1 \comp q_1 \rweq p_2 \comp q_2$.
    \item $p \rweq q$ implies $\inv{p} \rweq \inv{q}$.
    \item $p \rweq q$ implies $f_*(p) \rweq f_*(q)$ for any $f$.
    \item $p \rweq q$ implies $C[p] \rweq C[q]$ for any context $C$.
  \end{enumerate}
  Analogous congruence results hold for $\mapLeft$, $\mapRight$, $\mapTwo$,
  $\BiContext.\mapTwo$, $\DepContext.\mathrm{map}$, $\lamCongr$, $\prodMk$,
  and $\sigmaMk$.
\end{theorem}

\begin{proof}
  Each part follows by induction on the $\RwEq$ derivation, using the
  structural closure rules~\ref{rule:symm-congr}--\ref{rule:context-congr-closure}
  in the base case ($\RwEq.\mathrm{step}$).
\end{proof}

The congruence properties ensure that $\RwEq$ is a well-behaved equivalence
relation that respects the algebraic structure of paths.

\begin{theorem}[Groupoid Laws up to $\RwEq$]\label{thm:rweq-groupoid-laws}
  The following hold:
  \begin{enumerate}[label=(\roman*)]
    \item $\refl(a) \comp p \rweq p$ and $p \comp \refl(b) \rweq p$.
    \item $(p \comp q) \comp r \rweq p \comp (q \comp r)$.
    \item $p \comp \inv{p} \rweq \refl(a)$ and $\inv{p} \comp p \rweq \refl(b)$.
  \end{enumerate}
\end{theorem}

\begin{proof}
  Each is a single application of $\RwEq.\mathrm{step}$ to the
  corresponding rule from Group~I.
\end{proof}

\section{Normalization}
\label{sec:normalization}

\begin{definition}[Normal Form]\label{def:normal-form}
  A path $p : \Path_A(a,b)$ is \emph{normal} if $p = \ofEq(\toEq(p))$. The
  \emph{normalization function} is
  \[
    \normalize(p) \;=\; \ofEq(\toEq(p)) \;:\; \Path_A(a,b).
  \]
\end{definition}

Since $\toEq$ extracts the underlying equality proof and $\ofEq$ wraps it in a
single-step path, normalization discards the trace and replaces it with the
canonical one-step witness.

\begin{theorem}[Properties of Normalization]\label{thm:normalization}
  \begin{enumerate}[label=(\roman*)]
    \item $\normalize(p)$ is always normal.
    \item $p \rweq \normalize(p)$ for every path $p$.
    \item Two paths $p, q : \Path_A(a,b)$ are $\RwEq$-equivalent if and only if
      $\normalize(p) = \normalize(q)$.
  \end{enumerate}
\end{theorem}

\begin{proof}
  Part~(i) is immediate from the definition. Part~(ii): by soundness,
  $\toEq(p) = \toEq(\normalize(p))$, and both $p$ and $\normalize(p)$ can be
  connected via the rewrite rules (the groupoid rules and $\beta/\eta$-rules
  suffice to reduce any path to its normal form). Part~(iii): since
  $\normalize(p) = \ofEq(\toEq(p))$ and $\normalize(q) = \ofEq(\toEq(q))$,
  these are equal iff $\toEq(p) = \toEq(q)$, which holds by proof irrelevance
  of $\Eq$. The ``only if'' direction follows from soundness (\cref{thm:step-sound}).
\end{proof}

\begin{corollary}\label{cor:rweq-decidable}
  Rewrite equality of paths is decidable: $p \rweq q$ iff
  $\normalize(p) = \normalize(q)$, which can be checked by structural
  comparison.
\end{corollary}

\section{Termination}
\label{sec:termination}

\begin{theorem}[Termination]\label{thm:termination}
  The rewrite relation $\rews$ is well-founded: there are no infinite
  reduction sequences.
\end{theorem}

The proof uses a \emph{recursive path ordering} (RPO) adapted to the typed
rewriting setting.

\begin{definition}[Rule Precedence]\label{def:rule-precedence}
  The 76 rewrite rules are assigned a numeric rank
  $\mathrm{rank} : \mathrm{Rule} \to \Nat$, forming a well-founded
  precedence relation. The ranking is chosen so that rules introducing
  simpler path expressions (e.g., $\refl$) have lower rank than rules
  producing compound expressions.
\end{definition}

\begin{definition}[RPO Measure]\label{def:rpo}
  Each path $p$ is assigned a \emph{term} $T(p)$ in the RPO, comprising:
  \begin{itemize}
    \item A \emph{symbol} drawn from $\{\mathrm{nf}\} \cup \mathrm{Rule}
      \cup \{\mathrm{pathLen}(n) : n \in \Nat\}$, where $\mathrm{nf}$
      (normal form) is the least element.
    \item An aggregate weight $\mathrm{pathLenSum}(p) \in \Nat$.
  \end{itemize}
  The ordering $T(p) >_{\mathrm{RPO}} T(q)$ holds when the symbol rank
  of~$p$ strictly exceeds that of~$q$ and the aggregate weight does not
  increase.
\end{definition}

\begin{proposition}\label{prop:rpo-wf}
  The RPO ordering is well-founded.
\end{proposition}

\begin{theorem}\label{thm:rpo-decrease}
  Every application of a rewrite rule strictly decreases the RPO measure:
  if $p \rew q$ via rule $R$, then $T(p) >_{\mathrm{RPO}} T(q)$.
\end{theorem}

\begin{proof}
  By case analysis on the 75 rules. Each rule either reduces the symbol
  rank or maintains the rank while strictly decreasing the aggregate weight.
\end{proof}

\section{Confluence}
\label{sec:confluence}

\begin{definition}[Join]\label{def:join}
  A \emph{join} of $q$ and $r$ (where $p \rews q$ and $p \rews r$ for some
  common source~$p$) is a path $m$ together with witnesses $q \rews m$ and
  $r \rews m$.
\end{definition}

\begin{theorem}[Strip Lemma (Local Confluence)]\label{thm:strip-lemma}
  If $p \rew q$ and $p \rews r$, then $q$ and $r$ have a common reduct:
  there exists $m$ with $q \rews m$ and $r \rews m$.
\end{theorem}

\begin{proof}
  By induction on the derivation of $p \rews r$. The base case
  ($r = p$) is trivial. For the inductive case, suppose $p \rews r'$
  and $r' \rew r$. By the induction hypothesis applied to $p \rew q$
  and $p \rews r'$, we obtain a join of $q$ and $r'$ at some $m'$.
  We then perform a \emph{critical pair analysis}: for each pair of
  overlapping rules that could apply to~$r'$, we exhibit an explicit
  join. The analysis covers all pairs among the 75 rules.
\end{proof}

The critical pair analysis is the most technically demanding part of the
confluence proof. Representative cases include:

\begin{itemize}
  \item \textbf{Product $\fst$ overlap.}\;
    When $\fst(\prodMk(p, q))$ can be rewritten by both the $\beta$-rule
    (\ref{rule:prod-fst-beta}) and a structural closure rule, the two
    reducts join at~$p$.

  \item \textbf{Associativity--unit overlap.}\;
    When $((p \comp q) \comp r)$ where $r = \refl$ can be rewritten by
    either \ref{rule:tt}~(associativity) or \ref{rule:rrr}~(right unit),
    the join is $p \comp q$.

  \item \textbf{Context substitution overlap.}\;
    When $\substL(C, r, p) \comp t$ overlaps with the $\beta$-rule
    (\ref{rule:tsbll}) and the associativity rule (\ref{rule:tsblr}),
    the two reducts join at $r \comp \substR(C, p, t)$.
\end{itemize}

\begin{theorem}[Confluence]\label{thm:confluence}
  The rewrite system is confluent: for any paths $p, q, r$ with
  $p \rews q$ and $p \rews r$, there exists $m$ with $q \rews m$
  and $r \rews m$.
\end{theorem}

\begin{proof}
  By Newman's lemma~\cite{Newman42}: a terminating relation is confluent
  if and only if it is locally confluent. Termination is established in
  \cref{thm:termination}, and local confluence follows from the strip
  lemma (\cref{thm:strip-lemma}).
\end{proof}

\begin{corollary}[Unique Normal Forms]\label{cor:unique-nf}
  Every path has a unique normal form (up to structural equality), and
  two paths are $\RwEq$-equivalent if and only if they reduce to the
  same normal form.
\end{corollary}

\begin{proof}
  Existence of normal forms follows from termination. Uniqueness follows
  from confluence: if $p \rews m_1$ and $p \rews m_2$ with $m_1, m_2$
  normal, then by confluence there exists $m$ with $m_1 \rews m$ and
  $m_2 \rews m$; since $m_1$ and $m_2$ are normal, $m_1 = m = m_2$.
\end{proof}

\section{The Quotient $\PathQuot$}
\label{sec:path-quot}

\begin{definition}[Path Quotient]\label{def:path-quot}
  The \emph{path quotient} is the quotient type
  \[
    \PathQuot_A(a, b) \;=\; \Path_A(a, b) \,/\, {\rweq}.
  \]
  We write $[p]$ for the equivalence class of a path $p$.
\end{definition}

Since $\RwEq$ is a congruence (\cref{thm:rweq-congruence}), all path
operations descend to well-defined operations on the quotient:

\begin{theorem}[Well-Defined Quotient Operations]\label{thm:quot-ops}
  The following operations are well-defined on $\PathQuot$:
  \begin{align*}
    \mathrm{trans} &: \PathQuot_A(a, b) \to \PathQuot_A(b, c) \to \PathQuot_A(a, c), \\
    \mathrm{symm} &: \PathQuot_A(a, b) \to \PathQuot_A(b, a), \\
    f_* &: \PathQuot_A(a, b) \to \PathQuot_B(f(a), f(b)).
  \end{align*}
\end{theorem}

\begin{theorem}[Strict Groupoid Laws on the Quotient]\label{thm:quot-groupoid}
  On $\PathQuot$, all groupoid axioms hold as \textbf{strict equalities}
  (equalities of quotient elements):
  \begin{enumerate}[label=(\roman*)]
    \item $[\refl(a)] \comp [q] = [q]$ \quad and \quad $[p] \comp [\refl(b)] = [p]$.
    \item $([p] \comp [q]) \comp [r] = [p] \comp ([q] \comp [r])$.
    \item $[p] \comp [\inv{p}] = [\refl(a)]$ \quad and \quad
      $[\inv{p}] \comp [p] = [\refl(b)]$.
    \item $[\inv{(\inv{p})}] = [p]$.
  \end{enumerate}
\end{theorem}

\begin{proof}
  Each identity holds because the corresponding rewrite rule from Group~I
  provides a witness of $\RwEq$, which becomes an equality after quotienting.
\end{proof}

\begin{theorem}[Equivalence with the Identity Type]\label{thm:quot-equiv}
  The semantic projection $\toEq$ descends to a bijection
  \[
    \PathQuot_A(a, b) \;\cong\; (a =_A b).
  \]
\end{theorem}

\begin{proof}
  By \cref{thm:normalization}(iii), two paths are $\RwEq$-equivalent
  iff they have the same underlying equality proof (which is unique by
  UIP). Hence each equivalence class corresponds to exactly one element
  of the identity type.
\end{proof}

% ============================================================================
% Chapter 4: The Groupoid of Computational Paths
% ============================================================================
\chapter{The Groupoid of Computational Paths}
\label{ch:groupoid}

Having established the rewrite system and its metatheoretic properties, we
now show that the algebraic structure of computational paths gives rise to
categorical structures: a \emph{weak} groupoid on the raw path space, a
\emph{strict} groupoid on the quotient, and functorial transport of the
entire rewrite structure.

\section{Weak Category and Weak Groupoid}
\label{sec:weak-groupoid}

\begin{definition}[Weak Category]\label{def:weak-cat}
  A \emph{weak category} on a type $A$ consists of:
  \begin{itemize}
    \item A composition $\mathrm{comp} : \Path_A(a,b) \to \Path_A(b,c) \to
      \Path_A(a,c)$.
    \item An identity $\mathrm{id} : (a : A) \to \Path_A(a,a)$.
    \item Witnesses (in $\Rw$) of the associativity and unit laws:
      \begin{align*}
        &\mathrm{comp}(\mathrm{comp}(p, q), r) \rews
          \mathrm{comp}(p, \mathrm{comp}(q, r)), \\
        &\mathrm{comp}(\mathrm{id}(a), p) \rews p, \\
        &\mathrm{comp}(p, \mathrm{id}(b)) \rews p.
      \end{align*}
  \end{itemize}
\end{definition}

\begin{theorem}\label{thm:type-weak-cat}
  Every type $A$ carries a canonical weak category with
  $\mathrm{comp} = \mathrm{trans}$ and $\mathrm{id} = \refl$. The unit
  and associativity laws hold via single rewrite steps (hence a fortiori
  via $\Rw$).
\end{theorem}

\begin{proof}
  The three witnesses are provided by rules~\ref{rule:lrr} (left unit),
  \ref{rule:rrr} (right unit), and \ref{rule:tt} (associativity) from
  Group~I of the rewrite system.
\end{proof}

\begin{definition}[Weak Groupoid]\label{def:weak-gpd}
  A \emph{weak groupoid} on $A$ extends a weak category with:
  \begin{itemize}
    \item An inversion $\mathrm{inv} : \Path_A(a,b) \to \Path_A(b,a)$.
    \item Witnesses of the cancellation laws:
      \begin{align*}
        &\mathrm{comp}(\mathrm{inv}(p), p) \rews \mathrm{id}(b), \\
        &\mathrm{comp}(p, \mathrm{inv}(p)) \rews \mathrm{id}(a).
      \end{align*}
  \end{itemize}
\end{definition}

\begin{theorem}\label{thm:type-weak-gpd}
  Every type $A$ is a weak groupoid under computational paths, with
  $\mathrm{inv} = \symop$.
\end{theorem}

\begin{proof}
  The cancellation laws are provided by rules~\ref{rule:tsr}
  ($\inv{p} \comp p \rew \refl(b)$) and~\ref{rule:tr}
  ($p \comp \inv{p} \rew \refl(a)$).
\end{proof}

\begin{remark}\label{rem:weakness}
  The word ``weak'' is used precisely: the groupoid laws hold only up to
  the rewrite relation $\Rw$, not as strict equalities of $\Path$ records.
  The strict equalities of \cref{thm:strict-monoid} (unit and associativity)
  provide even stronger witnesses, but the cancellation laws genuinely
  require the rewrite system.
\end{remark}

\section{Strict Category and Strict Groupoid on the Quotient}
\label{sec:strict-groupoid}

\begin{definition}[Strict Category]\label{def:strict-cat}
  A \emph{strict category} on $A$ is a category in the usual sense: the
  associativity and unit laws hold as equalities (not merely up to~$\Rw$).
\end{definition}

\begin{definition}[Strict Groupoid]\label{def:strict-gpd}
  A \emph{strict groupoid} on $A$ extends a strict category with an
  inversion satisfying the cancellation laws as equalities.
\end{definition}

\begin{theorem}\label{thm:quot-strict-gpd}
  The quotient $\PathQuot_A(-,-)$ carries the structure of a strict groupoid.
\end{theorem}

\begin{proof}
  By \cref{thm:quot-groupoid}, all groupoid axioms hold as equalities of
  quotient elements. The operations $\mathrm{trans}$, $\symop$, and $\refl$
  descend to well-defined operations on $\PathQuot$ by the congruence
  property of $\RwEq$ (\cref{thm:rweq-congruence}).
\end{proof}

This result establishes a clean separation between two levels of structure:

\begin{center}
\begin{tabular}{lll}
  \toprule
  \textbf{Level} & \textbf{Object} & \textbf{Laws} \\
  \midrule
  Raw paths & $\Path_A(a,b)$ & Weak groupoid (laws up to $\Rw$) \\
  Quotient  & $\PathQuot_A(a,b)$ & Strict groupoid (laws as equalities) \\
  \bottomrule
\end{tabular}
\end{center}

The quotient recovers the standard identity type (\cref{thm:quot-equiv}),
while the raw path space carries the additional combinatorial structure
needed for higher-dimensional algebra.

\section{Equality Functors}
\label{sec:eq-functor}

\begin{definition}[Equality Functor]\label{def:eq-functor}
  An \emph{equality functor} from $A$ to $B$ consists of:
  \begin{itemize}
    \item An object map $\mathrm{obj} : A \to B$.
    \item A path map $\mathrm{map} : \Path_A(a,b) \to \Path_B(\mathrm{obj}(a),
      \mathrm{obj}(b))$.
    \item Functoriality witnesses:
      \begin{align*}
        \mathrm{map}(\refl(a)) &= \refl(\mathrm{obj}(a)), \\
        \mathrm{map}(p \comp q) &= \mathrm{map}(p) \comp \mathrm{map}(q).
      \end{align*}
  \end{itemize}
\end{definition}

\begin{proposition}\label{prop:congrArg-functor}
  For any function $f : A \to B$, the pair $(\mathrm{obj} = f,\;
  \mathrm{map} = f_*)$ is an equality functor. The functoriality
  witnesses hold as strict equalities by \cref{thm:congrArg-functor}.
\end{proposition}

\section{Rewrite Lifts}
\label{sec:rewrite-lifts}

A \emph{rewrite lift} transports not only paths but also the rewrite
structure from one type to another.

\begin{definition}[Rewrite Lift]\label{def:rewrite-lift}
  A \emph{rewrite lift} from $A$ to $B$ consists of:
  \begin{itemize}
    \item An object map $\mathrm{obj} : A \to B$.
    \item A path map $\mathrm{mapPath} : \Path_A(a,b) \to
      \Path_B(\mathrm{obj}(a), \mathrm{obj}(b))$.
    \item A step map: $\Step(p, q) \implies \Step(\mathrm{mapPath}(p),
      \mathrm{mapPath}(q))$.
  \end{itemize}
\end{definition}

\begin{theorem}\label{thm:lift-rw-rweq}
  Any rewrite lift transports both $\Rw$ and $\RwEq$:
  \begin{enumerate}[label=(\roman*)]
    \item $p \rews q$ implies $\mathrm{mapPath}(p) \rews \mathrm{mapPath}(q)$.
    \item $p \rweq q$ implies $\mathrm{mapPath}(p) \rweq \mathrm{mapPath}(q)$.
  \end{enumerate}
\end{theorem}

\begin{proof}
  Part~(i) by induction on the $\Rw$ derivation, using the step map at
  each tail step. Part~(ii) by induction on the $\RwEq$ derivation, using
  (i) for the step case and the closure properties of $\RwEq$.
\end{proof}

\begin{proposition}\label{prop:canonical-lifts}
  Each of the following produces a canonical rewrite lift:
  \begin{enumerate}[label=(\roman*)]
    \item Any function $f : A \to B$ (via $\congrArgOp(f, -)$).
    \item Any context $C : \Context(A, B)$ (via $C[-]$).
    \item Any binary context $K : \BiContext(A, B, C)$ with a fixed
      argument (via $K.\mapLeft(-, b)$ or $K.\mapRight(a, -)$).
    \item Any dependent context $C : \DepContext(A, B)$ (via $C.\mathrm{map}$).
  \end{enumerate}
\end{proposition}

\begin{proof}
  In each case, the step map is provided by the corresponding structural
  closure rule (\ref{rule:context-congr-closure}) from Group~VIII.
\end{proof}

Rewrite lifts compose: if $L_1 : A \to B$ and $L_2 : B \to C$ are rewrite
lifts, their composition $L_2 \circ L_1$ is a rewrite lift from $A$ to $C$.
This makes the collection of types with rewrite lifts into a category,
which refines the category of types with equality functors.

% ============================================================================
% Chapter 5: Higher-Dimensional Structure
% ============================================================================
\chapter{Higher-Dimensional Structure}
\label{ch:higher-dimensional}

We now ascend from the one-dimensional algebra of paths to the
higher-dimensional structure that constitutes the central contribution of this
work. Rewrite equalities between paths serve as \emph{two-cells}, and iterated
derivation structures provide cells at every dimension. The resulting tower
forms a weak $\omega$-groupoid, with a sharp contractibility threshold at
dimension~3.

\section{Two-Cells and the Bicategory of Paths}
\label{sec:two-cells}

\begin{definition}[Two-Cell]\label{def:two-cell}
  A \emph{two-cell} between paths $p, q : \Path_A(a,b)$ is a witness of
  rewrite equality:
  \[
    \eta : p \rweq q.
  \]
  Two-cells inhabit $\Prop$ (they are proof-irrelevant), since $\RwEq$ is a
  proposition.
\end{definition}

\begin{definition}[Two-Cell Operations]\label{def:two-cell-ops}
  Two-cells support the following operations:
  \begin{enumerate}[label=(\roman*)]
    \item \textbf{Identity.}\; $\mathrm{id}_p : p \rweq p$ \quad
      (via $\RwEq.\refl$).
    \item \textbf{Vertical composition.}\;
      $\eta \circ_v \theta : p \rweq r$ \quad for $\eta : p \rweq q$ and
      $\theta : q \rweq r$ \quad (via $\RwEq.\mathrm{trans}$).
    \item \textbf{Left whiskering.}\;
      $f \triangleright_L \eta : f \comp g \rweq f \comp h$ \\
      for $f : \Path_A(a,b)$ and $\eta : g \rweq h$ where
      $g, h : \Path_A(b,c)$ \\
      (via the congruence of $\mathrm{trans}$ in its second argument).
    \item \textbf{Right whiskering.}\;
      $\eta \triangleleft_R h : f \comp h \rweq g \comp h$ \\
      for $\eta : f \rweq g$ and $h : \Path_A(b,c)$ \\
      (via the congruence of $\mathrm{trans}$ in its first argument).
    \item \textbf{Horizontal composition.}\;
      $\eta \circ_h \theta : f \comp g \rweq f' \comp g'$ \\
      for $\eta : f \rweq f'$ and $\theta : g \rweq g'$ \\
      (defined as $(\eta \triangleleft_R g) \circ_v (f' \triangleright_L \theta)$).
  \end{enumerate}
\end{definition}

\subsection{Associator and Unitor Two-Cells}

The rewrite rules from Group~I provide canonical two-cells witnessing the
coherence data of a bicategory:

\begin{definition}[Associator]\label{def:associator}
  For composable paths $p : \Path_A(a,b)$, $q : \Path_A(b,c)$,
  $r : \Path_A(c,d)$, the \emph{associator} is the two-cell
  \[
    \alpha_{p,q,r} : (p \comp q) \comp r \;\rweq\; p \comp (q \comp r),
  \]
  given by $\RwEq.\mathrm{step}$ applied to rule~\ref{rule:tt}.
\end{definition}

\begin{definition}[Unitors]\label{def:unitors}
  The \emph{left} and \emph{right unitors} are:
  \begin{align*}
    \lambda_p &: \refl(a) \comp p \;\rweq\; p &
    &\text{(via rule~\ref{rule:lrr})}, \\
    \rho_p &: p \comp \refl(b) \;\rweq\; p &
    &\text{(via rule~\ref{rule:rrr})}.
  \end{align*}
\end{definition}

\subsection{Coherence Laws}

\begin{theorem}[Pentagon Coherence]\label{thm:pentagon}
  For composable paths $p, q, r, s$, the two canonical ways of
  reassociating the four-fold composite agree:
  \[
    \alpha_{p \comp q, r, s} \circ_v \alpha_{p, q, r \comp s}
    \;=\;
    (\alpha_{p,q,r} \triangleleft_R s) \circ_v
    \alpha_{p, q \comp r, s} \circ_v
    (p \triangleright_L \alpha_{q,r,s}).
  \]
  This equality of two-cells holds by proof irrelevance of $\RwEq$.
\end{theorem}

\begin{theorem}[Triangle Coherence]\label{thm:triangle}
  For composable paths $p : \Path_A(a,b)$ and $q : \Path_A(b,c)$:
  \[
    \alpha_{p, \refl(b), q} \circ_v (p \triangleright_L \lambda_q)
    \;=\;
    \rho_p \triangleleft_R q.
  \]
  Again, this holds by proof irrelevance.
\end{theorem}

\begin{theorem}[Interchange Law]\label{thm:interchange}
  For four two-cells $\eta_1, \eta_2, \theta_1, \theta_2$ arranged in a
  $2 \times 2$ grid:
  \[
    (\eta_1 \circ_h \theta_1) \circ_v (\eta_2 \circ_h \theta_2)
    \;=\;
    (\eta_1 \circ_v \eta_2) \circ_h (\theta_1 \circ_v \theta_2).
  \]
  This is the \emph{middle-four interchange}. It holds because both
  sides inhabit the same $\Prop$-valued type.
\end{theorem}

\begin{remark}\label{rem:proof-irrel-coherence}
  The pentagon, triangle, and interchange laws hold trivially (by
  $\mathsf{Subsingleton.elim}$) because two-cells are $\Prop$-valued.
  This is a feature, not a deficiency: it means that all coherence
  conditions at the two-cell level and above are \emph{automatically}
  satisfied. The non-trivial content of the theory lies at the
  one-cell level, where distinct paths encode genuinely different
  computational traces.
\end{remark}

\subsection{The Weak Bicategory and Weak 2-Groupoid}

\begin{definition}[Weak Bicategory]\label{def:weak-bicat}
  A \emph{weak bicategory} consists of:
  \begin{itemize}
    \item 0-cells (objects), 1-cells (morphisms), 2-cells.
    \item Composition and identity at the 1-cell level.
    \item Vertical and horizontal composition, whiskering, identity at
      the 2-cell level.
    \item Associator and unitors as invertible 2-cells.
    \item Pentagon and triangle coherences.
  \end{itemize}
\end{definition}

\begin{theorem}\label{thm:weak-bicat}
  Computational paths form a weak bicategory, with:
  \begin{center}
  \begin{tabular}{ll}
    0-cells: & elements of $A$, \\
    1-cells: & paths $p : \Path_A(a,b)$, \\
    2-cells: & rewrite equalities $\eta : p \rweq q$.
  \end{tabular}
  \end{center}
\end{theorem}

\begin{definition}[Weak 2-Groupoid]\label{def:weak-2-gpd}
  A \emph{weak 2-groupoid} extends a weak bicategory with:
  \begin{itemize}
    \item An inversion $\mathrm{inv}_1$ on 1-cells, with cancellation
      two-cells: $p \comp \inv{p} \rweq \refl(a)$ and $\inv{p} \comp p \rweq \refl(b)$.
    \item An inversion on 2-cells: $\eta : p \rweq q$ implies $\inv{\eta} : q \rweq p$.
  \end{itemize}
\end{definition}

\begin{theorem}\label{thm:weak-2-gpd}
  Computational paths form a weak 2-groupoid, with $\mathrm{inv}_1 = \symop$
  and inversion on 2-cells given by $\RwEq.\mathrm{symm}$.
\end{theorem}

\section{The Globular Tower}
\label{sec:globular-tower}

\begin{definition}[Globular Cell]\label{def:globular-cell}
  A \emph{globular cell} over a type $\beta$ is a triple
  \[
    c = (\mathrm{src}, \mathrm{tgt}, \mathrm{path})
    \quad\text{where}\quad
    \mathrm{src}, \mathrm{tgt} : \beta \quad\text{and}\quad
    \mathrm{path} : \Path_\beta(\mathrm{src}, \mathrm{tgt}).
  \]
  Globular cells carry reflexivity, symmetry, and composition operations
  inherited from $\Path$, satisfying the analogous algebraic laws.
\end{definition}

\begin{definition}[Globular Tower]\label{def:globular-tower}
  The \emph{globular tower} over a type $A$ is defined inductively:
  \begin{align*}
    \mathrm{Level}_0(A) &\;=\; A, \\
    \mathrm{Level}_{n+1}(A) &\;=\; \mathrm{GlobularCell}(\mathrm{Level}_n(A)).
  \end{align*}
  Each level carries $\refl$, $\symop$, and $\mathrm{trans}$ operations,
  as well as a functorial $\mathrm{map}$ operation that sends a function
  $f : A \to B$ to level-wise maps $\mathrm{Level}_n(f) :
  \mathrm{Level}_n(A) \to \mathrm{Level}_n(B)$.
\end{definition}

\begin{proposition}\label{prop:globular-tower-functorial}
  The $\mathrm{map}$ operation on globular levels satisfies:
  \begin{enumerate}[label=(\roman*)]
    \item $\mathrm{map}(\refl(x)) = \refl(\mathrm{map}(x))$.
    \item $\mathrm{map}(\symop(c)) = \symop(\mathrm{map}(c))$.
    \item $\mathrm{map}(\mathrm{trans}(p, q, h)) =
      \mathrm{trans}(\mathrm{map}(p), \mathrm{map}(q), f_*(h))$.
  \end{enumerate}
\end{proposition}

The globular tower provides the \emph{geometric} scaffolding for the
$\omega$-groupoid, but it does not encode the rewrite structure. For that,
we need the derivation cells.

\section{Derivation Cells and the Weak $\omega$-Groupoid}
\label{sec:omega-groupoid}

\subsection{Dimension 2: Derivations Between Paths}

\begin{definition}[Derivation$_2$]\label{def:derivation2}
  A \emph{derivation} (or \emph{type-valued two-cell}) between paths
  $p, q : \Path_A(a,b)$ is an element of the inductive type
  $\Derivation_2(p, q)$ with constructors:
  \begin{enumerate}[label=(\roman*)]
    \item $\Derivation_2.\refl(p) : \Derivation_2(p, p)$.
    \item $\Derivation_2.\mathrm{step}(s) : \Derivation_2(p, q)$ for
      $s : p \rew q$.
    \item $\Derivation_2.\mathrm{inv}(\delta) : \Derivation_2(q, p)$ for
      $\delta : \Derivation_2(p, q)$.
    \item $\Derivation_2.\mathrm{vcomp}(\delta_1, \delta_2) :
      \Derivation_2(p, r)$ for $\delta_1 : \Derivation_2(p, q)$ and
      $\delta_2 : \Derivation_2(q, r)$.
  \end{enumerate}
\end{definition}

Unlike $\RwEq$ (which is $\Prop$-valued), $\Derivation_2$ is
\emph{type-valued}: it carries explicit derivation structure, recording
which steps and closure operations were applied.

\begin{proposition}\label{prop:d2-projects}
  Every $\Derivation_2(p,q)$ projects to an $\RwEq(p,q)$ witness via a
  forgetful map $\Derivation_2(p,q) \to \RwEq(p,q)$. Moreover,
  $\Derivation_2(p,q)$ is inhabited if and only if $p \rweq q$.
\end{proposition}

$\Derivation_2$ supports horizontal operations:

\begin{definition}[Horizontal Operations on $\Derivation_2$]\label{def:d2-horizontal}
  \begin{enumerate}[label=(\roman*)]
    \item \textbf{Left whiskering.}\;
      $f \triangleright_L \delta : \Derivation_2(f \comp g, f \comp h)$
      for $\delta : \Derivation_2(g, h)$.
    \item \textbf{Right whiskering.}\;
      $\delta \triangleleft_R h : \Derivation_2(f \comp h, g \comp h)$
      for $\delta : \Derivation_2(f, g)$.
    \item \textbf{Horizontal composition.}\;
      $\delta_1 \circ_h \delta_2 : \Derivation_2(p \comp q, p' \comp q')$
      for $\delta_1 : \Derivation_2(p, p')$ and $\delta_2 : \Derivation_2(q, q')$.
  \end{enumerate}
\end{definition}

\subsection{Dimension 3: Meta-Steps and Derivations Between Derivations}

\begin{definition}[MetaStep$_3$]\label{def:metastep3}
  A \emph{primitive three-cell} $\mathrm{MetaStep}_3(\delta_1, \delta_2)$
  between derivations $\delta_1, \delta_2 : \Derivation_2(p, q)$ witnesses
  that $\delta_1$ and $\delta_2$ are ``equivalent as derivations.'' The
  constructors include:
  \begin{enumerate}[label=(\roman*)]
    \item \textbf{Groupoid laws for derivations:}
      \begin{align*}
        &\mathrm{vcomp}(\refl(p), \delta) \;\mapsto\; \delta, \\
        &\mathrm{vcomp}(\delta, \refl(q)) \;\mapsto\; \delta, \\
        &\mathrm{vcomp}(\mathrm{vcomp}(\delta_1, \delta_2), \delta_3)
          \;\mapsto\; \mathrm{vcomp}(\delta_1,
          \mathrm{vcomp}(\delta_2, \delta_3)), \\
        &\mathrm{inv}(\mathrm{inv}(\delta)) \;\mapsto\; \delta, \\
        &\mathrm{vcomp}(\mathrm{inv}(\delta), \delta) \;\mapsto\; \refl(q), \\
        &\mathrm{vcomp}(\delta, \mathrm{inv}(\delta)) \;\mapsto\; \refl(p), \\
        &\mathrm{inv}(\mathrm{vcomp}(\delta_1, \delta_2)) \;\mapsto\;
          \mathrm{vcomp}(\mathrm{inv}(\delta_2), \mathrm{inv}(\delta_1)).
      \end{align*}
    \item \textbf{Step coherence:} any two single-step derivations
      $\mathrm{step}(s_1)$ and $\mathrm{step}(s_2)$ for the same
      $p \rew q$ are connected.
    \item \textbf{$\RwEq$-coherence:} any two derivations projecting to
      the same $\RwEq$ witness are connected:
      $\delta_1.\mathrm{toRwEq} = \delta_2.\mathrm{toRwEq} \implies
      \mathrm{MetaStep}_3(\delta_1, \delta_2)$.
    \item \textbf{Bicategorical coherences:} pentagon and triangle for
      derivation-level associators.
    \item \textbf{Whiskering:} whiskering preserves meta-steps.
  \end{enumerate}
\end{definition}

\begin{definition}[Derivation$_3$]\label{def:derivation3}
  $\Derivation_3(\delta_1, \delta_2)$ is the free groupoid generated by
  $\mathrm{MetaStep}_3$: it has constructors $\refl$, $\mathrm{step}$
  (from $\mathrm{MetaStep}_3$), $\mathrm{inv}$, and $\mathrm{vcomp}$.
\end{definition}

\subsection{The Contractibility Theorem}

\begin{theorem}[Contractibility at Dimension $\geq 3$]\label{thm:contract3}
  For any two parallel derivations $\delta_1, \delta_2 : \Derivation_2(p,q)$,
  there exists a three-cell connecting them:
  \[
    \mathrm{contract}_3(\delta_1, \delta_2) \;:\; \Derivation_3(\delta_1, \delta_2).
  \]
\end{theorem}

\begin{proof}
  Since $\RwEq$ is $\Prop$-valued, the projections
  $\delta_1.\mathrm{toRwEq}$ and $\delta_2.\mathrm{toRwEq}$ are equal
  by $\mathsf{Subsingleton.elim}$. The $\RwEq$-coherence constructor of
  $\mathrm{MetaStep}_3$ then produces a primitive three-cell, which lifts
  to $\Derivation_3$ via the $\mathrm{step}$ constructor.
\end{proof}

\begin{corollary}[Loop Contraction]\label{cor:loop-contract}
  Every loop derivation $\delta : \Derivation_2(p, p)$ contracts to the
  identity:
  \[
    \Derivation_3(\delta,\; \Derivation_2.\refl(p)).
  \]
\end{corollary}

\begin{remark}[Critical Design Choice]\label{rem:contract-threshold}
  Contractibility starts at dimension~3, \textbf{not} at dimension~2. At
  dimension~2, $\Derivation_2(p,q)$ is inhabited only when $p \rweq q$;
  it does not connect arbitrary parallel paths. This is essential for
  capturing non-trivial fundamental groups. For example, on the circle
  $S^1$, the loop generator and $\refl$ are \emph{not} connected by a
  two-cell, which is what allows $\pi_1(S^1) \cong \ZZ$.
\end{remark}

\subsection{Dimensions 4 and Beyond}

The pattern continues uniformly:

\begin{definition}[Higher Derivation Cells]\label{def:higher-cells}
  \begin{enumerate}[label=(\roman*)]
    \item $\mathrm{MetaStep}_4$ and $\Derivation_4(\mu_1, \mu_2)$
      for $\mu_1, \mu_2 : \Derivation_3(\delta_1, \delta_2)$, with
      analogous groupoid laws, step coherence, and whiskering.
    \item For $n \geq 5$, $\mathrm{DerivationHigh}_n(c_1, c_2)$
      for $c_1, c_2 : \Derivation_4(\mu_1, \mu_2)$, parametrized by
      dimension.
  \end{enumerate}
\end{definition}

\begin{theorem}[Contractibility at Dimension $\geq 4$]\label{thm:contract4}
  For any two parallel three-cells $\mu_1, \mu_2 : \Derivation_3(\delta_1,
  \delta_2)$:
  \[
    \mathrm{contract}_4(\mu_1, \mu_2) \;:\; \Derivation_4(\mu_1, \mu_2).
  \]
  More generally, for all $k \geq 3$, any two parallel $(k-1)$-cells are
  connected by a $k$-cell.
\end{theorem}

\begin{proof}
  The same argument as \cref{thm:contract3}: the projection of each
  higher cell to its $\Prop$-valued counterpart is unique by proof
  irrelevance, and the coherence constructor lifts this to an explicit
  higher cell.
\end{proof}

\subsection{The Cell Types}

\begin{definition}[Cell Type at Each Dimension]\label{def:cell-types}
  \[
    \mathrm{Cell}_k(A) \;=\;
    \begin{cases}
      A & k = 0, \\
      \Sigma_{a,b : A}\; \Path_A(a,b) & k = 1, \\
      \Sigma_{a,b,p,q}\; \Derivation_2(p,q) & k = 2, \\
      \Sigma_{\ldots}\; \Derivation_3(\delta_1, \delta_2) & k = 3, \\
      \Sigma_{\ldots}\; \Derivation_4(\mu_1, \mu_2) & k = 4, \\
      \Sigma_{\ldots}\; \mathrm{DerivationHigh}_{k-5}(c_1, c_2) & k \geq 5.
    \end{cases}
  \]
\end{definition}

\subsection{The Main Structure Theorem}

\begin{definition}[Weak $\omega$-Groupoid]\label{def:omega-gpd}
  A \emph{weak $\omega$-groupoid} (in the sense of Batanin--Leinster
  \cite{Leinster04}) on a type $A$ consists of:
  \begin{itemize}
    \item Cells at every dimension: $\mathrm{Cell}_k(A)$ for $k \in \Nat$.
    \item Groupoid operations (identity, composition, inversion) at each
      dimension.
    \item Contractibility: for $k \geq 3$, any two parallel $(k-1)$-cells
      are connected by a $k$-cell.
    \item Bicategorical coherence: pentagon and triangle at the
      2-cell level, witnessed as 3-cells.
  \end{itemize}
\end{definition}

\begin{theorem}[Main Structure Theorem]\label{thm:omega-groupoid}
  For any type $A$, the tower
  \[
    A, \quad \Path, \quad \Derivation_2, \quad \Derivation_3, \quad
    \Derivation_4, \quad \ldots
  \]
  forms a weak $\omega$-groupoid with contractibility starting at
  dimension~3.
\end{theorem}

\begin{proof}
  The construction assembles:
  \begin{itemize}
    \item $\mathrm{contract}_3$ (\cref{thm:contract3}) for the
      contractibility at dimension~3.
    \item $\mathrm{contract}_4$ (\cref{thm:contract4}) for dimension~4.
    \item The parametrized $\mathrm{contractHigh}_n$ for dimensions
      $\geq 5$.
    \item The pentagon coherence (\cref{thm:pentagon}) as a 3-cell:
      $\mathrm{MetaStep}_3.\mathrm{pentagon}$ provides the pentagon
      equation between the two canonical reassociation derivations.
    \item The triangle coherence (\cref{thm:triangle}) as a 3-cell:
      $\mathrm{MetaStep}_3.\mathrm{triangle}$ provides the triangle
      equation.
  \end{itemize}
  Groupoid operations at each dimension are given by the constructors
  of the derivation types ($\refl$, $\mathrm{vcomp}$, $\mathrm{inv}$).
\end{proof}

\section{The Infinity-Groupoid Approximation}
\label{sec:infinity-gpd}

\begin{definition}[Coherence at Level $n$]\label{def:coherence-level}
  \[
    \mathrm{CoherenceAt}(A, n) \;=\;
    \begin{cases}
      \top & n \leq 1, \\
      \forall\, \delta_1, \delta_2 : \Derivation_2(p,q).\;
        \Derivation_3(\delta_1, \delta_2) & n = 2, \\
      \forall\, \mu_1, \mu_2 : \Derivation_3(\delta_1,\delta_2).\;
        \Derivation_4(\mu_1, \mu_2) & n = 3, \\
      \forall\, c_1, c_2.\; \mathrm{DerivationHigh}_{n-4}(c_1, c_2) & n \geq 4.
    \end{cases}
  \]
\end{definition}

\begin{theorem}\label{thm:infinity-gpd}
  For any type $A$, the canonical coherence witnesses at every level
  assemble into an $\infty$-groupoid structure:
  \[
    \mathrm{coherenceAt}(A, n) : \mathrm{CoherenceAt}(A, n)
    \quad\text{for all}\; n \in \Nat.
  \]
\end{theorem}

\begin{definition}[$n$-Groupoid Truncation]\label{def:n-truncation}
  The \emph{$n$-truncation} of the $\omega$-groupoid collapses all cells
  above dimension $n$ to the trivial type $\mathbf{1}$:
  \[
    \mathrm{TruncCell}_k(A, n) \;=\;
    \begin{cases}
      \mathrm{Cell}_k(A) & k \leq n, \\
      \mathbf{1} & k > n.
    \end{cases}
  \]
\end{definition}

\begin{theorem}\label{thm:1-truncation}
  The 1-truncation of the $\omega$-groupoid recovers the strict groupoid
  $\PathQuot$ of \cref{thm:quot-strict-gpd}.
\end{theorem}

\section{Double Groupoid and Symmetric Monoidal Structure}
\label{sec:enriched-structures}

The proof-irrelevance of two-cells enables additional algebraic structures
on the computational path space.

\subsection{Double Groupoid}

\begin{definition}[Double Groupoid]\label{def:double-gpd}
  A \emph{double groupoid} is a weak 2-groupoid equipped with an explicit
  interchange law: the two ways of composing a $2 \times 2$ grid of
  two-cells (vertical-then-horizontal vs.\ horizontal-then-vertical)
  are equal.
\end{definition}

\begin{theorem}\label{thm:double-gpd}
  Computational paths form a double groupoid. The interchange law holds
  by proof irrelevance of $\RwEq$.
\end{theorem}

\subsection{Groupoid-Enriched Category}

\begin{definition}[Groupoid-Enriched Category]\label{def:gpd-enriched}
  A \emph{groupoid-enriched category} is a weak bicategory whose
  hom-categories (the categories of 2-cells between fixed 1-cells) are
  groupoids---i.e., all 2-cells are invertible, and the groupoid axioms
  (associativity, units, inverses for vertical composition) hold as
  equalities.
\end{definition}

\begin{theorem}\label{thm:gpd-enriched}
  Computational paths form a groupoid-enriched category. All groupoid
  axioms for 2-cells hold by $\mathsf{Subsingleton.elim}$ on $\RwEq$.
\end{theorem}

\subsection{Symmetric Monoidal Structure}

\begin{definition}[Monoidal Path Algebra]\label{def:monoidal}
  Viewing path composition as a tensor product ($\otimes = \mathrm{trans}$)
  and the reflexive path as the unit ($I = \refl$), the computational path
  space carries a \emph{monoidal structure} with:
  \begin{itemize}
    \item Associator: $\alpha_{p,q,r} : (p \comp q) \comp r \rweq
      p \comp (q \comp r)$.
    \item Left unitor: $\lambda_p : \refl \comp p \rweq p$.
    \item Right unitor: $\rho_p : p \comp \refl \rweq p$.
    \item Pentagon and triangle coherences.
  \end{itemize}
\end{definition}

\begin{definition}[Braiding]\label{def:braiding}
  The \emph{braiding} is provided by the anti-homomorphism of symmetry
  (\cref{thm:strict-antihom}):
  \[
    \beta_{p,q} \;:\; \inv{(p \comp q)} \;\rweq\; \inv{q} \comp \inv{p}.
  \]
  This is a two-cell (rewrite equality) provided by rule~\ref{rule:stss}.
\end{definition}

\begin{theorem}[Symmetric Monoidal Path Algebra]\label{thm:sym-monoidal}
  The computational path space, equipped with the monoidal structure
  and braiding defined above, forms a \emph{symmetric monoidal}
  structure. The braiding satisfies:
  \begin{enumerate}[label=(\roman*)]
    \item $\inv{(\inv{q} \comp \inv{p})} \rweq p \comp q$ \;(inverse braiding).
    \item The hexagon identities (left and right) hold as equalities of
      $\Prop$-valued two-cells.
  \end{enumerate}
  Naturality of the braiding with respect to $\RwEq$ follows from the
  congruence property.
\end{theorem}

\medskip

This concludes Part~I of the paper. In Part~II, we develop the homotopy theory
built on these foundations: fundamental groups, covering spaces, fibrations,
exact sequences, and the computation of $\pi_1$ for standard spaces.

% ============================================================================

\bibliographystyle{alpha}
\begin{thebibliography}{DQOR18}

\bibitem[BG11]{vdBG11}
B.~van~den~Berg and R.~Garner.
\newblock Types are weak $\omega$-groupoids.
\newblock {\em Proc.\ London Math.\ Soc.}, 102(2):370--394, 2011.

\bibitem[BDJ95]{BaezDolan95}
J.~Baez and J.~Dolan.
\newblock Higher-dimensional algebra and topological quantum field theory.
\newblock {\em J.~Math.\ Phys.}, 36(11):6073--6105, 1995.

\bibitem[BCH14]{BeCH14}
M.~Bezem, T.~Coquand, and S.~Huber.
\newblock A model of type theory in cubical sets.
\newblock In {\em 19th International Conference on Types for Proofs and Programs (TYPES 2013)}, LIPIcs~26, pages 107--128, 2014.

\bibitem[DQOR16]{DQOR16}
R.\,J.\,G.\,B.~de Queiroz, A.\,G.~de Oliveira, and A.\,F.~Ramos.
\newblock Propositional equality, identity types, and direct computational paths.
\newblock {\em South Amer.\ J.\ Logic}, 2(2):245--296, 2016.

\bibitem[EH62]{EckHil62}
B.~Eckmann and P.\,J.~Hilton.
\newblock Group-like structures in general categories.\ {I}.
\newblock {\em Math.\ Annalen}, 145:227--255, 1962.

\bibitem[Hat02]{Hatcher02}
A.~Hatcher.
\newblock {\em Algebraic Topology}.
\newblock Cambridge University Press, 2002.

\bibitem[Lei04]{Leinster04}
T.~Leinster.
\newblock {\em Higher Operads, Higher Categories}.
\newblock London Math.\ Soc.\ Lecture Note Ser.~298. Cambridge University Press, 2004.

\bibitem[Lum10]{Lumsdaine10}
P.\,L.~Lumsdaine.
\newblock Weak $\omega$-categories from intensional type theory.
\newblock {\em Logical Methods in Computer Science}, 6(3):1--19, 2010.

\bibitem[May99]{May99}
J.\,P.~May.
\newblock {\em A Concise Course in Algebraic Topology}.
\newblock University of Chicago Press, 1999.

\bibitem[New42]{Newman42}
M.\,H.\,A.~Newman.
\newblock On theories with a combinatorial definition of ``equivalence''.
\newblock {\em Annals of Math.}, 43(2):223--243, 1942.

\bibitem[RDQO18]{RDQO18}
A.\,F.~Ramos, R.\,J.\,G.\,B.~de Queiroz, and A.\,G.~de Oliveira.
\newblock Explicit computational paths.
\newblock {\em South Amer.\ J.\ Logic}, 4(2):441--484, 2018.

\bibitem[UF13]{HoTTBook}
The {Univalent Foundations Program}.
\newblock {\em Homotopy Type Theory: Univalent Foundations of Mathematics}.
\newblock Institute for Advanced Study, 2013.

\end{thebibliography}

\end{document}
