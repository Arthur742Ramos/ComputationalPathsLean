\chapter{Conclusion}

Motivated by seeing the equality between two computational objects as a sequence of rewrites between then, we have proposed in this work a entity known as computational paths. We have also accomplished $3$ main objectives. The first one was the proposal of computational paths as an new entity of type theory. In this proposal, we pointed out the fact that computational paths should be seen as the syntax counterpart of the homotopical paths between terms of a type. We have also proposed a formalization of the identity type using computational paths. The second objective was the proposal of a mathematical structure for a type using computational paths. We have shown that using categorical semantics it was possible to induce a groupoid structure for a type and also a higher groupoid structure, using computational paths and a rewrite system. We have used this groupoid structure to show that computational paths also refutes the uniqueness of identity proofs. The last objective was to formulate and prove the main concepts and building blocks of homotopy type theory. We ended this last objective with a proof of the isomorphism between the fundamental group of the circle and the group of the integers.

The introduction of this work was focused on giving a brief introduction to the foundations of mathematics and a brief explanation why axiomatic set theory, $ZFC$, is not suitable of working as a foundation of computation. Thus, we have said that homotopy type theory is a suitable alternative, since it can be used as a foundation of mathematics and computation at the same time. We have also pointed out that the identity type is the main concept of homotopy type theory, since it is responsible for the homotopical interpretation of a type. Nevertheless, we said that this interpretation is limited to a semantical interpretation, since there is no entity in the syntax of type theory that represents those paths. Thus, we said that the focus of this work is to add such entity to type theory, which is known as computational path.

The second chapter has been focused on introducing the basic concepts of type theory. In this chapter we have introduced the constructions of the main types, showing the formation, introduction, elimination and computation rules. We have also shown the current approach for the identity type using the constructor $J$. We have also shown the use of those rules in practice, showing how one can use $J$ to construct basic types of type theory, such as the symmetry and transitivity properties of the identity type.

The third chapter focused on introducing the basic concepts of category theory. We have said that one of the objectives achieved here was the proposal of a mathematical structure for computational paths. To achieve that, we have used the framework of category theory. Thus, the needed concepts of this theory has been introduced in chapter $3$. We have also shown some concepts of higher category theory, since we use then in chapter $4$ to construct a higher groupoid.

The fourth chapter has been focused on introducing a new entity known as computational paths to the syntax of type theory. To achieve that, we have used concepts developed in chapter $2$ and some basic concepts of $\lambda$-calculus. After introducing the concept of computational paths, we have shown how one can formally define the identity type using this entity. To do that, we show the formation, introduction, elimination and computation rules for the identity type. To make our approach clearer, we have used this newly introduced rules to construct three basic types of equality, the reflexivity, symmetry and transitivity. We also showed that the process of obtaining those constructions was easier than using $J$. After that, we have shown the existence of a rewrite system that establishes equalities between two computational paths. We have also said that this system does that by mapping all possible redundancies that can appear in a computational path. We also pointed out that this system is confluent and terminates. Thus, we have said that it is possible to think of a strong normal form for a computational path. In the sequel, we have used the concepts of chapter $3$ to construct a groupoid model for those computational paths. We also have gone one step further, showing that it is possible to think of a higher groupoid structure. Moreover, we have put this groupoid structure together with the existence of a strong normal form to show that this approach also refutes the uniqueness of identity proofs.

The fifth chapter has been focused on accomplishing the third objective of this work. In this chapter we have developed the main building blocks of homotopy theory using computational paths. Instead of using path-induction in our proofs, we have used our algebra of paths based on our rewrite system. Proceeding that way, we have shown dozens of lemmas and theorems of homotopy type theory. Thus, we have shown that computational paths are capable of developing this theory. We finished this chapter showing one of the most classic proofs of algebraic topology: the fundamental group of the circle is isomorphic to the group of integers.

\section{Future Work}

Since we have successfully accomplished our main objectives, we hope that we have made a strong case for the power of computational paths in homotopy type theory. This approach seems promising, since it has been possible to prove many lemmas and theorems of homotopy type theory. Also, another promising aspect is that our mathematical structure mirrors the one obtained using the traditional identity type.

Thus, with those results in mind, it is possible to obtain further results in two fronts: we can focus on the mathematical interpretation of computational paths. In this work, we have limited our scope to bicategories, but it is possible, in the future, to go even further, adding more dimensions, eventually obtaining an important result: computational paths should be able of inducing a weak $\infty$-groupoid. We expect this fact, since this has been recently achieved using the traditional approach of the identity type \cite{lumsdaine1, Benno}. The other direction is to continue to develop the main concepts and theorems of homotopy theory. We have shown dozens in chapter $5$, but there are many more results. Thus, we could focus on this and use our algebra of computational paths to obtain further results. In fact, based on the concepts developed in chapter 5, we have constructed the fundamental group of many structures, including the M\"obius band, the cylinder, torus and the projective plane. These results appear in a still unpublished work "On the Calculation of Fundamental Groups in Homotopy Type Theory by Means of Computational Paths". A preprint version is available in \cite{Art5}.






