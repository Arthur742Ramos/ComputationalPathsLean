\chapter{Introduction}
\label{chp:introduction}

% \begin{quotation}[]{Poul Anderson}
% I have yet to see any problem, however complicated, which, when looked at in the
% right way, did not become still more complicated.
% \end{quotation}
\let\thefootnote\relax\footnote{This work has been written based on the author's master's thesis \cite{ArtMestrado}, three journal papers, two published \cite{Art3,Ruy1} and a third one still unpublished, and a conference talk \cite{Art4} based on the still unpublished paper.}

In this chapter, we introduce the main objective of this work. First, we show that the axiom system \gls{zfc}, the main theory currently used as foundations of mathematics, is not constructive and thus, modeling it using computers is not practicable. After that, we introduce a theory that can be used as a foundation for computation and mathematics at the same time, homotopy type theory. Then, we introduce the type responsible for this connection, the identity type. Given the importance given to this type, it will be the main entity of this work. The main objective is to show an alternative way of formalizing the identity type, using an entity known as computational path. To do that, we propose a mathematical model to computational paths and prove many results of homotopy type theory using it.


\section{Foundations of Mathematics and ZFC}

One can easily say that the XIX century caused a mathematical revolution. It is mainly responsible for the modern way of mathematical thinking \cite{avigad}. Before this century, mathematical practice was closely related to algorithmic processes. This century was marked by a sharp increase in abstraction in mathematics \cite{avigad}. For example, it was in this century that non-euclidean geometries were proposed. First, the hyperbolic geometry by Nikolai Lobachevsky in 1832 and then elliptic geometry by Bernhard Riemman in 1851.   

Given this high level of abstraction, a natural question has arisen. Where do mathematical objects come from? A mathematical  object already exists and is awaiting for someone to discover it, or is it a creation of human minds? This question divided the mathematical community and was responsible for the creation of a whole new area of research, called philosophy of mathematics. In one hand, there were some mathematicians that defended that there is already an abstract and immutable universe containing all mathematical objects. Thus, a mathematician's job was to discover the objects of this work. This vision is known as Platonism and is currently the main vision of most mathematicians. On the other hand, there were some that believed that a mathematical object is created at the exact moment that it is conceived in the mind of a mathematician. Thus, the objective is constructed by the mathematician. This vision is known as constructivism. Since computers uses algorithmic process  (programs) to obtain results, it is closely related to this constructive view of mathematics.  

The way one does mathematics can depend on the view one chooses to follow. We are going to give an classic example of two proofs \cite{Dummett}, one that is not accepted by constructivists and one that is. Consider the following proposition:

\begin{prop}
There exist two irrational numbers $a$ e $b$ such that ${a^b}$ is a rational number.
\end{prop}

First, we give a proof that is accepted by platonists but not by constructivists.  
\begin{proof}
Consider the number $\sqrt{2}$. It is clearly irrational. Thus,  $\sqrt{2}^{\sqrt{2}}$ is irrational or rational. If it is rational, then we conclude the proof. If not $(\sqrt{2}^{\sqrt{2}})^{\sqrt{2}} = \sqrt{2}^2 = 2$ \cite{Dummett}, also conclude the proof, since $\sqrt{2}^{\sqrt{2}}$ was considered to be irrational.
\end{proof}

This proof shows that one of the two cases solves the problem, but not determine which one does. Thus, this proof is not constructive. But, if one accepts the platonic point of view, this proof is completely acceptable. Consider this alternate proof: 


\begin{proof}
Consider numbers $\sqrt{2}$ and  $\log_{\sqrt{2}} 3$. Since both are irrationals and $\sqrt{2}^{\log_{\sqrt{2}} 3} = 3$, then our proof is complete.
\end{proof}

One can notice that we directly show two irrational numbers that can be used to construct a rational one. Thus, we consider this proof as a constructive one. 

Currently, most mathematicians are platonists, thus they accept proofs like the first one. One of the factors that may be responsible for this is the fact that $ZFC$ is widely accepted as a foundation for mathematics. $ZFC$, a abbreviation for Zermelo-Frankel with choice is an axiomatic theory proposed in 1908 and improved in 1920-1940 \cite{Stanford1}. that gave a mathematical formalization for set theory. It was born out of necessity, since Russel noticed that naive set theory led to a paradox.

In 1901, the philosopher and mathematician Bertrand Russell discovered a paradox that disrupted the mathematical community. In naive set theory, one can construct the following set: {$ R = \{ x| x \notin x\}$. Russel noticed that it led to a paradox  \cite{Stanford2}, if one thinks in the following way: If $R$ is not a member of itself, then by the definition of $R$ it should be a member of itself. But if $R$ is a member of itself, it would contradict directly its own definition. Thus, $R$ cannot exist. Thus, $ZFC$ was proposed to deal with problems like this one.

The main problem is that $ZFC$ is not constructive. Using the axiom of choice, one can show a theorem that states that every set can be well-ordered\cite{introSet}. Nevertheless, one is not able to construct directly a well-ordering for the reals. Therefore, this clearly shows that $ZFC$ admits results without an explicit construction.


\section{The Identity Type}

In this section, we show that it is possible to connect computer science and mathematics using the identity type. The identity type is arguably the most important entity of a theory known as type theory. Type theory is a construct theory proposed by the mathematician Martin-L\"of in 1971\cite{Martin1,Martin2,Martin4}. The fundamental concept of this theory is the concept of type. A type is defined by a description on how to construct and eliminate it \cite{Stanford4}. We show this theory in detail in \textbf{chapter 2}

Given any type $A$, we write $a : A$ to indicate that $a$ is a term of type $A$. The identity type captures the following idea: given any terms $a : A$ and $b : A$ and a proof $p$ that establishes that $ a = b$, then one can say that $p$ is term of the identity type $Id_{A}(a,b)$, i.e., the terms of type $Id_{A}(a,b)$ are proofs that establish that $ a = b$. That way, the identity type gives two main facts: that $a$ is equal to $b$ and why this equality holds.   

A groundbreaking result turned the identity type in one of the most studied topics of type theory: the direct relation between the identity type and homotopy type theory \cite{Vlad1abnt}. This came from the fact that one can semantically interpret a type $A$ as a topological space, the objects $a,b : A$ are seen as point of this space and a term $p : Id_{A}(a,b)$ is seen as a homotopical path between points $a$ and $b$\cite{hott}. This interpretation yielded groundbreaking results. One of the most important result is the fact that it connected homotopy theory with type theory, giving rise to homotopy type theory. Moreover, it raised the possibility of type theory as a foundation of mathematics. Also, since type theory is naturally constructive, it can be used as a foundation for computation.

The distinguished mathematician Vladimir Voevodsky made clear the advantages of using a constructive theory as a foundation for mathematics \cite{Vlad1abnt}. Voevodsky, which is a Fields Medal winner, proposes that the increasingly abstraction of mathematics makes the mathematician prone to committing errors when doing mathematics. To illustrate that, he uses his past experiences: he discovered in 2013 that a paper published by him in 1989 contained errors \cite{Vlad1abnt}. Thus, he argues that if one did mathematics using the help of an automatic theorem checker, errors like this one would not occur anymore. Thus, one practical advantage of homotopy type theory is the fact that all proofs can be modeled and checked by computers.

One of the disadvantages of the identity type is that it can be hard to understand. It is based on the fact that the only canonical proof of equality is the reflexivity, i.e., given a term $a : A$, we have a canonical proof $r(a) : a = a$. This leads to a complex elimination rule that gives rise to an induction known as path induction \cite{hott}. Although beautifully defined, we have noticed that proofs that uses the identity type can be sometimes a little too complex. The elimination rule of the intensional identity type encapsulates lots of information, sometimes making too troublesome the process of finding the reason that builds the correct type.

Inspired by the path-based approach of the homotopy interpretation, we believe that a similar approach can be used to define the identity type in type theory. Our main idea is to add computational paths to the formal syntax of type theory. That way, this new entity would be the syntax counterpart of semantical paths. In the sequel, we shall define formally the concept of a computational path. The main idea, i.e.\ proofs of equality statements as (reversible) sequences of rewrites, is not new, as it goes back to a paper entitled ``Equality in labeled deductive systems and the functional interpretation of propositional equality ", presented in December 1993 at the {\em 9th Amsterdam Colloquium\/}, and published in the proceedings in 1994\cite{Ruy4}.

One of the most interesting aspects of the identity type is the fact that it can be used to construct higher structures. This is a rather natural consequence of the fact that it is possible to construct higher identities. For any $a, b : A$, we have type $Id_{A}(a,b)$. If this type is inhabited by any $p, q:Id_{A}(a,b)$, then we have type $Id_{Id_{A}(a,b)}(p,q)$. If the latter type is inhabited, we have a higher equality between $p$ and $q$\cite{harper1}. This concept is also present in computational paths. One can show the equality between two computational paths $s$ and $t$ by constructing a third one between $s$ and $t$. We show in the sequel a system of rules used to establish equalities between computational paths\cite{Anjo1}. Then, we show that these higher equalities go up to the infinity, forming a $\infty$-globular-set. We also show that computational paths naturally induce a structure known as groupoid. We also go a step further, showing that computational paths are capable of inducing a higher groupoid structure.

After constructing this mathematical model, we also need to show that it is possible to use computational paths to construct concepts of homotopy type theory. To do that,  we investigate well established properties and concepts of the foundations of homotopy type theory. We are interested in the ones connected to the identity type. Our main objective is to show that these properties and theorems are valid in our approach for the identity type based on computational paths. In this sense, we show that one can use computational paths to define and develop concepts of homotopy type theory. We end this work with a proof using computational paths that the fundamental group of a circle is isomorphic to the integers.


\section{Objectives}

In the previous sections, we pointed out the importance of type theory to mathematics and computation. We have also said that the identity type is one of the main concepts of this theory and perhaps the most interesting one. With that in mind, we have said that we want to develop an alternative approach to the identity type, based on the fact that an equality proof can be seen as a sequence of rewrites between two computational objects. In this sense, this work has $3$ main objectives.

The first objective is to formally introduce to type theory an entity known as computational paths and, based on this entity, propose a formulation for the identity type. To do this, we revisit the main concepts of type theory and some important concepts of $\lambda$-calculus. Then, we introduce the notion of computational paths. We present this new entity in the traditional way of defining a type in type theory: we define formation, introduction, elimination and computation rules. We also establish a rewrite system that will work as an algebra of computational paths.


The second objective is to give mathematical meaning to the structure of computational paths. We do this using categorical semantics. Specifically, we are talking about the groupoid structure of a type. We use our computational path entity and the associated rewrite system to show that every type has an induced groupoid associated to it. We also go a step further, showing that it is possible to induce higher structures such as bicategories.

Our final objective is to establish the connection between computational paths and homotopy type theory. We use the theory developed in this work to show that many concepts and proofs of homotopy type theory can be achieved without the use of path-induction, using computational paths instead. In this sense, we show that our approach is capable of producing the main building blocks of homotopy theory. We end this objective with an important proof: we use computational paths to show that the fundamental groupoid of the circle is isomorphic to the group of the integers.

Thus, to achieve the first objective, we have mainly used a computational approach. The second one is mainly a mathematical approach. The third objective is a mix between the previous two.

\section{Structure}

This first chapter was meant as an introduction for this work. We have highlighted the importance of the importance of the identity type in type theory. We have also exposed the main objectives of this work.

The second chapter will be focused on type theory. In this chapter, we will introduce the basic concepts and the difference of definitional and propositional equality. We also show the classic approach for the identity type, showing the formation, introduction, elimination and computation rules. We also show how to use this approach in practice, showing how some basic types can be constructed.

The third chapter will be focused on category theory. We show the basic concepts of this theory and also some concepts of higher category. This chapter is important to understand the results of chapter $4$.

The fourth chapter is of great importance, since it is responsible for objectives $1$ and $2$. In this chapter, we introduce the concept of computational paths and establishes the connection with the identity type. Moreover, we introduce an extremely important rewrite system, responsible for establishing the equalities between computational paths. Thus, we use this system and categorical semantics to show that computational paths induce a mathematical structure known as groupoid. We finish this chapter establishing one fundamental result, the refutation of the uniqueness of identity proofs using computational paths.

The fifth chapter is responsible for objective $3$. In this chapter. we use the theory developed in chapter $4$ to define, construct and prove the main building blocks of homotopy theory. In the process, we prove dozens of lemmas and theorems. We end this chapter with one of the most classic proofs of algebraic topology: we use computational paths and the rewrite system to show that the fundamental group of the circle is isomorphic to the group of the integers.

The sixth chapter is the conclusion of this work. It is a short chapter in which we review and point out all results obtained in this work.
