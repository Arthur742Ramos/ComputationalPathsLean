
% resumo em inglês
\begin{resumo}[Abstract]
 \begin{otherlanguage*}{english}

The current work has three main objectives. The first one is the proposal of computational paths as a new entity of type theory. In this proposal, we point out the fact that computational paths should be seen as the syntax counterpart of the homotopical paths between terms of a type. We also propose a formalization of the identity type using computational paths. The second objective is the proposal of a mathematical structure for a type using computational paths. We show that using categorical semantics it is possible to induce a groupoid structure for a type and also a higher groupoid structure, using computational paths and a rewrite system. We use this groupoid structure to prove that computational paths also refutes the uniqueness of identity proofs. The last objective is to formulate and prove the main concepts and building blocks of homotopy type theory. We end this last objective with a proof of the isomorphism between the fundamental group of the circle and the group of the integers.



   \vspace{\onelineskip}
 
   \noindent 
   \textbf{Key-words}: 	Computational paths. Homotopy type theory. Identity type. Category theory. Term rewrite system. Uniqueness of identity proofs.
 \end{otherlanguage*}
\end{resumo}

% resumo em português
\begin{resumo}[Resumo]

O presente trabalho tem três objetivos principais. O primeiro é propor caminhos computacionais como uma nova entidade da teoria dos tipos. Nessa proposta, indicamos que os caminhos computacionais podem ser vistos como uma contrapartida sint\'atica dos caminhos homot\'opicos entre termos de um mesmo tipo. Tamb\'em propomos uma formaliza\c{c}\~ao do tipo identidade usando caminhos computacionais. O segundo objetivo \'e propor uma estrutura matem\'atica para um tipo usando os caminhos computacionais. Mostramos, usando sem\^antica categ\'orica, que \'e poss\'ivel induzir uma estrutura de grup\'oide de alta ordem para um tipo, utilizando os caminhos computacionais e um sistema de reescrita. Usamos o modelo de grupóide para provar que os caminhos computacionais tamb\'em refutam a unicidade de provas de identidade. O \'ultimo objetivo \'e formular e provar os principais conceitos da teoria homot\'opica dos tipos utilizando caminhos. Finalizamos esse \'ultimo objetivo com uma prova do isomorfismo entre o grupo fundamental do c\'irculo e o grupo dos inteiros.

A proposta principal do trabalho \'e formalizar computacionalmente e matematicamente uma \'algebra de igualdade entre provas, teoremas e objetos computationais. Fortemente baseado em uma \'area recente e promissora da matem\'atica e da computa\c{c}\~ao, a teoria homot\'opica dos tipos, o trabalho introduz igualdades a partir de um conceito computacional conhecido como caminhos computacionais. A partir desses caminhos, dezenas de teoremas e lemas s\~ao provados de forma totalmente original.

O tratamento da igualdade como um tipo na teoria dos tipos origina uma das mais interessantes estruturas dessa teoria, o tipo identidade. A ideia \'e que esse tipo representa uma prova de igualdade entre dois termos de um determinado tipo. O mais interessante \'e que, por fazer parte da pr\'opria teoria, pode-se pensar em at\'e mesmo provas de igualdade entre provas de igualdade. Nesse contexto, as propriedades interessantes desse tipo gerou a descoberta de uma \'area totalmente nova da computa\c{c}\~ao e da matem\`atica, a teoria homot\'opica dos tipos. A formaliza\c{c}\~ao dessa nova toeria gerou resultados te\'oricos extremamente importantes, assim como resultados pr\'aticos: \'e poss\'ivel usar teoria homot\'opica dos tipos para construir checadores autom\'aticos de teoremas. Dada essa import\^ancia, o foco desse trabalho ser\'a desenvolver uma teoria simples para o tipo identidade. O grande problema \'e que, como ser\'a mostrado, o tipo identidade da teoria atual \'e demasiadamente complexo e de dif\'icil utiliza\c{c}\~ao. O presente trabalho ir\'a propor uma forma mais simples de definir e desenvolver esse tipo baseada em conceitos computacionais.

Importante avan\c{c}os te\'oricos em teoria matem\'atica e computacional.
 \vspace{\onelineskip}
    
 \noindent
 \textbf{Palavras-chaves}: Caminhos computacionais. Teoria homot\'opica dos tipos. Tipo identidade.  Teoria das categorias. Sistema de reescrita de termos. Unicidade de provas de identidade.
\end{resumo}


